\section{Hinzuf"ugen neuer ASCII-Formate f"ur Vektordateien}
\label{fdl}

Vistra kann um benutzerdefinierte ASCII-Formate f"ur 
Trainingsvektor-Dateien erweitert werden.
Dazu sind folgende drei Schritte notwendig:

\begin{enumerate}
\item Erzeugen einer ASCII-Datei, die das neue Format in der 
Formatbeschreibungssprache FDL definiert.
Der Name dieser Datei mu"s auf {\tt .fmt} enden.
\item Kopieren der FDL-Datei in das Verzeichnis, das durch die 
Environment-Variable VISTRAFORMATS gegeben ist.
\item Neustarten von Vistra, um das neue Format einzubinden.
\end{enumerate}

\subsection{Die Formatbeschreibungssprache FDL}

Die Beschreibung eines Formats in FDL "ahnelt von der Struktur her 
einer Auspr"agung dieses Formats.

Der erste Unterschied besteht jedoch darin, da"s in der FDL-Beschreibung Angaben
"uber Dimensionen, die Anzahl der Vektoren etc. durch bestimmte
Steuersymbole, den sogenannten {\it Deskriptoren}, ersetzt werden.
Auch f"ur Klassensymbole, Eingabe- und Ausgabevektoren existieren
entsprechende Deskriptoren. \\ 
Der zweite Unterschied ist, da"s Wiederholungen (z.B.~eine Aufz"ahlung
aller Eingabevektoren) durch Schleifenkonstrukte dargestellt werden.

Zur Veranschaulichung soll eine FDL-Beschreibung des LVQ-Formats dienen.
Vistra be\-n"o\-tigt zwar keine FDL-Beschreibung dieses Standard-Formats, da
Lese- und Schreibroutinen festcodiert vorliegen.
Als einf"uhrendes Beispiel f"ur FDL ist sie jedoch sehr geeignet.   

\subsubsection*{Beispiel~1: FDL-Spezifikation des LVQ-Formats}
\begin{samepage} Eine FDL-Beschreibung des LVQ-Formats lautet:
\begin{verbatim}
%e <
{ %i %c < }
\end{verbatim}  
\end{samepage}
Die erste Zeile enth"alt zwei Deskriptoren.
{\tt \%e} gibt an, da"s an dieser Stelle die Dimensionalit"at der 
Eingabevektoren erscheint.
Der zweite Deskriptor {\tt <} bedeutet "`new-line"', d.h. "uberall wo
dieser Deskriptor auftaucht mu"s in der Vektor-Datei eine neue Zeile
beginnen.

Die zweite Zeile der obigen FDL-Beschreibung enth"alt eine 
{\it repetition}-Schleife.
Anfang bzw. Ende der Schleife sind durch {\tt \{} bzw. {\tt \}} markiert.
Der Rumpf besteht aus einer Liste von einem oder mehreren Deskriptoren,
in diesem Fall aus den Deskriptoren {\tt \%i}, {\tt \%c} und {\tt <}.  
Das {\it repetition}-Konstrukt repr"asentiert eine null- oder mehrmalige
Wiederholung der Deskriptoren innerhalb des Rumpfs.
Im Beispiel bedeutet dies, da"s in der Vektor-Datei zun"achst der
erste Eingabevektor gefolgt vom zugeh"origen Klassensymbol erscheint,
anschlie"send der zweite Eingabevektor mit zugeh"origem Klassensymbol
usw. 
Nach jedem Klassensymbol mu"s eine neue Zeile begonnen werden.

Neben Deskriptoren und {\it repetition}-Schleifen gibt es als letztes
FDL-Sprachmittel noch die sogenannten {\it matchOne}-Schleifen.
Eine {\it matchOne}-Schleife steht f"ur das null- oder mehrmalige
Ausw"ahlen einer Alternative.
Sie wird durch die Zeichen {\tt [} und {\tt ]} begrenzt und enth"alt im 
Rumpf eine oder mehrere Listen von Deskriptoren, die die
verschiedenen Alternativen darstellen und durch {\tt |} getrennt sind.
Die {\it matchOne}-Schleife wird verst"andlicher, wenn man sich ein
Beispiel ansieht.

\subsubsection*{Beispiel~2:}
Folgendes Format f"ur eine Vektor-Datei w"are denkbar:
\begin{itemize}
\item jede Zeile enth"alt einen Eingabevektor, einen Ausgabevektor oder
einfach einen Kommentar
\item eine Zeile, die einen Eingabevektor enth"alt, beginnt mit {\tt input:}
und wird gefolgt von den Vektorelementen
\item bei Ausgabevektoren beginnt die Zeile mit {\tt output:}
\item es spielt keine Rolle, ob zuerst die Eingabevektoren oder die
Ausgabevektoren erscheinen. 
Ein- und Ausgabevektoren d"urfen auch beliebig abwechseln.
\end{itemize}  
\begin{samepage}
Das Format w"urde in FDL wie folgt aussehen:
\begin{verbatim}
[ input:  %i < | 
  output: %o < | 
  * 
]
\end{verbatim}
\end{samepage}
Der {\tt *} ist ein Deskriptor, der f"ur einen beliebigen String bis
einschlie"slich dem
n"achsten {\it new-line}-Zeichen steht und im Beispiel~2 somit die
Kommentar-Zeilen repr"asentiert. \\
{\tt \%o} repr"asentiert einen Ausgabevektor. 

Folgende Vektor-Dateien des xor-Problems w"aren zul"assige
Auspr"agungen des Bei\-spiel-Formats~2:

\nopagebreak
\parbox[t]{6cm}{
{\sl Auspr"agung~1} \\[1ex]
{\tt Ausgabe-Muster \\
output: 0 \\
output: 1 \\
output: 1 \\
output: 0.0 \\ 
\\
Eingabe-Muster \\
input: 0.0 0.0 \\
input: 0 1 \\
input: 1.0 0 \\
und noch eins... \\
input: 1 1 }} 
\parbox[t]{6cm}{{\sl Auspr"agung~2} \\[1ex] 
{\tt die ersten 2 abwechselnd \\
input:   0  0 \\
output:  0 \\
input:   0  1 \\
output:  1 \\
die letzten 2 getrennt \\
input:   1  0 \\
input:   1  1 \\
output:  1 \\
output:  0 }
}

Bei der {\it matchOne}-Schleife ist zu beachten, da"s Vistra beim
Einlesen die Alternativen des Rumpfs in der Reihenfolge durchprobiert,
in der sie angegeben sind.
Ein {\tt *}, der eine Kommentarzeile repr"asentiert, sollte daher
stets die letzte Alternative darstellen, da dieser Deskriptor auf
jede Zeile der Vektordatei pa"st.
Im obigen Beispiel w"urden folglich auch die Zeilen mit den Vektoren
als Kommentarzeilen betrachtet und einfach "uberlesen werden.

\subsubsection*{Beispiel~3: das PAT-Format}

Als weiteres Beispiel dient ein Format, das vom SNNS (Stuttgarter
Neuronale Netze Simulator) der Universit"at Stuttgart f"ur die 
Trainingsvektoren verwendet wird. \\
Das PAT-Format kann textuell folgenderma"sen beschrieben werden:

\nopagebreak
\begin{itemize}
\item die Datei gliedert sich in einen Kopf und einen Rumpf
\item die ersten zwei Zeilen des Kopfs k"onnten wie folgt aussehen:
\begin{verbatim}
SNNS pattern definition file V1.4 
generated at Mon May 13 10:22:34 1993 
\end{verbatim}
Die erste Zeile enth"alt eine SNNS-Kennung plus Versionsnummer.
Die zweite Zeile gibt Datum und Uhrzeit an, zu der die Vektordatei
erzeugt wurde. 
\item Anschlie"send kann der Kopf folgende drei Muster enthalten:
\begin{itemize}
\item {\tt No. of patterns :} {\it numberOfPatterns}
\item {\tt No. of input units :} {\it inputDimension}
\item {\tt No. of output units :} {\it outputDimension}
\end{itemize} 
Dabei stehen die kursiv gedruckten W"orter f"ur Zahlen der 
entsprechenden Bedeutung.
\item der Rumpf besteht aus {\it numberOfPatterns} Bl"ocken der 
folgenden Form:
\begin{itemize}
\item ein Kommentar, der mit {\tt \#} beginnt
\item eine Zeile mit {\it inputDimension} Werten, die die Elemente
eines Eingabevektors darstellen 
\item eine Zeile mit {\it outputDimension} Werten, die die Elemente
des zugeh"origen Ausgabevektors darstellen 
\end{itemize}
\end{itemize}

Das xor-Problem k"onnte im PAT-Format wie folgt aussehen:

\nopagebreak
\begin{verbatim}
SNNS pattern definition file V1.4 
generated at Mon May 13 10:22:34 1993 

No. of patterns : 4
No. of input units : 2
No. of output units : 1
#1
0 0
0
#2
0 1
1
#3
1.0 0.0
1.0
#4
1 1
0
\end{verbatim}

Folgende Beschreibung in FDL w"are m"oglich:

\nopagebreak
\begin{verbatim}
SNNS pattern definition file V1.4 * 
generated at * \
[ No. of patterns : %n \ |
  No. of input units : %e \ |
  No. of output units : %a \ |
  *
]
{ #* 
  %i %o <
}
\end{verbatim}

Obige Beschreibung l"a"st durch den Deskriptor {\tt *} zu, da"s hinter
der Versionsnummer in der ersten Zeile noch weitere Zeichen stehen d"urfen.
Der {\tt *} in der zweiten Zeile ersetzt das Datum und die Uhrzeit.
Dabei wurde der Umstand ausgen"utzt, da"s der SNNS die Uhrzeit 
beim Einlesen von PAT-Dateien einfach ignoriert.
Eine Vektordatei, die durch Vistra mithilfe der obigen FDL-Spezifikation
geschrieben wird, kann aus diesem Grund auch ohne Zeitangabe vom SNNS
korrekt eingelesen werden.  

Desweiteren enth"alt obige FDL-Beschreibung von PAT zwei Deskriptoren, die
noch nicht erl"autert wurden.
{\tt \%n} repr"asentiert eine ganze Zahl $\geq 1$, die die Anzahl der 
Trainingsvektoren angibt, die in der Vektor-Datei enthalten sind.

Zur Erl"auterung des Deskriptors {\tt $\backslash$}, 
mu"s man mehr
"uber die Verwendung der FDL-Be\-schreib\-ungen durch Vistra wissen. \\
Vistra benutzt die FDL-Spezifikationen von Formaten sowohl zum Lesen
als auch zum Schreiben von Vektor-Dateien im betreffenden Format.
Dabei wird jeder Deskriptor bzw. jede Schleife als Anweisung f"ur
einen Interpreter betrachtet.
Es existiert jeweils ein Interpreter f"ur das Schreiben und f"ur das Lesen.
Je nach Deskriptor f"uhren die Interpreter unterschiedliche Aktionen 
auf der zu lesenden bzw. zu schreibenden Vektor-Datei aus.

Der Deskriptor {\tt $\backslash$} wird beim Lesen von Vektor-Dateien ignoriert.
Er stellt lediglich beim Schreiben eine Anweisung f"ur den Interpreter
dar.
Sie lautet: "`beginne eine neue Zeile"'.
Der Deskriptor {\tt <} hat beim Schreiben einer Datei genau die gleiche
Wirkung, im Gegensatz zu {\tt $\backslash$} jedoch auch beim Lesen 
eine Bedeutung, 
die wie folgt lautet: "`teste, ob das n"achste Wort in einer neuen Zeile
beginnt; falls nicht, erzeuge einen Fehler"'.

\subsection{Die Syntax von FDL}

Die Syntax von FDL kann in BNF-Notation wie folgt beschrieben werden:

\begin{tabular}[t]{lll}
{\it fdl\_description} & ::= & {\it stmt\_list} \\
{\it stmt\_list} & ::= & {\it fdl\_stmt stmt\_list} $|$ {\it fdl\_stmt} \\
{\it fdl\_stmt} & ::= & {\it descriptor\_list $|$ matchOne\_loop $|$
repetition\_loop} \\
{\it repetition\_loop} & ::= & {\bf '\{'} {\it descriptor\_list} {\bf '\}'} \\
{\it matchOne\_loop} & ::= & {\bf '['} {\it alternatives} {\bf ']'} \\
{\it alternatives} & ::= & {\it descriptor\_list} {\bf '$|$'} 
            {\it alternatives} $|$ {\it descriptor\_list} \\
{\it descriptor\_list} & ::= & {\it descriptor descriptor\_list} $|$
            {\it descriptor} \\
{\it descriptor} & ::= & {\it any\_string} $|$ {\bf '?'} $|$ {\bf '*'} $|$
            {\bf '$<$'} $|$ {\bf '$\backslash$'} $|$ \\ 
& &         {\bf '\%e'} $|$ {\bf '\%E'} $|$ {\bf '\%a'} $|$ {\bf '\%A'} $|$ \\
& &         {\bf '\%i'} $|$ {\bf '\%I'} $|$ {\bf '\%o'} $|$ {\bf '\%O'} $|$ \\
& &         {\bf '\%c'} $|$ {\bf '\%C'} $|$ {\bf '\%n'} $|$ {\bf '\%N'}
\end{tabular}
       
Die Terminale dieser Grammatik sind fettgedruckt.
Um sie besser von den Nichtterminalen in obiger Grammatik zu unterscheiden,
wurden sie zus"atzlich in einfache Anf"uhrungszeichen eingeschlossen, die
jedoch in einer FDL-Beschreibung wegzulassen sind.

\begin{sloppypar}
Eine FDL-Beschreibung besteht aus einer Liste von Deskriptoren, 
{\it repetition-} und {\it matchOne-}Schleifen, wobei die Schleifen
nicht geschachtelt werden d"urfen.
Die Terminale der Grammatik k"onnen durch null oder mehrere Blanks, 
Tabulatoren oder Newlines getrennt sein.
\end{sloppypar}

Das Nichtterminal {\it any\_string} steht f"ur einen beliebigen String,
der kein Terminal enth"alt.
Soll ein String ein Zeichen enthalten, das ein Terminal der Grammatik
ist, so mu"s diesem ein {\tt \%}-Zeichen vorangestellt werden, damit es nicht
als Grammatik-Symbol fehlinterpretiert wird.

\begin{samepage} Hier die Liste aller Zeichenkombinationen, 
die die Terminale der Grammatik als einfaches Textzeichen repr"asentieren:
\begin{verbatim}
%{ = {   %} = }   %[ = [   %| = |   %] = ] 
%? = ?   %* = *   %< = <   %\ = \   %% = % 
\end{verbatim}    
\end{samepage}

\subsection{Zusammenfassung der Deskriptoren}

Tabelle~\ref{descriptors} enth"alt eine vollst"andige Liste aller 
FDL-Deskriptoren.

\begin{table}[thb]
\begin{tabular}{l|p{5.8cm}|p{5.8cm}}
{\bf Deskriptor} & {\bf steht f"ur\ldots} & {\bf geschrieben wird\ldots} \\
\hline
{\it any\_string} & {\it any\_string} & {\it any\_string} \\
{\tt ?} & ein beliebiges Wort &
'? ' \\
{\tt *} & alle Zeichen bis einschlie"slich dem n"achsten Zeilenende & 
'$\backslash$n' \\
{\tt <} & ein Zeilenende & '$\backslash$n' \\
{\tt $\backslash$} & --- (der Deskriptor {\tt $\backslash$} wird beim
Einlesen ignoriert und hat nur f"ur das Schreiben von 
Trai\-nings\-vek\-tor-Dateien eine Bedeutung) & '$\backslash$n' \\
{\tt \%e}, {\tt \%E} & eine ganze Zahl $\geq 1$, die die Dimensionalit"at der
Ein\-gabe\-vek\-toren spezifiziert & die Dimensionalit"at der 
Ein\-gabe\-vektoren \\
{\tt \%a}, {\tt \%A} & eine ganze Zahl $\geq 0$, die die Dimensionalit"at der
Ausgabevektoren spezifiziert. 0 bedeutet, da"s keine Ausgabevektoren
existieren. & die Dimensionalit"at der Ausgabevektoren oder '0', falls 
keine vorhanden \\
{\tt \%i}, {\tt \%I} & einen Eingabevektor. Die Vektorelemente m"ussen durch
ein oder mehrere Newlines, Tabulatoren oder Blanks getrennt sein. 
& ein Eingabevektor \\
{\tt \%o}, {\tt \%O} & einen Ausgabevektor. Die Vektorelemente m"ussen durch
ein oder mehrere Newlines, Tabulatoren oder Blanks getrennt sein.
  & ein Ausgabevektor, falls einer 
existiert, sonst eine Klassennummer. \\
{\tt \%c}, {\tt \%C} & ein Wort, das ein Klassensymbol darstellt. &
ein Klassensymbol, falls eines existiert, sonst eine Klassennummer. \\
{\tt \%n}, {\tt \%N} & eine ganze Zahl $\geq 1$, die die 
Anzahl der Trainingsvektoren
spezifiziert. & die Anzahl der Trainingsvektoren     
\end{tabular}  
\caption{\label{descriptors} Die FDL-Deskriptoren}
\end{table}

Die zweite Spalte der Tabelle gibt an, was durch die jeweiligen Deskriptoren
repr"asentiert wird.
Trifft der Interpreter beim Einlesen einer Trainingsvektor-Datei auf
einen Deskriptor, so wird "uberpr"uft, ob die dem aktuellen Dateizeiger 
folgenden Zeichen der Vektor-Datei zum Deskriptor passen oder nicht.
Passen sie, werden diese Zeichen eingelesen und in
internen Programmvariablen gespeichert (f"ur {\tt \%n} wird z.B. eine
positive ganze Zahl eingelesen, die die Anzahl der Trainingsvektoren
spezifiziert und die die weiteren Aktionen des Interpreters beeinflu"st)
oder die Zeichen werden einfach "uberlesen (z.B. bei {\tt *}).   
Pa"st der Inhalt der Vektor-Datei nicht zum Deskriptor, so wird
eine Fehlermeldung generiert, die die darauf hinweist, da"s die einzulesende
Vektor-Datei ein falsches Format besitzt und die Trainingsvektoren
nicht eingelesen werden k"onnen.

Die dritte Spalte beschreibt, was beim Schreiben einer Trainingsvektor-Datei
anstelle eines Deskriptors ausgegeben wird.

Die beiden n"achsten Abschnitte stellen die Vorgehensweise der
FDL-Interpreter beim Lesen bzw. Schreiben von Trainingsvektor-Dateien
detailliert dar. 

\newpage
\subsection{Der Lese-Interpreter}

Wie bereits erw"ahnt, besteht eine FDL-Beschreibung aus einer 
Folge von {\it Deskriptoren, repetition-}Schleifen und 
{\it matchOne-}Schleifen.
Diese FDL-Konstrukte sind als Anweisungen f"ur einen Interpreter 
zu verstehen, der das Einlesen von Trainingsvektor-Dateien 
steuert, die in einem benutzerdefinierten Format vorliegen. 
Abbildung~\ref{lesen} zeigt ein Programmschema, das die Arbeitsweise 
des Lese-Interpreters vereinfacht darstellt. 

\begin{figure}[ht]
\centerline{\fbox{\parbox{15cm}{
{\bf for each} $fdl\_stmt$ {\bf do} \\
\hspace*{0.5cm} {\bf switch(}$fdl\_stmt${\bf )} \\
\hspace*{1cm} $fdl\_stmt = desc$: \\
\hspace*{2cm} {\bf if} Deskriptor $desc$ pa"st auf den 
Vektordatei-Inhalt {\bf then} \\ 
\hspace*{2.5cm} lese alle Zeichen ein, die durch den Deskriptor abgedeckt 
werden \\
\hspace*{3cm} (siehe Spalte~2 der Tabelle~\ref{descriptors}); \\
\hspace*{2.5cm} speichere dabei gelesene Vektoren, Dimensionsangaben etc.; \\
\hspace*{2.5cm} "uberpr"ufe die Konsistenz der Angaben und erzeuge ggf. \\
\hspace*{3cm} eine Fehlermeldung; \\
\hspace*{2cm} {\bf else} Fehler \\
\hspace*{1cm} $fdl\_stmt = \{\ desc_{1}\ desc_{2}\ \ldots\ desc_{n}\ \}$: \\ 
\hspace*{2cm} $exitLoop$ := FALSE; \\
\hspace*{2cm} {\bf while not} $exitLoop$ {\bf do} \\
\hspace*{2.5cm} {\bf if} $desc_{1}$ pa"st auf den Vektordatei-Inhalt
{\bf then} \\ 
\hspace*{3cm} lese $desc_{1}, \ldots , desc_{n}$ ein; \\
\hspace*{3cm} {\bf if} alle Vektoren, Dimensionsangaben oder Klassen 
eingelesen {\bf then} \\
\hspace*{3.5cm} $exitLoop$ := TRUE; \\  
\hspace*{2.5cm} {\bf else} $exitLoop$ := TRUE; \\
\hspace*{2cm} {\bf endwhile} \\ 
\hspace*{1cm} $fdl\_stmt = [\ alt_{1}\ |\ alt_{2}\ |\ \ldots\ |\ alt_{n}\ ]$: \\ 
\hspace*{2cm} {\bf while} der erste Deskriptor hinter der Schleife 
(also nach '$]$') nicht \\
\hspace*{2cm} zum Vektordatei-Inhalt pa"st {\bf do} \\
%\hspace*{2.5cm} $altMatched$ := FALSE; \\
%\hspace*{2.5cm} $i$ := 1; \\
%\hspace*{2.5cm} {\bf while not} $altMatched$ {\bf and} $i \leq n$ {\bf do} \\
\hspace*{2.5cm} vergleiche zuerst $alt_1$, dann $alt_2$ etc. mit dem 
Vektordatei-Inhalt; \\ 
\hspace*{2.5cm} lese erste Deskriptor-Liste $alt_i$ ein, die pa"st; \\ 
\hspace*{2.5cm} {\bf if} keine Alternative pa"st {\bf then} Fehler; \\
\hspace*{2cm} {\bf endwhile} \\ 
\hspace*{0.5cm} {\bf endswitch} \\    
{\bf endfor} \\
"uberpr"ufe, ob \\
\hspace*{0.5cm} -- die Vektor-Datei vollst"andig gelesen wurde \\
\hspace*{0.5cm} -- mindestens ein Eingabevektor gelesen wurde \\
\hspace*{0.5cm} -- die Angabe "uber die Anzahl der Trainingsvektoren (\%n) mit
der \\
\hspace*{1cm} Anzahl gelesener Eingabevektoren "ubereinstimmt \\
\hspace*{0.5cm} -- Klassensymbole oder Ausgabevektoren gelesen wurden \\
\hspace*{0.5cm} -- die Anzahl gelesener Klassensymbole bzw. Ausgabevektoren
mit der Anzahl von \\ 
\hspace*{1cm} Eingabevektoren "ubereinstimmt 
(sofern Klassen bzw. Ausgabevektoren vorhanden) \\
Trifft einer dieser Punkte nicht zu, so erzeuge einen Fehler 
}}}
\caption{\label{lesen} Die Arbeitsweise des Lese-Interpreters}
\end{figure}

Konsistenzpr"ufungen finden nicht nur zum Schlu"s statt, sondern 
auch w"ahrend des Interpretierens von Deskriptoren, die mit {\tt \%} beginnen.
Diese Deskriptoren repr"asentieren Daten, die von Vistra intern gespeichert
werden.
Wird eine Konsistenzbedingung verletzt, so wird der Lesevorgang abgebrochen
und eine Fehlermeldung erzeugt, die die Art der Konsistenzverletzung
spezifiziert und die Zeile in der Trainingsvektor-Datei angibt, in der
der Fehler festgestellt wurde. 

Beim Interpretieren von {\tt \%e} bzw. {\tt \%a} wird "uberpr"uft, ob die
Angaben "uber die Dimensionalit"at der Eingabevektoren bzw. der Ausgabevektoren
mit fr"uher gelesenen Werten f"ur {\tt \%e} bzw. {\tt \%a} "ubereinstimmen.
Au"serdem werden die Angaben auch noch mit der Dimensionalit"at bereits
gelesener Eingabe- bzw. Ausgabevektoren verglichen.

Wird "uber {\tt \%n} die Anzahl der Trainingsvektoren eingelesen, so
wird dieser Wert mit evtl. vorher gemachten Angaben verglichen.
Au"serdem wird sichergestellt, da"s die Anzahl bisher gelesener 
Vektoren oder Klassen diesen Wert nicht "ubersteigt.

Beim Einlesen von Vektoren "uber die Deskriptoren {\tt \%i} oder {\tt \%o}
sind zwei F"alle zu unterscheiden.
Ist die Dimensionalit"at $d$ der Vektoren bereits bekannt (weil 
bereits ein entsprechender Vektor gelesen wurde oder bereits eine
{\tt \%e}- bzw. {\tt \%a}-Angabe gemacht wurde), so werden die n"achsten $d$
durch Whitespace getrennte Zahlen gelesen.
Ist die Dimensionalit"at unbekannt, so werden alle aufeinanderfolgenden
Zahlen der Vektor-Datei gelesen bis eine Zeichenfolge angetroffen wird,
die keine Zahl darstellt.  

%\newpage
\subsection{Der Schreib-Interpreter} 

Wie f"ur das Einlesen existiert auch f"ur das Schreiben von Trainingsvektoren
ein FDL-Interpreter.

Sehr einfach beschrieben, arbeitet der Interpreter wie folgt:

\nopagebreak
\begin{itemize}
\item F"ur einen {\it Deskriptor} wird eine Folge von Zeichen in die
Ausgabedatei geschrieben (siehe dritte Spalte der Tabelle~\ref{descriptors}).
\item Eine {\it repetition-}Schleife wird solange wiederholt, bis
kein Deskriptor des Schleifenrumpfs neue Daten mehr liefert.
Unter neuen Daten sind dabei bisher noch nicht geschriebene Eingabevektoren
({\tt \%i}), Ausgabevektoren ({\tt \%o}) oder Klassen ({\tt \%c}) zu verstehen 
bzw. bisher noch nicht 
geschriebene Angaben "uber die Anzahl von Trainingsvektoren ({\tt \%n}),
Eingabe-Dimensionen ({\tt \%e}) oder Ausgabe-Dimensionen ({\tt \%a}). 
\item Eine {\it matchOne-}Schleife wird solange wiederholt, 
bis in einem Schleifendurchlauf keine Alternative mehr geschrieben wurde.
Dabei werden die Alternativen (Deskriptor-Listen) in der Reihenfolge, in 
der sie aufgef"uhrt sind, geschrieben, sofern sie neue Daten liefern.       
\end{itemize}

Abbildung~\ref{schreiben} enth"alt eine vereinfachte Darstellung der
Arbeitsweise des Schreib-Inter\-pre\-ters in Form eines Programmschemas.

\begin{figure}[ht]
\centerline{\fbox{\parbox{15cm}{
{\bf for each} $fdl\_stmt$ {\bf do} \\
\hspace*{0.5cm} {\bf switch(}$fdl\_stmt${\bf )} \\
\hspace*{1cm} $fdl\_stmt = desc$: \\
\hspace*{2cm} schreibe Deskriptor $desc$ gem"a"s Spalte~3 der
Tabelle~\ref{descriptors}; \\
\hspace*{1cm} $fdl\_stmt = \{ desc_{1}\ desc_{2}\ \ldots\ desc_{n}\ \}$: \\
\hspace*{2cm} {\bf while} Liste $desc_{1} \ldots desc_{n}$ neue Daten
liefert {\bf do} \\
\hspace*{2.5cm} schreibe $desc_{1} \ldots desc_{n}$ gem"a"s Spalte~3 der
Tabelle~\ref{descriptors}; \\
\hspace*{2cm} {\bf endwhile} \\ 
\hspace*{1cm} $fdl\_stmt = [\ alt_{1}\ |\ alt_{2}\ |\ \ldots\ |\ alt_{n}\ ]$: \\ 
\hspace*{2cm} {\bf do} \\
\hspace*{2.5cm} $descriptor\_written =$ FALSE; \\
\hspace*{2.5cm} {\bf for} $alt := alt_{1}$ {\bf to} $alt_{n}$ {\bf do} \\
\hspace*{3cm} {\bf if} $alt$ liefert neue Daten {\bf then} \\
\hspace*{3.5cm} schreibe die Deskriptoren von $alt$ gem"a"s Spalte~3 der
Tabelle~\ref{descriptors}; \\
\hspace*{3.5cm} $descriptor\_written$ := TRUE; \\
\hspace*{3cm} {\bf endif} \\ 
\hspace*{2.5cm} {\bf endfor} \\
\hspace*{2cm} {\bf while} $descriptor\_written$ = TRUE; \\
\hspace*{0.5cm} {\bf endswitch} \\    
{\bf endfor} \\
"Uberpr"ufe, ob alle Eingabevektoren, Ausgabevektoren und Klassen geschrieben
wurden; \\
gebe ggf. eine entsprechende Warnung aus \\
}}}
\caption{\label{schreiben} Die Arbeitsweise des Schreib-Interpreters}
\end{figure}





