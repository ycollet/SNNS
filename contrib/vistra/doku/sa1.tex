\section{Einf"uhrung}
\label{einf}

%Neuronale Netze stellen eine Alternative zu "`klassischen"' symbolischen
%oder numerischen Verfahren dar, die f"ur geeignete Aufgabenstellungen
%einfachere und bessere Ergebnisse liefern kann.
%Einsatzbereiche sind z.B. die Mustererkennung oder das Verstehen und
%Verarbeiten von Bildern.
%Charakteristisch f"ur neuronale Netze ist, da"s die L"osung eines Problems
%nicht {\sl programmiert}, sondern anhand von Daten {\sl gelernt} wird.

%Jedes neuronale Netz besteht aus einer Menge von Neuronen bzw. Zellen,
%die untereinander durch ein gewichtetes Netzwerk verbunden sind.
%Jedes Neuron nimmt dabei zu jedem Zeitpunkt einen bestimmten 
%Aktivierungszustand an und besitzt einen Ausgabewert, der sich aus
%dem Aktivierungszustand durch Anwendung einer zum Neuron geh"orenden
%Ausgabefunktion berechnet. 

%Neuronen k"onnen in Eingabezellen, Ausgabezellen und "`versteckte"'
%Zellen eingeteilt werden.  
%Das neuronale Netz wird {\sl trainiert}, indem Eingabemuster 
%an die Eingabezellen
%angelegt werden (d.h. die Aktivierungen der Eingabezellen gesetzt werden)
%und in mehreren Schritten die Aktivierungen aller Zellen aus 
%den Ausgabewerten ihrer Vorg"anger sowie der 
%Gewichte der zugeh"origen Verbindungen neu berechnet werden.
%Anschlie"send werden die Werte der Ausgabezellen mit den gew"unschten
%Werten verglichen und die Gewichte der Verbindungen gem"a"s einer
%Lernregel in Abh"angigkeit von den beobachteten Abweichungen modifiziert.
%Das neuronale Netz {\sl lernt}.

Aufgabe der Studienarbeit war der Entwurf und die Implementierung eines
Programms, mit dem Eingabe- und Ausgabemuster f"ur neuronale Netze
transformiert und visualisiert werden k"onnen.
Als Eingabemuster ist hierbei ein Vektor zu verstehen, dessen Elemente
die Aktivierungen der Eingabezellen des neuronalen Netzes repr"asentieren.
Analog ist ein Ausgabemuster ein Vektor der Ausgabewerte der 
Ausgabezellen.
Trainings- bzw. Testdaten eines Netzes sind also
eine Menge von Eingabevektoren mit den zugeh"origen Ausgabevektoren.
Zur Spezifikation der Ausgabe werden zus"atzlich zu oder anstelle von den
Ausgabevektoren Klassen eingesetzt, die durch Nummern oder Klassensymbole
repr"asentiert werden.

Bevor Trainings- oder Testdaten in einem Simulator f"ur neuronale
Netze (wie z.B. KNet oder SNNS -- Stuttgarter Neuronale Netze Simulator --
der Universit"at Stuttgart) benutzt werden, ist es h"aufig w"unschenswert,
sich die Daten zun"achst graphisch oder textuell anzusehen und 
ggf.~einen oder mehrere Vorverarbeitungsschritte auf sie anzuwenden.
Desweiteren ben"otigen Simulatoren f"ur neuronale Netze zum Teil
sehr unterschiedliche Dateiformate f"ur die einzulesenden bzw.
zu schreibenden Trainingsvektoren.

Im Rahmen dieser Studienarbeit wurde das Programm "`Vistra"' entwickelt, das
\begin{itemize}
\item Trainingsdaten unterschiedlicher Formate lesen und schreiben
kann (Konvertierungsfunktion),
\item Trainingsdaten textuell und graphisch darstellen kann
(Visualisierungsfunktion),
\item Trainingsdaten transformieren und statistisch
auswerten kann (Manipulationsfunktion) sowie
\item um neue ASCII-Formate f"ur Trainingsvektor-Dateien erweitert werden 
kann mittels Vistra's Format-Beschreibungssprache FDL.
\end{itemize}
Durch die Konvertierungsfunktion ist Vistra vollkommen unabh"angig 
vom verwendeten Simulator f"ur neuronale Netze. 
Das Programm kann ganz allgemein zur Visualisierung und Manipulation von
Trainingsvektoren verwendet werden.

Auch eine Trennung von Trainings- und Testdaten wird unterst"utzt,
indem alle Transformationen, die auf einer Menge von Vektoren
ausgef"uhrt werden, in einer LOG-Datei festgehalten werden k"onnen.
Zu einem sp"ateren Zeitpunkt k"onnen die gleichen Transformationen 
dann mit einer anderen Menge von Trainingsvektoren wiederholt werden.
Zu diesem Zweck besitzt Vistra neben der interaktiven Benutzeroberfl"ache
auch noch einen Batch-Modus, in dem die in der LOG-Datei registrierten
Transformationen ausgef"uhrt werden k"onnen. 

Vistra wurde mit dem Gnu-C-Compiler (GCC) implementiert und verwendet
X-Windows (X11 Release 5) sowie das Athena Widget Set zur Realisierung
der graphischen Benutzer\-oberfl"ache. 
Vistra kann auf UNIX-Workstations der Fabrikate SUN und DEC compiliert
werden.
