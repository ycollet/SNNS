\section{Datei-Formate}
\label{formate}

Dieses Kapitel beschreibt, welche Formate die LVQ-, N01- und 
Symboltabellen-Dateien haben bzw. haben
m"ussen, wenn sie von Vistra geschrieben bzw. gelesen werden
(siehe auch \cite{bayer}).
Das Format der LOG-Dateien wurde bereits in Kapitel~\ref{log} erl"autert.
Formatbeschreibungs-Dateien sind Thema des Kapitels~\ref{fdl}, 
in der die Syntax von FDL (Format Description Language) beschrieben
wird und einige Beispiele f"ur benutzerdefinierte ASCII-Dateiformate
gegeben werden.

\subsection{Das LVQ-Format}
\label{lvq}

Das LVQ-Format ist ein sehr einfaches Trainingsvektor-Format, das eine Menge von
Eingabevektoren enth"alt und jedem dieser Vektoren ein Klassensymbol
zuordnet.

Das LVQ-Format setzt sich aus einem Kopf und einem Rumpf zusammen.
Der Kopf besteht lediglich aus einer Zeile, die die Dimensionalit"at der
Eingabevektoren enth"alt.
Der Rumpf besitzt pro Eingabevektor eine Zeile, in der die Elemente
des jeweiligen Vektors gefolgt von seinem Klassensymbol aufgef"uhrt sind.

Folgendes Beispiel zeigt ein LVQ-File, das zweidimensionale
Trainingsvektoren enth"alt, die das xor-Problem definieren:

\nopagebreak
\begin{verbatim}
2
0 0 NULL
0 1 EINS
1.0 0.0 EINS
1 1.0 NULL
\end{verbatim}
   
\subsection{Das N01-Format}
\label{n01}
Das N01-Format ist ein bin"ares Format, das geschaffen wurde, um das
Einlesen bzw. Schreiben von Trainingsvektor-Dateien zu beschleunigen.
Bei sehr gro"sen Vektor-Dateien ist es daher empfehlenswert, 
anstelle eines ASCII-Formats dieses Format zu verwenden.

Die ersten 32 Bytes stellen den Kopf des N01-Formats dar.
Er hat folgende Gestalt:

\nopagebreak
\begin{tabular}[t]{l|l|l}
{\bf Bytes} & {\bf C-Typ} & {\bf Inhalt} \\ \hline
0~--~3 & char[4] & String ''N01'' mit abschlie"sender '$\backslash0$' \\
4~--~7 & unsigned int & Dimensionalit"at der Eingabevektoren \\
8~--~11 & unsigned int & Dimensionalit"at der Ausgabevektoren \\
12~--~15 & unsigned int & Anzahl der Trainingsvektoren \\
16~--~19 & unsigned int & Offset der Eingabedaten \\
20~--~23 & unsigned int & Offset der Ausgabedaten \\
24~--~27 & unsigned int & Offset der Klassennummern \\
28~--~31 & unsigned int & Anzahl unterschiedl.~Klassen 
\end{tabular}  

Die Offsets geben die File-Positionen innerhalb der N01-Datei an, an denen
die Eingabevektoren, Ausgabevektoren oder Klassennummern beginnen.
Die Eingabe- bzw. Ausgabevektoren sind als Felder von aufeinanderfolgenden
Flie"skommazahlen von jeweils 4 Bytes (C-Typ: {\tt float}) gespeichert.
In diesen Feldern erscheinen zun"achst alle Elemente des ersten 
Eingabe- bzw. Ausgabevektors, gefolgt von den Elementen des zweiten etc.
Die Klassennummern sind in einem Feld aufeinanderfolgender
Integer von jeweils 4 Bytes L"ange untergebracht (C-Typ: {\tt unsigned int}).
Die L"ange des Felds entspricht der Anzahl der Trainingsvektoren.

Das N01-Format wurde urspr"unglich f"ur DEC Workstations eingef"uhrt,
weshalb deren Maschinencodierung von {\tt float} bzw. {\tt unsigned} Typen
verwendet wird.
Hierbei erscheint das niederwertigste Byte als erstes,
das h"oherwertigste als letztes Byte innerhalb der Datei.

Vistra kann N01-Formate jedoch in jedem Fall lesen und schreiben
(unabh"angig davon, ob es auf einer SUN oder einer DEC l"auft).
Das Programm stellt fest, ob die Maschinencodierung des Rechners, auf dem
es l"auft, mit der DEC-Codierung "ubereinstimmt.
Ist dies nicht der Fall (wie bei den Maschinen der Marke SUN), so
werden alle Daten nach dem Einlesen bzw. vor dem Schreiben der N01-Datei
in geeigneter Weise konvertiert. 
    
\subsection{Das Format der Symboltabellen}
\label{symtab}

Symboltabellen-Dateien haben eine Zeilenstruktur.
Jede Zeile enth"alt genau ein Symbol, wobei 
die $i$-te Zeile der Datei das Klassensymbol zur Klassennummer $i$ enth"alt.

Abh"angig davon, ob die geladenen Trainingsvektoren explizite
Klassensymbole besitzen (die durch Strings repr"asentiert werden) oder nicht,
ist der Inhalt der einzelnen Zeilen verschieden.

Existieren explizite Klassensymbole, so werden diese als 
Zeileninhalte verwendet.
Wurde z.B.~ein LVQ-File wie das aus Kapitel~\ref{lvq} geladen, so existieren
solche Klassensymbole. \\ 
\begin{samepage} Die Symboltabellen-Datei s"ahe wie folgt aus:
\begin{verbatim}
NULL
EINS
\end{verbatim}
\end{samepage}

Existieren keine expliziten Klassensymbole, so werden die Ausgabevektoren
als solche betrachtet.
Identische Vektoren stellen dasselbe Klassensymbol dar.
Eine Zeile der Sym\-bol\-tabel\-len-Datei besteht dann gerade aus den Elementen
eines Ausgabevektors und die
Anzahl der Zeilen der Symboltabelle entspricht genau der Anzahl der
unterschiedlichen Ausgabevektoren.
Die Ausgabevektoren werden in der Reihenfolge ausgegeben, in der sie 
zum ersten Mal in den Trainingsvektoren auftauchen.

\begin{samepage}
Eine Symboltabelle f"ur Trainingsvektoren mit zweidimensionalen 
Ausgabevektoren, die keine expliziten Klassensymbole besitzen, k"onnte
wie folgt aussehen:
\begin{verbatim}
0.1 0.9
0.9 0.9
0.9 0.1
0.1 0.1
\end{verbatim}
\end{samepage}
