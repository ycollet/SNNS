\section{Batch-Modus und LOG-Dateien}
\label{log}

Der Aufruf von Vistra im Batch-Modus wurde bereits in Abschnitt~\ref{call}
behandelt.
Dieses Kapitel beschreibt, wie LOG-Dateien zur Spezifikation eines 
durchzuf"uhrenden Jobs zu verwenden sind.

Eine LOG-Datei stellt eine Liste von Kommandos zur Transformation
von Trainingsvektoren dar.
Alle Transformationen, die interaktiv durchgef"uhrt werden, k"onnen 
mithilfe von LOG-Dateien auch im Stapelbetrieb durchgef"uhrt werden.
F"ur jede Transformation steht daf"ur ein entsprechendes Batch-Kommando
zur Verf"ugung.
Der Batch-Modus von Vistra erh"alt als Eingabe eine Trainingsvektor-Datei
und eine LOG-Datei, arbeitet die Befehle der LOG-Datei der Reihe nach
ab und schreibt die transformierten Trainingsvektoren wieder in eine
Ausgabe-Datei.  

Dem Batch-Modus fallen zwei Aufgaben zu:
\begin{itemize}
\item Transformationen k"onnen sowohl interaktiv "uber die X-Windows 
Oberfl"ache als auch als Job im Stapelbetrieb durchgef"uhrt werden.
Dazu ist eine LOG-Datei von Hand zu erstellen.
\item Der Batch-Modus erm"oglicht eine Trennung in
Trainings- und Testdaten oder ganz allgemein eine Aufteilung von
Trainingsvektoren in separate Dateien.
Die Transformationen, die interaktiv auf einen Satz der Trainingsvektoren 
angewandt werden, k"onnen in einer LOG-Datei festgehalten
werden, indem das Flag "`Write to log file"' des "`File"'-Men"us
gesetzt wird.
Sp"ater k"onnen die anderen Dateien durch Angabe dieser LOG-Datei 
im Batch-Modus auf die gleiche Weise umgewandelt werden.
\end{itemize}  
   
\subsection{Das Format von LOG-Dateien}

LOG-Dateien besitzen eine Zeilenstruktur.
Jede Zeile enth"alt eine Anweisung bzw. eine durchzuf"uhrende 
Transformation und besteht
aus ein oder mehreren W"ortern bzw. Zahlen, die
durch Blanks oder Tabulatoren getrennt sind.
Die Syntax einer Befehlszeile ist:

{\tt ('i'|'I'|'o'|'O') <command> [ <arg1> <arg2> ... ]}  

oder

{\tt \% <comment>}

Das erste Wort einer Zeile mu"s also immer ein einzelnes Zeichen sein.
Ein {\tt i} bzw. {\tt I} bedeutet, da"s die Anweisung auf die Eingabevektoren
anzuwenden ist, ein {\tt o} bzw. {\tt O}, da"s sie sich auf die
Ausgabevektoren bezieht. 
Generell wird in LOG-Dateien nicht zwischen Gro"s- und
Kleinschreibung unterschieden. \\  
{\tt \%} leitet eine Kommentarzeile ein. 
Alle Zeichen dieser Zeile werden von Vistra ignoriert.   

{\tt <command>} steht f"ur ein Schl"usselwort, das angibt, welche
Aktion bzw. Transformation durchzuf"uhren ist.

Manche Anweisungen ben"otigen zur genaueren Spezifikation einen oder
mehrere Parameter, die in obiger Syntax durch die Liste 
{\tt <arg1> <arg2> ...} repr"asentiert werden. 
Parameter k"onnen sowohl Strings als auch Zahlen sein.
 
F"ur jede Aktion, die im interaktiven Modus
Trainingsvektoren ver"andert, steht ein entsprechendes Batch-Kommando
zur Verf"ugung.
Neben Kommandos f"ur die Transformationen des "`Transform"'-Men"us und
des "`Remove"'-Men"us"' existieren auch f"ur das Laden der vertikalen
und horizontalen Skalarfelder (also f"ur jeden Men"upunkt des "`Load"'-Men"us)
sowie f"ur die Vektor--Skalar Operationen ("`add"', "`subtract"', "`multiply"',
"`divide"' und "`replace\ldots"') entsprechende Batch-Anweisungen.
W"ahrend eine LOG-Datei abgearbeitet wird, merkt sich Vistra neben den
aktuellen Werten der Trainingsvektoren auch die Inhalte der vertikalen
und horizontalen Skalarfelder zu den verschiedenen Zeitpunkten.

{\bf Beispiel:} \\
Mithilfe einer LOG-Datei sollen die Elemente der Trainingsvektoren auf
den Wertebereich [-1,1] skaliert werden.
Anschlie"send ist von jedem Vektorelement der statistische Mittelwert
der zugeh"origen Dimension abzuziehen. 
Zum Schlu"s soll noch eine Hauptachsentransformation (PCA) durchgef"uhrt
werden. 

\begin{samepage}
Diese Transformationen k"onnten im Batch-Modus durch folgende LOG-Datei
vollzogen werden:
\begin{verbatim}
i scale -1 1
% Mittelwerte der Dimensionen sollen 0 sein
i loadHoriz avg
i subtract horiz
% zum Schluss: Hauptachsentransformation
i pca
\end{verbatim}  
\end{samepage}
 
Tabelle~\ref{batchcommands} zeigt alle Batch-Anweisungen plus zugeh"orige
Parameter.

\begin{table}
\begin{tabular}{c|c|p{7.5cm}}
{\tt <command>} & {\tt <arg1> <arg2> ...} & {\bf Aktion} \\ \hline
{\tt pca} & & Hauptachsentransformation \\
{\tt fft} & & 1-dimensionale diskrete Fourier Trans\-for\-ma\-tion \\
{\tt hlog} & & halblogarithmische Transformation \\
{\tt scale} & {\it min\quad max} & skaliere aufs Intervall 
[{\it min}, {\it max}] \\
{\tt normalize} & & normalisiere auf L"ange 1 \\
{\tt randomize} & & mische die Vektoren willk"urlich \\
{\tt classExpand} & & erweitere mit Klassenvektor \\
{\tt outputExpand} & & erweitere mit Ausgabevektor \\
{\tt refreshClasses} & & berechne die Klassen aus den
Aus\-ga\-be\-vek\-toren \\
{\tt rmDimRange} & {\it from\quad to} & entferne die Dimensionen {\it from} 
bis {\it to} \\
{\tt rmConstDims} & & entferne alle konstanten Dimensionen \\
{\tt rmDims} & $dim_1 \ldots dim_n$ & entferne die Dimensionen 
$dim_1, \ldots, dim_n$ \\ 
{\tt loadVert} & {\it loadparams} & lade die vertikalen Skalarfelder 
je nach {\it loadparams} mit bestimmten Werten (siehe unten) \\
{\tt loadHoriz} & {\it loadparams} & lade die horizontalen Skalarfelder 
je nach {\it loadparams} mit bestimmen Werten (siehe unten) \\
{\tt add} & {\it direction} & addiere die vertikalen 
({\it direction} = {\tt vert})
bzw. die horizontalen ({\it direction} = {\tt horiz}) Skalare zu den 
zugeh"origen Vektoren bzw. Dimensionen \\
{\tt subtract} & {\it direction} & subtrahiere die vertikalen 
({\it direction} = {\tt vert}) bzw. die horizontalen 
({\it direction} = {\tt horiz}) 
Skalare von den zugeh"origen Vektoren bzw. Dimensionen \\
{\tt multiply} & {\it direction} & multipliziere die vertikalen 
({\it direction} = {\tt vert}) bzw. die horizontalen 
({\it direction} = {\tt horiz}) 
Skalare mit den zugeh"origen Vektoren bzw. Dimensionen \\
{\tt divide} & {\it direction} & dividiere die Vektoren 
({\it direction} = {\tt vert})
bzw. die Dimensionen ({\it direction} = {\tt horiz}) durch die zugeh"origen 
vertikalen bzw. horizontalen Skalare \\
{\tt replace} & {\it direction\quad nr} & ersetze den Vektor Nummer~{\it nr} 
({\it direction} = {\tt horiz}) bzw. die Dimension~{\it nr} 
({\it direction} = {\tt vert}) durch die horizontalen 
bzw. vertikalen Skalare \\
{\tt addConst} & {\it num} & addiere {\it num} zu jedem Vektorelement \\
{\tt subConst} & {\it num} & subtrahiere {\it num} von jedem Vektorelement \\
{\tt multConst} & {\it num} & multipliziere jedes Vektorelement mit 
{\it num} \\
{\tt divConst} & {\it num} & dividiere jedes Vektorelement durch {\it num} \\ 
\end{tabular}
\caption{\label{batchcommands} Die Batch-Anweisungen}
\end{table}
  
Durch die Batch-Anweisungen {\tt loadVert} und {\tt loadHoriz} k"onnen
die vertikalen bzw. horizontalen Skalarfelder mit Werten geladen werden.
Unter anderem stehen alle Funktionen zur Verf"ugung, die auch interaktiv "uber
das "`Load"'-Men"u aufgerufen werden k"onnen. 

{\it loadparams} aus Tabelle~\ref{batchcommands} kann folgende Werte
annehmen: 

\nopagebreak
\begin{tabular}{c|c}
{\it loadparams} & {\bf entspricht "`Load"'-Men"upunkt\ldots} \\ \hline
{\tt min} & "`minimum"' \\
{\tt max} & "`maximum"' \\
{\tt avg} & "`average"' \\
{\tt length} & "`length"' \\
{\tt sum} & "`sum"' \\
{\tt stddev} & "`standard deviation"' \\
{\tt pattern} {\it nr} & "`pattern\ldots"' \\
{\tt overallMin} & "`overall minimum"' \\
{\tt overallMax} & "`overall maximum"' \\
{\tt overallAvg} & "`overall average"' \\
{\tt overallStddev} & "`overall std dev"' \\
{\tt const} {\it number} & "`const\ldots"' \\
{\tt values} $val_1 \ldots val_n$ & --- \\
{\tt dim} \enspace $nr$ \enspace $value$ & --- \\  
\end{tabular} 

F"ur die Anweisung {\tt const} gibt {\it number} die zu ladende Konstante an.
Die Anweisung {\tt pattern} l"adt den Vektor Nummer {\it nr}
in die horizontalen Skalarfelder ({\tt <command> = loadHoriz})
bzw. die {\it nr}-te Dimension in die vertikalen Felder 
({\tt <command> = loadVert}). \\
Durch {\tt values} k"onnen die Werte, mit denen die vertikalen bzw.
horizontalen Skalarfelder geladen werden sollen, direkt angegeben werden.
Diese Werte wurden in obiger Tabelle mit $val_1 \ldots val_n$ bezeichnet.
$n$ mu"s dabei genau der Anzahl der betreffenden Skalarfelder
entsprechen. \\
Wie f"ur {\tt values} gibt es auch f"ur die Anweisung {\tt dim} keine
Entsprechung im "`Load"'-Men"u.
Mit {\tt dim} wird das $nr$-te vertikale ({\tt <command> = loadVert})
bzw. das $nr$-te horizontale Skalarfeld ({\tt <command> = loadHoriz})
auf den Wert $value$ gesetzt.
{\tt dim} entspricht im interaktiven Modus somit einer "Anderung 
eines Skalarfeldes von Hand.
F"ur jedes ge"anderte Feld wird eine {\tt dim}-Anweisung in die LOG-Datei
geschrieben.   
 
\subsubsection*{Fehlerbehandlung im Batch-Modus}

Trifft Vistra in der LOG-Datei auf eine Zeile mit fehlerhafter Syntax,
so wird eine Warnung auf stdout ausgegeben und mit der 
n"achsten Zeile fortgefahren.
Gleiches gilt f"ur Anweisungen, die aus anderen Gr"unden 
nicht ausgef"uhrt werden k"onnen.
Eine halblogarithmische Transformation ben"otigt beispielsweise Vektoren
mit einer quadratischen Anzahl von Dimensionen.
Ein weiteres Beispiel ist die {\tt loadVert values}-Anweisung, die
eine Warnung erzeugt, wenn die Anzahl der Werte nicht der Anzahl
der Trainingsvektoren entspricht. 

\subsection{LOG-Anweisungen der interaktiven Aktionen}

Dieser Abschnitt besch"aftigt sich mit der Frage, welche LOG-Anweisungen
f"ur die Aktionen, die interaktiv im Hauptfenster von Vistra 
durchgef"uhrt werden, in die LOG-Datei geschrieben werden, solange 
"`Write to log file"' gesetzt ist.
Dabei ist zu beachten, da"s eine Trennung von Trainingsvektoren nur
dann sinnvoll unterst"utzt wird, wenn ein und derselbe Vektor durch
Ausf"uhrung der LOG-Datei in genau der gleichen Weise ver"andert wird wie 
durch die interaktiv durchgef"uhrten Transformationen.

Vektor-Transformationen, deren Ergebnisse nur vom jeweiligen Vektor 
abh"angen, bereiten unter diesem Gesichtspunkt keine Probleme
(z.B. liefert eine Fourier Transformation auf den gleichen Vektor angewandt
immer das gleiche Resultat).
Schwierigkeiten machen jedoch Aktionen, die von {\it allen} geladenen
Vektoren abh"angen.
Darunter f"allt beispielsweise das Entfernen konstanter Dimensionen.
Eine Dimension, die f"ur eine Menge von Vektoren konstant ist, mu"s
nicht auch f"ur andere Vektormengen konstant sein.
W"urde in diesem Fall einfach nur {\tt rmConstDims} in die LOG-Datei
geschrieben, so k"onnten den Vektoren einer Datei evtl. ganz andere Dimensionen
entfernt werden als den Trainingsvektoren einer anderen Datei. 
Abhilfe kann nur geschaffen werden, wenn die LOG-Datei  
die tats"achlich entfernten Dimensionen explizit enth"alt (mithilfe
von {\tt rmDims}).      

"Ahnlich verh"alt es sich, wenn z.B. die horizontalen Skalarfelder 
mit den Minima der einzelnen Dimensionen geladen werden sollen.
Die Minima k"onnen je nach Vektormenge anders ausfallen, weshalb auch
hier anstelle von {\tt loadHoriz min} 
das Minimum jeder Dimension explizit in die LOG-Datei geschrieben wird.

Folgende Tabelle enth"alt in der linken Spalte die 
interaktiven Aktionen und in der rechten die daf"ur geschriebenen 
LOG-Anweisungen:

\begin{tabular}{l|l}
{\bf Men"upunkt/Button} & {\bf geschriebene LOG-Anweisung(en)} \\ \hline
"`HLOG"' & {\tt hlog} \\
"`FFT"' & {\tt fft} \\
"`PCA"' & {\tt pca} \\
"`Normalize"' & {\tt normalize} \\
"`Scale"' & {\tt multConst} {\it stretch} \\
& {\tt addConst} {\it shift} \\
"`Randomize"' & {\tt randomize} \\
"`Expand with class vector"' & {\tt classExpand} \\
"`Expand with output vector"' & {\tt outputExpand} \\
"`Refresh class numbers"' & {\tt refreshClasses} \\
"`Remove dimensions"' & {\tt rmDimRange} {\it from to} \\
"`Remove constant dims"' & {\tt rmDims} $dim_1 \ldots dim_n$ \\
"`add"' & {\tt add} {\it direction} \\
"`subtract"' & {\tt subtract} {\it direction} \\
"`multiply"' & {\tt multiply} {\it direction} \\
"`divide"' & {\tt divide} {\it direction} \\
"`replace"' & {\tt replace} {\it nr} \\
\end{tabular}

Eine Skalierung eines Wertebereichs auf einen anderen kann auf
eine Multiplikation und eine Addition zur"uckgef"uhrt werden.
Der betreffende Faktor bzw. Summand ist oben als {\it stretch} bzw.
{\it shift} bezeichnet. \\
{\it direction} steht entweder f"ur {\tt vert} oder f"ur {\tt horiz}.

In obiger Tabelle fehlen die Aktionen des "`Load"'-Men"us.
Bei ihnen h"angen die geschriebenen LOG-Anweisungen davon ab, ob die vertikalen
oder die horizontalen Skalarfelder geladen wurden:

\begin{tabular}{l|l|l}
{\bf "`Load"'-Men"upunkt} & {\bf vertikal} & {\bf horizontal} \\ \hline
"`minimum"' & {\tt loadVert min} & {\tt loadHoriz values} {\it valueList} \\
"`maximum"' & {\tt loadVert max} & {\tt loadHoriz values} {\it valueList} \\
"`average"' & {\tt loadVert avg} & {\tt loadHoriz values} {\it valueList} \\
"`length"' & {\tt loadVert length} & {\tt loadHoriz values} {\it valueList} \\
"`standard deviation"' & {\tt loadVert stddev} & {\tt loadHoriz values} 
{\it valueList} \\
"`sum"' & {\tt loadVert sum} & {\tt loadHoriz values} {\it valueList} \\
"`pattern"' & {\tt loadVert values} {\it valueList} &
{\tt loadHoriz values} {\it valueList} \\
"`overall minimum"' & {\tt loadVert const} {\it value} &
{\tt loadHoriz const} {\it value} \\   
"`overall maximum"' & {\tt loadVert const} {\it value} &
{\tt loadHoriz const} {\it value} \\   
"`overall average"' & {\tt loadVert const} {\it value} &
{\tt loadHoriz const} {\it value} \\   
"`overall std dev"' & {\tt loadVert const} {\it value} &
{\tt loadHoriz const} {\it value} \\   
"`const"' & {\tt loadVert const} {\it value} & 
{\tt loadHoriz const} {\it value} \\ 
\end{tabular}

{\it valueList} repr"asentiert hierbei eine Liste von Zahlen, die jeweils 
durch ein Blank voneinander getrennt sind.
Eine letzte M"oglichkeit zur Manipulation der Skalarfelder ist 
das direkte "Andern eines Feldinhalts von Hand mithilfe der Tastatur.
In diesem Fall wird einer der folgenden beiden Anweisungen in die
LOG-Datei geschrieben (je nachdem, ob es sich um ein vertikales oder
ein horizontales Skalarfeld handelt):

{\tt loadVert dim} \enspace $nr$ \enspace $value$ \qquad oder \\
{\tt loadHoriz dim} \enspace $nr$ \enspace $value$  

Das $nr$-te vertikale bzw. horizontale Skalarfeld wird hierdurch auf
den Wert $value$ gesetzt.