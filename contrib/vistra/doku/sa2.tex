\section[Funktionalit"at, Installation und Aufruf]{Funktionalit"at, 
Installation und Aufruf von Vistra}
\label{funk}

Dieses Kapitel gibt einen "Uberblick "uber die Funktionalit"at
von Vistra. 
Die Funktionen werden an dieser Stelle jedoch nur aufgelistet.
Eine detaillierte Bedienungsanleitung des Programms ist 
Gegenstand der n"achsten Kapitel.
Anschlie"send wird beschrieben, wie Vistra installiert wird und
wie die Syntax des Programmaufrufs mit seinen Optionen lautet.
 
\subsection{Funktionalit"at}
Die Funktionen von Vistra k"onnen in die drei Kategorien
{\sl Datei-Funktionen (Konvertierung), Transformationen} und
{\sl Visualisierungen} eingeteilt werden.

\subsubsection*{Datei-Funktionen (Konvertierung)}

Abbildung~\ref{flow} zeigt alle Arten von Dateien, die von Vistra
gelesen bzw.~geschrieben werden:

\begin{figure}[ht]
\hspace*{0.3cm}
\begin{picture}(14.4,6)
\put(0,4.8){\framebox(4,1.2){\parbox{4cm}
         {\begin{center} Trainingsvektor- \\ dateien \end{center}}}}
\put(0,3.2){\framebox(4,1.2){\parbox{4cm}
         {\begin{center} LOG-Dateien \end{center}}}}
\put(0,1.6){\framebox(4,1.2){\parbox{4cm}
         {\begin{center} Symboltabellen- \\ dateien \end{center}}}}
\put(0,0){\framebox(4,1.2){\parbox{4cm}
         {\begin{center} Formatbeschreibungs- \\ dateien \end{center}}}}
\put(5.2,2){\framebox(4,2){\parbox{4cm}
         {\begin{center} {\Large {\bf Vistra}} \end{center}}}}
\put(10.4,4){\framebox(4,1.2){\parbox{4cm}
         {\begin{center} Trainingsvektor- \\ dateien \end{center}}}}
\put(10.4,2.4){\framebox(4,1.2){\parbox{4cm}
         {\begin{center} LOG-Dateien \end{center}}}}
\put(10.4,0.8){\framebox(4,1.2){\parbox{4cm}
         {\begin{center} Symboltabellen- \\ dateien \end{center}}}}
\put(4,0.6){\vector(2,3){1.2}}
\put(4,2.2){\vector(2,1){1.2}}
\put(4,3.8){\vector(2,-1){1.2}}
\put(4,5.4){\vector(2,-3){1.2}}
\put(9.2,2.6){\vector(1,-1){1.2}}
\put(9.2,3){\vector(1,0){1.2}}
\put(9.2,3.4){\vector(1,1){1.2}} 
\end{picture}
\caption{\label{flow} Eingabe und Ausgabe von Vistra}
\end{figure}

Folgende Aktionen sind m"oglich:
\begin{itemize}

\item Trainingsvektor-Dateien unterschiedlicher Formate k"onnen eingelesen
und geschrieben werden. 
Zwei Formate stehen standardm"a"sig zur Verf"ugung:
\begin{itemize}
\item das N01-Format und
\item das LVQ-Format.
\end{itemize}
Beide Formate werden u.a. von KNet der Universit"at Stuttgart,
einem parallelen Simulator f"ur die selbstorganisierenden Karten
von Teuvo Kohonen (Kohonen Feature Maps), verwendet
(siehe \cite{bayer}). 
Das LVQ-Format ist ein ASCII-Format und stammt urspr"unglich
aus dem LVQ\_PAK, einem Public-Domain Paket, das "uber 
Internet von der Kohonen-Gruppe zur Verf"ugung gestellt 
wird\footnote{erh"altlich "uber die Internet-Adresse cochlea.hut.fi}.

Beim N01-Format handelt es sich um ein bin"ares Format. 
Die Vektoren liegen in maschinencodierter Form vor.
Bin"are Formate haben den Vorteil gegen"uber ASCII-Formaten, da"s sie
schneller eingelesen bzw.~geschrieben werden k"onnen, da 
zeitaufwendige Konvertierungen von Strings in Flie"skommazahlen 
vermieden werden.
Nachteilig ist, da"s sie maschinenabh"angig sind und vom Menschen
nicht gelesen werden k"onnen. 
Obwohl SUN- und DEC-Workstations unterschiedliche Codierungen f"ur
Zahlen verwenden, kann Vistra N01-Formate auf beiden Fabrikaten
lesen und schreiben.

Trainingsvektoren k"onnen in einem Format gelesen und in einem anderen
Format geschrieben werden.
Dadurch ist Vistra in der Lage, eine Vektordatei
von einem Format in ein anderes zu konvertieren.
N01- und LVQ-Format sind in Kapitel~\ref{formate} detailliert beschrieben.

\item Symboltabellen k"onnen eingelesen und geschrieben werden. 
Anstelle eines Ausgabevektors, wird einem Eingabevektor h"aufig nur
ein Symbol (ein String) zur Klassifikation der Ausgabe zugeordnet.
Eine Symboltabellen-Datei enth"alt dann eine Liste dieser Klassensymbole. 
Das genaue Aussehen von Symboltabellen-Dateien kann Kapitel~\ref{formate}
entnommen werden.

\item Neben dem LVQ-Format kann Vistra um weitere ASCII\--For\-mate 
f"ur Trainings\-vektor-Dateien erweitert werden.
Voraussetzung ist, da"s sich das Format durch Vistra's 
Formatbeschreibungssprache FDL ({\bf F}ormat {\bf D}escription 
{\bf L}anguage) 
hinreichend beschreiben l"a"st.
Ist dies der Fall, so kann Vistra ein solches Format sowohl lesen als
auch schreiben.
Zu diesem Zweck wurde ein Interpreter f"ur FDL entwickelt. 
FDL ist ausf"uhrlich in Kapitel~\ref{fdl} beschrieben.  

\item Vistra kann die auf den Trainingsvektoren durchgef"uhrten
Transformationen in einer LOG-Datei festhalten.
Dadurch wird es z.B.~m"oglich, mit Trainings- und Testvektoren, die zu
unterschiedlichen Zeitpunkten bzw. in separaten Dateien zur Verf"ugung stehen, 
die gleichen Transformationen durchzuf"uhren. 
Neben dem interaktiven Modus verf"ugt Vistra deshalb "uber einen Batch-Modus,
der die Transformationen einer LOG-Datei im Stapelbetrieb ausf"uhrt. 
\end{itemize}     

\subsubsection*{Transformationen}

Vistra kann folgende Transformationen auf einer Menge von
Vektoren durchf"uhren:

\begin{itemize}
\item Hauptachsentransformationen (PCA -- Principle Component Analysis)
\item Halblogarithmische Transformationen (HLOG) 
\item Diskrete eindimensionale Fourier Transformationen (FFT)
\item Normalisieren der Vektoren (auf die L"ange 1)
\item Skalieren auf einen bestimmten Wertebereich
\item Mischen der Vektoren ("`Randomize"')
\item Erweitern der Eingabevektoren mit den zugeh"origen Ausgabevektoren
\item Erweitern der Vektoren mit einem Klassenvektor
\item elementweise Addition, Subtraktion, Multiplikation und Division der
Vektoren oder Dimensionen mit Skalaren 
\item Manuelles Editieren einzelner Vektoren oder Dimensionen
\item Entfernen von Dimensionen
\item Entfernen von Vektoren
\end{itemize}

Zu jedem Vektor bzw.~jeder Dimension k"onnen au"serdem statistische
Werte wie Minima, Maxima, Mittelwerte, Standardabweichungen, L"angen
etc.~berechnet werden.   
Diese Werte k"onnen dann auch als Skalare zur elementweisen
Verkn"upfung in einer der vier Grundrechenarten verwendet werden.

\subsubsection*{Visualisierungs-Formen}

Die Vektoren (Eingabe- oder Ausgabevektoren) k"onnen auf folgende
Arten dargestellt werden:
\begin{itemize}
\item textuell
\item als Grauwertbilder
\item als Farbbilder
\item als Histogramme
\item als Linienz"uge
\item als 2-dimensionale Projektion in einem gemeinsamen Koordinatensystem
\end{itemize}

Die Gr"o"se der graphischen Darstellungen kann innerhalb vorgegebener
Grenzen variiert werden.
Die Visualisierungsformen werden in Kapitel~\ref{benutzung} behandelt.  

\subsection{Installation}

Die Installation von Vistra l"auft in folgenden Schritten ab:
\begin{enumerate}
\item Das ausf"uhrbare Programm "`vistra"' wird in ein beliebiges
Verzeichnis kopiert.
\item Es wird sichergestellt, da"s sich das Programm "`pca"' auf dem
Befehlspfad befindet. \\
Vistra ruft "`pca"' zur Durchf"uhrung von Hauptachsentransformationen auf.
Dazu wird die vom Original leicht abgewandelte Version von "`pca"' ben"otigt,
die die Eingabe im LVQ-Format erwartet und die Ausgabe ebenfalls im
LVQ-Format schreibt\footnote{die Originalversion von "`pca"'
ist "uber die Internet-Adresse ftp.icsi.berkeley.ed erh"altlich.}.
\item Es wird, falls gew"unscht, ein neues Verzeichnis angelegt, das
die Formatbeschreibungs-Dateien f"ur die ASCII-Formate aufbewahrt, 
um die Vistra erweitert werden soll.
Anstelle eines neuen Verzeichnisses k"onnen die Formatbeschreibungen auch
in ein bereits existierendes Verzeichnis gestellt werden.
Es sollte dann jedoch sichergestellt sein, da"s keine anderen Dateien in
diesem Verzeichnis auf {\tt .fmt} enden, da Vistra alle Dateien mit dieser
Endung f"ur Formatbeschreibungs-Dateien h"alt.   
\item Es wird eine Environment-Variable namens "`VISTRAFORMATS"' eingerichtet,
die auf den Pfadnamen des Format-Verzeichnisses aus Schritt~3
(hier als {\tt <format\_dir>} bezeichnet), gesetzt wird:
\begin{verbatim}
setenv VISTRAFORMATS <format_dir>
\end{verbatim} 
Existiert die Environment-Variable oder das angegebene Verzeichnis nicht,
so wird Vistra w"ahrend des Programmstarts eine entsprechende Warnung
ausgeben und die benutzerdefinierten Formate nicht einbinden k"onnen.
\end{enumerate}

\subsection{Aufruf}
\label{call}

Dieser Abschnitt beschreibt, wie und mit welchen Optionen Vistra
gestartet wird.
Dabei wird zwischen dem Aufruf des interaktiven Modus und
dem Aufruf des Batch-Modus unterschieden.

%\newpage
\subsubsection*{Interaktiver Modus}

Der Aufruf im interaktiven Modus sieht wie folgt aus:

{\tt vistra [ <options> ]  [ <patternfile> ]}

{\tt <patternfile>} steht hierbei f"ur den Dateinamen einer
Trainingsvektor-Datei.
Mit den Vektoren dieser Datei wird das Programm zu Beginn geladen.
Wird {\tt <patternfile>} weggelassen, so erscheint das Hauptfenster 
von Vistra zun"achst ohne Vektoren. 
Der Benutzer hat dann lediglich die M"oglichkeit, eine
Trainingsvektor-Datei nachtr"aglich zu laden oder das Programm wieder
zu verlassen.
  
{\tt <options>} steht f"ur eine Liste von einer oder mehreren der
in Tabelle~\ref{interoptions} aufgef"uhrten Optionen. 

\begin{table}[ht]
\begin{tabular}[t]{lp{10.5cm}}
{\bf Option} & {\bf Bedeutung} \\ \hline
{\tt -rfmt} {\tt <readformat>} & Ist nur von Bedeutung, wenn der Name einer
Trainingsvektor-Datei angegeben wird. 
{\tt <readformat>} gibt dann das Format dieser Datei an.
M"ogliche Werte sind {\tt n01, lvq} sowie die Namen aller benutzerdefinierten
Formate, um die Vistra erweitert wurde. 
Im letzten Fall mu"s der Namen der Formatbeschreibungs-Datei ohne die
Endung {\tt .fmt} angegeben werden. 
 
{\sl Default:} Suffix von {\tt <patternfile>}. \\
{\tt -log} & Ist nur von Bedeutung, wenn der Name einer Trainingsvektor-Datei
angegeben wird.
Durch diese Option wird ein Schalter gesetzt, der bewirkt, da"s die 
Transformationen, denen die Trainingsvektoren von {\tt <patternfile>}
unterzogen werden, in eine LOG-Datei geschrieben werden. 
Diese Speicherungsfunktion kann sp"ater "uber einen Men"upunkt beliebig
an- und ausgeschaltet werden. 

Der Name der LOG-Datei ist {\tt <patternfile>.log}.
Die LOG-Datei wird in das Verzeichnis von {\tt <patternfile>} gestellt
(falls {\tt <patternfile>} sich in einem Verzeichnis befindet, f"ur
das der Benutzer keine Schreibberechtigung besitzt, so sollte 
in einem anderen Verzeichnis ein Link auf diese Datei eingerichtet
werden). \\
{\tt -nf} {\tt <font>} & Im Hauptfenster von Vistra sind entweder
die Eingabevektoren oder die Ausgabevektoren einer Vektordatei permanent
in textueller Form sichtbar. 

Um m"oglichst viele Vektoren auf einmal auf dem Bildschirm zu haben, 
kann es w"unschenswert sein, den Text-Font selbst festzulegen, der zur 
Darstellung der Vektorelemente verwendet wird.
Dies geschieht, indem man f"ur {\tt <font>} den Namen eines
X-Windows Text-Fonts angibt. 

{\sl Default:} {\tt 7x13} \\
\end{tabular}
\caption{\label{interoptions} Command-line Optionen f"ur den interaktiven Modus}
\end{table}

%\newpage
\subsubsection*{Batch-Modus}

Der Batch-Modus von Vistra bietet die M"oglichkeit, Transformationen,
die der Benutzer im interaktiven Modus durchgef"uhrt hat, zu einem
sp"ateren Zeitpunkt auf Testvektoren oder anderen Trainingsvektoren 
zu wiederholen.
Dazu werden die Modifikationen bzw.~Transformationen w"ahrend des
interaktiven Programmlaufs in eine LOG-Datei geschrieben, die sp"ater
als Eingabe f"ur Vistra's Batch-Modus dient. 
Kapitel~\ref{log} beschreibt das Format der LOG-Dateien sowie die 
internen Abl"aufe w"ahrend des Batch-Betriebs.

Vistra wird im Batch-Modus wie folgt gestartet:
   
\nopagebreak
{\tt vistra -batch <logfile> [ <options> ] <patternfile>}

\begin{sloppypar}
{\tt <logfile>} ist der Name einer LOG-Datei, deren Befehle auf die
Vektor-Datei {\tt <patternfile>} angewandt werden. \\
{\tt <options>} steht f"ur eine Liste von einer oder mehreren der
in Tabelle~\ref{batchoptions} aufgef"uhrten Optionen.
\end{sloppypar}
 
\begin{table}[ht]
\begin{tabular}[t]{lp{10.2cm}}
{\bf Option} & {\bf Bedeutung} \\ \hline
{\tt -out} {\tt <outputfile>} &
{\tt <outputfile>} gibt den Namen der Datei an, in die die transformierten
Trainingsvektoren geschrieben werden. 

{\sl Default:} {\tt <patternfile>.out} \\
{\tt -rfmt} {\tt <readformat>} &
{\tt <readformat>} gibt das Format der Vek\-tor-Da\-tei {\tt <patternfile>} an.
M"ogliche Werte sind {\tt n01, lvq} oder der Namen eines benutzerdefinierten
Formats ohne die Endung {\tt .fmt}. 

{\sl Default:} Suffix von {\tt <patternfile>}. \\
{\tt -wfmt} {\tt <writeformat>} &
{\tt <writeformat>} gibt den Namen des Ausgabeformats an.
Alle Werte f"ur {\tt <readformat>} sind auch f"ur {\tt <writeformat>}
zul"assig. 

{\sl Default:} Suffix von {\tt <outputfile>}, falls eine Ausgabedatei 
angegeben ist. 
Ansonsten erh"alt die Ausgabe-Datei das gleiche Format wie die Eingabe-Datei.
\end{tabular}
\caption{\label{batchoptions} Command-line Optionen f"ur den Batch-Modus}
\end{table}






