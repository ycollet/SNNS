\chapter{Parameter"ubergabe in der Kopfzeile}
\label{Parameter}


Es gibt in {\bf Nessus} die M"oglichkeit, ein Programm mit Parametern
zu versehen, deren Wert erst zur "Ubersetzungszeit festgelegt werden
mu"s. Diese Parameter sind dann innerhalb des Programms als Variablen
verf"ugbar. \index{Parameter!des Programms} \index{Programmparameter}

Diese zur "Ubersetzungszeit festgelegten Parameter erm"oglichen es in
Verbindung mit dem Einlesen von Dateikonstanten,
\index{Datei!physischer Name!als Programmparameter} {\bf
Nessus}-Programme zu schreiben, die eine ganze Reihe von Netzen
erzeugen k"onnen. Das ist dann sinnvoll, wenn

\begin{itemize}
  \item die Netze sich haupts"achlich durch die Anzahl von Zellen bzw. topologischer Strukturen
	unterscheiden, 
  \item die Verbindungsmuster gleichartig sind und f"ur die verschiedenen Netze mit den gleichen
	Sprachmitteln definiert werden k"onnen, oder wenn
  \item die Verbindungen "uber eine {\bf map}-Konstante\index{map} definiert werden, die dann aus einer Datei
	eingelesen werden kann.
\end{itemize}

\section{Kopfzeile eines {\bf Nessus}-Programms}
\label{Kopfzeile}

Die Parameter werden in der Kopfzeile\index{Kopfzeile} des Programms
deklariert. F"ur diese Parameter k"onnen auch
Standardwerte\index{Standardwert!f"ur Programmparameter} vordefiniert
werden.  Syntaxregel~\ref{syprogramheader} zeigt die Form des
Programmkopfes.\index{Programm!Kopfzeile}\index{Kopfzeile!eines
Programms}

Die Kopfzeile\index{Programm!Kopfzeile}\index{Kopfzeile!des Programms}
besteht aus dem Schl"usselwort {\bf network}\index{network} gefolgt
von dem Namen des Netzes. Danach wird in Klammern die Liste der
Parameter des Programms \index{Parameter!des Programms} angegeben, die

\begin{itemize}
  \item entweder leer ist, oder
  \item nur aus einer Liste von Parametern {\it ohne\/} Standardwerte, oder
  \item nur aus einer Liste von Parametern {\it mit\/} vordefinierten Standardwerten besteht.
\end{itemize}

Man beachte, da"s es sich dabei um Positions- und nicht um
Schl"usselparameter handelt.  Deshalb ist es nicht erlaubt, in der
Parameterliste Parameter ohne und solche mit Standardwerten zu
mischen, bei der Parameter"ubergabe die Reihenfolge der Parameter bei
der "Ubersetzung zu vertauschen, oder Parameter auszulassen.  Die
Angabe einer Liste von Parametern ohne Standardwerte erfolgt gem"a"s
Syntaxregel~\ref{syidlist}.  Standardwerte f"ur Parameter werden mit
dem Zuweisungsoperator (siehe Syntaxregel~\ref{sysimpleassignment})
definiert.

\begin{center}
\fbox{
\begin{minipage}{\textwidth}
{\small
\begin{center}
\begin{eqnarray}
  \mbox{program\_header} & ::= & \mbox{{\bf network\index{network} identifier} {\bf (} $[$parameter\_list$]$ {\bf )~;}} 
								\label{syprogramheader}\\[.3 cm]
  \mbox{parameter\_list} & ::= & \mbox{identifier\_list} \nonumber\\ \index{Parameter!des Programms!Definition}
          &  |  &  \mbox{default\_param\_list} \label{syparameterlist}\\[.3 cm]
  \mbox{default\_param\_list} & ::= & \mbox{default\_parameter \{{\bf ~,~}default\_parameter \}} 
								\label{sydefaultparamlist}\\[.3 cm]
  \mbox{default\_parameter} & ::= & \mbox{simple\_assignment} \label{sydefaultparameter}\\ \nonumber
\end{eqnarray}
\end{center}
}
\end{minipage}
}
\end{center}

Die Datentypen der Programmparameter
\index{Programmparameter!Datentypen} \index{Parameter!des
Programms!Datentypen} sind auf die einfachen Typen

\begin{itemize}
  \item {\bf int},\index{int}
  \item {\bf float}\index{float} und
  \item {\bf string}\index{string}
\end{itemize}

beschr"ankt. Allerdings kann ein Parameter des Typs {\bf
string}\index{string} ohne weiteres der (physische) Name einer
Datei\index{Datei!physischer Name!als Programmparameter} sein. So kann
sich hinter einem solchen Parameter also auch eine Feld- oder {\bf
map}-Konstante\index{Konstante!als Programmparameter} verbergen.

\begin{quote}
  An Parameter mit vordefinierten Standardwerten k"onnen zur "Ubersetzungszeit neue Werte zugewiesen
  werden. In diesem Fall m"ussen die Datentypen der neuen Werte mit denen der Standardwerte "ubereinstimmen! 
\end{quote}


\section{Definition von Parametern zur "Ubersetzungszeit}
\label{runtimeparameter}

Die Werte der Parameter eines {\bf Nessus}-Programms k"onnen zur
"Ubersetzungszeit festgelegt werden. Dadurch werden dann vordefinierte
Standardwerte dieser Parameter "uberschrieben. Da es sich um
Positionsparameter handelt, m"ussen aber dann die Werte f"ur alle
Parameter angegeben werden.

Die Parameterwerte werden zur "Ubersetzungszeit folgenderma"sen definiert:
\index{Parameter!des Programms!Definition zur "Ubersetzungszeit}
\index{Programmparameter!Definition zur "Ubersetzungszeit}

\begin{quote}
  Beim Aufruf des {\bf Nessus}-Compilers (siehe
Kapitel~\ref{CompilerAufruf})\index{Programm!"Ubersetzung} wird nach der Option 
  {\bf -d} die Liste der aktuellen Parameterwerte angegeben. Die
einzelnen Parameterwerte werden durch Leerzeichen getrennt. 
\end{quote}

Die Zuordnung der aktuellen Werte zu den Parametern in der
Kopfzeile\index{Kopfzeile!des Programms} des Programms erfolgt allein
durch ihre Reihenfolge. Darin liegt der Grund daf"ur, da"s weder die
Parameterwerte in vertauschter Reihenfolge angegeben noch irgendwelche
Parameterwerte ausgelassen werden d"urfen.

\begin{figure}[htbp]
  \begin{center}
    \mbox{
      \begin{minipage}{\textwidth}
           \begin{tabbing}\=nn\=\kill
	   {\bf network}~\verb&parameterBsp&\verb&(&\verb&intPar&~\verb&:=&~3\verb&,&~\verb&fPar&~\verb&:=&~$0.95$\verb&,&~\\
~~~~~~~~~~~~~~~~~~~~~~~~~~~~~~~~~~~\verb&datPar&~\verb&:=&~\verb&"stdFile.dat"&\verb&)&\verb&;&\\
~~~~{\small \{\verb{Parameterdatentypen sind jetzt festgelegt.{\}}\\[.15cm]
{\bf cons}\\
~~~~\verb&fileInput&~\verb&=&~{\bf file}~\verb&datPar&~{\bf of}~{\bf float}\verb&;&\\[.15cm]
{\bf begin}\\
{\bf end}\verb&.& \\

           \end{tabbing}
      \end{minipage}
    }
    \caption{\label{ParameterBsp} Beispiel: Parameterdefinition in der Kopfzeile}
  \end{center}
\end{figure}

Der Datentyp der Programmparameter \index{Parameter!des
Programms!Datentypen} \index{Programmparameter!Datentypen} wird als
Zeichenkette\index{Zeichenkette!als Programmparameter} eingelesen. Der
Wert und der Datentyp, den der Parameter dann erh"alt, wird daraus
nach folgenden Regeln ermittelt:

\begin{itemize}
  \item Enth"alt die Zeichenkette nur Ziffern (mit oder ohne
Vorzeichen), wird sie in eine ganze Zahl\index{Zahl!ganze!als
Programmparameter} ({\bf int}) umgewandelt.  
  \item Entspricht die Zeichenkette der Darstellung einer rationalen
Zahl\index{Zahl!rationale} in {\bf Nessus}, so wird sie in eine solche umgewandelt.
  \item Sonst wird die Zeichenkette als {\bf
string}-Konstante\index{Konstante!Datentyp wenn Programmparameter} aufgefa"st. 
\end{itemize}

Enthalten Zeichenketten Leer- oder Sonderzeichen, so m"ussen sie in
``~'' eingeschlossen werden. Das gilt insbesondere auch f"ur solche
Symbole, die von UNIX als Sonderzeichen aufgefa"st werden.

Abbildung~\ref{ParameterBsp} zeigt eine Kopfzeile\index{Kopfzeile!des
Programms} mit drei Standardwertdefinitionen f"ur Parameter.  Man
beachte, da"s der Datentyp eines jeden dieser Parameter dadurch
verbindlich festgelegt ist und auch durch eine "Uberschreibung des
Parameterwerts zur "Ubersetzungszeit nicht mehr ge"andert werden darf.

Dieses Programm stehe in einer Datei ``parameterBsp.n''. Dann
"uberschreibt ein Compiler-Aufruf\index{Programm!"Ubersetzung}
 
\centerline{{\bf nessus} parameterBsp.n {\bf -d} 229 -0.34 neueDatei.dat} 

die Standardwerte der Parameter. Bei der Auswertung des Programms
gelten dann die neuen Werte. Insbesondere liest das Programm jetzt
statt der vordefinierten Datei ``stdfile.dat'' die Datei
``neueDatei.dat'' ein.


\section{Beispiel: F"arbung von  Karten}

Die Parameter"ubergabe an ein {\bf Nessus}-Programm zur
"Ubersetzungszeit soll nun am Beispiel des Problems ``F"arbung einer
Karte mit vier Farben'' demonstriert werden.

Eine Karte soll mit vier Farben so gef"arbt werden, da"s keine zwei
benachbarten Regionen dieselbe Farbe haben.

Es soll nun ein neuronales Netz definiert werden, das dieses Problem
-- zuerst einmal f"ur eine gegebene Karte -- l"ost. In \cite{rcs86}
wird ein Netz folgender Struktur vorgeschlagen:

\begin{itemize}
  \item F"ur jede Region der Karte werden vier Zellen angelegt - eine f"ur jede der Farben
``rot'', 
	``blau'', ``gr"un'' und ``wei"s''.
  \item Die Zellen einer Region werden miteinander verbunden. Da sich die vier Farben
gegenseitig 
	ausschlie"sen sollen, ist das Gewicht dieser Verbindungen $-1.0$.
  \item Bei benachbarten Regionen werden die Zellen, die derselben Farbe entsprechen,
miteinander 
	verbunden. Das Gewicht dieser Verbindungen ist $-0.1$.
\end{itemize}

Dieses Netz f"uhrt zu einem einfachen {\bf Nessus}-Programm, in dem
die vier Zellen einer Region als Cliquen und die
Nachbarschaftsbeziehungen der Regionen mit einer {\bf
map}-Konstanten\index{map} dargestellt werden.

Damit erh"alt man ein Programm, das f"ur eine bestimmte Karte das
entsprechende Netz definiert. Will man ein Programm schreiben, das
statt des Netzes f"ur eine Karte die Struktur der Netze f"ur dieses
Problem definiert, mu"s man nur zwei "Anderungen vornehmen:

\begin{enumerate}
  \item Die Konstante, die die Anzahl\index{Zahl!rationale} der
Regionen angibt, wird zur "Ubersetzungszeit definiert 
	(Parameter in der Kopfzeile\index{Kopfzeile!des Programms} des Programms).
  \item Die {\bf map}-Konstante,\index{map} die die
Nachbarschaftsbeziehungen abbildet, wird aus einer Datei 
	eingelesen.  Der physische Name dieser Datei wird ebenfalls
zur "Ubersetzungszeit definiert. 
\end{enumerate} 

Dadurch entsteht das Programm, das Abbildung~\ref{ColorMapProg} zeigt.

\begin{figure}[htbp]
  \begin{center}
    \mbox{
      \begin{minipage}{\textwidth}
           \begin{tabbing}\=nn\=\kill
	   {\bf network}~\verb&colorMap&\verb&(&\verb&size&\verb&,&~\verb&regionFile&\verb&,&~\verb&borderFile&\verb&,&~~\verb&mSep&\verb&,&~\verb&lSep&\verb&)&\verb&;&\\
~~~~~~~~~~{\small \{\verb{size = no. of regions, regionFile = file of region names{\}}\\
~~~~~~~~~~{\small \{\verb{borderFile = file of borders (map){\}}\\[.15cm]
{\bf const}\\
~~\verb&colors&~\verb&=&~\verb&[&\verb&"red"&\verb&,&~\verb&"green"&\verb&,&~\verb&"blue"&\verb&,&~\verb&"white"&\verb&]&\verb&;&\\
~~\verb&regions&~\verb&=&~{\bf file}~\verb&regionFile&~{\bf of}~{\bf string}\verb&;&\\
~~\verb&regionNames&~\verb&=&~\verb&[&\verb&regions&\verb&]&\verb&;&~~~~~\\
~~~~~~~~~~{\small \{\verb{build array of region names{\}}\\
~~\verb&borders&~\verb&=&~{\bf file}~\verb&borderFile&~{\bf of}~{\bf map}\verb&;&\\
~~~~~~~~~~{\small \{\verb{logical file name for map of regions{\}}\\
~~\verb&borderMap&~\verb&=&~{\bf map}~{\bf string}~{\bf to}~{\bf string}\verb&:&\\
~~~~~~{\bf get}~\verb&(&\verb&mSep&\verb&,&~\verb&lSep&\verb&)&~{\bf from}~\verb&borders&\verb&;&\\
~~~~~~~~~~{\small \{\verb{domain and range are region names{\}}\\[.15cm]
{\bf typedef}\\
~~{\bf site}~{\bf with}~{\bf name}~\verb&"inhibitory"&\verb&,&~{\bf sitefunc}~\verb&Site_Pi&~\verb&:&~\verb&inhibSite&\verb&;&\\
~~{\bf unit}~{\bf with}~{\bf actfunc}~\verb&Act_Logistic&\verb&,&~{\bf outfunc}~\verb&Out_Identity&\verb&,&~{\bf act}~$1.0$\verb&,&~\\
~~~~~~~~~~~~~~~~~~~~~~~~~~~~~~~~~~~~~{\bf site}~\verb&inhibSite&\verb&:&~\verb&colorType&\verb&;&\\
~{\small \{\verb{actfunc should be a special function (UFColor){\}}\\[.15cm]
{\bf structure}\\
~~{\bf array}\verb&[&\verb&size&\verb&]&~{\bf of}~{\bf clique}\verb&[&4\verb&]&~{\bf of}~\verb&colorType&~~\\
~~~~~~{\bf get}~{\bf name}~{\bf from}~\verb&colors&~{\bf matrix}~\verb&(&2\verb&,&2\verb&)&~{\bf at}~\verb&(&1\verb&,&~10\verb&)&\\
~~~~~~{\bf by}~\verb&-&$1.0$~{\bf through}~\verb&inhibSite&~~\verb&:&~\verb&mapUnits&\verb&;&\\[.15cm]
{\bf var}\\
~~{\bf int}~~\verb&:&~\verb&region1No&\verb&,&~\verb&region2No&\verb&;&\\
~~{\bf unit}~\verb&:&~\verb&color1&\verb&,&~\verb&color2&\verb&;&\\[.15cm] 
{\bf begin}\\
~~~~~~~~~~{\small \{\verb{look for adjacent regions{\}}\\
~~{\bf for}~\verb&region1No&~\verb&:=&~0~{\bf to}~\verb&size&\verb&-&1~{\bf do}\\
~~~~{\bf for}~\verb&region2No&~\verb&:=&~0~{\bf to}~\verb&size&\verb&-&1~{\bf do}\\
~~~~~~~~~~{\small \{\verb{foreach pair of regions{\}}\\
~~~~~~{\bf if}~\verb&(&\verb&regionNames&\verb&[&\verb&region1No&\verb&]&\verb&,&~\verb&regionNames&\verb&[&\verb&region2No&\verb&]&\verb&)&~{\bf in}~\verb&borderMap&~\\
~~~~~~~~~~~~~~~~~{\bf and}~~\verb&region1No&~\verb&<>&~\verb&region2No&~{\bf then}\\
~~~~~~~~~~{\small \{\verb{adjacent regions: connect correspondent colors{\}}\\
~~~~~~~~{\bf foreach}~\verb&color1&~{\bf in}~\verb&mapUnits&\verb&[&\verb&region1No&\verb&]&\verb&,&~\verb&color2&~{\bf in}~\verb&mapUnits&\verb&[&\verb&region2No&\verb&]&~{\bf do}\\
~~~~~~~~~~\verb&color1&\verb&.&\verb&inhibSite&~\verb&<->&~\verb&color2&\verb&.&\verb&inhibSite&~\verb&:&~1\\
~~~~~~~~{\bf end}\\
~~~~~~{\bf end}\\
~~~~{\bf end}\verb&;&\\
~~~~{\bf foreach}~\verb&color1&~{\bf in}~\verb&mapUnits&\verb&[&\verb&region1No&\verb&]&~{\bf do}\\
~~~~~~\verb&color1&\verb&.&{\bf name}~\verb&:=&~\verb&regionNames&\verb&[&\verb&region1No&\verb&]&~\verb&|&~\verb&"_"&~\verb&|&~\verb&color1&\verb&.&{\bf name}\\
~~~~{\bf end}\\
~~{\bf end}\\
{\bf end}\verb&.&\\

           \end{tabbing}
      \end{minipage}
    }
    \caption{\label{ColorMapProg} Beispiel: Vier-F"arbung von Karten}
  \end{center}
\end{figure}
