\chapter{Aufbau eines Nessus-Programms}
\label{Programmaufbau}

Wenn man ein konnektionistisches Netz in {\bf Nessus} definieren will,
geht man so vor, da"s man das Netz zuerst in Teilnetze\index{Teilnetz}
zerlegt.  Merkmale dieser Teilnetze k"onnen zum Beispiel gleichartige
Zellen oder bestimmte Verbindungsmuster\index{Verbindung} sein.

Solche gleichartigen Teilnetze k"onnen in {\bf Nessus} besonders
bequem definiert werden.  Damit ist normalerweise schon ein gro"ser
Teil der Verbindungen des Netzes definiert.  F"ur Zellen mit gleichen
Parametern kann man Typen einf"uhren.

\begin{figure}[htb]
  \small
  \begin{center}
      \makebox[\textwidth][c]{
	   \setlength{\unitlength}{0.0125in}
\begin{picture}(265,184)(105,576)
\thicklines
\small
\put(105,731){\line( 1, 0){265}}
\put(105,586){\framebox(265,169){}}
\put(105,707){\line( 1, 0){265}}
\put(105,683){\line( 1, 0){265}}
\put(105,658){\line( 1, 0){265}}
\put(105,634){\line( 1, 0){265}}
\put(225,605){\makebox(0,0)[lb]{\raisebox{0pt}[0pt][0pt]{\bf Topologiedefinition}}}
\put(153,598){\makebox(0,0)[lb]{\raisebox{0pt}[0pt][0pt]{\it end}}}
\put(153,616){\makebox(0,0)[lb]{\raisebox{0pt}[0pt][0pt]{\it begin}}}
\put(153,737){\makebox(0,0)[lb]{\raisebox{0pt}[0pt][0pt]{\it network}}}
\put(153,689){\makebox(0,0)[lb]{\raisebox{0pt}[0pt][0pt]{\it typedef}}}
\put(153,640){\makebox(0,0)[lb]{\raisebox{0pt}[0pt][0pt]{\it var}}}
\put(153,713){\makebox(0,0)[lb]{\raisebox{0pt}[0pt][0pt]{\it const}}}
\put(153,664){\makebox(0,0)[lb]{\raisebox{0pt}[0pt][0pt]{\it structure}}}
\put(225,737){\makebox(0,0)[lb]{\raisebox{0pt}[0pt][0pt]{\bf Kopfzeile}}}
\put(225,713){\makebox(0,0)[lb]{\raisebox{0pt}[0pt][0pt]{\bf Konstantendefinitionen}}}
\put(225,689){\makebox(0,0)[lb]{\raisebox{0pt}[0pt][0pt]{\bf Typdefinitionen}}}
\put(225,640){\makebox(0,0)[lb]{\raisebox{0pt}[0pt][0pt]{\bf Variablendeklarationen}}}
\put(225,664){\makebox(0,0)[lb]{\raisebox{0pt}[0pt][0pt]{\bf Strukturdefinitionen}}}
\end{picture}

      }
    \caption{\label{KB1} Struktur\index{Struktur!eines Programms} eines {\bf Nessus}-Programms}\index{Programm!Aufbau}
  \end{center}
\end{figure}

Abbildung~\ref{KB1} zeigt die "ubliche Struktur\index{Struktur!eines
Programms} eines {\bf Nessus}-Programms\index{Programm!Aufbau}:

\begin{enumerate}
  
\item Im Konstantendefinitionsteil k"onnen Konstanten
\index{Konstante!Definition} definiert werden.  Dadurch wird die
Lesbarkeit eines Programms erh"oht.
  
\item Im Typdefinitionsteil k"onnen Zell- oder Site-Typen definiert
werden. Zell- oder Site-Typen sind im Prinzip nichts anderes als
benannte Kombinationen von Parameterwerten f"ur Zellen bzw. Sites
mit bestimmten Merkmalen.
  
\item Im Strukturdefinitionsteil werden Zellen zu bestimmten
vordefinierten Strukturen einschlie"slich ihrer impliziten
Verbindungen zusammengefa"st.
  
\item Im Variablendeklarationsteil m"ussen alle f"ur die Erg"anzung
der Topologie\index{Topologie} 	ben"otigten Variablen\index{Variable}
deklariert werden.
  
\item Der Topologiemodifikationsteil\index{Topologie} dient dann dazu,
die bisher definierten 	Strukturen zu erg"anzen und miteinander zu
verbinden.
\end{enumerate}

Alle diese Komponenten bis auf den Topologiemodifikationsteil
\index{Topologie} k"onnen entfallen.

\begin{center}
\fbox{
\begin{minipage}{\textwidth}
{\footnotesize
\begin{center}
\begin{eqnarray}
  \mbox{program} & ::= & \mbox{program\_header $[$const\_block$]$ $[$typedef\_block$]$}  \nonumber\\[.3 cm]
          &    & \mbox{$[$structure\_block$]$ $[$var\_block$]$ topology\_modif\_block}  \label{sytopology} \\ \nonumber
\end{eqnarray}
\end{center}
}
\end{minipage}
}
\end{center}
