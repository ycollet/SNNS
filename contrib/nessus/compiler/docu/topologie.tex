\chapter{Erg"anzung der Topologie}
\label{Topologie}


Die Strukturen mit vordefinierten Verbindungen,\index{Verbindung} die
im Strukturdefinitionsteil eines {\bf Nessus}-Programms definiert
werden k"onnen, nehmen dem Programmierer viele der n"otigen
Verbindungsdefinitionen ab. Diese Strukturen
\index{Struktur!Erg"anzung} m"ussen aber noch
 
\begin{itemize}
  \item untereinander verbunden und eventuell
  \item erg"anzt
\end{itemize}

werden. Die Definition der dazu notwendigen
Verbindungen\index{Verbindung}\index{Verbindung!Abfrage} erfolgt in
einem Block von Anweisungen, der in {\bf begin} und {\bf
end}\index{end} eingeschlossen wird.

\begin{center}
\fbox{
\begin{minipage}{\textwidth}
{\footnotesize
\begin{center}
\begin{eqnarray}
  \mbox{modification\_block} & ::= & \mbox{{\bf begin}\index{begin}\index{Topologie!Beginn des Definitionsblocks} statement\_list {\bf end}\index{end} } \label{symodificationblock}\\[.3 cm]
  \mbox{statement\_list} & ::= & \mbox{statement \{ {\bf ;} statement \} } \label{systatementlist}\\[.3 cm]
  \mbox{statement} & ::= & \mbox{assignment\_stmt}        \nonumber \\
                   &  |  & \mbox{connection\_definition}  \nonumber \\
                   &  |  & \mbox{control\_stmt}  \nonumber \\
                   &  |  & \mbox{empty}  \label{systatement} \\ \nonumber
\end{eqnarray}
\end{center}
}
\end{minipage}
}
\end{center}

Regel~\ref{symodificationblock} zeigt die Syntax dieses
Anweisungsblocks. Die einzelnen Anweisungen werden jeweils durch ein
Semikolon getrennt und strikt sequentiell in ihrer textuellen
Reihenfolge ausgef"uhrt.

\section{Zuweisungsanweisung}
\label{assignmentstatement}

\begin{center}
\fbox{
\begin{minipage}{\textwidth}
{\footnotesize
\begin{center}
\begin{eqnarray}
  \mbox{assignment\_stmt} & ::= & \mbox{variable {\bf :=} expression}  \nonumber\\
   & | & \mbox{variable {\bf .~iotype} {\bf :=} topology\_key} \label{syassignmentstmt} \\ \nonumber
\end{eqnarray}
\end{center}
}
\end{minipage}
}
\end{center}

Den Elementen eines Felds\index{Feld} von Zellen ({\bf array of unit})
oder Strukturen darf kein neuer Wert zugewiesen werden, es d"urfen
aber die Komponenten {\bf act}, {\bf iotype} und {\bf name} dieser
Zellen ver"andert werden, soweit sie keine Typparameter sind.



\section{Verbindungsdefinition}
\label{Verbindungsdefinition}

Eine Verbindung\index{Verbindung}\index{Verbindung!Definition} in {\bf
Nessus} geht immer von einer Quellzelle

\begin{itemize}
  \item zu einer Site einer\index{Site} Zielzelle oder
  \item direkt zur Zielzelle, falls diese keine Sites\index{Site} besitzt.
\end{itemize}

Au"ser der Quelle und dem Ziel einer Verbindung \index{Verbindung}
\index{Verbindung!Quellzelle} mu"s ihre Richtung (vorw"arts oder
bidirektional)\index{bidirectional} und ihr Gewicht angegeben werden.
Diese Angaben sind dieselben, die auch gemacht werden m"ussen, um die
interne Verbindung\index{Verbindung!Richtung} einer Struktur zu
spezifizieren, allerdings gibt es hier keine R"uckw"artsverbindungen.
\index{Verbindung!Definition} Syntaxregel~\ref{syconnectiondefinition}
zeigt den Aufbau einer Verbindungsdefinition.\index{Verbindung}

\begin{center}
\fbox{
\begin{minipage}{\textwidth}
{\footnotesize
\begin{center}
\begin{eqnarray}
  \mbox{connection\_definition} & ::= & \mbox{variable conn\_operator variable}  \nonumber \\
          &     & \mbox{$[$ {\bf ~:~}expression\_or\_random \index{random} $]$} \label{syconnectiondefinition}\\[.3 cm]
  \mbox{conn\_operator} & ::= & \mbox{{\bf \verb&->&}} \nonumber\\
          &  |  & \mbox{{\bf \verb&<->&}} \label{syconnnoperator} \\ \nonumber
\end{eqnarray}
\end{center}
}
\end{minipage}
}
\end{center}

Der Operator zur Definition einer neuen Verbindung \index{Verbindung}
\index{Verbindung!Definition} legt auch die Richtung der Verbindung
\index{Verbindung!Richtung} fest:

\begin{itemize}
  \item {\bf \verb&->&}~~definiert eine unidirektionale Verbindung in Vorw"artsrichtung  (Richtung
	des Pfeils).  \item {\bf \verb&<->&}~definiert eine
bidirektionale\index{bidirectional} Verbindung.\index{Verbindung}
\end{itemize}

Die Quelle einer einfachen vorw"artsgerichteten
Verbindung\index{Verbindung!Quellzelle} ist immer eine Zelle. Das Ziel
der Verbindung besteht nur aus der Angabe der Zielzelle, falls diese
keine Sites\index{Site} hat. Besitzt die Zielzelle Sites,\index{Site}
so wird das Ziel der Verbindung\index{Verbindung!Zielzelle} in der
Form \\ 	 
\centerline{unit\_specification{\bf ~.~}site\_type\_name} angegeben.
Dabei ist ``site\_type\_name'' der {\it Bezeichner \index{Bezeichner}
des Typs\/}, der f"ur die Site eingef"uhrt wurde, nicht etwa der Name
der Site.

Bei der Definition bidirektionaler Verbindungen sind beide Operatoren
von {\bf \verb&<->&} Zielzellen. Das hei"st, in diesem Fall m"ussen
f"ur beide Zellen die Sites angegeben werden, falls sie welche
besitzen.

Die Angabe des Gewichts der Verbindung (``{\bf ~:~}
expression\_or\_random'') ist optional. Wird das Gewicht nicht
angegeben, wird bei der "Ubersetzung der daf"ur vorgesehene
Standardwert 0.0 eingesetzt.


\section{Kontrollstrukturen}
\label{controlstatements}

In {\bf Nessus} kann die Abarbeitung des Anweisungsblocks durch
Wiederholungsanweisungen\index{Iteration}\index{Wiederholungsanweisung}
und bedingte Anweisungen beeiflu"st  werden.

\begin{center}
\fbox{
\begin{minipage}{\textwidth}
{\footnotesize
\begin{center}
\begin{eqnarray}
  \mbox{control\_stmt} & ::= & \mbox{loop\_stmt} \nonumber\\
          &  |  & \mbox{conditional\_stmt} \label{sycontrolstmt}\\[.3cm]
  \mbox{loop\_stmt} & ::= & \mbox{loop\_header statement\_list {\bf end}}
		\index{end} \label{syloopstatement} \\[.3cm] 
  \mbox{loop\_header} & ::= & \mbox{while\_header} \nonumber \\
                      &  |  & \mbox{for\_header} \nonumber \\
                      &  |  & \mbox{foreach\_header}\index{foreach} \label{syloopheader} \\ \nonumber
\end{eqnarray}
\end{center}
}
\end{minipage}
}
\end{center}

Diese Kontrollstrukturen d"urfen geschachtelt werden. Eine
Wiederholungsanweisung (Schleife)\index{Schleife} darf bedingte
Anweisungen (Verzweigungen) enthalten, die wiederum andere
Wiederholungsanweisungen\index{Iteration}\index{Wiederholungsanweisung}
enthalten d"urfen.


\subsection{Wiederholungsanweisungen}
\label{loopstatements}

Wie Syntaxregel~\ref{syloopheader} zeigt, gibt es drei Arten von
Wiederholungsanweisungen:\index{Iteration}\index{Wiederholungsanweisung} 

\begin{enumerate}
  \item Die abweisende Schleife:\index{Schleife!while} die Anweisungen des Rumpfes werden solange
	ausgef"uhrt, wie eine beliebig formulierte Bedingung wahr ist.
  \item Die Z"ahlschleife:\index{Z"ahlschleife}\index{Schleife!for} die Anweisungen des Rumpfes werden f"ur eine vorgegebene Reihenfolge von
	Werten einer Schleifenvariablen\index{Schleifenvariable} ausgef"uhrt.
  \item Die Abbarbeitung eines Felds:\index{Feld!Behandlung mit foreach} die Anweisungen des Rumpfes werden f"ur die Werte einer
	Schleifenvariablen\index{Schleifenvariable}\index{Schleife!foreach} durchgef"uhrt, die nacheinander die Werte aller Elemente eines Felds
	annimmt. 
\end{enumerate}


\subsubsection{{\bf while}-Schleife}
\label{whileloop}

\index{Iteration!while}\index{Wiederholungsanweisung!while}\index{Schleife!while} 
\begin{center}
\fbox{
\begin{minipage}{\textwidth}
{\footnotesize
\begin{center}
\begin{eqnarray}
  \mbox{while\_header} & ::= & \mbox{{\bf while}\index{while} logical\_expr {\bf do}}\index{do} \label{sywhileheader} \\ \nonumber
\end{eqnarray}
\end{center}
}
\end{minipage}
}
\end{center}

Die Anweisungen im Rumpf der {\bf while}-Schleife werden solange
iteriert, wie der logische Ausdruck im Kopf der Schleife wahr ist. Ist
der Kopf schon vor der ersten Iteration \index{Iteration}
\index{Wiederholungsanweisung} falsch, werden die Anweisungen des
Rumpfes gar nicht ausgef"uhrt.

Bei der Verwendung einer {\bf while}-Schleife\index{while} sollte
immer genau gepr"uft werden, ob die Schleife auch terminiert. Sonst
ger"at der {\bf Nessus}-Compiler\index{Programm!"Ubersetzung} beim
"Ubersetztungsvorgang in eine Endlosschleife, da er keine M"oglickeit
hat, die Terminierung\index{Terminierung} einer solchen Schleife zu
pr"ufen.
\index{Iteration!while}\index{Wiederholungsanweisung!while}\index{Schleife!while}

\subsubsection{{\bf for}-Schleife}
\label{forloop}

\index{Iteration!for}\index{Wiederholungsanweisung!!for}
\begin{center}
\fbox{
\begin{minipage}{\textwidth}
{\footnotesize
\begin{center}
\begin{eqnarray}
  \mbox{for\_header} & ::= & \mbox{{\bf for}\index{for} simple\_assignment {\bf to}\index{to} expression {\bf do}}\index{do}
			     \label{syforheader} \\[.3cm] 
  \mbox{simple\_assignment} & ::= & \mbox{identifier~{\bf :=~} expression}
			     \label{sysimpleassignment} \\ \nonumber
\end{eqnarray}
\end{center}
}
\end{minipage}
}
\end{center}
\index{Iteration!for}\index{Wiederholungsanweisung!!for}\index{Schleife!for} 

Als Schleifenvariable\index{Schleifenvariable!Datentyp} sind nicht
Variable aller Datentypen zul"assig, sondern nur derjenigen, die auch
in einer Bereichsdefinition verwendet werden d"urfen. Dies sind

\begin{itemize}
\index{Iteration!for}\index{Wiederholungsanweisung!!for}\index{Schleife!for} 
  \item ganze Zahlen ({\bf int})\index{int} und
  \item Zeichenketten\index{Zeichenkette!als Schleifenvariable} ({\bf
	string}),\index{string} die aus genau einem der f"ur Bereichsdefinitionen zul"assigen 
	Zeichen bestehen (siehe Abschnitte~\ref{Zeichenketten} und~\ref{internarraydef}).
\end{itemize}

Der Wert der Schleifenvariablen\index{Schleifenvariable!Ver"anderung}
darf innerhalb der Schleife {\it nicht\/} ver"andert werden. Eine
Variable, die Bestandteil des Ausdrucks ``final\_expr'' ist, darf zwar
ver"andert werden, das hat aber auf die Schleife keinen Einflu"s:

\begin{quote}
  Bei der Abarbeitung der Z"ahlschleife\index{Z"ahlschleife} wird zu Beginn aus dem Anfangs- und dem Endwert der
  Schleifenvariablen\index{Schleifenvariable!Anfangswert} die Anzahl
der Iterationen\index{Iteration}\index{Wiederholungsanweisung} bestimmt. 
\end{quote}

Der Wert der Schleifenvariablen\index{Schleifenvariable!Endwert}
bleibt nach der Abarbeitung der Schleife erhalten.
\index{Iteration!for} \index{Wiederholungsanweisung!!for}
\index{Schleife!for}



\subsubsection{{\bf foreach}-Schleife}
\label{foreachloop}

\index{Iteration!foreach} \index{Wiederholungsanweisung!!foreach}
\index{Schleife!foreach} Die {\bf foreach}-Schleife\index{foreach}
dient dazu, alle Elemente eines Felds\index{Feld} oder mehrerer Felder
auf gleiche Art und Weise zu behandeln. Regel~\ref{syforeachheader}
zeigt die Syntax dieser Schleife.

\begin{center}
\fbox{
\begin{minipage}{\textwidth}
{\footnotesize
\begin{center}
\begin{eqnarray}
  \mbox{foreach\_header} & ::= & \mbox{{\bf foreach}\index{foreach} loop\_variable\_list {\bf do}}\index{do} 
					\label{syforeachheader} \\[.3 cm]
  \mbox{loop\_variable\_list} & ::= & \mbox{loop\_variable \{{\bf ~,~}loop\_variable\}} 
					\label{syloopvariablelist} \\[.3 cm]
  \mbox{loop\_variable} & ::= & \mbox{identifier {\bf in} element\_group}  \label{syloopvariable} \\ \nonumber
\end{eqnarray}
\end{center}
}
\end{minipage}
}
\end{center}
\index{Iteration!foreach}
\index{Wiederholungsanweisung!!foreach}
\index{Schleife!foreach} 

Die Besonderheit der {\bf foreach}-Schleife besteht darin, da"s
mehrere Schleifenvariablen synchron inkrementiert werden. Dazu mu"s
``loop\_variable\_list'' folgenden Bedingungen gen"ugen:

\begin{itemize}
  \item In jeder ``loop\_variable''-Definition mu"s die Variable vom Basistyp des Felds\index{Feld}
	sein, wie in ``inclusion\_expression'' (siehe Syntaxregel~\ref{syinclusionexpression}) auch.
  \item Besteht ``loop\_variable\_list'' aus mehreren solcher ``loop\_variable''-Definitionen,
	m"ussen die Felder, die die Wertemengen der
	Schleifenvariablen\index{Schleifenvariable!in {\bf foreach}}
	initialisieren, alle gleich lang sein.\footnote{In der
jetzigen Implementierung werden unterschiedlich gro"se Wertemengen der
Schleifenvariablen jedoch toleriert. Bei der "Ubersetzung wird eine
Warnung ausgegeben und die Schleife beendet, sobald die {\it erste\/}
Schleifenvariable alle ihre vorgesehenen Werte erreicht hat.}
\end{itemize}
\index{Iteration!foreach}\index{Wiederholungsanweisung!!foreach}\index{Schleife!foreach} 

Eine {\bf foreach}-Schleife\index{foreach} wird folgenderma"sen ausgef"uhrt:
\begin{enumerate}
  \item Alle Schleifenvariablen,\index{Schleifenvariable!in {\bf
foreach}!Inkrementierung} die in der ``loop\_variable\_list'' definiert wurden, werden 
	mit dem ersten Element ihrer jeweiligen Wertemenge initialisiert. 
  \item Der Anweisungsblock im Rumpf der {\bf
foreach}-Schleife\index{foreach} wird f"ur diese Wertekombination der 
	Schleifenvariablen\index{Schleifenvariable!in {\bf foreach}!Wertekombination} ausgef"uhrt.
  \item Es wird gepr"uft, ob die  Schleifenvariablen schon alle Werte ihrer jeweiligen
	Wertemenge angenommen haben:
	\begin{itemize}
	  \item Wenn sie noch nicht alle Werte angenommen haben, erhalten jetzt {\it alle
		Schleifenvariablen\/} den n"achsten Wert ihrer jeweiligen Wertemenge. Die
		Abarbeitung wird mit 2. fortgesetzt.
	  \item Wenn die Schleifenvariablen schon alle ihre vorgesehenen Werte angenommen haben, ist
		die Ausf"uhrung der Schleife beendet.
	\end{itemize}
\end{enumerate}
\index{Iteration!foreach}\index{Wiederholungsanweisung!!foreach}\index{Schleife!foreach} 


\subsection{Bedingte Anweisungen}
\label{condstatement}

\begin{center}
\fbox{
\begin{minipage}{\textwidth}
{\footnotesize
\begin{center}
\begin{eqnarray}
  \mbox{conditional\_stmt} & ::= & \mbox{{\bf if}\index{if} logical\_expr {\bf then}\index{then} statement\_list} \nonumber\\
                           &     & \mbox{$[$else\_if\_part$]$ $[$else\_stmt$]$\index{else} {\bf end}}\index{end} 
					\label{syconditionalstatement}\\[.3 cm]
  \mbox{else\_if\_part} & ::= & \mbox{else\_if\_stmt \{else\_if\_stmt\}} \label{syelseifpart} \\[.3 cm]
  \mbox{else\_if\_stmt} & ::= & \mbox{{\bf elseif}\index{elseif} logical\_expr {\bf then}\index{then} statement\_list}  \label{syelseifstmt} \\[.3 cm]
  \mbox{else\_stmt}     & ::= & \mbox{{\bf else}\index{else}  statement\_list} \label{syelsestmt}\\ \nonumber
\end{eqnarray}
\end{center}
}
\end{minipage}
}
\end{center}
\index{Anweisung!bedingte}\index{Verzweigungsanweisung}

Die bedingte Anweisung wird folgenderma"sen  abgearbeitet:
\begin{enumerate}
  \item Es werden nacheinander die Bedingungen ``logical\_expr'' ausgewertet, bis die erste davon
	wahr ist.
  \item Wurde eine solche wahre Bedingung gefunden, wird die Liste der Anweisungen, die dem
	dazugeh"origen {\bf then}\index{then} folgt, ausgef"uhrt. Dann ist die bedingte Anweisung beendet,
	alle nachfolgenden Bedingungen und Anweisungen werden ignoriert.
  \item Wurde keine wahre Bedingung gefunden, werden die Anweisungen hinter dem {\bf else}\index{else}
	ausgef"uhrt. 
\end{enumerate}

Es wird in einer bedingten Anweisung also h"ochstens einer der
Anweisungsbl"ocke nach einer Bedingung ausgef"uhrt. Ist keine der
Bedingungen wahr und kein {\bf else}-Teil vorhanden, so bewirkt die
bedingte Anweisung nichts.





\section{Beispiel: Necker-W"urfel}
\label{NeckerBsp}

Hier soll ein {\bf Nessus}-Programm gezeigt werden, das ein Netz, das
ein bekanntes Vexierbild - den Necker-W"urfel - darstellt, definiert.
Die perspektivische Darstellung des W"urfels wie in
Abbildung~\ref{WuerfelBild} l"a"st zwei verschiedene Deutungen zu:

\begin{enumerate}
  \item Man sieht den W"urfel von vorn, das hei"st, die vertikale Kante B-C ist dem Betrachter zugewandt.
  \item Man sieht den W"urfel schr"ag von unten, das hei"st, die untere Kante H-G ist dem Betrachter zugewandt.
\end{enumerate}

\begin{figure}[htb]
  \begin{center}
%    \mbox{
      \begin{minipage}{\textwidth}
%        \vspace{6cm}
	\screendump{0.7}{necker.ps}
      \end{minipage}
%    }
    \caption{\label{WuerfelBild} Netzwerk zur Darstellung des Necker-W"urfels}
  \end{center}
\end{figure}

Bei solchen Vexierbildern springt normalerweise der menschliche
Betrachter von einer Deutung zur anderen. In \cite{fel01,fel02} wird
ein Netzwerk f"ur den Necker-W"urfel vorgestellt, das als Ergebnis die
Entscheidung f"ur eine der beiden Betrachtungsweisen liefert.

Dieses Netz besteht aus zwei miteinander konkurrierenden Strukturen.
Jede dieser Strukturen stellt eine der Betrachtungsweisen des W"urfels
dar und besteht aus dreizehn Zellen:

\begin{itemize}
  \item Acht Zellen, die den Ecken des W"urfels entsprechen. Diese Ecken werden mit ``A'' \verb&..&
	``H'' bezeichnet. 
  \item Vier Zellen, die angeben, welche Ecken des W"urfels dem Betrachter zugewandt sind. 
  \item Eine Zelle, die das Ergebnis liefert. Sie ist aktiviert, wenn
	sich das Netz f"ur die Betrachtungsweise entscheidet, die von derjenigen der beiden 
	Strukturen, zu der sie geh"ort, dargestellt wird.  
\end{itemize}

Abbildung~\ref{WuerfelBild}  zeigt dieses Netz in der Darstellung von SNNS.

In diesem Fall bietet es sich an, das Teilnetz, das f"ur jede der
Betrachtungsweisen ja gleich ist, in einem eigenen Programm zu
definieren und dann diese beiden Teilnetze miteinander zu verbinden.
Abbildung~\ref{NeckerProg} zeigt die Programme, die das Netz
definieren.

\begin{figure}[htbp]
  \begin{center}
    \mbox{
      \begin{minipage}{\textwidth}
	   \small
           \begin{tabbing}\=nn\=\kill
	   {\bf network}~\verb&neckerCube&\verb&(&\verb&pathName&~\verb&:=&~\verb&"./"&\verb&)&\verb&;&\\
~~{\footnotesize \{\verb{version Feldman '88{\}}\\
{\bf const}\\
~~\verb&inDat&~\verb&=&~\verb&pathName&~\verb&|&~\verb&"NeckerView.n"&\verb&;&\\
~~\verb&subFile&~\verb&=&~{\bf file}~\verb&inDat&~{\bf of}~{\bf subnet}\verb&;&\\
{\bf structure}\\
~~{\bf subnet}~\verb&subFile&~{\bf at}\verb&(&0\verb&,&~1\verb&)&~\verb&:&\verb&BCloser&\verb&;&~{\footnotesize \{\verb{subnet represents view with closer B edge{\}}\\
~~{\bf subnet}~\verb&subFile&~{\bf at}\verb&(&2\verb&,&~1\verb&)&~\verb&:&\verb&HCloser&\verb&;&~{\footnotesize \{\verb{subnet represents view with closer H edge{\}}\\
{\bf var}\\
~~{\bf unit}~\verb&:&~\verb&x&\verb&,&~\verb&y&\verb&;&\\
~~{\bf array}~{\bf of}~{\bf unit}~\verb&:&~\verb&bStruct&\verb&,&~\verb&hStruct&\verb&;&\\
{\bf begin}\\
~~{\footnotesize \{\verb{connect subnets by inhibitory connections{\}}\\
~~{\bf foreach}~\verb&bStruct&~{\bf in}~\verb&[&\verb&BCloser&\verb&.&\verb&FrontEdges&\verb&,&~\verb&BCloser&\verb&.&\verb&BackEdges&\verb&]&\verb&,&\\
~~~~~~~~~~\verb&hStruct&~{\bf in}~\verb&[&\verb&HCloser&\verb&.&\verb&FrontEdges&\verb&,&~\verb&HCloser&\verb&.&\verb&BackEdges&\verb&]&~{\bf do}\\
~~~~{\bf foreach}~\verb&x&~{\bf in}~\verb&bStruct&\verb&,&~\verb&y&~{\bf in}~\verb&hStruct&~{\bf do}\\
~~~~~~\verb&x&~\verb&<->&~\verb&y&~\verb&:&~\verb&-&~$.575$~~~~~~{\footnotesize \{\verb{inhibitory connections among edges of both cubes{\}}\\
~~~~{\bf end}\\
~~{\bf end}\verb&;&\\
~~{\bf foreach}~\verb&x&~{\bf in}~\verb&BCloser&\verb&.&\verb&Relations&\verb&,&~\verb&y&~{\bf in}~\verb&HCloser&\verb&.&\verb&Relations&~{\bf do}\\
~~~~~~\verb&x&~\verb&<->&~\verb&y&~\verb&:&~\verb&-&~$.8$~~~~~~~~{\footnotesize \{\verb{inhibitory connections - relational units of both cubes{\}}\\
~~{\bf end}\verb&;&\\
~~\verb&BCloser&\verb&.&\verb&Result&~\verb&<->&~\verb&HCloser&\verb&.&\verb&Result&~\verb&:&~\verb&-&$1.0$\verb&;&\\
~~\verb&BCloser&\verb&.&\verb&Result&\verb&.&{\bf name}~\verb&:=&~\verb&"bCloser"&\verb&;&\\
~~\verb&HCloser&\verb&.&\verb&Result&\verb&.&{\bf name}~\verb&:=&~\verb&"hCloser"&\\
{\bf end}\verb&.&\\[.5 cm]

{\bf network}~\verb&neckerView&\verb&(&\verb&)&\verb&;&\\
~~{\footnotesize \{\verb{version Feldman '88 -- subnet for one view{\}}\\
{\bf const}~\\
~~\verb&names&~\verb&=&~\verb&[&\verb&"A"&\verb&..&\verb&"H"&\verb&]&\verb&;&\\
~~\verb&relNam&~\verb&=&~\verb&[&\verb&"A<<>>E"&\verb&,&~\verb&"B<<>>F"&\verb&,&~\verb&"C<<>>G"&\verb&,&~\verb&"D<<>>H"&\verb&]&\verb&;&\\
{\bf structure}\\
~~{\bf ring}\verb&[&4\verb&]&~{\bf with}~{\bf act}~{\bf random}~{\bf get}~{\bf name}~{\bf from}~\verb&names&\verb&[&0\verb&..&3\verb&]&~\\
~~~~~~~~~~{\bf matrix}~\verb&(&2\verb&,&2\verb&)&~{\bf at}~\verb&(&3\verb&,&~5\verb&)&~\verb&<->&~{\bf by}~$.200$~\verb&:&~\verb&FrontEdges&\verb&;&\\
~~{\bf ring}\verb&[&4\verb&]&~{\bf with}~{\bf act}~{\bf random}~{\bf get}~{\bf name}~{\bf from}~\verb&names&\verb&[&4\verb&..&7\verb&]&\\
~~~~~~~~~~~~~~~{\bf matrix}~\verb&(&2\verb&,&2\verb&)&~{\bf at}~\verb&(&4\verb&,&~3\verb&)&~\verb&<->&~{\bf by}~$.200$~\verb&:&~\verb&BackEdges&\verb&;&\\
~~{\bf ring}\verb&[&4\verb&]&~{\bf with}~{\bf act}~{\bf random}~{\bf get}~{\bf name}~{\bf from}~\verb&relNam&~\\
~~~~~~~~~~~~~~~{\bf matrix}~\verb&(&2\verb&,&2\verb&)&~{\bf at}~\verb&(&2\verb&,&~2\verb&)&~~\verb&<->&~{\bf by}~$.300$~\verb&:&~\verb&Relations&\verb&;&\\
~~{\bf unit}~{\bf with}~{\bf iotype}~{\bf output}\verb&,&~{\bf act}~{\bf random}~{\bf at}~\verb&(&2\verb&,&~0\verb&)&~\verb&:&~\verb&Result&\verb&;&\\
{\bf var}\\
~~{\bf unit}\verb&:&~\verb&x&\verb&,&~\verb&y&\verb&,&~\verb&z&\verb&,&~\verb&res&\verb&;&\\
{\bf begin}\\
~~{\footnotesize \{\verb{define connections between front and back face edges of cube{\}}\\
~~{\bf foreach}~\verb&x&~{\bf in}~\verb&FrontEdges&\verb&,&~\verb&y&~{\bf in}~\verb&BackEdges&\verb&,&~\verb&z&~{\bf in}~\verb&Relations&~{\bf do}\\
~~~~\verb&x&\verb&<->&\verb&y&~\verb&:&~$.2$\verb&;&~~~~~~~~{\footnotesize \{\verb{connect front and back faces of cube{\}}\\
~~~~\verb&x&\verb&<->&\verb&z&~\verb&:&~$.3$\verb&;&~~~~~~~~{\footnotesize \{\verb{connect front face to relational units{\}}\\
~~~~\verb&y&\verb&<->&\verb&z&~\verb&:&~$.3$~~~~~~~~~{\footnotesize \{\verb{connect front face to relational units{\}}\\
~~{\bf end}\verb&;&~\\
~~{\bf foreach}~\verb&x&~{\bf in}~\verb&Relations&~{\bf do}\\
~~~~\verb&x&\verb&<->&\verb&Result&~\verb&:&~$.5$~~~~~~~{\footnotesize \{\verb{connect relational units to result unit{\}}\\
~~{\bf end}\\
{\bf end}\verb&.&\\
           \end{tabbing}
      \end{minipage}
    }
    \caption{\label{NeckerProg} Beispielprogramm: Necker-W"urfel}
  \end{center}
\end{figure}

