\chapter{Ausdr"ucke und Variablendeklarationen}
\label{EinfacheAusdruecke}
\label{LogischeAusdruecke}


Ausdr"ucke sind in {\bf Nessus} streng typgebunden und bestimmte
Operatoren sind auf einzelne Datentypen beschr"ankt.
Syntaxregel~\ref{syexpression} zeigt die Syntax eines Ausdrucks.

\section{Vorzeichen und bin"are Operatoren}
\label{numExpr}

\begin{center}
\fbox{
\begin{minipage}{\textwidth}
{\footnotesize
\begin{center}
\begin{eqnarray}
  \mbox{expression} & ::= & \mbox{num\_expression} \nonumber\\
          &  |  & \mbox{string\_expression} \label{syexpression}\\[.3 cm]
  \mbox{num\_expression} & ::= & \mbox{num\_expression \verb&+& term} \nonumber\\
          &  |  &  \mbox{num\_expression $-$ term} \nonumber\\
          &  |  &  \mbox{term} \label{synumexpression}\\[.3 cm]
  \mbox{term} & ::= & \mbox{term \verb&*& factor} \nonumber\\
          &  |  &  \mbox{term $/$ factor} \nonumber\\
          &  |  &  \mbox{term {\bf div}\index{div} factor} \nonumber\\
          &  |  &  \mbox{term {\bf mod}\index{mod} factor} \nonumber\\
          &  |  &  \mbox{factor} \label{syterm}\\  \nonumber
\end{eqnarray}
\end{center}
}
\index{Addition} \index{Subtraktion} \index{Multiplikation} \index{Division!ganzzahlige} \index{Division!rationale} 
\index{Vorzeichen} \index{Modulo}
\end{minipage}
}
\end{center}


Dabei binden $+$ und $-$ \index{Bindung}\index{Vorrang} weniger stark
als $*$, $/$, {\bf div}\index{div} oder {\bf mod}. Ausdr"ucke mit
mehrerern Operatoren gleicher Bindung werden von links nach rechts
ausgewertet (linksassoziativ).

\index{Ausdr"ucke!einfache} \index{Addition} \index{Subtraktion}
\index{Multiplikation} \index{Division!ganzzahlige}
\index{Division!rationale} \index{Vorzeichen} \index{Modulo}

\index{Ausdr"ucke!einfache} \index{Ausdruck!numerischer} In {\bf
Nessus} wird strikt zwischen der Division ganzzahliger und der
Division rationaler Zahlen\index{Zahl!rationale!Division}
unterschieden.  \index{Addition} \index{Subtraktion}
\index{Multiplikation} \index{Division!ganzzahlige}
\index{Division!rationale} \index{Vorzeichen} \index{Modulo}

\begin{itemize}
  \item In einem Ausdruck ``x {\bf div}\index{div} y'' m"ussen x und y ganze
Zahlen\index{Zahl!ganze!Division} sein. Das Ergebnis ist 
	ebenfalls ganzzahlig.  \index{Ausdruck!numerischer}
  \item In ``x $/$ y'' d"urfen x und y ganzzahlig oder rational sein, aber das Ergebnis
wird immer - 
	unabh"angig vom Typ der Operanden - rational sein. 
\end{itemize}

Die Operanden des Modulo-Operators {\bf mod}\index{mod} m"ussen
ebenfalls ganze Zahlen sein. Das Ergebnis ist immer eine ganze
Zahl.\index{Zahl!ganze!Modulo-Operator} ``x {\bf mod}\index{mod} y''
hat x bez"uglich der Restklasse y zum Ergebnis.
\index{Ausdruck!numerischer} \index{Addition} \index{Subtraktion}
\index{Multiplikation} \index{Division!ganzzahlige}
\index{Division!rationale} \index{Vorzeichen} \index{Modulo}

F"ur die Operatoren $+$, $-$ und $*$ gilt die Regel, da"s das Ergebnis
eine ganze Zahl ist, wenn beide Operanden ganze Zahlen sind. Ist einer
der Operanden eine rationale Zahl,\index{Zahl!rationale} ist auch das
Ergebnis rational.  \index{Ausdruck!numerischer}

\begin{center}
\fbox{
\begin{minipage}{\textwidth}
{\footnotesize
\begin{center}
\begin{eqnarray}
  \mbox{factor} & ::= & \mbox{{\bf (} expression {\bf )}} \nonumber\\
          &  |  & \mbox{sign factor}  \nonumber\\
          &  |  & \mbox{function\_expression}  \nonumber\\
          &  |  & \mbox{variable} \label{syfactor}\\[.3 cm]
  \mbox{sign} & ::= & +\;\;|\;\; - \label{sysign}\\[.3 cm]
  \mbox{variable} & ::= & \mbox{simple\_constant} \nonumber\\
          &  |  & \mbox{identifier} \nonumber\\
          &  |  & \mbox{variable {\bf ~.~} selection\_field} \nonumber\\
          &  |  & \mbox{variable\index{Variable} field\_index} \label{syvariable}\\[.3cm]
  \mbox{simple\_constant} & ::= & \mbox{intliteral}\index{Literal!ganzzahliges} \\
          & | & \mbox{floatliteral}\index{Literal!rationales} \\
          & | & \mbox{stringliteral}\index{Literal!Zeichenkette}
			\label{syunsignedconstant} \\[.3cm]
  \mbox{string\_expression} & ::= & \mbox{string\_expression ``$|$'' variable} \nonumber\\
          &  |  & \mbox{variable} \label{systringexpression}\\ \nonumber
\end{eqnarray}
\end{center}
}
\end{minipage}
}
\end{center}

Sind die Operanden Zeichenketten, so liefert x $|$ y die
Konkatenation\index{Konkatenation} von x und y. So ist zum Beispiel

	\centerline{``nessus'' $|$ ``programm'' $=$ ``nessusprogramm''.}



\section{Variablen}

\subsection{Zugriff auf einzelne Elemente eines Felds}\index{Feld}

Die Angabe des Feldelements\index{Feld!Zugriff auf Feld} erfolgt wie
in Syntaxregel~\ref{syfieldindex} in eckigen Klammern. Dabei mu"s man
aber beachten, da"s

\begin{itemize}
  \item der Datentyp des Feldindexes\index{Feld!Index} {\bf int}\index{int} sein mu"s und
  \item der erste Index eines Felds\index{Feld!erstes Element} immer Null und nicht Eins ist.
\end{itemize}

Der zweite Punkt f"uhrt dazu, da"s zum Beispiel auf das letzte Element
eines als {\bf ``cluster}$[$100$]$ : unitSet'' vereinbarten Felds mit
dem Ausdruck ``unitSet$[$99$]$'' zugegriffen werden mu"s und
``unitset$[$100$]$'' gar nicht definiert ist.
\index{Ausdruck!Feldelement}

\begin{center}
\fbox{
\begin{minipage}{\textwidth}
{\footnotesize
\begin{center}
\begin{eqnarray}
  \mbox{field\_index} & ::= & \mbox{``${\bf [}$'' expression ``${\bf
]}$''} \label{syfieldindex} \\ \nonumber 
\end{eqnarray}
\end{center}
}
\end{minipage}
}
\end{center}


\subsection{Selektion}
\label{Selektion}

\subsubsection{Selektion von Komponenten von Zellen}

Die Selektion\index{Selektion} dient dazu, um auf

\begin{itemize} \index{Ausdruck!Selektion}
  \item eine Komponente einer Zelle oder
  \item einen Bezeichner\index{Bezeichner} eines Teilnetzes\index{Teilnetz!Zugriff auf Komponenten} 
\end{itemize}

zuzugreifen. Damit kann zum Beispiel der Name einer Zelle in einem
Ausdruck von Zeichenketten verwendet werden.

Der Zugriff auf die Komponente einer Zelle erfolgt dadurch, da"s man
nach dem Selektionsoperator das entsprechende Schl"usselwort angibt.
Es kann jedoch auf alle diejenigen Komponenten nicht zugegriffen
werden, die "uber Bezeichner\index{Bezeichner} angegeben werden und
die nicht von einem der einfachen Datentypen sind. Dies sind {\bf
actfunc} und {\bf outfunc}.

Man beachte aber:
\begin{quote}
  Wenn der Aktivierungswert einer Zelle "uber das Schl"usselwort {\bf
random} definiert wurde, \index{random}\index{Aktivierungswert}
ist er im restlichen {\bf Nessus}-Programm nicht verf"ugbar, das
hei"st die Selektion\index{Selektion} dieses Aktivierungswerts liefert
``undefiniert''.
\end{quote}

Nur bei der Angabe der Zielsite einer Verbindung wird hinter dem
Ausdruck, der die Zelle spezifiziert mit dem Selektionsoperator der
Typname der Zielsite angegeben. Der Ausdruck

\centerline{unit\_specification~{\bf .}~site\_type\_name} \index{Ausdruck!Selektion}

definiert die Site, f"ur die der Typ ``site\_type\_name'' eingef"uhrt
wurde, der durch ``unit\_specification'' angegebenen Zelle als Ziel
einer Verbindung.  Man beachte, da"s nur in Zusammenhang mit
Verbindungsdefinitionen auf einen Sitetyp einer Zelle zugegriffen
werden kann. Es ist nicht m"oglich, auf diesen Sitetyp andere
Operatoren anzuwenden.

\begin{center}
\fbox{
\begin{minipage}{\textwidth}
{\footnotesize
\begin{center}
\begin{eqnarray}
  \mbox{selection\_field} & ::= & \mbox{{\bf name}\index{name} $|$ {\bf act}} \nonumber \\
          &  |  & \mbox{{\bf identifier}} \label{syselectionfield}\\ \nonumber
\end{eqnarray}
\end{center}
}
\end{minipage}
}
\end{center}


\subsubsection{Selektion von Komponenten eines Teilnetzes}

Soll auf eine Komponente eines Teilnetzes \index{Teilnetz}
\index{Teilnetz!Zugriff auf Komponenten} zugegriffen werden, so
schreibt man nach dem \index{Ausdruck!Selektion} Selektionsoperator
\index{Selektion} den Bezeichner dieser Komponente im {\bf
Nessus}-Programm, das das betreffende Teilnetz\index{Teilnetz}
definiert. Vor dem Selektionsoperator mu"s der Bezeichner
\index{Bezeichner} stehen, der f"ur das Teilnetz \index{Teilnetz}
eingef"uhrt wurde (Bezeichner der Struktur vom Typ {\bf subnet}).
\index{Ausdruck!Selektion}

Die Namen in externen {\bf Nessus}-Programmen, die als
Teilnetze\index{Teilnetz} eingebunden werden, und die
Bezeichner\index{Bezeichner} des Netzdefinitionsprogramms sind
voneinander v"ollig unabh"angig und d"urfen gleich sein. Die
dazugeh"origen Objekte sind durch den vorangestellten Bezeichner des
Teilnetzes\index{Teilnetz} eindeutig festgelegt. Wo dieser
vorangestellte Bezeichner fehlt, ist ebenso eindeutig, da"s ein
Bezeichner des Netzdefinitionsprogramms gemeint ist.

Abbildung~\ref{SelektionBsp}\index{Selektion} zeigt an einem Beispiel
die Anwendung des Selektionsoperators. \index{Ausdruck!Selektion}
 
\begin{figure}[htbp]
  \begin{center}
    \mbox{
      \begin{minipage}{\textwidth}
           \begin{tabbing}\=nn\=\kill
	   {\bf network}~\verb&selectionBsp&\verb&(&\verb&)&\verb&;&\\[.15cm]
{\bf cons}\\
~~\verb&extFile&~\verb&=&~{\bf file}~\verb&"SubNet.n"&~{\bf of}~{\bf subnet}\verb&;&\\[.15cm]
{\bf struct}\\
~~{\bf subnet}~\verb&extfile&~ {\bf at} (10, 10) \verb&:&~\verb&subId&\verb&;&\\
~~~~~~{\small \{\verb{Damit ist das Teilnetz jetzt definiert.{\}}\\[.15cm]
{\bf var}\\
~~{\bf string}~\verb&:&~\verb&newString&\verb&;&\\[.15cm]
{\bf begin}\\
~~\verb&newstring&~\verb&:=&~\verb&subId&\verb&.&\verb&unitSet&\verb&[&83\verb&]&\verb&.&{\bf name}\verb&;&\\
{\bf end}\verb&.&\\ 

           \end{tabbing}
      \end{minipage}
    }
    \caption{\label{SelektionBsp} Beispiel: Zugriff auf Objekte eines Teilnetzes}
  \end{center}
\end{figure}

In diesem Beispiel soll der Name einer Zelle einer {\bf
string}-Variablen\index{string} zugewiesen werden. Die Datei
``SubNet.n'' enth"alt ein {\bf Nessus}-Programm, das ein Netz
definiert. Dieses Netz wird als Teilnetz in ``selectionBsp'' eingebaut
und erh"alt den logischen Namen ``subId''.

Damit der Selektionsausdruck\index{Selektion} in dem Beispiel
semantisch korrekt ist, mu"s ``subId'' folgende Bedingungen erf"ullen:
\begin{itemize}
  \item In ``subId'' gibt es eine Struktur\index{Struktur!in Teilnetz} von Zellen, die in dem Programm, das ``subId'' definiert,
	mit ``unitSet'' bezeichnet wird.
  \item Diese Struktur\index{Struktur!in Teilnetz} ``unitSet'' besteht aus mindestens 84 Zellen.
\end{itemize}

Das Beispiel zeigt auch, da"s Selektion\index{Selektion} und die
Auswahl eines Feldelements\index{Feldelement!Auswahl} gleich stark
binden.\index{Bindung}\index{Vorrang} Gemischte Ausdr"ucke wie der des
Beispiels werden strikt von links nach rechts bewertet.



\subsubsection{Sites von Teilnetzen}\index{Teilnetz!Sites}
\label{subnetsites}

Die Unabh"angigkeit der Bezeichner\index{Bezeichner} verschiedener
{\bf Nessus}-Programme endet da, wo die Namen auch f"ur die
Simulation durch SNNS eine Rolle spielen. SNNS sieht -- im Gegensatz
zu {\bf Nessus} nicht den Typ einer Site, sondern nur den Namen dieser
Site.  Das f"uhrt zu der folgenden Konsequenz: Auch "uber verschiedene
{\bf Nessus}-Programme hinweg mu"s -- sobald die Netze verkn"upft
werden -- die Zuordnung ``Sitename'' auf ``Sitefunktion'' eindeutig
sein. Es ist also genauso wie in einem einzelnen Programm nicht
erlaubt, zwei Sitetypen mit demselben Sitenamen aber mit verschiedenen
Sitefunktionen zu definieren.

{\bf Verboten} ist demnach 
  \begin{center}
    \mbox{
      \begin{minipage}{\textwidth}
           \begin{tabbing}\=nn\=\kill
	   {\bf network} mainNet{\bf ();} \\
	   {\bf ~~typedef} \\
	   ~~~~{\bf site with name} "`uniqueSite"`{\bf,~sitefunc}~mainFunc{\bf~:~}mainSite{\bf ;} \\
	   ~~~~~~~~~~~~~~~~~~~~~~~~~~~~~~~~~~~~{\bf \ldots} \\[.5cm]
	   {\bf network} subNet{\bf ();} \\
	   {\bf ~~typedef} \\
	   ~~~~{\bf site with name} "`uniqueSite"'{\bf ,~sitefunc}~newSiteFunc{\bf ~:~}subNetSite{\bf ;}
	   ~~~~~~~~~~~~~~~~~~~~~~~~~~~~~~~~~~~~{\bf \ldots} \\
           \end{tabbing}
      \end{minipage}
    }
  \end{center}




\section{Numerische Funktionen}

Die in {\bf Nessus} verf"ugbaren Funktionen eines Arguments sind alle
numerisch, die einer Argumentliste bilden eine Liste von Argumenten
auf ein Ergebnis des Typs ``simple\_type'' ab.
\index{Ausdruck!Funktion}

\begin{center}
\fbox{
\begin{minipage}{\textwidth}
{\footnotesize
\begin{center}
\begin{eqnarray}
  \mbox{function\_expression} & ::= & \mbox{simple\_function} \nonumber\\
          &  |  &  \mbox{list\_function} \label{syfunctionexpression}\\[.3 cm]
  \mbox{simple\_function} & ::= & \mbox{simple\_func\_key {\bf (} expression {\bf )}}
		\label{sysimplefunction}\\[.3 cm] 
  \mbox{simple\_func\_key} & ::= & \mbox{{\bf sin}\index{sin} $|$ {\bf cos}\index{cos} $|$
		{\bf arctan}}\index{arctan} \nonumber\\ 
          &  |  & \mbox{{\bf exp}\index{exp} $|$ {\bf ln}\index{ln} $|$ {\bf
		sqr}\index{sqr} $|$ {\bf sqrt}}\index{sqrt} \label{sysimplefunckey}\\
		\nonumber\index{Funktion!sin} \index{Funktion!cos}\index{Funktion!arctan}  
\end{eqnarray}
\end{center}
}
\end{minipage}
}
\end{center}

Die verf"ugbaren trigonometrischen Funktionen sind {\bf sin}, {\bf
cos} und {\bf arctan}.  Das Argument dieser Funktionen kann
ganzzahlig\index{Zahl!ganze!als Funktionsargument} oder rational sein.
Der Ergebnistyp dieser Funktionen \index{Ausdruck!Funktion} ist immer
{\bf float}.\index{float}

Au"serdem gibt es noch die transzendenten Funktionen {\bf exp}, {\bf
ln}, {\bf sqr} und {\bf sqrt}. Auch die Argumente dieser Funktionen
k"onnen ganzzahlig oder rational sein.  Der Ergebnistyp von {\bf exp},
{\bf ln}\index{ln} und {\bf sqrt}\index{sqrt} ist immer vom Typ {\bf
float}.\index{float} Der Ergebnistyp der Quadratfunktion {\bf
sqr}\index{sqr} ist der Typ des Arguments.


\section{Listenfunktionen}

Es gibt in {\bf Nessus} zwei  Listenfunktionen, die aus einer Liste
mehrerer Argumente eines ausw"ahlen und als Ergebnis liefern. Dies sind

\begin{itemize}
  \item die Minimumfunktion und
  \item die Maximumfunktion.
\end{itemize}  \index{Ausdruck!Funktion}
Regel~\ref{sylistfunction} zeigt die Syntax dieser Listenfunktionen.

\begin{center}
\fbox{
\begin{minipage}{\textwidth}
{\footnotesize
\begin{center}
\begin{eqnarray}
  \mbox{list\_function} & ::= & \mbox{list\_func\_key {\bf (} expression\_list {\bf )}} \label{sylistfunction}\\[.3 cm]
  \mbox{list\_func\_key} & ::= & \mbox{{\bf min}\index{min} $|$ {\bf max}\index{max}} \label{sylistfunckey} \\ \nonumber\index{Funktion!exp} \index{Funktion!ln} \index{Funktion!sqr} \index{Funktion!sqrt}
		\index{Funktion!sin} \index{Funktion!cos}\index{Funktion!arctan} \index{Funktion!min} \index{Funktion!max}
\end{eqnarray}
\end{center}
}
\end{minipage}
}
\end{center}

Die Syntax von ``expression\_list'' wurde schon in
Regel~\ref{syexprlist} beschrieben. Es gelten f"ur die Liste der
Argumente die folgenden semantischen Bedingungen:

\begin{itemize}
  \item Die Argumente der Funktionen {\bf min} oder {\bf max} m"ussen von einem Datentyp sein, auf
	dem eine Ordnung definiert ist. Dies ist auf den Typen
	\begin{itemize}
	  \item {\bf int},
	  \item {\bf float} und \index{Funktion!min} \index{Funktion!max}
	  \item der in Abschnitt~\ref{Zeichenketten}\index{Zeichenkette!Funktion concat} definierten Untermenge von {\bf string}
	\end{itemize}
	der Fall.
  \item Diese Funktionen k"onnen auch auf ein Feld eines  ad"aquaten Datentyps
	angewendet werden. In diesem Fall enth"alt ``expression\_list'' genau ein Argument, n"amlich
	den Bezeichner\index{Bezeichner} des Felds.   \index{Ausdruck!Funktion}
\end{itemize}


Boolesch-wertige Ausdr"ucke erh"alt man durch den Vergleich einfacher
Ausdr"ucke, wie sie in Kapitel~\ref{EinfacheAusdruecke} beschrieben
worden sind. Diese Ausdr"ucke k"onnen dann untereinander wieder mit
logischen Operatoren verkn"upft werden. Au"serdem gibt es zwei
logische Funktionen, die eine Liste logischer Ausdr"ucke evaluieren.
\index{Ausdruck!logischer}

Syntaxregel~\ref{sylogicalexpr} zeigt den Aufbau eines logischen
Ausdrucks. ``boolean\_expr'' bezeichnet boolesch-wertige
Vergleichsausdr"ucke. Die Syntaxregeln spiegeln auch die
unterschiedliche Bindung der logischen Operatoren wieder. Operatoren
gleicher Bindung sind linksassoziativ.

\begin{center}
\fbox{
\begin{minipage}{\textwidth}
{\small
\begin{center}
\begin{eqnarray}
  \mbox{logical\_expr} & ::= & \mbox{logical\_expr {\bf or}\index{or} logical\_term} \nonumber\\
                       &  |  & \mbox{logical\_term} \label{sylogicalexpr}\\[.3 cm]
  \mbox{logical\_term} & ::= & \mbox{logical\_term {\bf and}\index{and} logical\_factor} \nonumber\\
                       &  |  & \mbox{logical\_factor} \label{sylogicalterm}\\[.3 cm]
  \mbox{logical\_factor} & ::= & \mbox{{\bf not}\index{not} logical\_factor} \nonumber\\
                       &  |  & \mbox{logical\_function} \nonumber\\
                       &  |  & \mbox{{\bf (} logical\_expr {\bf )}} \nonumber\\
                       &  |  & \mbox{boolean\_expr} \label{sylogicalvar}\\ \nonumber
\end{eqnarray}
\end{center}  \index{Ausdruck!logischer}
}
\end{minipage}
}
\end{center}

Ein boolesch-wertiger Ausdruck ist ein Vergleich zweier einfacher
Ausdr"ucke. Die Vergleichsoperatoren sind nicht assoziativ. Das geht
auch aus Syntaxregel~\ref{sybooleanexpr} hervor.
\index{Ausdruck!boolesch-wertiger}

\begin{center}
\fbox{
\begin{minipage}{\textwidth}
{\small
\begin{center}
\begin{eqnarray}
  \mbox{boolean\_expr} & ::= & \mbox{comparison\_expression} \nonumber\\
          &  |  &  \mbox{inclusion\_expression}  \nonumber\\
          &  |  &  \mbox{relation\_expression} 	\label{sybooleanexpr}\\[.3cm]
  \mbox{comparison\_expression} & ::= & \mbox{expression relational\_oper expression} \label{sycomparisonexpression}\\[.3cm]
  \mbox{relational\_oper} & ::= & \mbox{$>$} \; \; | \; \;  \mbox{$>=$} \nonumber \\
          &  |  &  \mbox{$<$} \; \; |  \; \; \mbox{$<=$} \nonumber\\
          &  |  &  \mbox{$=$} \; \; | \; \; \mbox{$<>$} \label{syrelationaloper} \\ \nonumber
\end{eqnarray}
\end{center}
}
\end{minipage}
}
\end{center}

Dabei bedeutet
\begin{itemize} \index{Ausdruck!boolesch-wertiger}
  \item ``comparison\_expression'' den  Vergleich zweier einfacher Ausdr"ucke,
  \item ``inclusion\_expression''  die Abfrage, ob der Wert eines einfachen Ausdrucks Element eines
	Felds ist, und 
  \item ``relation\_expression''\index{Relation!Abfrage} die Abfrage, ob durch eine {\bf map}-Konstante\index{Konstante!Relation!Verwendung}\index{Konstante!Relation!in logischem Ausdruck} eine Relation\index{Relation!in logischem Ausdruck} zwischen
	den Werten zweier einfacher Ausdr"ucke definiert wird.
\end{itemize}
Die Syntax von ``expression'' wurde in Kapitel~\ref{EinfacheAusdruecke}
(Syntaxregel~\ref{syexpression}) beschrieben. 
 \index{Ausdruck!boolesch-wertiger}


\section{Vergleich}

Die Ausdr"ucke, die miteinander verglichen werden sollen, m"ussen vom
gleichen Datentyp sein. Das gilt auch f"ur numerische Ausdr"ucke. Ein
Vergleich von Ausdr"ucken des Typs {\bf int}\index{int} mit solchen
des Typs {\bf float}\index{float} ist nicht zul"assig.

Au"ser numerischen Ausdr"ucken k"onnen noch
Zeichenketten\index{Zeichenkette!Vergleich} miteinander verglichen
werden, die nur aus Zeichen bestehen, auf denen die
lexikalische\index{Reihenfolge!lexikalische}\index{lexikalische
Reihenfolge} Reihenfolge aus Abschnitt~\ref{Zeichenketten} definiert
ist.

Ausdr"ucke anderer Datentypen, zum Beispiel Zellen, d"urfen nicht
miteinander verglichen werden, da auf diesen Typen keine Ordnung
definiert ist. \index{Ausdruck!boolesch-wertiger}
\index{Ausdruck!Vergleich}





\section{Mengeninklusion}
\index{Inklusion}



\begin{center}
\fbox{
\begin{minipage}{\textwidth}
{\small
\begin{center}
\begin{eqnarray}
  \mbox{inclusion\_expression}   & ::= &  \mbox{expression {\bf in}\index{in} element\_group} 
							\label{syinclusionexpression}\\ \nonumber
\end{eqnarray}
\end{center}
}
\end{minipage}
}
\end{center}

Ein Ausdruck ``expr$_{1}$'' {\bf in}\index{in} ``expr$_{2}$'' ist
genau dann wahr, wenn ``expr$_{1}$'' Element der Elementliste bzw. des
Felds ``expr$_{2}$'' ist. Die Syntax von ``element\_group'' ist in
Regel~\ref{syelementgroup} dargestellt.

Man beachte aber, da"s ``expr$_{1}$'' auf jeden Fall vom
Basisdatentyp, das hei"st, vom Typ der Elemente von ``expr$_{2}$''
sein mu"s -- sonst liefert der Ausdruck nicht etwa ``falsch'' sondern
einen Typfehler.  \index{Ausdruck!boolesch-wertiger}
\index{Ausdruck!Inklusion}


\section{Relation}


\begin{center}
\fbox{
\begin{minipage}{\textwidth}
{\small
\begin{center}
\begin{eqnarray}
  \mbox{relation\_expression} & ::= & \mbox{{\bf (} expression, expression {\bf )} {\bf in}\index{in} element\_group} 
          						 \label{syrelationexpression}\\ \nonumber
\end{eqnarray}
\end{center} \index{Ausdruck!boolesch-wertiger} \index{Ausdruck!Relation} 
}
\end{minipage}
}
\end{center}

\label{mapcond} Die Regel~\ref{syrelationexpression} definiert die
Syntax einer Abfrage, ob die durch eine {\bf map}-Konstante
\index{map} \index{Konstante!Relation!Verwendung}
\index{Konstante!Relation!in logischem Ausdruck}
\index{Relation!Verwendung} \index{Relation!Abfrage} definierte
Relation f"ur ein Paar von Werten einfacher Ausdr"ucke existiert oder
nicht. Es ist

\centerline{{\bf (} expr$_{1}$, expr$_{2}$ {\bf )} {\bf in}\index{in} mapExpr $\Longleftrightarrow$ expr$_{1}$ mapExpr
expr$_{2}$ gilt.} \index{Ausdruck!boolesch-wertiger}

``element\_group'' mu"s in diesem Fall 

\begin{itemize}
  \item entweder der Bezeichner\index{Bezeichner} einer {\bf map}-Konstanten\index{map} oder
  \item ein Bereich von Zeilen einer {\bf map}-Konstanten\index{map} sein.
\end{itemize}

Ein Bereich\index{Bereich!von Zeilen einer Relation} von Zeilen einer
{\bf map}-Konstanten \index{map}
\index{Konstante!Relation!Zeilenbereich} wird nach der
Syntaxregel~\ref{syfieldrange} angegeben, das hei"st, nach dem
Bezeichner der {\bf map}-Konstanten\index{map} gibt man in eckigen
Klammern den Bereich der Zeilen an, auf die die
Relation\index{Relation} eingeschr"ankt werden soll.
\index{Ausdruck!Relation}


\section{Variablendeklarationen}
\label{Variablendeklaration}
\label{Variablendeklarationen}

Variablen\index{Variable} m"ussen in {\bf Nessus} deklariert werden.
Dies geschieht in einem Block von Deklarationen, der auf den
Strukturdefinitionsblock\index{Struktur!vs. Variable} folgt, aber auch
entfallen kann.  Syntaxregel~\ref{syvarblock} zeigt den Aufbau des
Deklarationsteils.

\begin{center}
\fbox{
\begin{minipage}{\textwidth}
{\footnotesize
\begin{center}
\begin{eqnarray}
  \mbox{var\_block} & ::= & \mbox{{\bf var}\index{var} variable\_declaration\index{Variable}}  \nonumber\\
                        &     & \mbox{\{ variable\_declaration \}} \label{syvarblock}\\[.3 cm]
  \mbox{variable\_declaration}\index{Variable} & ::= & \mbox{variable\_type {\bf ~:~} identifier\_list {\bf ~;} }
                                       \label{syvariabledeclaration}\\[.3 cm]
  \mbox{variable\_type}\index{Variable} & ::= & \mbox{singular\_type } \nonumber\\
                        &  |  & \mbox{{\bf array of} array\_var\_type } \label{syvariabletype}\\[.3 cm]
  \mbox{array\_var\_type}\index{Variable} & ::= & \mbox{{\bf structure}\index{unit} } \nonumber\\
                        &  |  & \mbox{singular\_type} \label{syarrayvartype}\\[.3 cm]
  \mbox{singular\_type}\index{Variable} & ::=  & \mbox{{\bf unit}\index{unit} } \nonumber\\
                        &  |  & \mbox{simple\_type} \label{sysingulartype}\\ \nonumber
\end{eqnarray}
\end{center}
}
\end{minipage}
}
\end{center}

Wird eine Variable vom Typ {\bf unit}, {\bf array of unit} oder {\bf
array of structure} deklariert, so entstehen dadurch keine
eigenst"andigen Zellen, die in das Netz "ubernommen werden. Solche
Zellen k"onnen nur bereits definierte Strukturen oder Zellen als Wert
annehmen.

Hinter ``simple\_type'' verbergen sich (siehe
Syntaxregel~\ref{sysimpletype}) die einfachen Datentypen
	\begin{itemize}
	  \item {\bf int},\index{int}
	  \item {\bf float}\index{float} und
	  \item {\bf string}.\index{string}
	\end{itemize}


