\chapter{Einleitung}

Das vorliegenden Benutzerhandbuch ist eine Einf"uhrung in die
Programmierung mit der Sprache {\bf Nessus} ({\underline {Ne}}tzwerk -
{\underline S}pezifikations - {\underline S}prache der {\underline
U}niversit"at {\underline S}tuttgart). {\bf Nessus} dient dazu, die
Topologie neuronaler Netze auf einem hohen Abstraktionsniveau
beschreiben. Der zugeh"orige Compiler erzeugt aus Programmen der
Sprache eine Repr"asentation der Netze, die als Eingabe f"ur den
Simulator-Kern von SNNS\index{SNNS} dient. SNNS (\underline{S}tuttgart
\underline{N}eural \underline{N}etwork \underline{S}imulator) ist ein
Simulator f"ur konnektionistische Netzwerke, der am Institut f"ur
Parallele und Verteilte H"ochstleistungsrechner (IPVR) der
Universit"at Stuttgart entwickelt wurde.

Konnektionistische Modelle sind Modelle f"ur informationsverarbeitende
Systeme, die sich aus vielen einfachen, parallel arbeitenden Einheiten
zusammensetzen. Diese tauschen Information in Form ihrer
Aktivierung\index{Aktivierung} "uber ein Netz von Verbindungen aus.
Diese Modelle sind anderen Ans"atzen der KI "uberlegen in Bezug auf
Lernf"ahigkeit und Parallelit"at.

Eine ausf"uhrliche Darstellung der Grundlagen und Konzepte neuronaler
Netze kann in \cite{rum861} nachgelesen werden.  Eine Kurzfassung
steht in~\cite{kem88}.

Neuronale Netze k"onnen unter zwei Aspekten betrachtet werden. Der
{\em statische}\/ Aspekt besteht aus der Topologie der Netzwerks und
seinen Elementen, der {\em dynamische}\/ Aspekt aus den Funktionen,
die die Informationsverarbeitung in einer zeitlichen Abfolge
bestimmen.

Ein {\bf Nessus}-Programm dient dazu, die Topologie eines
konnektionistischen Netzes zu definieren und seine dynamischen
Eigenschaften durch die Angabe der Zellfunktionen festzulegen. Diese
Funktionen allerdings k"onnen in der vorliegenden Version von {\bf
Nessus} noch nicht in der Sprache selbst definiert werden.  Solche
Funktionen m"ussen in C (siehe~\cite{ker781}) definiert und zu SNNS
dazugebunden werden. Dieser Vorgang wird in~\cite{SNNSDoku}
ausf"uhrlich erkl"art.

\paragraph{Das Netzwerkmodell von {\bf Nessus}} ~~\\ Die
Topologie\index{Topologie} eines Netzes besteht aus einer Menge von
Zellen und Verbindungen\index{Verbindung} zwischen diesen Zellen. Das
Netzwerkmodell,\index{Netzwerkmodell} das SNNS\index{SNNS} und {\bf
Nessus} zugrunde liegt, hat folgende Eigenschaften:

\begin{itemize}
  \item Eine Zelle kann beliebig viele benannte Eing"ange besitzen.
        Diese werden im folgenden als Sites\index{Site} bezeichnet.
  \item Die Verbindungen\index{Verbindung} gehen immer von einer Zelle
	zu einer Site ihrer\index{Site} Nachfolgerzelle  
        (wenn diese mehrere Eing"ange hat). Dadurch ist es m"oglich,
	da"s zwischen zwei Zellen des Netzes mehrere Verbindungen 
	unterschiedlicher Bedeutung existieren.
  \item Es k"onnen nicht mehrere Verbindungen von einer Zelle zum
	selben Eingang derselben Nachfolgerzelle definiert werden.
  \item Die Zellen k"onnen topologisch in
	\begin{itemize}
          \item Eingabezellen,
          \item Ausgabezellen oder
          \item interne Zellen
        \end{itemize}
	unterteilt werden. Es ist aber auch erlaubt, da"s Zellen
	Eingabe- und Ausgabezellen gleichzeitig sind, wie es zum 
	Beispiel in Auto-Assoziator-Netzen\index{Auto-Assoziator-Netz} 
        notwendig ist.
  \item Zellen besitzen bei SNNS\index{SNNS} und {\bf Nessus}
	au"serdem folgende Merkmale:
	\begin{itemize}
          \item Jede Zelle hat einen Namen, der aber entfallen darf
		und auch nicht eindeutig sein mu"s.
          \item Zu jeder Zelle geh"ort ein Aktivierungswert\index{Aktivierung}
(Potential),\index{Potential} der bei SNNS\index{SNNS} eine 
                rationale Zahl\index{Zahl!rationale} aus dem Intervall $[-1.0,1.0]$ sein
mu"s. SNNS\index{SNNS} arbeitet mit  
                einer Genauigkeit von f"unf Dezimalstellen. 
          \item Denselben Wertebereich\index{Wertebereich!rationale Zahl} hat auch die Ausgabe einer Zelle.
        \end{itemize}  
  \item Der Aktivierungswert\index{Aktivierung} einer Zelle wird aus ihrer bisherigen
Aktivierung\index{Aktivierung} und 
                den Werten, die an ihren Eing"angen anliegen, von der
Aktivierungsfunktion\index{Aktivierungsfunktion}   
                berechnet.
  \item Aus diesem Aktivierungswert\index{Aktivierung} berechnet die Ausgabefunktion den Ausgabewert
                der Zelle.
  \item Die Werte, die an den Eing"angen einer Zelle anliegen, sind das Ergebnis
                der Eingangsfunktion (site function). Diese Funktion bestimmt den aktuellen
                Wert des Eingangs aus den gewichteten Ausgabewerten aller Vorg"angerzellen
                des Eingangs.
\end{itemize}


\paragraph{Das Sprachkonzept von {\bf Nessus}}
~~\\
Mit SNNS\index{SNNS} sollen vor allem kleine bis mittelgro"se Netze
simuliert werden.  Schon die mittelgro"sen Netzwerke k"onnen aus mehr
als tausend Zellen und oft mehr als zehntausend Verbindungen bestehen.
Solche Netzwerke k"onnen ohne m"achtige Hilfsmittel nicht mehr
definiert, dargestellt oder ver"andert werden. Um mit
konnektionistischen Netzwerken\index{konnektionistische Netzwerke} zu
experimentieren, ist es deshalb hilfreich, mittels einer auf diesen
Zweck zugeschnittenen Sprache Netze definieren zu k"onnen. Daf"ur gibt
es verschiedene Ans"atze:

\begin{itemize}
  
\item  Beim weitverbreiteten {\bf Rochester Connectionist Simulator}
(RCS)\index{RCS (Rochester Connectionist Simulator)} (siehe
\cite{fel01,fel02,god87}) werden dem Benutzer in C eingebettete
Funktionen zur Verf"ugung gestellt, mit denen weitgehend
daten\-unab\-h"angig Zellen und Verbindungen definiert werden k"onnen.

\item Beim Simulator {\bf P3}\index{P3-Simulator} wurde eine Sprache
definiert, die speziell auf diesen Zweck zugeschnitten ist (siehe
\cite{zip86}).

\item Verschiedene kleinere Simulatoren bieten nur sehr beschr"ankte
Hilfsmittel (zum Beispiel in \cite{rum863}).

\item An der Universit"at Z"urich-Irchel wurde die prozedurale
Sprache {\bf Condela-3} (siehe 	auch~\cite{alm90,koe88}) entworfen,
mit der nicht nur die Topologie, sondern auch das 	Verhalten
neuronaler Netze definiert werden kann. Das System ist so
implementiert, da"s ein 	"Ubersetzer aus den
Condela-3-Programmen C-Programme erzeugt, die die Netzwerktopologie
erzeugen und das Verhalten des Netzes simulieren.
\end{itemize}

Wir sind der Meinung, da"s die Methode, neuronale Netze in einer
allgemeinen Programmiersprache mittels einfacher Funktionen, die nur
die Datenstrukturen des Simulator-Kerns verbergen, zu definieren, sehr
m"uhsam und fehleranf"allig ist.  Aus diesem Gedanken heraus ist {\bf
Nessus} entstanden.

In {\bf Nessus} wurde der Versuch unternommen, Sprachmittel
bereitzustellen, die der Aufgabe des Programms, n"amlich topologische
Strukturen zu definieren, m"oglichst nahe kommen. Dazu geh"oren unter
anderem

\begin{itemize}
  \item Typdefinitionen f"ur Zellen und Sites,
  \item Strukturen mit impliziten Verbindungen nach vordefinierten 
	oder explizit definierbaren Verbindungstopologien, sowie
  \item ein Teilnetzkonzept, das es erlaubt, Netze aus verschiedenen {\bf
	Nessus}-Programmen zusammenzusetzen.
\end{itemize}

Dadurch kann man in {\bf Nessus} die Verbindungstopologie eines Netzes
sehr effizient und sicher definieren. Zus"atzlich kann man f"ur die
Strukturen und Teilnetze die graphische Topologie in der Darstellung
der graphischen Benutzeroberf"ache von SNNS definieren. Der {\bf
Nessus}-Compiler erzeugt dann daraus das Layout des Netzes.

\paragraph{Gliederung}
~~\\
Dieses Benutzerhandbuch ist wie folgt gegliedert: 
In den Kapiteln~\ref{Programmaufbau}~-~\ref{Parameter} wird die
Programmiersprache {\bf Nessus} vorgestellt.

In Kapitel~\ref{Umgebung} folgt die Beschreibung der
Programmierumgebung unter UNIX und des Aufrufs des {\bf
Nessus}-Compilers.

Anhang~\ref{Syntax} enth"alt die vollst"andige Syntax von {\bf Nessus}
in EBNF.\index{EBNF}

Es wurde darauf verzichtet, in diesem Handbuch die Definition der
Schnittstelle zwischen dem {\bf Nessus}-Compiler und SNNS
wiederzugeben. Sie ist in~\cite{SNNSDoku} ausf"uhrlich beschrieben.


\paragraph{Konventionen}
~~\\
In einigen Kapiteln der Dokumentation wird die erweiterte BNF
(EBNF)\index{EBNF} verwendet, um die Syntax von {\bf Nessus}
darzustellen. Dabei bedeutet

\begin{itemize}
  \item ${\bf |}$ Alternativen,
  \item in  ${\bf []}$ werden Optionen (gar nicht oder einmal auftretende Teile)
        eingeschlossen
  \item in ${\bf \{\}}$ werden Optionen, die auch mehrfach auftreten k"onnen, eingeschlossen.
\end{itemize}

Diese Sonderzeichen sind also nicht Bestandteile der Sprache selbst.
Wo diese Symbole Zeichen von {\bf Nessus} bezeichnen, werden sie in
``~'' geschrieben.  Terminalsymbole sind fett gedruckt. Au"ser
Schl"usselw"ortern, Operatoren und Sonderzeichen sind

\begin{itemize}
  
\item Bezeichner\index{Bezeichner} ({\bf identifier}),
  
\item ganzzahlige\index{Zahl!ganze}\index{Zahl!rationale}
Literale\index{Literal!ganzzahliges} ({\bf intliteral}), 

  
\item rationale Literale\index{Literal!rationales} ({\bf
floatliteral}) und

\item Zeichenketten\index{Zeichenkette} ({\bf stringliteral})
\index{Literal!Zeichenkette}

\end{itemize}
Terminalsymbole. 

Die Terminalsymbole werden durch regul"are Ausdr"ucke
\index{Ausdruck!regul"arer} im Stil einer Lex-Eingabe
(siehe~\cite{les75,sch85}) beschrieben. Dabei haben folgende Zeichen
eine spezielle Bedeutung:

\begin{itemize}
  
\item ${\bf |}$ trennt Alternativen.
  
\item In  ${\bf []}$ werden Klassen von Zeichen eingeschlossen. Das
Minuszeichen zwischen zwei Zeichen bezeichnet innerhalb einer solchen
Klasse einen Bereich einschlie"slich der beiden Zeichen. ${\bf [A-Z]}$
steht also f"ur alle Gro"sbuchstaben, ${\bf [0-9]}$ f"ur alle Ziffern.
  
\item Beginnt eine Klasse von Zeichen mit {\bf \verb&^&}, so
bezeichnet die Klasse jedes einzelne Zeichen, das {\it nicht} durch
den nachfolgenden regul"aren Ausdruck repr"asentiert wird. So
bezeichnet zum Beispiel {\bf \verb&[^"}"]&} die Klasse aller Zeichen
au"ser {\bf \verb&"}"&}.
  
\item Ein ${\bf *}$, das einem regul"aren Ausdruck folgt, bedeutet,
da"s die entsprechenden Zeichen beliebig oft oder auch gar nicht
auftreten d"urfen.
  
\item Ein ${\bf +}$, das einem regul"aren Ausdruck folgt, bedeutet,
da"s die entsprechenden Zeichen beliebig oft (aber mindestens einmal)
vorkommen k"onnen.
  
\item {\bf \{n\}} nach einem regul"aren Ausdruck bedeutet, da"s die
dadurch definierte Zeichenkette genau n-mal vorkommen mu"s.
  
\item Runde Klammern\index{Klammerung!in regul"aren Ausdr"ucken} {\bf
()} dienen der Gruppierung regul"arer
Ausdr"ucke.\index{Ausdruck!regul"arer} So definiert zum Beispiel ${\bf
(a|b)*}$ alle Zeichenketten einschlie"slich der leeren, die nur aus
den Zeichen {\bf a} oder {\bf b} in beliebiger Reihenfolge bestehen.
Da {\bf *}, {\bf +} und {\bf \{n\}} st"arker
binden\index{Bindung}\index{Vorrang} als ${\bf |}$, mu"s geklammert
werden: im Gegensatz dazu definiert ${\bf a|b*}$ alle Zeichenketten,
die entweder nur aus dem Zeichen ${\bf a}$ oder aus beliebig vielen
${\bf b}$, einschlie"slich der leeren Zeichenkette, bestehen.
  
\item Treten diese Sonderzeichen innerhalb von Mustern auf, werden sie
 in ``~'' eingeschlossen.

\end{itemize}


