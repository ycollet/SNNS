\chapter{Strukturdefinitionen}
\label{Strukturdefinitionen}

Im Strukturdefinitionsblock\index{Struktur!Definitionsblock} werden
alle Zellen definiert und benannt, die Bestandteil des Netzes 
werden sollen.

\begin{center}
\fbox{
\begin{minipage}{\textwidth}
{\footnotesize
\begin{center}
\begin{eqnarray}
  \mbox{structure\_block} & ::= & \mbox{{\bf structure}\index{struct} structure\_definition}  \nonumber\\
  	  &     & \mbox{\{ structure\_definition \} }  \label{systructure}\\[.3 cm]
  \mbox{structure\_definition} & ::= & \mbox{struct\_specification
{\bf ~:~} identifier\_list {\bf ~;~}}\label{systructdef}\\[.3 cm] 
  \mbox{struct\_specification } & ::= & \mbox{single\_unit} \nonumber\\
  	  &  |  & \mbox{linear\_struct} \nonumber\\
          &  |  &  \mbox{matrix\_struct} \label{systructspec} \\[.3 cm]
  \mbox{identifier\_list} & ::= & \mbox{identifier \{ {\bf ~,~} identifier \} } \label{syidlist}\\ \nonumber
\end{eqnarray}
\end{center}
}
\end{minipage}
}
\end{center}

Eine Strukturdefinition besteht aus der Definition der
Struktur\index{Struktur} und einer Liste von
Bezeichnern\index{Bezeichner} (siehe Regel~\ref{systructdef}).  F"ur
{\it jeden\/} Bezeichner\index{Bezeichner!mehrere f"ur eine Struktur}
wird die Struktur einmal angelegt.

Syntaxregel~\ref{systructspec} zeigt die m"oglichen Struktur-Typen:

\begin{enumerate}
\item ``single\_unit'' bezeichnet  die Spezifikation einer einzelnen
Zelle. 

\item Hinter ``linear\_struct'' verbirgt sich die Spezifikation eines
Felds\index{Feld} von Zellen, die untereinander nach bestimmten
Mustern verbunden sein k"onnen.
  
\item ``matrix\_struct'' steht f"ur die Spezifikation von
Feldern\index{Feld} linearer Strukturen.\index{Struktur!Typen} 
\end{enumerate}

F"ur jede Struktur\index{Struktur!graphische Darstellung} mu"s bei
ihrer Definition angegeben werden, wie sie w"ahrend der Simulation
graphisch\index{graphische Darstellung!einer Struktur} dargestellt
werden soll. Die Definitionen dieser Strukturtopologien folgen direkt
auf die Spezifikation der Zellen, aus denen die Struktur besteht.
Welche Strukturtopologien erlaubt sind und wie sie definiert werden,
wird in Abschnitt~\ref{StrukturTopologie}\index{Topologie} erkl"art.



\section{Einzelne Zellen}
\label{unitstruct}

F"ur eine Struktur,\index{Struktur!einzelne Zelle} die nur aus einer
einzelnen Zelle besteht, m"ussen nur die Parameter dieser Zelle und
ihre Position festgelegt werden.

\begin{center}
\fbox{
\begin{minipage}{\textwidth}
{\footnotesize
\begin{center}
\begin{eqnarray}
  \mbox{single\_unit} & ::= & \mbox{{\bf unit}\index{unit} $[$struct\_unit\_type$]$ } \\ \nonumber
                      &     & \mbox{$[$unit\_param\_list$]$
position\_specification}\index{Position} \label{sysingleunit}\\[.3 cm]  
  \mbox{struct\_unit\_type} & ::= & \mbox{{\bf of}\index{of} identifier} \label{systructunittype}\\ \nonumber
\end{eqnarray}
\end{center}
}
\end{minipage}
}
\end{center}

Die explizite Definition von Parametern folgt
Syntaxregel~\ref{syunitparamlist}. Die Syntax ist dieselbe wie bei der
Typdefinition, es gibt aber semantische Unterschiede:

\begin{enumerate}
  
\item Es gibt keine Parameter,\index{Parameter!einer Zelle!in
Struktur} die definiert werden {\it m"ussen\/}. Es kann also z. B. die
Definition der Aktivierungsfunktion\index{Aktivierungsfunktion}
weggelassen werden.
  
\item F"ur alle nicht definierten Parameter\index{Parameter!einer
Zelle!Standardwerte} einer Zelle werden von SNNS Standardwerte
eingesetzt.  

\item F"ur die Sites\index{Site} einer Zelle gibt es keinen
Standardwert. Wird eine Zelle im Strukturdefinitionsblock ohne
Sites\index{Site} definiert, wird angenommen, da"s sie auch keine hat.

\item Die Menge der Sites einer Zelle darf grunds"atzlich nach der
Strukturdefinition nicht mehr ver"andert werden. Alle anderen in der
Strukturspezifikation 	festgelegten Parameterwerte
\index{Parameter!einer Zelle!variable} d"urfen nachtr"aglich
ver"andert werden. Darin unterscheiden sie sich von den in einer
Typdefinition festgelegten Parameterwerten.
\end{enumerate}

Die Angabe eines Typs f"ur eine einzelne Zelle folgt
Regel~\ref{systructunittype}. Es d"urfen auch nach der Angabe eines
Zelltyps noch andere Parameter\index{Parameter!einer Zelle} definiert,
aber keine in der Typdefinition festgelegten Merkmale "uberschrieben
werden.

Im Strukturdefinitionsblock des Beispiels\\

\begin{tabbing}bbbb\=bbbb\=bbbb\=bbbb\=bbbb\=bbbb\=\kill
{\bf typedef}\\
~~{\bf unit}~{\bf with}~{\bf actfunc}~\verb&Act_Sum&\verb&,&~{\bf outfunc}~\verb&Out_Identity&~\verb&:&~\verb&stdUnitType&\verb&;&\\
\\
{\bf structure}\\
~~{\bf unit}~{\bf of}~\verb&stdUnitType&~{\bf with}~{\bf act}~{\bf random}\verb&,&~{\bf name}~\verb&"Zelle"&~{\bf at}\verb&(&1\verb&,&1\verb&)&~\verb&:&~\verb&u1&\verb&,&~\verb&u2&\verb&,&~~\verb&u3&\verb&;&\\
~~{\bf unit}~{\bf at}\verb&(&2\verb&,&1\verb&)&~\verb&:&~\verb&v1&\verb&,&~\verb&v2&\verb&;&\\
\end{tabbing}


werden f"unf Zellen definiert. ``u1'', ``u2'' und ``u3'' sind Zellen
mit der Aktivierungsfunktion ``Act\_Sum'' und der Ausgabefunktion
``Out\_Identity''. Bei ``v1'' und ``v2'' bleiben die Funktionen
undefiniert. SNNS setzt daf"ur beim Laden des Netzes Standardwerte
ein. Diese Standardwerte sind ``Act\_Logistic'' f"ur die Aktivierungs-
und ``Out\_Identity'' f"ur die Ausgabefunktion.

Man beachte, da"s bei ``u1'', ``u2'' und ``u3'' die
Aktivierungsfunktion nicht mehr "uberschrieben werden darf, da sie
Merkmal des Zelltyps ``stdUnitType'' ist.



\section{Eindimensionale Strukturen}
\label{EinDimStrukturen}

Eindimensionale Strukturen\index{Struktur!eindimensionale} sind
Felder\index{Feld} von Zellen. Es gibt drei verschiedene Typen solcher
linearer Strukturen (siehe Syntaxregel~\ref{sylinearstruct}):

\begin{enumerate}
  \item Einfache Felder\index{Feld!einer Struktur} von Zellen ohne
Verbindungen\index{Verbindung} zwischen den Zellen (``cluster\_struct'')\index{cluster}
dienen 
	der Definition gr"o"serer Mengen gleichartiger Zellen.
  \item Felder\index{Feld einer Struktur} von Zellen mit impliziten
	Verbindungen,\index{Verbindung!implizite} die den in {\bf
	Nessus} vordefinierten Topologieformen entsprechen, sind in
	Regel~\ref{sylinearstruct} unter ``connected\_struct'' zusammmengefa"st. 
  \item Teilnetze\index{Teilnetz}\index{Teilnetz}\index{Teilnetz!Sites} k"onnen als
	selbst"andige 
 	{\bf Nessus}-Programme definiert und dann in das 
	Netzwerk eingebunden werden (``subnet\_struct'').
\end{enumerate}

\begin{center}
\fbox{
\begin{minipage}{\textwidth}
{\footnotesize
\begin{center}
\begin{eqnarray}
  \mbox{linear\_struct} & ::= & \mbox{cluster\_struct}\index{cluster} \nonumber\\
 	  &  |  & \mbox{connected\_struct} \nonumber\\
          &  |  & \mbox{subnet\_struct} \label{sylinearstruct}\\ \nonumber
\end{eqnarray}
\end{center}
}
\end{minipage}
}
\end{center}

Einfache Felder und solche\index{Feld} mit Verbindungen
\index{Verbindung} \index{Verbindung!implizite!Parameter} zwischen den
Zellen unterscheiden sich auch durch die Angaben, die zu ihrer
Definition gemacht werden m"ussen:

\begin{itemize}
  \item In beiden F"allen m"ussen die
Feldl"ange\index{Feld!L"ange},die\index{L"ange!eines Felds} Parameterwerte der Zellen und 
	die graphische\index{graphische Darstellung!einer Struktur}
Darstellung der Struktur\index{Struktur!lineare} definiert werden.  
  \item Bei Feldern\index{Feld} mit impliziten Verbindungen\index{Verbindung} m"ussen zus"atzlich noch
 	\begin{itemize}
  	  \item die Verbindungsst"arke,\index{Verbindung!implizite!Gewicht}
  	  \item die Zielsites\index{Site} der impliziten Verbindungen\index{Verbindung!implizite!Zielsite} und
  	  \item in einigen F"allen die Verbindungsrichtung\index{Verbindung!implizite!Richtung}
	\end{itemize}
	angegeben werden.
\end{itemize}


\subsection{Einfache Felder von Zellen}
\label{EinfacheFelder}

Einfache Felder\index{Feld!einfaches!von Zellen} von Zellen werden,
wie Syntaxregel~\ref{syclusterstruct} zeigt, durch das Schl"usselwort
{\bf cluster}\index{cluster} definiert, dem die Anzahl der
Feldelemente folgen mu"s.  Die L"ange eines Felds \index{L"ange!eines
Felds} mu"s eine Konstante\index{Konstante!als Feldl"ange} vom Typ
{\bf int}\index{int} sein.

\begin{center}
\fbox{
\begin{minipage}{\textwidth}
{\footnotesize
\begin{center}
\begin{eqnarray}
  \mbox{cluster\_struct} & ::= & \mbox{{\bf cluster}\index{cluster}
array\_size $[$unit\_specification$]$}  \nonumber \\ 
                             &     & \mbox{topology\_specification} \label{syclusterstruct}\\[.3 cm]
  \mbox{array\_size}\index{array!Gr"o"se eines}\index{L"ange!eines
Felds}  & ::= & \mbox{``$[$'' expression ``$]$''} \label{syarraysize}\\[.3 cm] 
  \mbox{unit\_specification} & ::= & \mbox{struct\_unit\_type $[$unit\_param\_list$]$} \nonumber \\
                             &     & \mbox{$[$var\_param\_list$]$} \label{syunitspecification} \\ \nonumber
\end{eqnarray}
\end{center}
}
\end{minipage}
}
\end{center}



Bei einfachen Feldern\index{Feld} m"ussen nur die Feldl"ange und die
Merkmale der einzelnen Zellen des Felds definiert werden. Dabei gibt
es zwei verschiedene Arten, diese Merkmale zu definieren:

\begin{itemize}
  \item Es k"onnen Parameterwerte\index{Parameter!einer Zelle!mit {\bf
with}}  definiert werden, die f"ur alle Zellen des Felds\index{Feld} gleich sind. Das 
	geschieht 
	\begin{itemize}
	  \item mittels einer Typangabe (gem"a"s Regel~\ref{systructunittype} f"ur
	        ``struct\_unit\_type'') oder
	  \item "uber eine explizite Definition der
Parameterwerte\index{Parameter!einer Zelle!mit {\bf get}} (gem"a"s Regel~\ref{syunitparamlist} f"ur 
	        ``unit\_param\_list''),
	\end{itemize}
	analog zur Definition der Parameterwerte einer einzelnen Zelle.
  \item Die Parameter {\bf name}\index{name} und {\bf act}\index{act}
	k"onnen mit Werten belegt
	werden, die f"ur jede Zelle des Felds verschieden sind. Wie diese Definition erfolgt,
	wird im n"achsten Abschnitt erkl"art.
\end{itemize}

So definiert zum Beispiel das Programmst"uck\\
\begin{center}
\begin{tabbing}
{\bf typedef}\\
~~{\bf unit}~{\bf with}~{\bf actfunc}~\verb&Act_Logistic&\verb&,&~~{\bf
	outfunc}~\verb&Out_Identity&\verb&,&~~~~\\
~~~~~~~~~~~~~~~~~~~~~{\bf iotype}~{\bf output}~~\verb&:&~\verb&outUnitType&\verb&;&\\ 
{\bf structure}\\[.15cm]
~~{\bf cluster}\verb&[&10\verb&]&~{\bf of}~\verb&outUnitType&~~{\bf with}~{\bf act}~1.0~{\bf plane}~{\bf at}~
	\verb&(&1\verb&,&50\verb&)&~\verb&:&~\verb&u1&~\verb&,&~\verb&u2&~\verb&,&~\verb&u3&\verb&;&\\
\end{tabbing}
\end{center}

ein die drei Felder ``u1'', ``u2'' und ``u3'' aus aus je zehn Zellen
des Zelltyps ``outUnitType''.

\subsubsection{Einzelbelegung der Parameterwerte der Zellen eines
Felds}
\index{Feld}

Die Belegung von Parametern mit unterschiedlichen Werten f"ur jede
Zelle eines Felds erfolgt nach Syntaxregel~\ref{syvarparamlist} in
einer Liste von Definitionen, die mit dem Schl"usselwort {\bf get}
\index{get} beginnt. Jede solche Definition legt die Werte f"ur ein
bestimmtes Merkmal der Zellen fest (Regel~\ref{syvarparameter}).
\index{Parameter!einer Zelle!mit {\bf get}}

Die Merkmale, die so definiert werden d"urfen, sind
\begin{itemize}
  \item {\bf name}:\index{name} der Name der Zelle und
  \item {\bf act}:\index{act} der anf"angliche
Aktivierungswert\index{Aktivierung} der Zelle. 
\end{itemize}

F"ur jedes so definierte Merkmal wird ein Ausdruck nach
Syntaxregel~\ref{syvarparameter} geschrieben. Dabei wird vor dem
Schl"usselwort {\bf from}\index{from} das Schl"usselwort des Merkmals
angegeben, dahinter ein Feld,\index{Feld} das die Werte des Merkmals
f"ur die einzelnen Zellen enth"alt. Dieses Feld kann auf verschiedene
Arten angegeben werden (siehe Regel~\ref{syelementgroup}):

\begin{itemize}
  \item Es kann ein konstantes Feld\index{Feld} angegeben werden, wie in Syntaxregel~\ref{syarray}
	beschrieben. Die Werte k"onnen insbesondere, wie auch bei der Definition einer
	Feldkonstanten, aus einer Datei\index{Datei!explizite Parameterwerte} eingelesen werden.
  \item Es kann der Name einer bereits definierten Feldkonstanten\index{Feld} angegeben werden.
  \item Es kann ein zusammenh"angender Bereich\index{Bereich!von Feldelementen} von
	Elementen einer bereits definierten Feldkonstanten angegeben werden. 
\end{itemize}

 Die L"ange des so definierten Wertefelds\index{Feld} mu"s mit der
L"ange des Zellenfelds\index{L"ange!eines Felds} "ubereinstimmen.
Werden verschiedene (unterschiedlich benannte)
Strukturen\index{Struktur!explizite Parameter} des so definierten
Zellenfelds erzeugt, so sind diese alle identisch.

\begin{center}
\fbox{
\begin{minipage}{\textwidth}
{\footnotesize
\begin{center}
\begin{eqnarray}
    \mbox{var\_param\_list} & ::= & \mbox{{\bf get}\index{get}
var\_parameter\index{Parameter!einer Zelle!mit {\bf get}} \{ {\bf ~,~}
var\_parameter \}}\label{syvarparamlist}\\[.3cm]  
  \mbox{var\_parameter} & ::= & \mbox{ parameter\_key {\bf
from}\index{from} element\_group}\label{syvarparameter}\\[.3cm] 
  \mbox{parameter\_key} & ::= & \mbox{ {\bf act}\index{act} $|$ {\bf
name}\index{name} }\label{syparameterkey}\\[.3cm] 
  \mbox{element\_group} & ::= & \mbox{array\_constant} \nonumber \\
                        &  |  & \mbox{identifier}\nonumber\\
                        &  |  & \mbox{field\_range}\label{syelementgroup} \\[.3cm]
  \mbox{field\_range}   & ::= & \mbox{variable}\index{Variable} \nonumber\\
                        &  |  & \mbox{variable\index{Variable} ``$[$''
expression {\bf \verb&..&} expression ``$]$''}  
						 \label{syfieldrange} \\ \nonumber
\end{eqnarray}
\end{center}
}
\end{minipage}
}
\end{center}

Abbildung~\ref{BspStructCluster} zeigt die Definition eines Netzes,
bestehend aus $4 * 100$ Ausgabe- und $4 * 50$ Eingabezellen. 

\begin{figure}[htb]
  \begin{center}
    \mbox{
      \begin{minipage}{\textwidth}
           \begin{tabbing}\=nn\=\kill
	   {\bf network}~\verb&clusterStructBsp&\verb&(&\verb&)&\verb&;&\\[.3cm]
{\bf const}\\[.15cm]
~~\verb&actFile&~\verb&=&~{\bf file}~\verb&"../input/ActivationList.dat"&~{\bf of}~{\bf float}\verb&;&\\[.15cm]
~~\verb&nameFile&~\verb&=&~{\bf file}~\verb&"../input/NameList.dat"&~{\bf of}~{\bf string}\verb&;&\\[.15cm]
~~\verb&nameList&~\verb&=&~\verb&[&\verb&nameFile&\verb&]&\verb&;&\\[.3cm]
{\bf typedef}\\[.15cm]
~~{\bf unit}~{\bf with}~{\bf actfunc}~\verb&Logistic&\verb&,&~~{\bf
outfunc}~\verb&Identity&\verb&,&~~{\bf iotype}~{\bf output}~~\verb&:&~\verb&outType&\verb&;&\\[.3cm]
{\bf structure}\\[.15cm]
~~{\bf cluster}\verb&[&100\verb&]&~{\bf of}~\verb&outType&~~{\bf get}~{\bf act}~{\bf from}~\verb&[&\verb&actFile&\verb&]&\verb&,&~\\[.15cm]
~~~~~~~~{\bf name}~{\bf from}~\verb&nameList&\verb&[&17\verb&..&116\verb&]&~{\bf plane}~{\bf at}~\verb&(&1\verb&,&50\verb&)&~\verb&:&~\verb&out1&~\verb&,&~\verb&out2&~\verb&,&~\verb&out3&~\verb&,&~\verb&out4&\verb&;&\\[.15cm]
~~{\bf cluster}\verb&[&50\verb&]&~{\bf with}~{\bf iotype}~{\bf input}\verb&,&~{\bf act}~\verb&-&$.25$\verb&,&~{\bf actfunc}~\verb&Logistic&\verb&,&~\\[.15cm]
~~~~~~~~{\bf outfunc}~\verb&Identity&~{\bf plane}~{\bf at}~\verb&(&20\verb&,&50\verb&)&~\verb&:&~\verb&in1&\verb&,&~\verb&in2&\verb&,&~\verb&in3&\verb&,&~\verb&in4&\verb&;&\\[.3cm]
{\bf begin}\\[.15cm]
{\bf end}\verb&.&
           \end{tabbing}
      \end{minipage}
    }
    \caption{\label{BspStructCluster} Beispiel: Felder einfacher Zellen}
  \end{center}
\end{figure}




\subsection{Felder mit impliziten Verbindungen}
\label{StructFelder}

Felder\index{Feld} mit impliziten Verbindungen\index{Verbindung} sind
Felder von Zellen mit einem vordefinierten Muster von Verbindungen
zwischen diesen Zellen. Die Topologie der internen Verbindungen des
Felds wird durch das dazugeh"orige Schl"usselwort festgelegt. Es gibt
zwei M"oglichkeiten, die Verbindungen innerhalb des Felds zu
definieren:

\begin{enumerate}
\item Man w"ahlt eine der vordefinierten Topologien von
Verbindungen. \index{Verbindung!implizite!vordefinierte Topologie}

\item Man definiert eine {\bf map}-Konstante,\index{map} die {\bf
int} \index{int} auf {\bf int} abbildet. Es k"onnen dann alle die
Feldelemente miteinander verbunden werden, zwischen deren Indizes die
durch die {\bf map}-Konstante\index{map} definierte
Relation\index{Relation!in Topologiedefinition} existiert.
\end{enumerate}

Folgende  Topologieformen sind in  {\bf Nessus} vordefiniert:

\begin{itemize} 
\index{Verbindung!implizite!star} 
\index{Topologieform!star}
\index{Topologieform!ring}  
\index{Topologieform!chain}  
\index{Topologieform!clique}
\index{Verbindung!implizite!clique}
\index{Verbindung!implizite!chain}
\index{Verbindung!implizite!ring}
  
\item {\bf clique}\index{clique} ist ein Feld von Zellen, die
vollst"andig untereinander verbunden sind.
  
\item {\bf chain}\index{chain} ist ein Feld von Zellen mit
impliziten Verbindungen\index{Verbindung} von der ersten zur zweiten
Zelle, der zweiten zur dritten, usw. Es gibt aber keine Verbindung von
der letzten zur"uck zur ersten Zelle.

\item {\bf ring}\index{ring} ist ein Feld\index{Feld} von Zellen mit
impliziten Verbindungen wie bei dem Typ {\bf chain},\index{chain} aber
mit der R"uckw"artsverbindung von der letzten zur ersten Zelle
(geschlossene Kette).
  
\item {\bf star}\index{star} ist ein Feld von Zellen mit impliziten
sternf"ormigen Verbindungen\index{Verbindung} einer ausgezeichneten
Zelle zu allen anderen.
\end{itemize}

Die Topologieformen sind in Abbildung~\ref{StructTopolBild}
dargestellt. Die Felder bestehen aus je vier Zellen und die
Verbindungen\index{Verbindung} sind vorw"artsgerichtet.

\begin{figure}[htbp]
  \begin{center}
     \footnotesize
	\setlength{\unitlength}{0.0065in}
\begin{picture}(320,435)(160,250)
\thicklines
\small
\put(360,575){\oval(170, 50)[tr]}
\put(360,575){\oval(170, 50)[tl]}
\put(275,575){\vector( 0,-1){0}}
\put(265,635){\circle{22}}
\put(325,635){\circle{22}}
\put(385,635){\circle{22}}
\put(445,635){\circle{22}}
\put(329,440){\circle{22}}
\put(369,470){\circle{22}}
\put(329,505){\circle{22}}
\put(264,470){\circle{22}}
\put(265,370){\circle{22}}
\put(345,370){\circle{22}}
\put(265,310){\circle{22}}
\put(345,310){\circle{22}}
\put(265,565){\circle{22}}
\put(325,565){\circle{22}}
\put(385,565){\circle{22}}
\put(445,565){\circle{22}}
\put(275,360){\vector(-3, 2){  0}}
\put(275,360){\vector( 3,-2){ 60}}
\put(275,320){\vector(-3,-2){  0}}
\put(275,320){\vector( 3, 2){ 60}}
\put(280,310){\vector(-1, 0){  0}}
\put(280,310){\vector( 1, 0){ 50}}
\put(345,355){\vector( 0, 1){  0}}
\put(345,355){\vector( 0,-1){ 30}}
\put(265,355){\vector( 0, 1){  0}}
\put(265,355){\vector( 0,-1){ 30}}
\put(280,370){\vector(-1, 0){  0}}
\put(280,370){\vector( 1, 0){ 50}}
\put(275,465){\vector( 2,-1){ 40}}
\put(275,470){\vector( 1, 0){ 80}}
\put(275,475){\vector( 2, 1){ 40}}
\put(395,635){\vector( 1, 0){ 35}}
\put(335,635){\vector( 1, 0){ 35}}
\put(275,635){\vector( 1, 0){ 35}}
\put(160,260){\framebox(320,420){}}
\put(275,565){\vector( 1, 0){ 35}}
\put(335,565){\vector( 1, 0){ 35}}
\put(395,565){\vector( 1, 0){ 35}}
\put(180,335){\makebox(0,0)[lb]{\raisebox{0pt}[0pt][0pt]{\bf clique}}}
\put(180,465){\makebox(0,0)[lb]{\raisebox{0pt}[0pt][0pt]{\bf star}}}
\put(180,560){\makebox(0,0)[lb]{\raisebox{0pt}[0pt][0pt]{\bf ring}}}
\put(180,630){\makebox(0,0)[lb]{\raisebox{0pt}[0pt][0pt]{\bf chain}}}
\end{picture}

    \caption{\label{StructTopolBild} Vordefinierte Verbindungstopologien}
  \end{center}
\end{figure}

\begin{center}
\fbox{
\begin{minipage}{\textwidth}
{\footnotesize
\begin{center}
\begin{eqnarray}
  \mbox{connected\_struct} & ::= & \mbox{chained\_struct}\index{chain} \nonumber\\
          &  |  & \mbox{clique\_struct}\index{clique} \nonumber\\
          &  |  & \mbox{star\_struct} \nonumber\\
          &  |  & \mbox{map\_struct} \label{syconnectedstruct}\\ \nonumber
\end{eqnarray}
\end{center}
}
\end{minipage}
}
\end{center}


\subsubsection{Die Topologieformen chain und ring}
\index{Verbindung!implizite!chain}
\index{Verbindung!implizite!ring}
\index{Topologieform!chain}
\index{Topologieform!ring}

Bei den beiden Topologieformen {\bf chain}\index{chain} und {\bf
ring}\index{ring} m"ussen zus"atzlich zur Spezifikation der Zellen
noch folgende Angaben "uber die Verbindungen gemacht werden (siehe
Syntaxregel~\ref{sychainedstruct}):

\begin{itemize}
  \item Die St"arke der Verbindungen zwischen den Zellen kann  eine {\bf float}-Konstante\index{float} oder das
	Schl"usselwort {\bf random}\index{random} sein (siehe Regel~\ref{syrandomexpression}).
  \item Hinter dem Schl"usselwort {\bf through}\index{through} kann der Typname f"ur die Site der\index{Site} Zielzellen, an
	der die Verbindungen ankommen, angegeben werden. 
  \item Danach kann noch die Richtung der Verbindung\index{Verbindung} festgelegt werden (siehe Regel~\ref{sydirection}).
\end{itemize}

\begin{center}
\fbox{
\begin{minipage}{\textwidth}
{\footnotesize
\begin{center}
\begin{eqnarray}
  \mbox{chained\_struct} & ::= & \mbox{connection\_key array\_size $[$unit\_specification$]$ } \nonumber\\
                         &     & \mbox{topology\_specification  $[$direction$]$ } \nonumber\\
                         &     & \mbox{$[${\bf by}\index{by}
expression\_or\_random$]$ $[$site\_specification$]$}  \label{sychainedstruct}\\[.3 cm]
  \mbox{site\_specification} & ::= & \mbox{{\bf through}\index{through} identifier} \label{sysitespecification}\\[.3 cm]
  \mbox{direction} & ::= & \mbox{{\bf \verb&->&}}\index{forward!Verbindung!Richtung einer}~~~|~~~
			\mbox{{\bf \verb&<-&}}\index{backward!Verbindung!Richtung einer}~~~|~~~\mbox{{\bf
\verb&<->&}}\index{bidirectional!Verbindung!Richtung einer} \label{sydirection}\\[.3 cm] 
  \mbox{connection\_key} & ::= & \mbox{{\bf chain}} \index{chain}~~~|~~~ \mbox{{\bf ring}}\index{ring}\index{ring} \label{syconnectionkey} \\  \nonumber
\end{eqnarray}
\end{center}
}
\end{minipage}
}
\end{center}

Diese Angaben "uber die Verbindungen zwischen den Elementen des Felds
d"urfen entfallen. Wenn das Gewicht der Verbindung nicht angegeben
wird, wird bei der "Ubersetzung der Standardwert 0.0 eingesetzt.

{\bf \verb&->&} gibt an, da"s die impliziten Verbindungen in
Vorw"artsrichtung definiert werden sollen. Analog dazu definiert {\bf
\verb&<-&} r"uckw"artsgerichtete und {\bf \verb&<-&} bidirektionale
Verbindungen.

Fehlt die Angabe der Verbindungsrichtung,\index{Verbindung} wird der
Standardwert {\bf \verb&->&} eingesetzt. Die Standardwerte f"ur
Verbindungsgewicht und -richtung sind in {\bf Nessus} definiert.

Die Angabe der Site der\index{Site} Zielzelle, an der die Verbindung
ankommt, mu"s genau dann fehlen, wenn die Zellen des Felds keine
Sites\index{Site} haben. Es ist ein Fehler, wenn Verbindungen zu
nichtexistenten Sites\index{Site} definiert oder bei vorhandenen
Sites\index{Site} die Zielsites nicht angegeben werden.
\index{Verbindung!implizite!chain} \index{Verbindung!implizite!ring}

"Andert man das vorherige Beispiel ab zu\\
\begin{tabbing}
{\bf typedef}\\
~~{\bf unit}~{\bf with}~{\bf actfunc}~\verb&Act_Logistic&\verb&,&~~{\bf
	outfunc}~\verb&Out_Identity&\verb&,&\\
~~{\bf iotype}~{\bf output}~~\verb&:&~\verb&outUnitType&\verb&;&\\[.15cm]
{\bf structure}\\
~~{\bf ring}\verb&[&10\verb&]&~{\bf of}~\verb&outUnitType&~~{\bf with}~{\bf act}~1.0~{\bf plane}~{\bf at}~
	\verb&(&1\verb&,&50\verb&)&~\verb&->&~{\bf by random}~\verb&:&~\verb&r1&~\verb&,&\verb&;&\\
~~{\bf ring}\verb&[&10\verb&]&~{\bf of}~\verb&outUnitType&~~{\bf with}~{\bf act}~1.0~{\bf plane}~{\bf at}~
	\verb&(&1\verb&,&50\verb&)&~\verb&<-&~{\bf by random}~\verb&:&~\verb&r2&\verb&;&\\
\end{tabbing}
werden zwei Ringe ``r1'' und ``r2'' aus je zehn Zellen definiert. Die
Zellen von ``r1'' sind so miteinander verbunden, da"s gilt\\
\centerline{r1$[$0$]$ \verb&->& r1$[$1$]$ \verb&->& r1$[$2$]$
\verb&->& \ldots \verb&->& r1$[$9$]$ \verb&->& r1$[$0$]$.} Die Zellen
von ``r2'' sind gerade umgekehrt miteinander verbunden. Es gilt\\
\centerline{r1$[$9$]$ \verb&->& r1$[$8$]$ \verb&->& r1$[$7$]$
\verb&->& \ldots \verb&->& r1$[$0$]$ \verb&->& r1$[$9$]$.}

\subsubsection{Die Topologieform {\bf clique}}

\index{Verbindung!implizite!clique} \index{Topologieform!clique} Wie
Syntaxregel~\ref{sycliquestruct} zeigt, ist die Definition der
Topologieform {\bf clique}\index{clique} der der Form {\bf
chain}\index{chain} oder {\bf ring}\index{ring} sehr "ahnlich. Der
einzige Unterschied: da bei der Form {\bf clique}\index{clique} jede
Zelle mit allen anderen Zellen des Felds au"ser sich selbst verbunden
wird, sind alle Verbindungen\index{Verbindung} von vorneherein
bidirektional.\index{bidirectional} Die Richtung der Verbindung darf
bei der Definition einer Clique nicht angegeben werden.

\begin{center}
\fbox{
\begin{minipage}{\textwidth}
{\footnotesize
\begin{center}
\begin{eqnarray}
  \mbox{clique\_struct} & ::= & \mbox{{\bf clique}\index{clique} array\_size $[$unit\_specification$]$  } \nonumber\\
     &     & \mbox{topology\_specification $[${\bf by}\index{by} expression\_or\_random$]$\index{random} } \nonumber\\
     &     & \mbox{$[$site\_specification$]$} \label{sycliquestruct}\\ \nonumber 
\end{eqnarray}
\end{center}
}
\end{minipage}
}
\end{center}

\index{Verbindung!implizite!clique}




\subsubsection{Die Topologieform {\bf star}}

\index{Verbindung!implizite!star} \index{Topologieform!star} Die
Topologieform {\bf star}\index{star} wird so definiert, wie die Formen
{\bf chain}\index{chain} und {\bf ring},\index{ring} aber es mu"s noch
angegeben werden, welches Element des Felds das Zentrum -- also die
Zelle, die mit allen anderen verbunden wird -- werden soll (siehe
Syntaxregel~\ref{systarstruct}). Das Zentrum wird "uber den Index des
entsprechenden Feldelements definiert. Der Ausdruck nach {\bf
center}\index{center} in Syntaxregel~\ref{systarcenter} mu"s eine
Konstante\index{Konstante!als Zentrum einer Struktur} vom Typ {\bf
int} sein.

Man beachte dabei, da"s alle Felder\index{Feld} in {\bf Nessus} mit
dem Index 0 beginnen !

Wird das Zentrum der Struktur\index{Struktur!star} nicht definiert,
wird das {\it erste\/} Feldelement als Zentrum genommen, also das
Feldelement mit Index 0.

\index{Verbindung!implizite!star}
\begin{center}
\fbox{
\begin{minipage}{\textwidth}
{\footnotesize
\begin{center}
\begin{eqnarray}
  \mbox{star\_struct} & ::= & \mbox{{\bf star}\index{star} array\_size $[$unit\_specification$]$ topology\_specification} \nonumber\\
                         &     & \mbox{$[$direction$]$ $[${\bf by}\index{by} expression\_or\_random$]$\index{random} $[$site\_specification$]$}  \nonumber\\
                         &     & \mbox{$[$star\_center$]$}\index{center} \label{systarstruct}\\
  \mbox{star\_center} & ::= & \mbox{{\bf center}\index{center} expression} \label{systarcenter} \\ \nonumber 
\end{eqnarray}
\end{center}
}
\end{minipage}
}
\end{center}

Durch das Programmst"uck\\
\begin{tabbing}
{\bf typedef}\\[.15cm]
~~{\bf unit}~{\bf with}~{\bf actfunc}~\verb&Act_Logistic&\verb&,&~~{\bf
	outfunc}~\verb&Out_Identity&\verb&,&~\verb&:&~\verb&UnitType&\verb&;&\\[.3cm] 
{\bf structure}\\[.15cm]
~~{\bf star}\verb&[&5\verb&]&~{\bf of}~\verb&UnitType&~~{\bf with}~{\bf ellipse}(5,5)~{\bf at}~
	\verb&(&3\verb&,&3\verb&)&~\verb&->&~{\bf by random}~{\bf
center}~0~~\verb&:&~\verb&s1&~\verb&,&\verb&;&\\[.15cm] 
~~{\bf star}\verb&[&5\verb&]&~{\bf of}~\verb&UnitType&~~{\bf with}~{\bf plane}~{\bf at}~
	\verb&(&1\verb&,&50\verb&)&~\verb&<-&~{\bf by random}~{\bf
center}~2~~\verb&:&~\verb&s2&\verb&;&\\[.15cm]
\end{tabbing}

werden die zwei in Abbildung~\ref{StarBild} dargestellten Strukturen
definiert.

\begin{figure}[htbp]
  \begin{center}
    \mbox{
      \begin{minipage}{\textwidth}
        \begin{center}
	   \setlength{\unitlength}{0.01in}
\begin{picture}(280,215)(20,530)
\thicklines \small
\put( 57,610){\circle{18}}
\put(109,609){\circle{18}}
\put(119,654){\circle{18}}
\put( 50,658){\circle{18}}
\put( 85,684){\circle{18}}
\put(235,684){\circle{18}}
\put(200,658){\circle{18}}
\put(269,654){\circle{18}}
\put(259,609){\circle{18}}
\put(207,610){\circle{18}}
\put(250,610){\vector(-1, 0){ 35}}
\put(260,645){\vector(-3,-2){ 45}}
\put(200,650){\vector( 0,-1){ 30}}
\put(230,680){\vector(-1,-3){ 20}}
\put( 85,675){\vector( 1,-3){ 15}}
\put( 80,675){\vector(-1,-3){ 15}}
\put( 90,680){\vector( 4,-3){ 20}}
%\multiput( 75,680)(-0.40000,-0.40000){26}{\makebox(0.4444,0.6667){\sevrm .}}
\put( 65,670){\vector(-1,-1){0}}
\put( 20,540){\framebox(280,200){}}
\put(230,575){\makebox(0,0)[b]{\raisebox{0pt}[0pt][0pt]{\bf s2}}}
\put( 80,575){\makebox(0,0)[b]{\raisebox{0pt}[0pt][0pt]{\bf s1}}}
\put( 55,605){\makebox(0,0)[b]{\raisebox{0pt}[0pt][0pt]{\bf 2}}}
\put(120,650){\makebox(0,0)[b]{\raisebox{0pt}[0pt][0pt]{\bf 3}}}
\put(110,605){\makebox(0,0)[b]{\raisebox{0pt}[0pt][0pt]{\bf 4}}}
\put( 50,655){\makebox(0,0)[b]{\raisebox{0pt}[0pt][0pt]{\bf 1}}}
\put( 85,680){\makebox(0,0)[b]{\raisebox{0pt}[0pt][0pt]{\bf 0}}}
\put(235,680){\makebox(0,0)[b]{\raisebox{0pt}[0pt][0pt]{\bf 0}}}
\put(200,655){\makebox(0,0)[b]{\raisebox{0pt}[0pt][0pt]{\bf 1}}}
\put(260,605){\makebox(0,0)[b]{\raisebox{0pt}[0pt][0pt]{\bf 4}}}
\put(270,650){\makebox(0,0)[b]{\raisebox{0pt}[0pt][0pt]{\bf 3}}}
\put(205,605){\makebox(0,0)[b]{\raisebox{0pt}[0pt][0pt]{\bf 2}}}
\end{picture}

	\end{center}
      \end{minipage}
    }
    \caption{\label{StarBild} Beispiel: Strukturen des Typs {\bf star}}
  \end{center}
\end{figure}


\subsubsection{Felder mit unregelm"a"siger Topologie}
\label{mapstruct}

\index{Verbindung!implizite!irregular} \index{Topologieform!irregular}
Entspricht das Muster der Verbindungen\index{Verbindung} zwischen den
Elementen eines Felds einer {\bf map}-Konstanten, \index{map}
\index{Konstante!Relation!als irregular-Topologie} so kann die
Relation\index{Relation!als Verbindungsmuster} direkt zur Definition
der impliziten Verbindungen innerhalb der Struktur herangezogen
werden.

Sei {\it arrUnit\/} ein Feld von Zellen und {\it intMap\/} eine {\bf map}-Konstante\index{map} vom Typ {\bf int to int}. Dann
ist die Semantik dieser Felddefinition folgende: 
\begin{quote}
  F"ur zwei Indizes x und y  (Typ {\bf int}\index{int} und $\geq$ 0) gilt: \\
  $\exists$ Verbindung von {\it arrUnit\/}$[$x$]$~ 
	nach~{\it arrUnit\/}$[$y$]$ $\Longleftrightarrow$~x~{\it intMap\/}~y.
\end{quote}

Man kann also Felder\index{Feld} von Zellen mit unregelm"a"siger
interner Topologie sehr bequem dadurch definieren, da"s man eine {\bf
map}-Konstante\index{map} definiert, die diese Topologie darstellt.

\index{Verbindung!implizite!irregular}
\begin{center}
\fbox{
\begin{minipage}{\textwidth}
{\footnotesize
\begin{center}
\begin{eqnarray}
  \mbox{map\_struct} & ::= & \mbox{{\bf irregular}\index{irregular} array\_size $[$unit\_specification$]$ } \nonumber \\
                         &     & \mbox{topology\_specification  $[$direction$]$}  \nonumber\\
                         &     & \mbox{ $[${\bf by}\index{by} expression\_or\_random$]$ $[$site\_specification$]$ }\index{random}\nonumber\\
                         &     & \mbox{$[$map\_specification$]$} \label{symapstruct}\\
  \mbox{map\_specification} & ::= & \mbox{{\bf map}\index{map} identifier} \label{symapspec} \\ \nonumber 
\end{eqnarray}
\end{center}
}
\end{minipage}
}
\end{center}
\index{Verbindung!implizite!irregular}

\begin{figure}[htb]
  \begin{center}
    \mbox{
      \begin{minipage}{\textwidth}
        \begin{center}
	   \setlength{\unitlength}{0.0125in}
\begin{picture}(230,215)(30,590)
\thicklines
\put(202,773){\circle{22}}
\put(213,706){\circle{22}}
\put(169,657){\circle{22}}
\put(117,623){\circle{22}}
\put( 80,675){\circle{22}}
\put(160,721){\circle{22}}
\put(115,775){\circle{22}}
\put( 80,720){\circle{22}}
\put( 30,600){\framebox(230,200){}}
\put( 85,685){\vector( 2, 1){ 60}}
\put(125,775){\vector( 1, 0){ 60}}
\put( 90,730){\vector( 3, 1){ 90}}
\put(210,720){\vector( 0, 1){ 40}}
\put(160,735){\vector( 1, 1){ 30}}
\put(170,670){\vector( 1, 3){ 30}}
\put(165,670){\vector( 0, 1){ 35}}
\put(175,665){\vector( 1, 1){ 30}}
\put(125,630){\vector( 3, 2){ 30}}
\put(115,635){\vector( 0, 1){125}}
\put(115,635){\vector(-1, 3){ 25}}
\put(110,635){\vector(-2, 3){ 20}}
%\put(165,650){\makebox(0,0)[lb]{\raisebox{0pt}[0pt][0pt]{\twlbf 7}}}
%\put(210,700){\makebox(0,0)[lb]{\raisebox{0pt}[0pt][0pt]{\twlbf 6}}}
%\put(155,715){\makebox(0,0)[lb]{\raisebox{0pt}[0pt][0pt]{\twlbf 5}}}
%\put(200,770){\makebox(0,0)[lb]{\raisebox{0pt}[0pt][0pt]{\twlbf 4}}}
%\put(110,770){\makebox(0,0)[lb]{\raisebox{0pt}[0pt][0pt]{\twlbf 3}}}
%\put( 75,715){\makebox(0,0)[lb]{\raisebox{0pt}[0pt][0pt]{\twlbf 2}}}
%\put( 75,670){\makebox(0,0)[lb]{\raisebox{0pt}[0pt][0pt]{\twlbf 1}}}
%\put(115,620){\makebox(0,0)[lb]{\raisebox{0pt}[0pt][0pt]{\twlbf 0}}}
\end{picture}

	\end{center}
      \end{minipage}
    }
    \caption{\label{MapStructBild} Beispiel: Gerichteter Graph}
  \end{center}
\end{figure}

%\clearpage

Soll zum Beispiel der in Abbildung~\ref{MapStructBild} gezeigte
gerichtete Graph durch ein Netz dargestellt werden, kann man dieses
durch das nachfolgende Programmst"uck definieren.
\begin{tabbing}bbbb\=bbbb\=bbbb\=bbbb\=bbbb\=bbbb\=\kill
{\bf const}\\
~~\verb&graph&~\verb&=&~{\bf map}~{\bf int}~{\bf to}~{\bf int}\verb&:&\\
~~~~~~~~~~0~{\bf to}~\verb&[&1\verb&,&~2\verb&,&~3\verb&,&~7\verb&]&\verb&,&\\
~~~~~~~~~~\verb&[&1\verb&,&7\verb&]&~{\bf to}~5\verb&,&\\
~~~~~~~~~~7~{\bf to}~\verb&[&5\verb&,&~6\verb&]&\verb&,&\\
~~~~~~~~~~\verb&[&2\verb&,&~3\verb&,&~5\verb&,&~6\verb&,&~7\verb&]&~{\bf to}~4\verb&;&\\[.15cm]
{\bf structure}\\
~~{\bf irregular}\verb&[&8\verb&]&~{\bf with}~{\bf act}~{\bf random}~\\
~~~~{\bf ellipse}\verb&(&4\verb&,&4\verb&)&~{\bf
at}\verb&(&4\verb&,&4\verb&)&~\verb&->&~{\bf by}~$0.5$~{\bf map}~\verb&graph&~\verb&:&~\verb&net&\verb&;&\\
\end{tabbing}
Man beachte dabei, da"s die Verbindung von net$[$7$]$ nach net$[$5$]$
durch die {\bf map}-Konstante nur einmal definiert wird, obwohl sich
die Teilrelationen graph$[$1$]$ und graph$[$2$]$ "uberschneiden.



\subsection{Teilnetze}
\label{subnetstruct} \label{Teilnetze}

 \index{Topologieform!subnet}
Es ist in {\bf Nessus} m"oglich, ein Netz  in einzelne Teilnetze\index{Teilnetz}\index{Teilnetz} zu zerlegen, um diese in
eigenst"andigen Programmen zu definieren, zu "ubersetzen und zu testen. Das hat unter anderem
folgende Vorteile: 
\begin{itemize}
  \item Ein gro"ses komplexes Netz kann nach den Prinzipien des ``top-down''-Designs und
	``information-hiding'' definiert werden. Diese
Entwurfsprinzipien haben sich in der Programmierung 
	seit langem bew"ahrt. Die Vorteile der Anwendung dieser Entwurfsprinzipien sind beim Entwurf
	konnektionistischer Modelle im Prinzip dieselben, wie bei der Programmierung anderer
	Systeme: 
	\begin{itemize}
	  \item bessere "Uberschaubarkeit des Modells,
	  \item klar definierte Schnittstellen zwischen einzelnen Komponenten (Teilnetzen),
	  \item M"oglichkeit, eine Teilnetzdefinition\index{Teilnetz} mehrfach in ein Netz einzubinden.
	\end{itemize}
  \item Teilnetze sind ein sehr gutes Hilfsmittel, um die M"oglichkeiten der Sprache zu erweitern.
	So kann man Teilnetze beliebiger Topologien entwerfen und in Netze einbinden.
  \item Man kann sich Sammlungen von Netzen zusammenstellen, die bestimmte Funktionen erf"ullen.
\end{itemize}

Teilnetze gelten in {\bf Nessus} als Struktur. Sie werden also im
Strukturdefinitionsblock\index{Struktur!Teilnetz} in das Programm
eingebunden. Der logische Name der Datei,\index{Datei!als Teilnetz} in
der sich das {\bf Nessus}-Programm befindet, das als
Teilnetz\index{Teilnetz} fungieren soll, mu"s zuvor als
Konstante\index{Konstante!Teilnetzdatei} definiert worden sein (siehe
Syntaxregel~\ref{syfiletype}).

\begin{center}
\fbox{
\begin{minipage}{\textwidth}
{\footnotesize
\begin{center}
\begin{eqnarray}
  \mbox{subnet\_struct} & ::= & \mbox{{\bf subnet}\index{subnet} identifier position\_specification }\nonumber\\
                        &     & \mbox{$[$parameter\_specification$]$}\index{Parameter} \label{sysubnetstruct}\\[.3 cm]
  \mbox{parameter\_specification}\index{Parameter!"ubergabe an Teilnetz} & ::= & \mbox{{\bf var}\index{param} parameter\_value}\label{syparamspec}\\[.3 cm]
  \mbox{parameter\_value} & ::= & \mbox{``$[$''~expression \{ {\bf ~,~} expression \}~``$]$''}\label{syparamval}\\ \nonumber
\end{eqnarray}
\end{center}
}
\end{minipage}
}
\end{center}

Der Ausdruck ``parameter\_specification''\index{Parameter!"Ubergabe an
Teilnetz} ist optional. Es gibt die M"oglichkeit, an Programme zur
"Ubersetzungs- bzw. an Teilnetze zur Einbindungszeit Parameter zu
"ubergeben. Diese Parameter werden in der
Kopfzeile\index{Kopfzeile!des Programms} des Programms definiert.
Dient dieses Programm dann als Definition eines Teilnetzes, so wird in
``parameter\_specification''\index{Parameter!"Ubergabe an Teilnetz}
angegeben, welche Werte die Parameter annehmen sollen.

\begin{figure}[htbp]
\begin{center}
%    \mbox{
      \begin{minipage}{\textwidth}
%	\vspace{4cm}
        \screendump{0.5}{zug.ps}
      \end{minipage}
%    }
    \caption{\label{SubnetStructBild} Beispiel: Netz f"ur Zugverbindungen}
  \end{center}
\end{figure}

So besteht zum Beispiel das in Abbildung~\ref{SubnetStructBild}
gezeigte Netz, das bestimmte Zugverbindungen in Baden-W"urttemberg und
Bayern darstellen soll, aus zwei "ahnlichen Komponenten. Diese beiden
Teilgraphen kann man durch das folgende Teilnetz definieren:

\begin{center}
    \mbox{
      \begin{minipage}{\textwidth}
           \begin{tabbing}\=nn\=\kill
{\bf network}~\verb&mapBsp&\verb&(&\verb&size&\verb&,&~\verb&cities&\verb&,&~\verb&sMap&\verb&)&\verb&;&\\
{\bf const}\\
~~\verb&cityNames&~\verb&=&~{\bf file}~\verb&cities&~{\bf of}~{\bf string}\verb&;&\\
~~\verb&cityList&~\verb&=&~\verb&[&\verb&cityNames&\verb&]&\verb&;&\\
~~\verb&stateMap&~\verb&=&~{\bf file}~\verb&sMap&~{\bf of}~{\bf map}\verb&;&\\
~~\verb&graph&~\verb&=&~{\bf map}~{\bf int}~{\bf to}~{\bf int}\verb&:&\\
%~~~~~~~~~~{\bf get}~\verb&(&\verb&"to"&\verb&,&~\verb&"&$\backslash$n\verb&"&\verb&)&~{\bf from}~\verb&stateMap&\verb&;&\\
~~~~~~~~~~{\bf get}{\tt ("`to"`, "`$\backslash$n"`)}~{\bf from}~\verb&stateMap&\verb&;&\\
{\bf structure}\\
~~{\bf irregular}\verb&[&\verb&size&\verb&]&~{\bf with}~{\bf act}~$0.6935$~{\bf get}~{\bf name}~{\bf from}~\verb&cityList&\\
~~~~{\bf ellipse}\verb&(&4\verb&,&4\verb&)&~{\bf at}\verb&(&4\verb&,&4\verb&)&~\verb&->&~{\bf by}~$0.5$~{\bf map}~\verb&graph&~\verb&:&~\verb&net&\verb&;&\\
{\bf begin}\\
{\bf end}\verb&.&\\
           \end{tabbing}
      \end{minipage}
    }
  \end{center}

Dieses Teilnetzprogramm erwartet als Parameter 

\begin{itemize}
  \item eine ganze Zahl (Anzahl der Zellen in der Struktur),
  \item den Namen einer Datei, die Zeichenketten - hier die Namen der St"adte - enth"alt, und
  \item den Namen der Datei, in der die {\bf map}-Konstante f"ur die Verbindungsdefinition steht.
\end{itemize}
Das ganze Netz wird dann durch folgendes Programm erzeugt:
\begin{tabbing}bbbb\=bbbb\=bbbb\=bbbb\=bbbb\=bbbb\=\kill
{\bf network}~\verb&main&\verb&(&\verb&)&\verb&;&\\
{\bf const}\\
~~\verb&graphNetz&~\verb&=&~{\bf file}~\verb&"graphNetz.n"&~{\bf of}~{\bf subnet}\verb&;&\\[.15cm]
{\bf structure}\\
~~{\bf subnet}~\verb&gNetz&~{\bf at}\verb&(&0\verb&,&~0\verb&)&~{\bf var}~\verb&[&8\verb&,&~\verb&"Baden-Wuertt.dat"&\verb&,&~\verb&"Baden-Wuertt.map"&\verb&]&\verb&:&~\verb&BW_graph&\verb&;&\\
~~{\bf subnet}~\verb&gNetz&~{\bf at}\verb&(&3\verb&,&~0\verb&)&~{\bf var}~\verb&[&6\verb&,&~\verb&"Bayern.dat"&\verb&,&~\verb&"Bayern.map"&\verb&]&~\verb&:&~\verb&Bayern_graph&\verb&;&\\[.15cm]
{\bf begin}\\
{\bf end}\verb&.&
\end{tabbing}

Dabei m"ussen \verb&"Baden-Wuertt.dat"& bzw. \verb&"Bayern.dat"&
Dateien sein, die die Namen der St"adte enthalten, und
\verb&"Baden-Wuertt.map"& bzw. \verb&"Bayern.map"& die {\bf
map}-Konstante, die die Zugverbindungen beschreibt.  Der Mechanismus
der Parameter"ubergabe und der Parameterdeklaration
\index{Parameter!"Ubergabe an Teilnetz} in der Kopfzeile eines
Programms wird in Kapitel~\ref{Parameter} vorgestellt.



\section{Felder eindimensionaler Strukturen}
\label{Matrizen}

Es gibt in {\bf Nessus} noch die M"oglichkeit, Felder der bisher
vorgestellten Strukturen\index{Struktur!zweidimensionale} zu
definieren. Dies ist dann sinnvoll, wenn man eine gr"o"sere Anzahl
regelm"a"siger Strukturen definieren will, ohne jede dieser Strukturen
mit einem eigenen Bezeichner\index{Bezeichner} versehen zu m"ussen.
Wie Syntaxregel~\ref{symatrixstruct} zeigt, wird dazu der
Strukturdefinition der Ausdruck {\bf array\index{array!einer Struktur}
of} vorangestellt.

\begin{center}
\fbox{
\begin{minipage}{\textwidth}
{\footnotesize
\begin{center}
\begin{eqnarray}
  \mbox{matrix\_struct} & ::= & \mbox{{\bf array} array\_size {\bf of}\index{of}
               linear\_struct}\label{symatrixstruct}\\ \nonumber
\end{eqnarray}\index{array}
\end{center}
}
\end{minipage}
}
\end{center}

Die Elemente eines solchen zweidimensionalen Felds sind
eindimensionale Felder von Zellen. Variable, die diese Feldelemente
als Wert annehmen sollen, m"ussen deshalb vom Typ {\bf array of unit}
sein (siehe auch Kapitel~\ref{Variablendeklarationen}).

Zum Beispiel erzeugt das Programmst"uck
\begin{tabbing}bbbb\=bbbb\=bbbb\=bbbb\=bbbb\=bbbb\=\kill
~~{\bf array}\verb&[&4\verb&]&~{\bf of}~{\bf star}\verb&[&6\verb&]&~{\bf with}~{\bf
act}~$0.7$~{\bf ellipse}\verb&(&3\verb&,&4\verb&)&~{\bf at}\verb&(&2\verb&,&2\verb&)&~\verb&->&~\verb&:&~\verb&netUnits&\verb&;&\\
\end{tabbing}
das in Abbildung~\ref{ArrayStar} dargestellte Netz.

\begin{figure}[htbp]
  \begin{center}
%    \mbox{
      \begin{minipage}{\textwidth}
%	\vspace{5cm}
        \screendump{0.5}{linear.ps}
      \end{minipage}
%    }
    \caption{\label{ArrayStar} Beispiel: Feld linearer Strukturen}
  \end{center}
\end{figure}



\section{Die graphische Darstellung von Strukturen}
\label{StrukturTopologie}

F"ur jede Struktur\index{Struktur!graphische Darstellung} mu"s bei
ihrer Definition angegeben werden, wie sie w"ahrend der Simulation
graphisch\index{graphische Darstellung!einer Struktur} dargestellt
werden soll.  Die Definitionen dieser Strukturtopologien folgen direkt
auf die Spezifikation der Zellen, aus denen die Struktur besteht. Die
vollst"andige Spezifikation der graphischen\index{graphische
Darstellung!einer Struktur} Darstellung besteht aus zwei Komponenten,
n"amlich

\begin{itemize}
  \item der Angabe, welche Form die Struktur\index{Struktur!graphische Form} haben soll und
  \item der Angabe der Stelle der graphischen Benutzeroberf"ache, an der sie dargestellt
 	werden soll.
\end{itemize}

Dabei darf bei Teilnetzen (Struktur {\bf subnet}) und einzelnen Zellen
(Struktur {\bf unit}) die Form der Struktur nicht angegeben werden.
Eine einzelne Zelle ben"otigt keine Angabe einer Form, da sie immer
als einzelner Punkt dargestellt wird, und ein Teilnetz enth"alt in dem
{\bf Nessus}-Programm, durch das es definiert wird, schon die
Formdefinitionen der Strukturen\index{Struktur!graphische Darstellung}
- deshalb ist es falsch, diese f"ur das gesamte Teilnetz nochmals
festlegen zu wollen.

Die m"oglichen graphischen\index{graphische Darstellung!einer
Struktur} Darstellungsformen werden schon aus
Syntaxregel~\ref{sytopologyspecification} deutlich. Es gibt drei
Formen:
\begin{description}
  \item[{\bf plane}:~~~]\index{plane} Durch dieses Schl"usselwort wird angegeben, da"s die Zellen alle
	in einer Ebene dargestellt werden sollen. Nach {\bf plane}\index{plane} kann angegeben werden,
	ob diese Ebene waagerecht (``{\bf -}'') oder senkrecht (``{\bf $|$}'') dargestellt
	werden soll. Der Standardwert f"ur die Orientierung der Darstellung ist ``{\bf -}''.
  \item[{\bf matrix}:~~]\index{matrix} Dies bewirkt, da"s die Zellen der Struktur in Form einer Matrix
	dargestellt werden. Dabei mu"s nach dem Schl"usselwort {\bf matrix}\index{matrix} die Dimension
	der Matrix angegeben werden, etwa (3,8) f"ur eine Matrix mit drei Spalten und acht
	Zeilen. 
  \item[{\bf ellipse}:~]\index{ellipse} Die Zellen der Struktur sollen in Form einer
Ellipse\index{ellipse} dargestellt 
	werden. Nach dem Schl"usselwort mu"s angegeben werden, welche Form diese Ellipse\index{ellipse}
	haben soll. Die beiden Koordinaten\index{Koordinaten} beziehen sich dabei auf ihre beiden
	Achsen. Sind beide gleich, ist die Darstellungsform ein Kreis.
\end{description}

Man beachte, da"s bei einer Matrix die inneren
Positionen\index{Position!einer Struktur} des durch sie definierten
Rechtecks belegt werden, das hei"st, eine {\bf matix}(4,4) kann
sechzehn Zellen aufnehmen. Das ist bei der Definition einer
Ellipse\index{ellipse} nicht der Fall. Sie gibt nur die Form an
(Umfang der Ellipse), in der die Zellen angeordnet werden.

\begin{center}
\fbox{
\begin{minipage}{\textwidth}
{\footnotesize
\begin{center}
\begin{eqnarray}
  \mbox{topology\_specification} & ::= & \mbox{area\_specification position\_specification} 
		\label{sytopologyspecification}\\[.3 cm]
  \mbox{area\_specification} & ::= & \mbox{{\bf plane}\index{plane} orientation} \nonumber \\
                             &  |  & \mbox{ area\_key coordinates} \label{syareaspecification}\\[.3 cm]
  \mbox{orientation} & ::= & \mbox{{\bf $-$}}\;\;|\;\;\mbox{{\bf ``$|$''}} \label{syorientation}\\[.3 cm]
  \mbox{area\_key} & ::= & \mbox{{\bf matrix}}\index{matrix} \nonumber \\
                             &  |  & \mbox{{\bf ellipse}}\index{ellipse} \label{syareakey}\\[.3 cm]
  \mbox{position\_specification}\index{Position!einer Struktur} & ::= & \mbox{{\bf at}\index{at} coordinates} 
		\label{sypositionspecification}\\[.3 cm]
  \mbox{coordinates} & ::= & \mbox{{\bf (} expression{\bf ~,~}expression{\bf )}} \label{sycoordinates} \\ \nonumber 
\end{eqnarray}
\end{center}
}
\end{minipage}
}
\end{center}

Die Position\index{Position!einer Struktur} der
Struktur\index{Struktur!graphische Darstellung!Position} definiert
immer ihren Mittelpunkt. Sie wird nach dem Schl"usselwort {\bf
at}\index{at} angegeben.

Bei Teilnetzen gibt diese Positionsangabe\index{Position!einer
Struktur} den Ursprung der Teilnetzkoordinaten\index{Koordinaten} an.

Ist die angegebene Gr"o"se einer Matrix oder Ellipse\index{ellipse}
f"ur eine Struktur\index{Struktur} nicht ausreichend, so wird diese
von der Struktur belegte Fl"ache bei der "Ubersetzung vergr"o"sert.
Dabei werden die Proportionen der Ausdehnung in x- und in y-Richtung
bewahrt.

Sei n die Anzahl der Zellen einer Struktur, die als Ellipse
dargestellt werden soll, und a und b ihre beiden Achsen, so mu"s $$n <
\pi(\frac{3(a + b)}{4} - \frac{1}{2}\sqrt{ab})$$ gelten, damit nicht
mehrere Zellen an derselben Position dargestellt werden.

Der erste der beiden Ausdr"ucke in der Regel~\ref{sycoordinates} f"ur
``coordinates'' kennzeichnet die Position\index{Position!einer
Struktur} auf der X-Achse (horizontale Verschiebung), die zweite
Angabe bezieht sich auf die Y-Achse (vertikale Verschiebung).

Der Ursprung des SNNS-Koordinatenkreuzes\index{Koordinaten!Ursprung}
liegt in der linken oberen Ecke eines Fensters.  Die Koordinaten sind
positiv, wenn man sich von dort aus nach unten und rechts bewegt.

Ein Fenster hat etwa eine Gr"o"se von 30 auf 30. Diese Gr"o"se ist
variabel. Dabei bezieht sich diese Angabe - wie auch die
Koordinaten\index{Koordinaten!vs. Pixels} - nicht auf Pixels, sondern
auf Positionen\index{Position!einer Struktur} von Zellen. Die Zellen
d"urfen in diesem Koordinatenschema also dicht gepackt werden, auf dem
Bildschirm wird trotzdem ein Abstand zwischen ihnen bestehen.

Werden Strukturen mehrfach angelegt, so bleiben ihre Formen erhalten,
aber ihr Mittelpunkt mu"s verschoben werden. Diese Verschiebung
geschieht immer in der Richtung, in der die Ausehnung der Struktur
geringer ist -- die andere Koordinate bleibt konstant.  Die
Koordinate, die ver"andert werden mu"s, wird gerade soweit erh"oht,
da"s sich die Kopien der Struktur nicht "uberlagern. Eine solche
Verschiebung kann dadurch verursacht werden, da"s f"ur eine Struktur
mehrere Namen definiert wurden, oder auch dadurch, da"s ein Feld
solcher Strukturen angelegt wurde.


\subsection{Beispiel}

Definiert man die Strukturen 

  \begin{center}
    \mbox{
      \begin{minipage}{\textwidth}
           \begin{tabbing}\=nn\=\kill
	   \>{\bf unit} \ldots {\bf at} (12,9) {\bf :} u1, u2, u3{\bf ;} \\
	   \>{\bf cluster}$[$8$]$ \ldots {\bf plane -} {\bf at} (10,6) {\bf :} horizCluster{\bf ;} \\
	   \>{\bf cluster}$[$5$]$ \ldots {\bf plane $|$} {\bf at} (3,7) {\bf :} vertCluster{\bf ;} \\
	   \>{\bf array}$[$6$]$ {\bf of cluster}$[$5$]$  \ldots {\bf matrix} (2,3) {\bf at} (1,2) : outUnits{\bf ;} \\
	   \>{\bf cluster}$[$12$]$  \ldots {\bf ellipse} (12,4) {\bf at} (11,13) : inUnits{\bf ;} \\
           \end{tabbing}
      \end{minipage}
    }
  \end{center}

so entsteht die in Abbildung~\ref{TopologieBild} gezeigte Darstellung.

\begin{figure}[htb]
  \begin{center}
    \fbox{
      \begin{minipage}{\textwidth}
        \begin{center}
	   \setlength{\unitlength}{0.008in}
\begin{picture}(421,372)(75,390)
\thicklines
\put(420,560){\circle*{10}}
\put(380,560){\circle*{10}}
\put(340,560){\circle*{10}}
\put(280,500){\circle*{10}}
\put(260,480){\circle*{10}}
\put(260,460){\circle*{10}}
\put(280,440){\circle*{10}}
\put(320,440){\circle*{10}}
\put(360,440){\circle*{10}}
\put(380,460){\circle*{10}}
\put(380,480){\circle*{10}}
\put(360,500){\circle*{10}}
\put(320,500){\circle*{10}}
\put(160,560){\circle*{10}}
\put(160,580){\circle*{10}}
\put(160,600){\circle*{10}}
\put(160,620){\circle*{10}}
\put(160,640){\circle*{10}}
\put(380,620){\circle*{10}}
\put(360,620){\circle*{10}}
\put(340,620){\circle*{10}}
\put(320,620){\circle*{10}}
\put(300,620){\circle*{10}}
\put(280,620){\circle*{10}}
\put(260,620){\circle*{10}}
\put(240,620){\circle*{10}}
\put(120,680){\circle*{10}}
\put(140,700){\circle*{10}}
\put(120,700){\circle*{10}}
\put(140,720){\circle*{10}}
\put(120,720){\circle*{10}}
\put(180,720){\circle*{10}}
\put(200,720){\circle*{10}}
\put(180,700){\circle*{10}}
\put(200,700){\circle*{10}}
\put(180,680){\circle*{10}}
\put(240,720){\circle*{10}}
\put(260,720){\circle*{10}}
\put(240,700){\circle*{10}}
\put(260,700){\circle*{10}}
\put(240,680){\circle*{10}}
\put(300,720){\circle*{10}}
\put(320,720){\circle*{10}}
\put(300,700){\circle*{10}}
\put(320,700){\circle*{10}}
\put(300,680){\circle*{10}}
\put(360,680){\circle*{10}}
\put(380,700){\circle*{10}}
\put(360,700){\circle*{10}}
\put(380,720){\circle*{10}}
\put(360,720){\circle*{10}}
\put(420,720){\circle*{10}}
\put(440,720){\circle*{10}}
\put(420,700){\circle*{10}}
\put(440,700){\circle*{10}}
\put(420,680){\circle*{10}}
\multiput(100,460)(9.04762,0.00000){43}{\makebox(0.4444,0.6667){\small .}}
\multiput(100,480)(9.04762,0.00000){43}{\makebox(0.4444,0.6667){\small .}}
\multiput(100,680)(9.04762,0.00000){43}{\makebox(0.4444,0.6667){\small .}}
\multiput(100,720)(9.04762,0.00000){43}{\makebox(0.4444,0.6667){\small .}}
\multiput(100,600)(9.04762,0.00000){43}{\makebox(0.4444,0.6667){\small .}}
\multiput(100,500)(9.04762,0.00000){43}{\makebox(0.4444,0.6667){\small .}}
\multiput(100,440)(9.04762,0.00000){43}{\makebox(0.4444,0.6667){\small .}}
\multiput(100,620)(9.04762,0.00000){43}{\makebox(0.4444,0.6667){\small .}}
\multiput(160,740)(0.00000,-8.94737){39}{\makebox(0.4444,0.6667){\small .}}
\multiput(300,740)(0.00000,-8.94737){39}{\makebox(0.4444,0.6667){\small .}}
\multiput(340,740)(0.00000,-8.94737){39}{\makebox(0.4444,0.6667){\small .}}
\multiput(120,740)(0.00000,-8.94737){39}{\makebox(0.4444,0.6667){\small .}}
\multiput(320,740)(0.00000,-8.94737){39}{\makebox(0.4444,0.6667){\small .}}
\multiput(100,560)(9.04762,0.00000){43}{\makebox(0.4444,0.6667){\small .}}
\multiput(100,700)(9.04762,0.00000){43}{\makebox(0.4444,0.6667){\small .}}
\put(240,620){\line( 1, 0){140}}
\put(160,640){\line( 0,-1){ 80}}
\put(420,680){\framebox(20,40){}}
\put(360,680){\framebox(20,40){}}
\put(300,680){\framebox(20,40){}}
\put(240,680){\framebox(20,40){}}
\put(180,680){\framebox(20,40){}}
\put(120,680){\framebox(20,40){}}
\put(100,740){\vector( 1, 0){380}}
\put(100,740){\vector( 0,-1){340}}
\put(120,745){\line( 0,-1){  5}}
\put(140,745){\line( 0,-1){  5}}
\put(160,745){\line( 0,-1){  5}}
\put(220,745){\line( 0,-1){  5}}
\put(200,745){\line( 0,-1){  5}}
\put(180,745){\line( 0,-1){  5}}
\put(240,745){\line( 0,-1){  5}}
\put(260,745){\line( 0,-1){  5}}
\put(280,745){\line( 0,-1){  5}}
\put(340,745){\line( 0,-1){  5}}
\put(320,745){\line( 0,-1){  5}}
\put(300,745){\line( 0,-1){  5}}
\put(360,745){\line( 0,-1){  5}}
\put(380,745){\line( 0,-1){  5}}
\put(400,745){\line( 0,-1){  5}}
\put(460,745){\line( 0,-1){  5}}
\put(440,745){\line( 0,-1){  5}}
\put(420,745){\line( 0,-1){  5}}
\put( 95,720){\line( 1, 0){  5}}
\put( 95,700){\line( 1, 0){  5}}
\put( 95,720){\line( 1, 0){  5}}
\put( 95,680){\line( 1, 0){  5}}
\put( 95,620){\line( 1, 0){  5}}
\put( 95,660){\line( 1, 0){  5}}
\put( 95,640){\line( 1, 0){  5}}
\put( 95,660){\line( 1, 0){  5}}
\put( 95,600){\line( 1, 0){  5}}
\put( 95,580){\line( 1, 0){  5}}
\put( 95,600){\line( 1, 0){  5}}
\put( 95,560){\line( 1, 0){  5}}
\put( 95,500){\line( 1, 0){  5}}
\put( 95,540){\line( 1, 0){  5}}
\put( 95,520){\line( 1, 0){  5}}
\put( 95,540){\line( 1, 0){  5}}
\put( 95,480){\line( 1, 0){  5}}
\put( 95,460){\line( 1, 0){  5}}
\put( 95,480){\line( 1, 0){  5}}
\put( 95,440){\line( 1, 0){  5}}
\put( 85,440){\makebox(0,0)[rb]{\raisebox{0pt}[0pt][0pt]{bf 15}}}
\put( 85,540){\makebox(0,0)[rb]{\raisebox{0pt}[0pt][0pt]{bf 10}}}
\put( 85,640){\makebox(0,0)[rb]{\raisebox{0pt}[0pt][0pt]{bf 5}}}
\put( 85,660){\makebox(0,0)[rb]{\raisebox{0pt}[0pt][0pt]{bf 4}}}
\put( 85,680){\makebox(0,0)[rb]{\raisebox{0pt}[0pt][0pt]{bf 3}}}
\put( 85,700){\makebox(0,0)[rb]{\raisebox{0pt}[0pt][0pt]{bf 2}}}
\put( 85,720){\makebox(0,0)[rb]{\raisebox{0pt}[0pt][0pt]{bf 1}}}
\put(400,750){\makebox(0,0)[b]{\raisebox{0pt}[0pt][0pt]{bf 15}}}
\put(300,750){\makebox(0,0)[b]{\raisebox{0pt}[0pt][0pt]{bf 10}}}
\put(200,750){\makebox(0,0)[lb]{\raisebox{0pt}[0pt][0pt]{bf 5}}}
\put(180,750){\makebox(0,0)[lb]{\raisebox{0pt}[0pt][0pt]{bf 4}}}
\put(160,750){\makebox(0,0)[lb]{\raisebox{0pt}[0pt][0pt]{bf 3}}}
\put(140,750){\makebox(0,0)[lb]{\raisebox{0pt}[0pt][0pt]{bf 2}}}
\put(120,750){\makebox(0,0)[lb]{\raisebox{0pt}[0pt][0pt]{bf 1}}}
\put( 85,420){\makebox(0,0)[lb]{\raisebox{0pt}[0pt][0pt]{bf y}}}
\put(490,750){\makebox(0,0)[lb]{\raisebox{0pt}[0pt][0pt]{bf x}}}
\end{picture}

	\end{center}
      \end{minipage}
    }
    \caption{\label{TopologieBild} Beispiel: Graphische Darstellung von Strukturen}
  \end{center}
\end{figure}

In der Abbildung entspricht eine Einheit einer Einheit in der
Definition. Da diese Einheiten immer Zellpositionen darstellen, kann
es vorkommen, da"s die Positionen der einzelnen Zellen gerundet werden
m"ussen. Das ist insbesondere dann der Fall, eine Dimension einer
Matrix gerade ist. In diesem speziellen Fall verursacht das Runden der
Positionen\index{Position!einer Struktur} eine Verschiebung aller
Zellen um eine halbe Position nach rechts (siehe Strukturen
``outUnits'', ``horizCluster'' in Abbildung~\ref{TopologieBild}).


\section{Beispiel: Erkennung von Buchstaben}

Hier sollen jetzt ein Beispiel gezeigt werden, das von den
M"oglichkeiten der Strukturdefinitionen Gebrauch macht. Es soll ein
Musterklassifikator definiert werden, der in der Lage ist, in einer
5x7-Matrix dargestellte Gro"sbuchstaben zu erkennen.
Abbildung~\ref{LetterNetBild} zeigt das Netz im Display von SNNS f"ur
den Buchstaben ``B''.

\begin{figure}[htbp]
  \begin{center}
%    \mbox{
      \begin{minipage}{\textwidth}
%	\vspace{10cm}
        \screendump{0.5}{buchstabe.ps}
      \end{minipage}
%    }
    \caption{\label{LetterNetBild} Beispiel: Netz zur Erkennung von Gro"sbuchstaben}
  \end{center}
\end{figure}

Als Ausgabe dient ein ``winner-takes-all'' Netz aus 26 Ausgabezellen -
eine f"ur jeden Buchstaben.  Dazwischen liegt noch eine Schicht aus
zehn internen Zellen, die die Abbildung der Ein- auf die Ausgabezellen
leisten. Innerhalb der Zwischenschicht bilden sich beim Lernen des
Netzes Verbindungen zwischen den Zellen. Die einzelnen Schichten sind
untereinander vollst"andig verbunden.

Es ist naheliegend, in einem solchen Fall jede Schicht als eine
Struktur des {\bf Nessus}-Programms zu definieren und danach - im
Topologieerg"anzungsteil - die einzelnen Schichten miteinander zu
verbinden.

Abbildung~\ref{LetterNetProg} zeigt das Programm zur Definition dieses
Netzes. Die Anweisungen im Topologiemodifikationsblock werden in
Kapitel~\ref{Topologie} erkl"art.

\begin{figure}[htbp]
  \begin{center}
    \mbox{
      \begin{minipage}{\textwidth}
           \begin{tabbing}\=nn\=\kill
	   {\bf network}~\verb&recogLetter&\verb&(&\verb&)&\verb&;&\\
~~~~~{\small \{\verb{This network recognizes a letter displayed by a 5x7 input matrix.{\}}\\[.15cm]
{\bf const}~\\
~~\verb&Letters&~\verb&=&~\verb&[&\verb&"A"&\verb&..&\verb&"Z"&\verb&]&\verb&;&~~~~~~~~~{\small \{\verb{ output units are named "A", "B", ...{\}}\\[.15cm]
{\bf typedef}\\
~~{\bf unit}~{\bf with}~{\bf actfunc}~\verb&Act_Logistic&\verb&,&~{\bf outfunc}~\verb&Out_Threshold05&\verb&,&~{\bf act}~{\bf random}\verb&:&~\verb&stdUnit&\verb&;&\\
~~~~~~~~~~~~~~~~~~~~~~~~~~~~~~~~{\small \{\verb{unit type without sites - same for all units{\}}\\[.15cm]
{\bf structure}\\
~~{\bf cluster}\verb&[&35\verb&]&~{\bf of}~\verb&stdUnit&~{\bf with}~{\bf iotype}~{\bf input}~{\bf matrix}~\verb&(&5\verb&,&7\verb&)&~{\bf at}~\verb&(&2\verb&,&7\verb&)&~\verb&:&~\verb&inLayer&\verb&;&~~~~~\\
~~~~~~~~~~~~~~~~~~~~~~~~~~~~~~~~{\small \{\verb{matrix 5x7 to display letters{\}}\\
~~{\bf clique}\verb&[&10\verb&]&~{\bf of}~\verb&stdUnit&~{\bf plane}~\verb&|&~{\bf at}~\verb&(&8\verb&,&7\verb&)&~\verb&:&~\verb&hidLayer&\verb&;&~~~\\
~~~~~~~~~~~~~~~~~~~~~~~~~~~~~~~~{\small \{\verb{hidden layer of ten units, type hidden is default{\}}\\
~~{\bf clique}\verb&[&26\verb&]&~{\bf of}~\verb&stdUnit&~{\bf with}~{\bf iotype}~{\bf output}~~{\bf get}~{\bf name}~{\bf from}~\verb&Letters&\\
~~~~~~~~~~~~~~~~~~~~~~~~~~~~~~~~~~{\bf matrix}\verb&(&2\verb&,&13\verb&)&~{\bf at}~\verb&(&13\verb&,&7\verb&)&~{\bf by}~\verb&-&$1.0$~\verb&:&~\verb&outLayer&\verb&;&\\
~~~~~~~~~~~~~~~~~~~~~~~~~~~~~~~~{\small \{\verb{wta-net, one unit p/ letter{\}}\\[.15cm]
{\bf var}\\
~~{\bf unit}~\verb&:&~\verb&x&\verb&,&~\verb&y&\verb&;&~\\[.15cm]
{\bf begin}~~~~~~~~~~~~~~~~~~~~~~~~~~~{\small \{\verb{define connections between layers{\}}\\
~~{\bf foreach}~\verb&x&~{\bf in}~\verb&inLayer&~{\bf do}\\
~~~~{\bf foreach}~\verb&y&~{\bf in}~~\verb&hidLayer&~{\bf do}\\
~~~~~~\verb&x&~\verb&->&~\verb&y&~\verb&:&~$1.0$~~~~\\
~~~~{\bf end}\\
~~{\bf end}\verb&;&\\
~~{\bf foreach}~\verb&x&~{\bf in}~\verb&hidLayer&~{\bf do}\\
~~~~{\bf foreach}~\verb&y&~{\bf in}~\verb&outLayer&~{\bf do}\\
~~~~~~\verb&x&~\verb&->&~\verb&y&~\verb&:&~$1.0$~~~~~~~~~{\small \{\verb{connect input to hidden and then hidden to output layer{\}}\\
~~~~{\bf end}\\
~~{\bf end}\\
{\bf end}\verb&.&\\

           \end{tabbing}
      \end{minipage}
    }
    \caption{\label{LetterNetProg} Beispiel: Musterklassifikator}
  \end{center}
\end{figure}
