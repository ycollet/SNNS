\chapter{Der {\bf Nessus}-Compiler}
\label{Compiler}
\label{Umgebung}

\section{Aufruf des Compilers} 
\label{Aufruf}\label{CompilerAufruf}


Der {\bf Nessus}-Compiler wird mit

\centerline{  {\bf nessus} source\_file $[$options$]$}

aufgerufen. Dabei steht

\begin{itemize}
  \item ``source\_file'' steht f"ur den Namen der Datei,\index{Quellprogramm} die das {\bf Nessus}-Programm
	enth"alt\footnote{Vorschlag: {\bf Nessus}-Quelldateien immer mit Extension {\bf '.n'}
	bezeichnen.} und\index{Dateiname!Quelle}
  \item ``options'' f"ur die Optionen. Diese Angabe darf entfallen,  kann aber auch eine oder
	mehrere der in den n"achsten Abschnitten beschriebenen  Optionen sein.
\end{itemize}

Die Reihenfolge, in der die in den n"achsten Abschnitten beschriebenen
Optionen angegeben werden, ist beliebig.


\subsection{Spezifikation der Ausgabedatei}
	 
Die Option 

\centerline{{\bf -o} target\_output\_file}

spezifiziert die Datei,\index{Datei!Ausgabe} in die der vom Compiler
erzeugten Code geschrieben werden soll.	 Existiert diese Datei
bereits, wird sie "uberschrieben, und zwar {\it ohne vorherige
R"uckfrage\/}. Wird diese Option nicht 	angegeben oder kann der
Compiler die Ausgabedatei nicht erzeugen\footnote{Ursache davon ist
normalerweise ein Fehler bei der Pfadangabe.}, wird die Ausgabe in
eine Datei ``source\_file{\bf .snns}'' geschrieben.
\index{Dateiname!Ausgabe} 	\addtocounter{footnote}{-1}


\subsection{Definition aktueller Parameterwerte}

Die Werte der Parameter eines {\bf Nessus}-Programms k"onnen -- wie
schon in Abschnitt~\ref{runtimeparameter} beschrieben -- zur
"Ubersetzungszeit festgelegt werden.  Wenn Parametern, f"ur die kein
Standardwert vereinbart wurde, auch zur "Ubersetzungszeit kein Wert
zugewiesen wird, sind sie undefiniert. Jede Verwendung dieser
Parameter im Programm f"uhrt dann zu einer Fehlermeldung. Die Option

\centerline{{\bf -d} constant \{ constant \}}

"ubergibt aktuelle  Werte an die Parameter des Programms. Die Anzahl
der Werte und der Parameter des Programms mu"s genau "ubereinstimmen.

Die Werte ``constant'' m"ussen als einfache Literale\index{Literal!als
Parameterwert} angegeben werden. Ausdr"ucke wie zum Beispiel ``3 + 5''
sind nicht erlaubt. Die Literale m"ussen von einem der Datentypen

\begin{itemize}
  \item ganze Zahlen ({\bf int}),\index{int}
  \item rationale Zahlen\index{Zahl!rationale} ({\bf float})\index{float} oder
  \item Zeichenketten\index{Zeichenkette!als Programmparameter} ({\bf string})\index{string} 
\end{itemize}

sein. Zeichenketten d"urfen ohne ``~''  geschrieben werden, wenn sie
keine Sonderzeichen enthalten. Die Hochkommas werden bei der
Parameter"ubergabe von UNIX entfernt. 



\subsection{Auflisten des Quellprogramms}

Die Option 

\centerline{{\bf -l} $[$listing\_output\_file$]$}

spezifiziert die Datei,\index{Datei!Compiler-Listing} in die die vom
Compiler erzeugte Auflistung des Quellprogramms geschrieben werden
soll. In diese Auflistung werden auch die Fehlermeldungen des
Compilers eingetragen.

Wird der Dateiname\index{Datei!Listing} in dieser Option nicht
angegeben oder kann der Compiler die Ausgabedatei nicht erzeugen,
erscheint die 	Auflistung auf dem Bildschirm ({\it standard
output\/}). Sie kann auch durch geeignete Betriebssystemkommandos in
eine andere Datei umgelenkt werden (etwa ``$>$'' unter Unix oder
``assign'' unter anderen Betriebssystemen). Fehlt diese Option ganz,
wird das Quellprogramm nicht aufgelistet.


\subsection{Auflisten des Quellprogramms in \TeX-Format}

Die Option 

\centerline{{\bf -t}}

legt fest, da"s der Compiler das Quellprogramm in
\TeX-Format\index{TeX@\TeX}\index{LaTeX@\LaTeX}
(siehe~\cite{knu84,lam86}) auflisten soll.  	Diese Option war
zun"achst nur f"ur diese Dokumentation gedacht, bleibt jetzt aber
implementiert, um {\bf Nessus}-Programmierern die Dokumentation {\it
ihrer\/} Programme zu 	erleichtern. Bei der Verwendung dieser Option
m"ussen folgende Punkte beachtet werden:

\begin{itemize}
	  \item Die Option ``{\bf -t}'' sollte nur gemeinsam mit der Option ``{\bf -l}'' angegeben
		werden. Sonst wird das Quellprogramm gar nicht aufgelistet und die \TeX-Option\index{TeX@\TeX}\index{LaTeX@\LaTeX}
		bleibt wirkungslos.
	  \item Fehlermeldungen, die ja auch in der Auflistung des Quellprogramms eingetragen
		werden, erscheinen nicht in \TeX-Format.\index{TeX@\TeX}\index{LaTeX@\LaTeX} Das hei"st,  mit \LaTeX~\index{TeX@\TeX}\index{LaTeX@\LaTeX} k"onnen nur
		korrekte Programme formatiert werden. 
	  \item Die Auflistung des Quellprogramms wird durch  die \TeX-Option\index{TeX@\TeX}\index{LaTeX@\LaTeX} im einzelnen
		folgenderma"sen beeinflu"st:
		\begin{itemize}
		  \item Die Zeilen des Quellprogramms werden nicht numeriert,
		  \item Schl"usselworte werden fett gedruckt und
		  \item Kommentare\index{Kommentar!in \TeX-Format} werden in kleiner Schrift gedruckt.
		\end{itemize}
		Die Auflistung des Quellprogramms sieht dann nach der Formatierung mit \LaTeX~\index{TeX@\TeX}\index{LaTeX@\LaTeX} so
		aus, wie die Beispiele in dieser Dokumentation. 
	  \item Das Format des Quellprogramm, also Einr"ucken von geschachtelten Anweisungen und
		"Ahnliches, bleibt erhalten. Allerdings 
		\begin{itemize}
		  \item werden mit Tabulator-Stops formatierte Eingaben nicht korrekt einger"uckt,
			ein ``$\backslash$t'' wird in ein Leerzeichen umgewandelt,
		  \item sollten Kommentare nicht "uber mehrere Zeilen gehen, sie erscheinen sonst in
			der Auflistung in einer Zeile und werden dadurch normalerweise zu lang,
		  \item d"urfen Kommentare und Zeichenketten nicht das Symbol {\bf \&}
			enthalten\footnote{Kommentare und Zeichenketten werden in {\bf
			$\backslash$verb\&\dots\&} eingeschlossen.},
		  \item k"onnen sich durch Leerzeichen untereinander geschriebene Texte durch den
			\LaTeX-Blocksatz\index{TeX@\TeX}\index{LaTeX@\LaTeX} verschieben, wenn davor unterschiedlicher Text kommt -- das
			kann zum Beispiel bei rechtsb"undig geschriebenen Kommentaren\index{Kommentar!in \TeX-Format} am Ende von
			{\bf Nessus}-Anweisungen der Fall sein\footnote{Zur Anschauung dienen auch
			hier die Beispiele dieser Dokumentation: die Kommentare in den
			Quellprogrammen {\it waren\/} alle b"undig geschrieben.},
		  \item newline-Symbole werden in {\bf $\backslash\backslash$} "ubersetzt. Mehrere
			solcher Symbole hintereinander sind jedoch in \TeX~\index{TeX@\TeX}\index{LaTeX@\LaTeX} nicht zul"assig. Deshalb
			mu"s  die \TeX-Ausgabe\index{TeX@\TeX}\index{LaTeX@\LaTeX} eines Quellprogramms, das Leerzeilen enth"alt,
			nachtr"aglich bearbeitet werden.
		\end{itemize}
	\end{itemize}


\clearpage


\section{Fehlermeldungen}

Es gibt drei Klassen von Fehlern, die der {\bf Nessus}-Compiler in
einem Programm entdecken kann:

\begin{description}
  \item[{\bf fatal error}:]  So werden Syntaxfehler bezeichnet, die so gravierend sind, da"s der Parser
		Schwierigkeiten haben kann, die Programmerkennung fortzusetzen. Ein solcher Fehler
		\begin{itemize}
		  \item kann zu Folgefehlern f"uhren, insbesondere wenn er ein wichtiges Satzzeichen
			oder Schl"usselsymbol betrifft,
		  \item verhindert alle  weiteren semantischen Aktionen (Typ"uberpr"ufung,
			Netzwerkerzeugung) und
		  \item kann im schlimmsten Fall sogar die "Ubersetzung sofort abbrechen.
		\end{itemize}
  \item[{\bf serious error}:]  Das ist die Bezeichnung f"ur semantische Fehler, die 
		\begin{itemize}
		  \item verhindern, da"s das Netzwerk erzeugt wird, aber doch noch weitere
			semantische Aktionen wie Typ"uberpr"ufung oder Auswertung von Ausdr"ucken
			zulassen,
		  \item normalerweise weitere semantische Fehler zur Folge haben, weil
			Ausdr"ucke nicht ausgewertet werden konnten, aber
		  \item die "Ubersetzung nicht unterbrechen.
		\end{itemize}
  \item[{\bf warning}:] So werden Meldungen bezeichnet, die der Compiler ausgibt, sobald er eine
		Anomalie entdeckt, die er selber korrigieren zu k"onnen glaubt. Das k"onnen 
		\begin{itemize}
		  \item behebbare Typfehler, etwa {\bf float}\index{float} statt {\bf int},\index{int}
		  \item Mehrfachdefinitionen von Konstanten\index{Konstante!als Programmparameter!Neudefinition} (alter Wert wird "uberschrieben) oder
			Verbindungen\index{Verbindung!mehrfach definierte} (schon definierte Verbindung wird vorher entfernt)
		\end{itemize}
		und "ahnliche Fehler sein. Diese Fehler haben auf die  "Ubersetzung keinen Einflu"s,
		das Netz wird erzeugt. Allerdings k"onnen sie ein Indiz f"ur semantische Fehler des
		Programms sein. In einem solchen Fall sollte man genau pr"ufen, ob auch
		wirklich das beabsichtigte Netz erzeugt wurde.
\end{description}

Au"serdem gibt es noch durch die Umgebung verursachte Fehler. Diese
Fehler sind "ublicherweise

\begin{itemize}
  \item Warnungen, etwa, da"s keine Auflistung des Quellprogramms erzeugt werden kann, oder
  \item katastrophale Fehler, wie nicht gefundenes Quellprogramm oder Platzmangel.
\end{itemize} 

Bei katastrophalen Fehlern wird die "Ubersetzung sofort abgebrochen.
Die Warnungen dagegen sind normalerweise nur Sch"onheitsfehler bei der
"Ubersetzung, die auf das erzeugte Netz keinen Einflu"s haben.



\section{Zusammenspiel mit Emacs}

Unter vielen Unix-Systemen ist der Emacs\footnotemark-Editor
verf"ugbar.  \footnotetext{Copyright (C) 1988 Free Software
Foundation, Inc.} Dieser Editor verf"ugt "uber einen eingebauten
Parser, der in der Lage ist, Fehlermeldungen, die er von einem anderen
Proze"s empf"angt, zu analysieren. Au"serdem ist es m"oglich,
"Ubersetzungsprozesse in Emacs-Fenstern zu aktivieren. Damit entsteht
eine sehr komfortable Programmierumgebung:

\begin{enumerate}
  \item Starte {\bf emacs} source\_file.
  \item Ediere die Datei.
  \item Erzeuge ein neues Fenster im Editor mit {\bf ESC-x compile}.
Emacs fragt jetzt nach dem "Ubersetzungskommando. 
  \item Gib das "Ubersetzungskommando ein, wie sonst auch (siehe Kapitel~\ref{Aufruf}).
  \item Mit {\bf $<$ctrl$>$-x-`} wird der Parser aufgerufen, der die
Fehlermeldungen analysiert. Emacs springt 
        automatisch zu der Zeile, in der der erste Fehler gefunden wurde.
  \item Mit {\bf $<$ctrl$>$-x-`} sucht Emacs dann den jeweils n"achsten Fehler.
  \item Der "Ubersetzungsproze"s l"auft parallel zur Fehlerkorrektur weiter.
\end{enumerate}

Besonders wenn ein Netz aus verschiedenen {\bf Nessus}-Programmen
besteht ist diese Programmierumgebung sehr hilfreich, da Emacs einer
Fehlermeldung des {\bf Nessus}-Compilers auch den jeweiligen
Dateinamen einer Teilnetzdefinition entnehmen kann. Auch falls der
{\bf Nessus}-Compiler Fehler in einer Teilnetzdefinition entdeckt,
kann man mit {\bf $<$ctrl$>$-x-`} zu den Fehlerstellen springen, da
Emacs die entsprechende Datei l"adt, in der der Fehler gefunden wurde.

Dieser Mechanismus ist auch f"ur die Dateien implementiert, die in
{\bf Nessus}-Programmen eingelesen werden. Wenn etwa in einer {\bf
map}-Konstanten, die aus einer Datei eingelesen wird, ein Fehler
auftritt, kann Emacs die betreffende Datei laden und an die genaue
Fehlerposition springen.


