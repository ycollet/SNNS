\chapter{Konstanten, Bezeichner und Typen}
\label{Konstanten}

\section{Bezeichner und Kommentare}

F"ur Zell- bzw. Sitetypen, Strukturen und Variablen m"ussen eindeutige
Bezeichner\index{Bezeichner} eingef"uhrt werden.  Diese
Bezeichner\index{Bezeichner} sind "uber das ganze Programm an jeweils
einen Datentyp\index{Datentyp!Bindung an Objekt} gebunden.

\label{Namen}

Bezeichner\index{Bezeichner} d"urfen keine Schl"usselworte
\index{Schl"usselwort!vs.  Bezeichner} sein. Die Schl"usselworte in
{\bf Nessus} sind in Abbildung~\ref{SchlWorte} dargestellt.

\begin{figure}[htb]
\footnotesize
\begin{center}
\fbox{
\begin{minipage}{\textwidth}
\begin{center}
\begin{tabular}{lllllll} 
\index{backward}\index{backward!Verbindung!Richtung einer}
  {\bf act}\index{act}          & {\bf clique}\index{clique}  & {\bf exp}\index{exp}         & {\bf input}\index{input}    
& {\bf name}\index{name}        &{\bf sin}\index{sin}          & {\bf through}\index{through} \\

  {\bf actfunc}\index{actfunc}  & {\bf cluster}\index{cluster}& {\bf file}\index{file}       & {\bf int}\index{int}     
& {\bf network}\index{network}  & {\bf site}\index{site}       & {\bf to}\index{to}  \\

  {\bf and}\index{and}          & {\bf const}\index{const}    & {\bf float}\index{float}     & {\bf iotype}\index{iotype} 
& {\bf not}\index{not}          & {\bf sitefunc}\index{sitefunc}& {\bf typedef}\index{typedef} \\

  {\bf arctan}\index{arctan}    & {\bf cos}\index{cos}        & {\bf for}\index{for}         & {\bf irregular}\index{irregular}
& {\bf of}\index{of}            & {\bf sqr}\index{sqr}        & {\bf unit}\index{unit}  \\ 

  {\bf array}\index{array}      & {\bf div}\index{div}        & {\bf foreach}\index{foreach} & {\bf ln}\index{ln}      
&{\bf or}\index{or}             & {\bf sqrt}\index{sqrt}      & {\bf var}\index{var}  \\

  {\bf at} \index{at}           & {\bf do}\index{do}          & {\bf from}\index{from}       & {\bf map}\index{map}   
& {\bf outfunc}\index{outfunc}  & {\bf star}\index{star}      & {\bf while}\index{while} \\

  {\bf begin}                   & {\bf ellipse}\index{ellipse}& {\bf get}\index{get}         & {\bf matrix}\index{matrix}
& {\bf output}\index{output}    & {\bf string}\index{string}  & {\bf with}\index{with} \\ 

  {\bf by}\index{by}            & {\bf else}\index{else}      & {\bf hidden}\index{hidden}   & {\bf max}\index{max}     
& {\bf plane}\index{plane}      & {\bf structure}\index{structure}   \\

  {\bf center}\index{center}    & {\bf elseif}\index{elseif}  & {\bf if}\index{if}           & {\bf min}\index{min}   
& {\bf random}\index{random}    & {\bf subnet}\index{subnet} \\

  {\bf chain}\index{chain}      & {\bf end}\index{end}        & {\bf in}\index{in}           & {\bf mod}\index{mod}  
& {\bf ring}\index{ring}        & {\bf then}\index{then} \\

\end{tabular}
\end{center}                         
\end{minipage}
}
\caption{\label{SchlWorte} Schl"usselworte in {\bf Nessus}}
\end{center}\index{Schl"usselworte}
\end{figure}

Ein Bezeichner\index{Bezeichner} besteht aus einer beliebigen Anzahl
von Buchstaben\index{Buchstaben} und Ziffern. Das erste Zeichen mu"s
ein Buchstabe sein, auch der Unterstrich ``\_'' ist als erstes Zeichen
verboten. Der regul"are Ausdruck f"ur Bezeichner ist\\ \centerline{
${\bf [a-zA-Z]}$ ${\bf [a-zA-Z0-9\_]^*}$} Jeder in einem
Nessus-Programm verwendete Bezeichner\index{Bezeichner} mu"s textuell
vor seiner ersten Verwendung definiert werden.
Variablen\index{Variable} m"ussen deklariert werden.

In {\bf Nessus} wird Gro"s- und Kleinschreibung unterschieden.
Schl"usselworte d"urfen nur Kleinbuchstaben\index{Gro"s- und
Kleinschreibung} enthalten.

\label{Kommentar} 

An jeder Stelle des Programms d"urfen Kommentare\index{Kommentar}
stehen. Sie erstrecken sich immer von einer "offnenden ``{\bf \{}''
bis zur ersten schlie"senden geschweiften
Klammer\index{Klammerung!eines Kommentars} ``{\bf \}}''.
Kommentare\index{Kommentar} k"onnen beliebige Zeichen au"ser ``{\bf
\}}'' enthalten und "uber mehrere Zeilen gehen.

\centerline{\small {\bf \{} Das ist ein Kommentar. {\bf \}}}


\section{Definition von Konstanten}

Die Konstantendefinitionen\index{Konstante} eines {\bf
Nessus}-Programms sind in einem Block zusammengefa"st (const\_block in
Syntaxregel~\ref{syConsBl}), der mit dem Schl"usselwort {\bf
cons}\index{cons} beginnt und mindestens eine Konstantendefinition
\index{Konstantendefinition} enthalten mu"s.  Dieser Block kann aber
auch ganz entfallen, falls keine Konstanten definiert werden sollen.
Die Syntax der Konstantendefinitionen
\index{Konstantendefinition!Syntax} lautet:

\begin{center}
\fbox{
\begin{minipage}{\textwidth}
{\footnotesize
\begin{center}
\begin{eqnarray}
  \mbox{const\_block} & ::= & \mbox{{\bf const}\index{const} constant\_def \{constant\_def\}} \label{syConsBl} \\[.3 cm]
  \mbox{constant\_def}   & ::= & \mbox{{\bf identifier} {\bf \verb&=&} constant {\bf \verb&;&}}\label{syConsDef}\\[.3 cm]
  \mbox{constant}        & ::= & \mbox{expression} \nonumber \\
                         &  |  & \mbox{file\_ constant} \nonumber \\
                         &  |  & \mbox{array\_\index{array} constant} \nonumber \\
                         &  |  & \mbox{map\_ constant} \label{syConstant}\\ \nonumber
\end{eqnarray}
\end{center}
}
\end{minipage}
}
\end{center}

Der Wert einer Konstanten kann
\begin{itemize}
  
\item ein Ausdruck\index{Ausdruck!zur Konstantendefinition} eines der
einfachen Datentypen\index{Datentyp!einfacher} {\bf int},\index{int}
{\bf float}\index{float} oder {\bf string},\index{string}
  
\item der logische Name einer Datei,\index{Datei!logischer Name}
  
\item ein Feld\index{Feld}\index{Feld} eines der einfachen Datentypen oder
  
\item eine Relation\index{Relation} (in {\bf Nessus}: {\bf
map}-Konstante) \index{map}
\end{itemize}
sein.

\subsection{Einfache Datentypen}
\subsubsection{Ganze und rationale Zahlen}

Einfache Datentypen sind in {\bf Nessus} ganze
Zahlen\index{Zahl!ganze} ({\bf int}),\index{int} rationale
Zahlen\index{Zahl!rationale} ({\bf float})\index{float} und
Zeichenketten\index{Zeichenkette} ({\bf string}).\index{string} Die
Ausdr"ucke k"onnen beliebig aus den f"ur diese Datentypen erlaubten
Operatoren und Funktionen (siehe
Kapitel~\ref{EinfacheAusdruecke}\index{Ausdruck!einfacher Datentyp}
zusammengesetzt sein, vorausgesetzt sie enthalten nur
Literale\index{Literal} und Konstanten,\index{Konstante!einfacher
Datentyp} deren Definition ihrer Verwendung textuell vorausgeht.

Der Wertebereich\index{Wertebereich!numerische Konstante} dieser
Konstanten\index{Konstante!numerische} ist nicht in {\bf Nessus}
definiert. Die darstellbaren Werte der ganzen bzw. die Genauigkeit der
rationalen Zahlen\index{Zahl!rationale!Genauigkeit} entsprechen der
von {\it int\/}- bzw. {\it float\/}-Zahlen der Sprache C
(siehe~\cite{ker781}) auf dem jeweiligen System. Bei {\it
float\/}-Zahlen\index{float} werden von SNNS\index{SNNS} sowieso nur
f"unf Stellen nach dem Komma ber"ucksichtigt.
 
Numerische Konstanten werden so geschrieben, wie in C auch. Ganze
Zahlen mit einem Exponenten werden als rationale Zahlen aufgefa"st.
\label{Zeichenketten} Zeichenketten \index{Zeichenkette} werden in
{\bf \verb&" "&} eingeschlossen und d"urfen beliebige Zeichen bis auf
{\bf \verb&"&} enthalten.

\centerline{"' Dies ist eine Zeichenkette "'}

Innerhalb der Zeichenkette\index{Zeichenkette} wird {\bf "'} als
{\bf $\backslash$"'} geschrieben: \\

\centerline{ {\bf "'}So schreibt man {\bf $\backslash$"'}
in einer Zeichenkette.{\bf "'}. }

Es gibt in {\bf Nessus} keinen Datentyp ``character'', auch einzelne
Buchstaben\index{Buchstabe!als Zeichenkette} m"ussen als Zeichenketten
geschrieben werden. Es gibt zwei Klassen von Symbolen, aus denen
Zeichenketten bestehen k"onnen. Die eine Klasse entha"alt solche
Symbole, auf denen eine lexikalische Reihenfolge definiert ist. Die
anndere Klasse enth"alt alle Symbole ohne eine solche lexikalische
Reihenfolge.  	 Die lexikalische\index{Reihenfolge!lexikalische}
Reihenfolge ist auf allen Buchstaben\index{Buchstaben!lexikalische
Reihenfolge} und Ziffern definiert, nicht jedoch auf Sonderzeichen
\index{Zeichenkette!Sonderzeichen} und dem Unterstrich ``\_''. Auf
diesen Symbolen ist die lexikalische\index{lexikalische Reihenfolge}
Reihenfolge

\begin{center}
	  $"'$0$"'$ \verb&..& $"'$9$"'$~~kleiner als~~
          $"'$A$"'$ \verb&..& $"'$Z$"'$~~kleiner als~~
	  $"'$a$"'$ \verb&..& $"'$z$"'$
\end{center}

definiert. Dadurch ist die lexikalische Reihenfolge in der Sprache und
damit systemunabh"angig definiert.

Es gibt in {\bf Nessus} keine leeren Zeichenketten
\index{Zeichenkette!leere}, jede Kette mu"s aus mindestens einem
Zeichen bestehen.

Zeichenketten, \index{Zeichenkette} die von SNNS gelesen werden, also
die Namen von Zellen oder Sites, d"urfen keine Leerzeichen
\index{Zeichenkette} und nicht den senkrechten Strich ``$|$''
enthalten.


\pagebreak[1]
\subsection{Dateien}
\label{Dateien}

In {\bf Nessus} ist es m"oglich, Daten aus Dateien einzulesen und im
Programm zu verwenden. Im Gegensatz zu anderen Sprachen ist die Ein-
und Ausgabe Teil der Sprache und systemunabh"angig definiert.

Damit Daten eingelesen werden k"onnen, mu"s ein logischer Name f"ur
die Eingabedatei\index{Datei!logischer Name} vereinbart werden.

Syntaxregel~\ref{syfile} zeigt die Syntax der Dateivereinbarung.
\index{Datei} ``expression'' mu"s vom Datentyp {\bf string}
\index{string} sein und bezeichnet den physischen Dateinamen.
\index{Dateiname!physischer}

\index{Datei!Typ}
\index{Datei!Typ!int}
\index{Datei!Typ!float}
\index{Datei!Typ!map}
\index{Datei!Typ!subnet}

\begin{center}
\fbox{
\begin{minipage}{\textwidth}
{\footnotesize
\begin{center}
\begin{eqnarray}
  \mbox{file\_constant} & ::= & \mbox{{\bf file}\index{file} expression {\bf of}\index{of} file\_type} \label{syfile} \\[.3cm]
  \mbox{file\_type}      & ::= & \mbox{simple\_type} \nonumber\\
                        & | & \mbox{{\bf map}}\index{map} \nonumber \\
                        & | & \mbox{{\bf subnet}}\index{subnet} \label{syfiletype} \\[.3cm]
  \mbox{simple\_type}    & ::= & \mbox{{\bf int}}\index{int}~~~|~~~\mbox{{\bf
float}}\index{float}~~~|~~~\mbox{{\bf string}}\index{string} \label{sysimpletype} \\ \nonumber 
\end{eqnarray}
\end{center}
}
\end{minipage}
}
\end{center}

F"ur das Format von Dateien der einfachen Typen {\bf int},\index{int} {\bf
float}\index{float} und {\bf string}\index{string} gilt, da"s

\begin{itemize}
\item alle Buchstaben\index{Buchstaben!aus Datei} A-Z und a-z,
\item alle Ziffern 0-9 und
\item der Unterstrich 
\end{itemize}

als g"ultige Eingabesymbole aufgefa"st werden. Alle anderen Zeichen,
\index{Zeichenkette!Format in Datei} der Zeilentrenner eingeschlossen,
werden als Trennzeichen\index{Trennzeichen!in Datei} aufgefa"st. Diese
Trennzeichen werden beim Einlesen der Datei
\index{Datei!Eingabesymbole} ignoriert und d"urfen auch mehrfach
hintereinander auftreten. Es darf allerdings nie der Punkt zur
Trennung von Zahlen\index{Zahl!Format in Datei} verwendet werden, da
er als Dezimalpunkt einer rationalen Zahl interpretiert w"urde.

Zeichenketten in Dateien\index{Datei!Typ string} werden nicht in
"' "' eingeschlossen.

Dateien\index{Datei!Verwendung} k"onnen in {\bf Nessus}-Programmen auf
zwei Arten verwendet werden:

\begin{enumerate}
\item Die Datei wird als Konstante\index{Konstante!ganzzahlige} eines
der Typen {\bf 	array\index{array} of} simple\_type (siehe
Syntaxregel~\ref{sysimpletype}) oder als Abbildungsdefinition einer
{\bf 	map}-Konstanten\index{map} aufgefa"st. In beiden F"allen
werden die 	Dateielemente\index{Datei!als Relation} bei der
Vereinbarung dieser Konstanten implizit eingelesen. Die Konstanten
werden dann verwendet, 	als w"aren ihre Werte intern im Programm
definiert worden.  

\item Die Datei enth"alt ein {\bf Nessus}-Programm.  Das damit
definierte Netz soll als Teilnetz\index{Teilnetz}\index{Teilnetz}
eingelesen werden.
\end{enumerate}


\subsection{Felder von Konstanten}
\label{KonstantenFelder}

In {\bf Nessus} k"onnen Felder\index{Feld} der einfachen Datentypen
{\bf int},\index{int} {\bf float}\index{float} und {\bf string}
\index{string} (simple\_type gem"a"s Syntaxregel~\ref{sysimpletype})
gebildet werden.

\begin{center}
\fbox{
\begin{minipage}{\textwidth}
{\footnotesize
\begin{center}
\begin{eqnarray}
  \mbox{array\_constant}\index{array} & ::= & \mbox{``$[$'' array\_elements ``$]$''} \label{syarray}\\[.3 cm]
  \mbox{array\_elements} & ::= & \mbox{range\_definition} \nonumber\\
                         &  |  & \mbox{expression\_list} \label{syarrayelements}\\[.3 cm]
  \mbox{expression\_list} & ::= & \mbox{expression $\{$ {\bf ,} expression $\}$} \label{syexprlist}\\[.3 cm]
  \mbox{range\_definition} & ::= & \mbox{expression {\bf \verb&..&} expression} \label{syrange}\\ \nonumber
\end{eqnarray}
\end{center}
}
\end{minipage}
}
\end{center}

Alle Elemente eines Felds\index{Feld} m"ussen Konstanten
\index{Konstante!Datei} desselben Datentyps sein. Die L"ange
\index{L"ange!eines Felds} eines Felds wird nicht explizit angegeben.
Jedes Feld mu"s aber aus mindestens einem Element bestehen -- leere
Felder \index{Feld} sind verboten.

\label{internarraydef}
\index{Feld!interne Definition}

Bei der Definition eines Felds durch Aufz"ahlung werden die einzelnen
Feldelemente\index{Feld}\index{Feldelement!explizite Definition}
explizit angef"uhrt. Die einzelnen Elemente werden durch Kommata
getrennt.

Bei der Definition eines Felds dyurch die Angabe eines
Wertebereichs\index{Bereich!Definition!Feld} m"ussen die Begrenzer des
Wertebereichs Elemente von einem der aufz"ahlbaren Datentypen

\begin{itemize}
\item {\bf int}\index{int} oder  \index{Feldelement!Definition durch
Bereichsangabe}

\item {\bf string}\index{string}, bestehend aus genau einem der
Zeichen, \index{Zeichenkette!Bereichsdefinition} auf denen die vorher
beschriebene lexikalische \index{Reihenfolge!lexikalische}
\index{lexikalische Reihenfolge} Reihenfolge definiert ist
\end{itemize}

sein. Bei der Definition eines Wertebereichs
\index{Bereich!Definition!Obergrenze}
\index{Bereich!Definition!Untergrenze} darf die untere Grenze nicht
gr"o"ser als die obere Grenze des Wertebereichs sein.  L"ange
\index{L"ange!eines Felds} Null definiert.

\begin{center}
    \mbox{
      \begin{minipage}{\textwidth}
           \begin{tabbing}\=nn\=\kill
{\bf cons}\\
~~{\bf \verb&enumerate&}~{\bf \verb&=&}~{\bf \verb&[&}~{"'unit1"'}~, {
"'unit2"'}~, {"'new"'}~, {"'old"'}~{\bf \verb&]&}{\bf \verb&;&}\\
%~~{\bf \verb&range&}~{\bf \verb&=&}~{\bf \verb&[&}~{"'A"'}~{\bf
%\verb&..&}~{\verb&"Z"&}~{\bf \verb&]&}{\bf \verb&;&}\\
~~{\bf  range = ["'A"' .. "'Z"' ]; }\\
           \end{tabbing}
      \end{minipage}
    }
  
\end{center}

Im obigen Beispiel wird ``enumerate'' als ein Feld von vier
Zeichenketten definiert.  ``range'' ist ein Feld mit den Zeichenketten
{\verb&"A"&}, {\verb&"B"&}, \ldots, {\verb&"Z"&}.

\label{filearray}\index{file}
\index{Feld!aus Datei}

Soll die Aufz"ahlung der Elemente eines Felds\index{Feld} aus einer
Datei eingelesen werden, schreibt man innerhalb eckiger Klammern den
Namen der Dateikonstanten,\index{Konstante!Feld!aus Datei} d.h., den
logischen Namen der Datei. Diese Datei wird dann bei der "Ubersetzung
eingelesen und ihre Elemente werden zu den Elementen des
Felds.\index{Feld}

Der Konstantendefinitionsteil 
  \begin{center}
      \mbox{
	\begin{minipage}{\textwidth}
          \begin{center}
	   \begin{tabbing}Abbb\=bbbb\=ton\=\kill
		{\bf const}\\
~~{\bf \verb&inFile&}~{\bf \verb&=&}~{\bf file}~{\bf \verb&"input.dat"&}~{\bf of}~{\bf string}{\bf \verb&;&}\\
~~{\bf \verb&nameList&}~{\bf \verb&=&}~{\bf \verb&[&}{\bf \verb&inFile&}{\bf \verb&]&}{\bf \verb&;&}\\

          \end{tabbing}
         \end{center}
        \end{minipage}
       }
  \end{center}

definiert ``nameList'' als das Feld\index{Feld} der in
``input\verb&.&dat'' enthaltenen Zeichenketten.\index{Zeichenkette!aus
Datei}


\subsection{Relationen}
\label{Relationen}

Mittels Relationen -- sie werden im Rest dieser Dokumentation als {\bf
map}-Konstanten\index{Konstante!Relation} bezeichnet -- kann man eine
Abbildung von Werten eines einfachen Datentyps (simple\_type nach
Syntaxregel~\ref{sysimpletype}) auf Werte eines anderen oder desselben
einfachen Datentyps definieren. In einem Programm zur Definition
neuronaler Netze ist das dann sinnvoll, wenn

\begin{itemize}
\item ein unregelm"a"siges konstantes Muster von
Verbindungen,\index{Verbindung!implizite!Definition durch Relation}
das sich nicht durch wenige geschlossene Ausdr"ucke (etwa:
verbinde x mit y wenn \dots) darstellen l"a"st, oder

\item eine tats"achliche Relation\index{Relation} zwischen zwei Datentypen 
\end{itemize}
im Programm definiert werden mu"s.

Die Semantik der {\bf map}-Konstanten\index{map} ist die einer
Relation, das hei"st, alle Mengen d"urfen sich beliebig
"uberschneiden, sowohl Bild und Urbild, als auch verschiedene Bilder
oder Urbilder.

Auf die Werte von Bild und Urbild einer {\bf map}-Konstanten
\index{map} kann sp"ater nicht mehr zugegriffen werden.  Insbesondere
ist es gleichg"ultig, ob in einer Zeile ein einzelnes Element mit oder
ohne Mengenklammern geschrieben wurde. Die Relation
\index{Relation!Semantik} gilt immer f"ur Paare (x,y) mit
x~$\in$~Bildmenge und y~$\in$~Urbildmenge. F"ur je zwei solche Paare
kann dann abgefragt werden, ob die Relation\index{Relation} existiert.

Die so definierte Relation weist a priori keine zus"atzlichen
Eigenschaften au"ser den explizit definierten Abbildungen auf.
Insbesondere ist sie weder symmetrisch noch transitiv.

Die Verwendung einer solchen {\bf map}-Konstanten \index{map}
\index{Konstante!Feld} in einem Programm wird in den
Abschnitten~\ref{mapstruct} und \ref{mapcond} genauer beschrieben.

Eine {\bf map}-Konstante\index{map} kann intern in einem {\bf
Nessus}-Programm definiert oder aus einer Datei\index{Datei!als
Relation} eingelesen werden.

\begin{center}
\fbox{
\begin{minipage}{\textwidth}
{\footnotesize
\begin{center}
\begin{eqnarray}
  \mbox{map\_constant} & ::= & \mbox{{\bf map}\index{map} {\bf of}\index{of} simple\_type
{\bf to}\index{to} simple\_type {\bf :}} \nonumber\\ 
                       &     & \mbox{map\_def {\bf ;}}  \label{symap} \\[.3 cm]
  \mbox{map\_def}      & ::= & \mbox{map\_file}~~~|~~~\mbox{map\_block}  \label{symapdef} \\[.3 cm]
  \mbox{map\_file}     & ::= & \mbox{ {\bf get}}\index{get}\mbox{~``$($'' expression $[$,
expression$]$ ``$)$''~}\mbox{{\bf from}\index{from} {\bf identifier} } \label{symapfile}\\[.3 cm]  
  \mbox{map\_block} & ::= & \mbox{map\_line {\bf \{ ,} map\_line {\bf \}}} \label{symapblock}\\[.3 cm]
  \mbox{map\_line}  & ::= & \mbox{map\_domain {\bf to}\index{to} map\_domain} \label{symapline}\\[.3 cm]
  \mbox{map\_domain}& ::= & \mbox{array\_constant\index{array}}~~~|~~~\mbox{expression}
\label{symapdomain}\\ \nonumber 
\end{eqnarray} 
\end{center}
}  
\end{minipage}
}
\end{center}

\begin{figure}[htbp]
  \begin{center}
    \mbox{
      \begin{minipage}{\textwidth}
           \begin{tabbing}\=nn\=\kill
~~{\bf \verb&mapCons&}~{\bf \verb&=&}~{\bf map}~{\bf int}~{\bf to}~{\bf string}{\bf \verb&:&}\\
~~~~~~~~~~~~0~~~~~~~~~~~{\bf to}~{\bf \verb&[&} "'x"' {\bf \verb&..&} "'z"' {\bf \verb&]&}{\bf \verb&,&}\\
~~~~~~~~~~~~{\bf \verb&[&}0{\bf \verb&..&}10{\bf \verb&]&}~~{\bf to}~{\bf \verb&[&}
"'a"'{\bf \verb&..&} "'d"' {\bf \verb&]&}{\bf \verb&,&}\\
           \end{tabbing}
      \end{minipage}
    }
    \caption{\label{ZeilenBsp} Beispiel: Zeilenweise Relationsdefinition}
  \end{center}
\end{figure}

Abbildung~\ref{ZeilenBsp} zeigt anhand eines Beispiels die zeilenweise
Definition von Relationen\index{Relation!zeilenweise Definition}. Man
beachte, da"s die Zeile\\ \centerline{{\bf \verb&[&}0{\bf
\verb&..&}10{\bf \verb&]&}~~{\bf to}~{\bf \verb&[&}
"'a"'{\bf \verb&..&} "'d"' {\bf \verb&]&}}
jedes Paar (a,b) mit a in $[$0\verb&..&10$]$ und b in {\bf \verb&[&}
"'a"'{\bf \verb&..&} "'d"' {\bf \verb&]&} in
die Relation aufnimmt.

Eine intern definierte Relation besteht aus folgenden Komponenten:

\begin{itemize}
\item Im Kopf der Relationsdefinition\index{Relation} werden die
Datentypen der Elemente von Urbild- und Bildmenge festgelegt. Diese
Datentypen m"ussen zu den einfachen Typen geh"oren (simple\_type).
  
\item Die {\bf map}-Konstante wird zeilenweise definiert, die Anzahl
der Zeilen mu"s nicht explizit angegeben werden
(Syntaxregel~\ref{symapblock}). Die Zeilen werden durch Kommata
getrennt.

\item Jede dieser Zeilen definiert eine Relation zwischen zwei
Teilmengen der beiden Datentypen.  Besteht eine der Teilmengen
einer Zeile aus nur einem Element, d"urfen die eckigen Klammern
weggelassen werden, wie Regel~\ref{symapdomain} zeigt.
\end{itemize}

Es ist m"oglich, Relationen\index{Relation} aus feineren oder
disjunkten anderen Relationen\index{Relation!Verfeinerung}
zusammenzusetzen, ohne die Abbildungen nochmals schreiben zu m"ussen.
In den Kapiteln~\ref{LogischeAusdruecke}, \ref{mapstruct}
und~\ref{mapcond} wird gezeigt, da"s auch Zeilenbereiche
\index{Bereich!Definition!Relation} einer {\bf map}-Konstanten
\index{Konstante!Relation!Verfeinerung} als Teilrelation
\index{Teilrelation} angesprochen werden k"onnen. In obigem Beispiel
w"are\\

\centerline{``mapCons$[$0{\bf ..}1$]$''~~~$\equiv$~~~``mapCons'',}

da ``mapCons'' nur aus den beiden Zeilen mit Indizes 0 und 1 besteht.

\label{filemap}

Eine {\bf map}-Konstante\index{map}\index{Konstante!Relation!aus
Datei} kann aus einer Datei\index{Datei!als Relation} eingelesen
werden (siehe Syntaxregel~\ref{symapfile}).  Daf"ur mu"s die Konstante
``identifier'' nach dem Schl"usselwort {\bf from} gem"a"s
Regel~\ref{syfile} zuvor als eine Datei vom Typ {\bf map}\index{map}
definiert worden sein.

In den Kopfzeilen\index{Kopfzeile!einer {\bf map}-Konstanten} der aus
Dateien einzulesenden {\bf map}-Konstanten \index{map}
\index{Konstante!Relation!Kopfzeile} werden wie bei explizit im
Programm definierten die Datentypen von Bild und Urbild festgelegt.
Anstatt dann jedoch die einzelnen Zeilen der {\bf map}-Konstanten
\index{map} zu schreiben, mu"s nun angegeben werden, wie die Zeilen
aus der Datei eingelesen werden sollen (siehe
Syntaxregel~\ref{symapfile}):

\begin{itemize}
  
\item ``identifier'' bezeichnet die zu der {\bf map}-Datei\index{map}
geh"orende Dateikonstante.

  
\item Innnerhalb der Klammern mu"s angegeben werden, wie Bild und
Urbild bzw. die einzelnen Zeilen getrennt werden:
     \begin{itemize} 
            \item Dr erste Ausdruck definiert die Zeichenkette, die in
		  der Datei Bild und Urbild einer Zeile der {\bf
		  map}-Konstanten trennt. Diese Angabe mu"s
                  durch eine Konstante vom Typ {\bf
		  string}\index{string} erfolgen und darf
                  nicht entfallen.
            \item Mit dem zweiten Ausdruck {\it darf\/} angegeben
		  werden,wie die
                  einzelnen Zeilen der {\bf map}-Konstanten\index{map}
		  getrennt werden. Die Angabe einer speziellen 
                  Zeichenkette\index{Relation!Trennsymbol in Datei}
		  zur Trennung von Zeilen darf entfallen, sie mu"s aber unbedingt in den 
                  F"allen angegeben werden, in denen die Zeile der {\bf map}-Konstanten\index{map} selbst ein
                  oder mehrere  ``$\backslash$n'' enth"alt, etwa weil sie l"anger ist als die maximale
                  Zeilenl"ange des Edierers, mit dem die Datei erzeugt wurde.
	\end{itemize}
  \item Die einzelnen Elemente von Bild oder Urbild k"onnen durch ein beliebiges der Zeichen
        getrennt werden, die in Abschnitt~\ref{Dateien} als Trennzeichen\index{Trennzeichen!externe Relation} aufgef"uhrt wurden.
	Die  Trennzeichen\index{Relation!Trennzeichen in externer} zwischen den Elementen m"ussen nat"urlich andere sein, als die zwischen
        Bild und Urbild bzw. zwischen den einzelnen Zeilen.
\end{itemize}

\begin{figure}[htbp]
  \begin{center}
    \mbox{
      \begin{minipage}{\textwidth}
           \begin{tabbing}\=nn\=\kill
	   {\bf network}~\verb&mapBsp&\verb&(&\verb&)&\verb&;&\\
\\
{\bf const}\\
~~\verb&graph&~\verb&=&~{\bf map}~{\bf int}~{\bf to}~{\bf int}\verb&:&\\
~~~~~~~~~~0~{\bf to}~\verb&[&1\verb&,&~2\verb&,&~3\verb&,&~7\verb&]&\verb&,&\\
~~~~~~~~~~\verb&[&1\verb&,&7\verb&]&~{\bf to}~5\verb&,&\\
~~~~~~~~~~7~{\bf to}~\verb&[&5\verb&,&~6\verb&]&\verb&,&\\
~~~~~~~~~~\verb&[&2\verb&,&~3\verb&,&~5\verb&,&~6\verb&,&~7\verb&]&~{\bf to}~4\verb&;&\\
\\
{\bf structure}\\
~~{\bf irregular}\verb&[&8\verb&]&~{\bf with}~{\bf act}~{\bf random}~\\
~~~~{\bf ellipse}\verb&(&4\verb&,&4\verb&)&~{\bf at}\verb&(&4\verb&,&4\verb&)&~\verb&->&~{\bf by}~$0.5$~{\bf map}~\verb&graph&~\verb&:&~\verb&net&\verb&;&\\
\\
{\bf begin}\\
{\bf end}\verb&.&
           \end{tabbing}
      \end{minipage}
    }
    \caption{\label{DatIntBsp} Beispiel: {\bf map}-Konstanten aus Dateien einlesen}
  \end{center}
\end{figure}

Die Datei ``LetterDigitRelation.map'' k"onnte etwa folgenden Inhalt haben:
  \begin{center}
    \fbox{
      \begin{minipage}{\textwidth}
           \begin{tabbing}\=nn\=\kill \\
		~~~~~~~~~~~~{\footnotesize \verb& 0 1 2 3 4 5 6 7 8 9 10 :  a, b, c, d, e; &}\\
		~~~~~~~~~~~~{\footnotesize \verb& 11 12 13 14 15 16 17 18 19 20 : &}\\
		~~~~~~~~~~~~~~~~~~{\footnotesize \verb& f, g, h, i, j, k, l, m, n, o, p, q, r, s, t ; &}\\
		~~~~~~~~~~~~{\footnotesize \verb& 21 22 23 24 25 26 : u, v, w, x, y, z; &}\\
		~~~~~~~~~~~~{\footnotesize \verb& 0 1 2 3 4 5 6 7 8 9 10 11 12 13 14 15 16 17 18 19 20 &}\\
		~~~~~~~~~~~~~~~~~~{\footnotesize \verb& 21 22 23 24 25 26 : x ; &}\\
		~~~~~~~~~~~~{\footnotesize \verb& 0 1 2 3 4 5 6 7 8 9 : y ; &}\\
           \end{tabbing}
      \end{minipage}
    }
  \end{center}

Feldelemente, die aus einer Datei eingelesen werden, m"ussen in dieser
Datei immer aufgez"ahlt sein. Die Definition der Feldelemente durch
die Angabe eines Wertebereichs ist nicht zul"assig.
 

\section{Typdefinitionen}
\label{Typdefinitionen}

Es gibt in {\bf Nessus} zwei Arten von Typdefinitionen:
\begin{itemize}
  \item Definition eines  Site-Typs\index{Site} und
  \item Definition eines Zell-Typs.
\end{itemize}

\begin{center}
\fbox{
\begin{minipage}{\textwidth}
{\footnotesize
\begin{center}
\begin{eqnarray}
  \mbox{type\_block} & ::= & \mbox{{\bf typedef}\index{typedef} type\_definition \{type\_definition\}}
					 \label{sytypeblock} \\[.3 cm]
  \mbox{type\_definition} & ::= & \mbox{type\_expression {\bf :}~ identifier {\bf ;}} \label{sytypedef}\\[.3 cm]
  \mbox{type\_expression} & ::= & \mbox{site\_type}\index{Site} \nonumber\\
          &  |  & \mbox{unit\_type} \label{sytypeexpr}\\ \nonumber
\end{eqnarray}
\end{center}
}
\end{minipage}
}
\end{center}

\subsection{Site-Typen}\index{Site}
\label{Sites}\index{Site}

Da Sites unver"anderliche Merkmale einer Zelle sind, mu"s f"ur jede
Site, die in einem {\bf Nessus}-Programm vorkommen soll, ein Site-Typ
definiert werden. Die Sites werden im "ubrigen Programm nur "uber den
Namen dieses Typs angesprochen. Au"serdem mu"s der Name einer Site
eindeutig sein -- es d"urfen nie zwei Typen f"ur Sites definiert
werden, deren Name gleich ist.

Enth"alt ein Netz Teilnetze \index{Teilnetz} \index{Teilnetz}
\index{Teilnetz!Sites} aus externen {\bf Nessus}-Programmen, so
m"ussen in diesen zwar f"ur die Sites lokale Typnamen eingef"uhrt
werden, aber wenn der Site-Name in zwei verschiedenen
Teilnetzen\index{Teilnetz} oder in einem Teilnetz\index{Teilnetz} und
im Hauptprogramm derselbe ist, mu"s auch die Sitefunktion dieselbe
sein. Eine genaue Erkl"arung zum Umgang mit Sites in
Teilnetzen\index{Teilnetz}\index{Teilnetz!Sites} findet sich in
Kapitel~\ref{Teilnetze}.

\begin{center}
\fbox{
\begin{minipage}{\textwidth}
{\footnotesize
\begin{center}
\begin{eqnarray}
  \mbox{site\_type} & ::= & \mbox{{\bf site\index{site} with} name\_definition {\bf ~,~}
site\_function}\label{sysitetype} \\[.3 cm] 
  \mbox{name\_definition} & ::=  & \mbox{{\bf name}\index{name} expression} \label{synamedef} \\[.3 cm]
  \mbox{site\_function} & ::=  & \mbox{{\bf sitefunc}\index{sitefunc} identifier}
				 \label{sysitefunc} \\ \nonumber
\end{eqnarray}
\end{center}
}
\end{minipage}
}
\end{center}

Die Merkmale einer Site \index{Site} von {\bf Nessus} und SNNS
bestehen aus zwei Parametern\index{Parameter!einer Site} (siehe
Regel~\ref{sysitetype}):
\begin{itemize}
  \item dem Namen der Site (vom Typ {\bf string}) und
  \item dem Namen der Sitefunktion (identifier).
\end{itemize}


\subsection{Zelltypen}
\label{unitTypes}

Ein konnektionistisches Netzwerk besteht im allgemeinen aus wenigen
gr"o"seren Gruppen von gleichartigen Zellen. Die Zellen innerhalb
einer solchen Gruppe unterscheiden sich oft nur durch wenige Merkmale.

In solchen F"allen ist es sinnvoll, wenn man nicht f"ur jede Zelle
alle ihre Merkmale neu festlegen mu"s. In {\bf Nessus} gibt es deshalb
die M"oglichkeit, Zelltypen einzuf"uhren. Ein Zelltyp ist nichts
anderes als ein benanntes Muster von Parameterwerten
\index{Parameter!einer Zelle} einer Zelle. Definiert wird die Belegung
bestimmter Merkmale mit Werten, die dann f"ur alle Zellen dieses Typs
gleich sind.

Es m"ussen bei der Definition von Zelltypen folgende Einschr"ankungen
beachtet werden:
\begin{itemize}
  
\item Die Merkmale ``act\_function'', ``out\_function'' und
``sites\_list'' m"ussen immer definiert werden, wenn ein Zelltyp
eingef"uhrt wird. Die Definition von ``act\_initial''\index{act} und
`name\_definition'' hingegen ist fakultativ. \index{Parameter!einer
Zelle!obligatorische}
  
\item Die Merkmale einer Zelle, die "uber die Typdefinition festgelegt
werden, d"urfen in den 	restlichen Anweisungen des {\bf
Nessus}-Programms nicht mehr ver"andert werden. \index{Parameter!einer
Zelle!konstante}
\end{itemize}

``initial\_act''\index{act} ist die Definition eines anf"anglichen
Aktivierungswerts\index{Aktivierung} f"ur die Zelle. Der
Aktivierungswert einer Zelle ist f"ur SNNS auf $[-1.0, 1.0]$
beschr"ankt. Statt einer {\bf float}-Konstanten\index{float} kann auch
das Schl"usselwort {\bf random}\index{random} angegeben werden. In
diesem Fall wird f"ur jede sp"ater definierte Zelle dieses Typs eine
Zufallszahl\index{Zahl!zuf"allige}\index{Zufallszahl} aus dem
Intervall $[-1.0, 1.0]$ erzeugt.

\begin{center}
\fbox{
\begin{minipage}{\textwidth}
{\footnotesize
\begin{center}
\begin{eqnarray}
  \mbox{unit\_type} & ::= & \mbox{{\bf unit}\index{unit} unit\_param\_list}\label{syunittype}\\[.3 cm]
  \mbox{unit\_param\_list} & ::= & \mbox{{\bf with}\index{with}
unit\_parameter \{ {\bf ~,~} unit\_parameter \} } \label{syunitparamlist}\\[.3 cm] 
  \mbox{unit\_parameter}\index{Parameter!einer Zelle} & ::= & \mbox{name\_definition} \nonumber\\
          &  |  & \mbox{initial\_act}\nonumber\\\index{act}
          &  |  & \mbox{topology\_type}\nonumber\\
          &  |  & \mbox{act\_function}\nonumber\\
          &  |  & \mbox{out\_function}\nonumber\\
          &  |  & \mbox{sites\_definition} \label{syunitparam}\\[.3cm]
  \mbox{initial\_act}\index{act} & ::= & \mbox{{\bf act} expression\_or\_random} \label{syactinit} \\[.3cm]
  \mbox{act\_function} & ::= & \mbox{{\bf actfunc}\index{actfunc} {\bf identifier}} \label{syactfunc} \\[.3cm]
  \mbox{out\_function} & ::= & \mbox{{\bf outfunc}\index{outfunc} {\bf
identifier}} \label{syoutfunc}\\  \nonumber
\end{eqnarray}
\end{center}
}
\end{minipage}
}
\end{center}

\pagebreak[2]

\begin{center}
\fbox{
\begin{minipage}{\textwidth}
{\footnotesize
\begin{center}
\begin{eqnarray}
  \mbox{sites\_definition} & ::= & \mbox{{\bf site}\index{site} sites\_list} \label{sysitesdef} \\[.3cm]
  \mbox{sites\_list} & ::= & \mbox{{\bf identifier}}\nonumber \\
          &  |  & \mbox{ ``$[$'' {\bf identifier} \{ {\bf ~,~} {\bf
identifier} \} ``$]$''}\label{sysiteslist}\\[.3cm] 
  \mbox{expression\_or\_random} & ::= & \mbox{expression}\;\;|\;\;\mbox{{\bf random}}\index{random} \label{syrandomexpression} \\[.3cm]
  \mbox{topology\_type} & ::= & \mbox{{\bf iotype}\index{type} topology\_key}\label{syrtopologytype} \\[.3cm]
  \mbox{topology\_key} & ::= & \mbox{{\bf hidden}\index{hidden} }\nonumber \\
          &  |  & \mbox{{\bf input}\index{input} $[$ {\bf output}\index{output} $]$ }\nonumber\\
          &  |  & \mbox{{\bf output}\index{output} $[$ {\bf
input}\index{input} $]$ } \label{sytopologykey}\\ \nonumber 
\end{eqnarray}
\end{center}
}
\end{minipage}
}
\end{center}



\subsection{Beispiel}

Das nachfolgende Beispiel zeigt zwei Site- und zwei
Zelltypdefinitionen.  Man beachte, da"s s"amtliche angegebenen
Funktionsnamen Bezeichner\index{Bezeichner} und keine Zeichenketten
sind und Sites "uber den f"ur den Sitetyp eingef"uhrten Bezeichner
\index{Bezeichner} und nicht "uber den Sitenamen angesprochen werden.

\begin{center}
    \mbox{
      \begin{minipage}{\textwidth}
           \begin{tabbing}\=nn\=\kill
	   {\bf typedef}\\
~~{\bf site}~{\bf with}~{\bf name}~{\bf \verb&"inhibit"&}{\bf \verb&,&}~{\bf sitefunc}~{\bf \verb&Identity&}~{\bf \verb&:&}~{\bf \verb&inSiteType&}{\bf \verb&;&}\\
~~{\bf site}~{\bf with}~{\bf name}~{\bf \verb&"excite"&}{\bf \verb&,&}~{\bf sitefunc}~{\bf \verb&WeightedSum&}~{\bf \verb&:&}~{\bf \verb&exSiteType&}{\bf \verb&;&}\\[.3cm]
~~{\bf unit}~{\bf with}~{\bf name}~{\bf \verb&"in_Unit"&}{\bf \verb&,&}~{\bf actfunc}~{\bf \verb&Logistic&}{\bf \verb&,&}~~{\bf outfunc}~{\bf \verb&Identity&}{\bf \verb&,&}~\\
~~~~~~~~~~~~{\bf act}~{\bf random}{\bf \verb&,&}~{\bf iotype}~{\bf input}~~{\bf \verb&:&}~{\bf \verb&inUnitType&}{\bf \verb&;&}\\
~~~~~~~~~~~~{\small \{\verb{Zelltyp fuer Eingabezellen, bei dem alle Merkmale definiert wurden.{\}}\\
~~~~~~~~~~~~{\small \{\verb{Zellen dieses Typs haben keine Sites.{\}}\\[.3cm]
~~{\bf unit}~{\bf with}~{\bf actfunc}~{\bf \verb&Logistic&}{\bf \verb&,&}~~{\bf outfunc}~{\bf \verb&Identity&}{\bf \verb&,&}~{\bf site}~{\bf \verb&[&}{\bf \verb&inSite&}{\bf \verb&,&}~{\bf \verb&exSite&}{\bf \verb&]&}{\bf \verb&,&}\\
~~~~~~~~~~~~{\bf iotype}~{\bf output}~~{\bf \verb&:&}~{\bf \verb&outUnitType&}{\bf \verb&;&}\\
~~~~~~~~~~~~{\small \{\verb{Zelltyp fuer Ausgabezellen mit zwei Sites.{\}}\\

           \end{tabbing}
      \end{minipage}
    }
\end{center}

Dadurch, da"s der Parameter {\bf sites}\index{sites} in der Definition
von ``inUnitType'' fehlt, wird festgelegt, da"s eine Zelle dieses Typs
keine Sites hat. Dagegen hat jede Zelle vom Typ ``outUnitType'' die
beiden Sites ``inSiteType'' und ``exSiteType''.

Diejenigen Parameter\index{Parameter!einer Zelle!variable} eines
Zelltyps, deren Werte nicht in der Typdefinition definiert wurden,
k"onnen sp"ater f"ur Zellen dieses Typs nachdefiniert werden. Die
Parameterwerte, die in einem Zelltyp definiert werden, umfassen in
{\bf Nessus} oft nur gemeinsame Parameter vieler Zellen.  Man beachte,
da"s die Parameterwerte, die in der Typdefinition festgelegt wurden,
nicht mehr "uberschrieben werden d"urfen.


