% %W%	%G%

\part{\ENZO Implementation Guide} % ==================================================

%\noch{zus"atzliches init; msgv[1] == \&reference in EVOLUTION\_INIT}

%\noch{Kap. zsf. "anderungen am SNNS}

%\noch{Kap. zsf. "anderungen kernel\_shell: sh\_netinfo, sh\_no\_pattern etc }

%\noch{"anderung der BLOCK-gr"o"sen im SNNS zwecks Speicherersparnis (kr\_def.h)}

%\noch{ein jeder entwickler mu"s das benutzerhandbuch kennen}

%\\noch{man darf sich nicht darauf verlassen, da"s das gerade aktive Netz
 %     in anderen Modulen nicht ge"andert wird!}

%\\noch{es gibt eine stop-fkt, die abbricht, wenn mit *parents/*offsprings
%      was nicht stimmt}

\section{Design}
The kernel of \ENZO\ has a simple structure:

\begin{verbatim}
    init-modules

    read-control-file

    {pre-evolution}
    
    {evaluation}
    {history}
    {survival-of-the-fittest}
    
    while not {stop-evolution} do
    begin
        {selection}
        {crossover}
        {mutation}
        {optimization}
        {evaluation}
        {history}
        {survival-of-the-fittest}
    end
    {post-evolution}

    exit-modules
\end{verbatim}

The names in brackets correspond to the evolutionary operators,
which consist of several modules.
All modules of one operator type are linked to a library, e.g.,
each operator has a corresponding library.
This allows easy extension, and since all functionality is implemented
in the libraries, one can use the skeleton implementation of
the algorithm to optimize other things than neural networks by exchanging the
libraries.
The flexible design also allows for a good parallelization, if necessary.

\ENZO\ was implemented in ANSI-C to allow  portability to
other systems. On the other hand it would be very appealing to implement
a genetic algorithm (as well as neural network simulators)
in a object oriented language like C++.
To achieve some of the flexibility of an object-oriented design,
all modules have a standardized interface, and  their functions are referred through
function pointer, stored in a module table.


\subsection{Extension of \ENZO by own modules}

For extension exists a sample module, which should allow you
to integrate an own module specialized for your purposes.
Just try it ...

To add an own module, the following things have to be done:
\begin{enumerate}
  \item Design of the interface functions.
  \item Implementation of the module.
  \item Test the module extensively!
  \item Add the new module in the global module table, and change
        the makefile such that your code is compiled and linked to the program.
\item	Don't forget the documentation of key words and functionality of the module.
\end{enumerate}



\subsection{Dependencies between modules \label{abh} ?}

\centerline{\em Ideally, there should be no dependencies !}

Some kind of dependencies between certain operators seem still
suggestive, e.g. during pre-evolution one might also want to
perform a local optimization. The corresponding module is part
of the optimization library. By sharing the keywords\footnote{This is similar to
a message dispatcher in a window-based environment}
the pre-evolution module
can find the optimization function through the module table
and call it via the module interface.
To provide this functionality a module should indicate if
it wants to use a command line exclusively, i.e., it should be removed
and not further dispatched.



\subsection{Sharing data}

The nepomuk library provides a user data field additionally to the
network information, e.g., the structure \verb+networkData+ is used
to store the data of the evolution process  with the individuals.
This is necessary to append the fitness to the network, but it could also
be used to optimize the parameters of the evolution !
One should be careful with this global-like data. Before adding a new field
to this structure, alternatives that do not need global data exchange
should be evaluated.



\section{Interfaces}

There are three levels of interfaces to distinguish:
the standardized module interfaces, the nepomuk interface and
the interface to the neural network simulator.
They are discussed in the following sections.

\subsection{Module interfaces \label{opschnitt}}

Every module has to implement the following functions,which are
made public through the module table.
Also has each module to define a keyword,
which activates the module during initialization.
The syntax is simply: \verb+keyword+ .\\

\begin{description}
  \item[\tt int module\_init( ModuleTableEntry *self, int msgc, char *msgv{[]} )]%
    ~\\
    This function handles the initialization of the module.	\\
    \verb+message = "initialize" :+  for general  initialization;\\
    \verb+message = "exit" :+       for general exit;\\


    \verb+other keywords :+ specific  initialization.\\
    \verb+self+ the entry in the global module table.\\

     \verb+return value:+\\
       $0:$ The message was not used.\\
       $1:$ The message was used.\\
       $< 0:$ A warning should be generated.\\
       $> 1:$ An  error message should be generated.\\

	


  \item[\tt %
        int module\_work( PopID *parents, PopID *offsprings, PopID *ref )]~\\
	This is the working horse of every module.\\
	These functions should not manipulate neither the  parents nor the
	reference networks, but only the offsprings.
	During selection, the concerning networks are copied to an offspring
	population, which can be manipulated.


  \item[\tt char *module\_errMsg( int err\_code )]~\\
     Returns a error message depending on   \verb+err_code+ .
\end{description}

The activation of modules is done with the function 
\verb+enzo_actModule()+ .
Furthermore the function  \verb+enzo_logprint( char *fmt, ... )+ allows
for output in the log file (as \verb+printf()+).



\subsection{The \nepomuk library} % --------------------------------------------------


All  \nepomuk interface functions have the prefix \verb+kpm_+,
shortcut for {\bf K}arlsruher {\bf P}opulations-{\bf M}anager;

Initialization and cleanup of \nepomuk:

\begin{description}

  \item[\tt kpm\_err kpm\_initialize( int max\_nets, int max\_pats )]~\\
	Initializes the internal data structures and fields of \nepomuk.
        The number of networks to manage is given by \verb+max_nets+
	and the number of pattern sets by \verb+max_pats+.

  \item[\tt kpm\_err kpm\_exit( void )]~\\
	For a clean  exit this function needs to be called. It frees allocated
	memory and resets \nepomuk.

\end{description}

Network management:
\begin{description}

  \item[\tt kpm\_err kpm\_setCurrentNet( NetID id )]~\\
    Selects the network with the given \verb+id+ for further manipulation.
	
  \item[\tt NetID   kpm\_getCurrentNet( void )]~\\
    Returns the \verb+NetID+ of the currently activated network.

  \item[\tt NetID   kpm\_loadNet( char *filename, void *usr\_data )]~\\
    Loads a new network from the file \verb+filename+ , makes it the active network
and returns its \verb+NetID+ . The given \verb+usr_data+ is appended to the structure.


  \item[\tt kpm\_err kpm\_saveNet( NetID id, char *filename, char *netname )]~\\
	Saves the network referred by \verb+id+ in the file \verb+filename+.

  \item[\tt kpm\_err kpm\_deleteNet( NetID id )]~\\
	Deletes the complete data structures of the network referred by \verb+id+,
	i.e., the internal representation as well as the  \verb+userData+.
        

  \item[\tt kpm\_sortNets( CmpFct netcmp )] ~\\
	Sort the networks in \nepomuk\ using the function \verb+netcmp+.
	The function works as \verb+strcmp+ and keeps the
        networks sorted. Is \verb+netcmp == NULL+ no sorting is done.

  \item[\tt NetID   kpm\_copyNet( NetID id, void *usr\_data )]~\\
	Creates a copy of the network, appends the given  \verb+usr_data+
 	and returns the \verb+NetID+.

  \item[\tt NetID   kpm\_newNet ( void *usr\_data )]~\\
	Creates a new (empty) network, appends the given  \verb+usr_data+
 	and returns the \verb+NetID+.
	
  \item[\tt void   *kpm\_getData( NetID id )]~\\
	Returns the \verb+usr_data+ of the active network.


  \item[\tt kpm\_err kpm\_getNetDescr( NetID id, NetDescr *n )]~\\
	Returns a description of the active network, e.g. number of units,
	weights, etc.

\end{description}

Management of subpopulations:

\begin{description}
  \item[\tt PopID kpm\_newPopID( void )]~\\
	Returns a new, unused population identification.

  \item[\tt kpm\_err kpm\_validPopId( PopID id )]~\\
	Checks, if the identification is valid. Returns \verb+KPM_NO_ERROR+
 	if valid and  \verb+KPM_INVALID_POPID+ otherwise.

  \item[\tt kpm\_err kpm\_setPopMember( NetID n\_id, PopID p\_id )]~\\
	Makes the network referred by \verb+n_id+ a member of the
	population referred by \verb+p_id+ . 

  \item[\tt PopID kpm\_getPopID( NetID id )]~\\
	Returns the identification of the population the network referred by
	\verb+n_id+  belongs to.

  \item[\tt NetID kpm\_popFirstMember( PopID p\_id )]~\\
	Returns the network identification of the first network in the
	population referred by \verb+p_id+.
	If no network exists, \verb+NULL+ is returned, otherwise the network
	is made the active network.
	The first network is the smallest network with respect to  function
	passed to \verb+kpm_init+ .

  \item[\tt NetID kpm\_popNextMember( PopID p\_id, NetID n\_id )]~\\
	Returns the network identification of the network direct after
	\verb+n_id+ in the population \verb+p_id+.
	If none exists, \verb+NULL+, else the network is mad the active network.
\end{description}

Pattern management:

\begin{description}
  \item[\tt PatID kpm\_loadPat( char *filename, void *usr\_data )]~\\
	Loads a new pattern set from the file \verb+filename+, appends the
	\verb+usr_data+ and returns its \verb+PatID+ .

  \item[\tt kpm\_err  kpm\_setCurrentPattern( PatID id )]~\\
	Sets the pattern set referred by \verb+id+.


  \item[\tt PatID kpm\_getCurrentPattern( void )]~\\
	Returns the  \verb+PatID+ of the current pattern set.

  \item[\tt void *kpm\_getPatData( PatID id )]~\\
	Returns the \verb+usr_data+ of the  pattern set referred by \verb+id+.

  \item[\tt PatID kpm\_getFirstPat( void )]~\\
	Returns the \verb+PatID+  of the first pattern set, or \verb+NULL+
	if none exists.


  \item[\tt PatID kpm\_getNextPat( PatID pat )]~\\
	Returns the \verb+PatID+  of the pattern set following the set referred by
	\verb+pat+ , or \verb+NULL+ if none exists.

  \item[\tt kpm\_err kpm\_setPatName( PatID id, char *name )]~\\
	A name can be assigned to pattern set for identification, e.g.,
    	"`learn"', "`test"', "`cross validation"', \dots .
	This functions copies the given string to store it with the pattern set.


  \item[\tt char *kpm\_getPatName( PatID id )]~\\
	Returns the name of the pattern set referred by \verb+id+ or \verb+NULL+
	if none exists.

\end{description}

\subsection{Interface to a neural network simulator} % --------------------------------


All  interface functions have the prefix \verb+ksh_+. They are located
in the file \verb+kr_shell.c+ which is linked to the nepomuk library.
There are about 100 interface functions for network manipulation request, e.g.,
to delete units, to add units, to perform learning, etc...

Every simulator that offers a large set of manipulation functions, as the SNNS simulator does,
could be used with \ENZO.
The functions of the interface do almost nothing except calling the
appropriate functions of the neural network simulator. 





\section{Implementation internals}

\subsection{\ENZO }

\ENZO is no stand alone program, but uses a neural network simulator for
network manipulation. It provides an interface to the simulator that should
allow for easy communication with most simulators, that provide an interface
for network manipulation. Figure \ref{design} shows the design and the interfaces of
\ENZO in connection with the SNNS simulator,
figure \ref{link} shows which libraries and objects are linked together to
the executable \verb+enzo+.

\begin{figure}[thb]
\centerline{\psfig{figure=../bilder/link.eps,width=15cm,height=12cm} }
\caption[Linking the executable ]{ \label{link}
{\small The figure shows all libraries and objects which are linked
together to form the executable \verb+enzo+. 
}}
\end{figure}

The interface functions of the simulator are called solely through \ENZO's interface
module (kr\_shell.c). These interface functions are, in turn, called from the modules
forming the evolutionary operators or from the population manager \nepomuk.
The handling of networks and patterns 
is done exclusively via the \nepomuk interface.
The main loop is linked with the other parts through function pointers.
It uses a module table to store all functions, and activates dependent on the command file
the specialized modules.
The function pointers forming an operator are  stored in a corresponding table. Note that
a pointer could appear here several times.

Summarizing, \ENZO consists of three main parts: the main loop,
the population manager library \nepomuk
and the specialized modules forming the evolutionary operators libraries.

\subsection{\nepomuk}

The population is managed with an array of constant size; the number of
population members is fixed at initialization.
Note that the functions for subpopulation management are designed for
rather small population (linear search), i.e., several hundred members
would be impossible. This would also lead to a memory problem.

For an efficient population management, free and used array elements are
stored in lists, i.e., requirements can be satisfied in O(1).
See also  figure \ref{nepstruct}.

\begin{figure}[ht]
\setlength{\unitlength}{0.0125in}%
\begin{picture}(426,373)(54,435)
\put(50,435){\epsffile{../bilder/pop_list.pstex}}
\put( 54,687){\makebox(0,0)[lb]{\raisebox{0pt}[0pt][0pt]{\tenrm Population}}}
\put( 75,717){\makebox(0,0)[lb]{\raisebox{0pt}[0pt][0pt]{\tenrm free-list}}}
\put( 66,801){\makebox(0,0)[lb]{\raisebox{0pt}[0pt][0pt]{\tenrm firstUsed}}}
\put(159,801){\makebox(0,0)[lb]{\raisebox{0pt}[0pt][0pt]{\tenrm firstFree}}}
\put( 75,657){\makebox(0,0)[lb]{\raisebox{0pt}[0pt][0pt]{\tenrm used-list}}}
\put(240,630){\makebox(0,0)[b]{\raisebox{0pt}[0pt][0pt]{\tenrm usedSucc/Pred}}}
\put(324,687){\makebox(0,0)[b]{\raisebox{0pt}[0pt][0pt]{\tenrm \dots}}}
\put(405,738){\makebox(0,0)[b]{\raisebox{0pt}[0pt][0pt]{\tenrm freeSucc}}}
\put(195,500){\parbox{7cm}{\small
\begin{itemize}
  \item int    used;      % TRUE if a valid net is stored  
  \item int    subPop;    % indicates the subpop membership
    
  \item memNet snnsNet;   % snns-specific Network-Information
    
  \item NetID  freeSucc;  % pointer to the successor in freeList
  \item NetID  usedSucc,  % dto. for the usedList
  \item        usedPred;  % and the predecessor in usedList

  \item void  *userData;  % this is what the user wants to store
\end{itemize}}}
\end{picture}
\caption{ \label{nepstruct} \sl Data structure used in \nepomuk. }
\end{figure}

\subsection{Necessary changes  in SNNS modules}

Two little changes had to be made to the original SNNS software.
Since we are working with a population of networks, the allocated memory
for units, etc... should be in a descent range. For that, the default block size
for memory allocation in the file \verb+kr_def.h+ were decreased.

Furthermore, the handling of several networks needs additional interface
functions to copy network information into the data structures of the SNNS kernel.
The functions located in the file \verb+enzo_kr_mem.c+ provide this functionality.
It's included in the SNNS file \verb+kr_mem.c+.
To activate this extensions you have to compile the SNNS kernel library
with the flag \verb+ -D__ENZO__+. 




