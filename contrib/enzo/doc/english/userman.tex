%  %W%	%G%


\part{\ENZO User Manual} % ==================================================


\section{General Introduction}

%********* Please insert now the abstract.                 ******************
\subsection*{Summary}
The construction of a neural network to a given problem specification
is a difficult optimization problem: Which topology (number of layers,
number of units per layer, connectivity of units) and which values for
the network coefficients (weights, threshold) gains the optimal
performance?  Our evolutionary network optimizing system (\ENZO) uses
the paradigm of evolution for optimizing the topology and the paradigm
of learning for optimizing the coefficients.  Particularly, \ENZO
evolves a population of networks by generating offsprings thru
mutating the topology of the parent network and by optimizing the
coefficients with our fast gradient descent algorithm RPROP.  For
measuring the performance (resp. the fitness) we can use different
criteria: learning error (error on learning set), generalization
capability (error on the test set), hardware complexity (number of
units and weights), runtime (number of layers), etc.  Using several
heuristics for speeding up the training time of the offsprings \ENZO
can efficiently optimize even large networks with 5 000 weights and 50
000 training patterns.  
 

\subsection{Introduction }                                        

The basic principles of evolution as a search heuristic may be
summarized as follows. The search points or candidate solutions are
interpreted as individuals. Since there is a population of
individuals, evolution is a multi-point search. The optimization
criterion has to be one-dimensional and is called the fitness of the
individual. Constraints can be embedded in the fitness function as
additional penalty terms. New candidate solutions, called offsprings,
are created using current members of the population, called parents.
The two most used operators are mutation and recombination
(crossover). Mutation means that the offspring has the same properties
as its single parent but small variations. Whereas recombination
(crossover) means that the offspring's properties are mixed from two
parents. The selection of the parents is randomly but biased,
preferring the fitter ones, i.e. fitter individuals produce more
offsprings. For each new inserted offspring another population member
has to be removed in order to maintain a constant population
size. This selection can be done randomly or according the fitness of
each member. Particularly, that means in the first case that the
expected lifetime is equal for all members, whereas in the second case
the fitter will live longer, i.e. the fittest may even survive for
ever.

The design of neural networks incorporates two optimization
problems. First of all the topology, i.e. number of hidden units and
their interconnection structure, and second the tuning of the net work
parameters i.e. weights. Therefore, we have to solve a mixed
optimization problem, discrete for the topology and continous for the
network parameters. The standard approach is to use the intuition of
the designer for defining the topology and to use a learning
algorithms (e.g. gradient descent) for adjusting the free parameter.

The published results for using evolutionary algorithms for the
parameter optimization instead of gradient descent like
backpropagation suggest, that this is only efficient, when gradient
descent is not possible (e.g. activation function of neurons is not
differentiable or the interconnection structure is not feed forward
but contains cycles) or unsuccessful for sparse topologies, since
gradient descent is much faster.

On the other hand, optimization of the topology incorporates
optimization of the parameters, since evaluating the fitness of a
topology means evaluation of the network behavior for which we need an
optimal instantiation of the according network parameters (training).
Meanwhile, there are some published approaches which use evolution for
the discrete topology optimization and use gradient descent just for
the fitness evaluation of the topology. Since training rsp. gradient
descent of a neural network is even for middle sized networks a time
consuming task, these investigations are limited to small
networks. In the following we describe an hybrid approach combining
evolution and gradient descent such that even large networks with over
5000 conne ctions and training sets with over 50 000 patterns (ca. 3
Mbyte) can be efficiently handled.



%###########################################################################

\subsection{\ENZO, - Our Evolutionary Approach}

%###########################################################################


Every heuristic for searching the global optimum of difficult
optimization problems has to handle the dilemma between exploration
and exploitation.  Priorizing the exploitation (as hill-climbing
strategies do) bears the danger of getting stuck in a poor local
optimum. On the other hand, full explorative search which guarantees
to find the global optimum, uses vast computing power.  Evolutionary
algorithms avoid getting stuck in a local optimum by parallelizing the
search using a population of search points (individuals) and by
stochastic search steps, i.e. stochastic selection of the parents and
stochastic generation of the offsprings (mutation, crossover). On the
other hand, this explorative search is biased towards exploitation by
biasing the selection of the parents preferring the fitter ones.

This approach has proven to be a very efficient tool for solving many
difficult combinatorial optimization problems \cite{goldberg89,reeves93,schwefel95}.
A big advantage of this approach is
its general applicability. There are only two problem dependent
issues: The representation of the candidate solutions as a string
(genstring = chromosome) and the computation of the fitness.  Even
though the choice of an adequate representation seems to be crucial
for the efficiency of the evolutionary algorithm, it is obvious that
in principle both conditions are fulfilled for every computable
optimization problem.

On the other hand, this problem independence neglects problem
dependent knowledge as e.g. gradient information. Therefore the pure
use of evolutionary algorithms may have only modest results in
comparison to other heuristics, which can exploit the additional
information. For the problem of optimizing feedforward neural networks
we can easily compute the gradient by backpropagation. Using a
gradient descent algorithm we can tremendously diminish the search
space by restricting the search to the set of local optima.

This hybrid approach uses two time scales. Each coarse step of the
evolutionary algorithm is intertwined with a period of fine steps for
the local optimization of the offspring. For this approach there seems
to be biological evidence, since at least for higher animals nature
uses the very same strategy: Before evaluating the fitness for mating,
the offsprings undergo a longer period of fine tuning called learning.
Since the evolutionary algorithm uses the fine tuning heuristic as a
subtask, we can call it a meta-heuristic. Obviously, this
meta-heuristic is at least as successful as the underlying fine tuning
heuristic, because the offsprings are optimized by that. Our
experimental investigations will show, that the results of this
meta-heuristic are not only as good but impressively superior to the
underlying heuristic. 

In the natural paradigm the genotype is an algorithmic description for
developing the phenotype, which seems not to be an invertible process,
i.e. it is not possible to use the improvements stemming from learning
(fine tuning) for improving the genotype as Lamarck erroneously
believed. In our application however, there is no difference between
genotype and phenotype, because the matrix of weights, which determines the
neural network, can be linearly noted and interpreted as a chromosome
(genstring). In this case Lamarck's idea is fruitful, because the whole
knowledge gained by learning in the fine tuning period can be
transferred to the offsprings (Lamarckism). 

The strengths of our approach stem mainly from this effect in two
ways: Firstly, since the topology of the offsprings is very similar to
the topology of the parents transferring the weights from the parents
to the offsprings diminishes impressively the learning time by 1-2
orders of magnitude (in comparison to learning from the scratch with
random starting weights).  This also implies, that we can generate 1-2
orders of magnitude more offsprings in the same computation time.
Secondly, the average fitness of these offsprings is much higher: the
fitness distribution for the training of a network topology with
random initial weights will be more or less Gaussian distributed with
a modest average fitness value whereas starting near the parental
weights (remind the topology of the offspring is similar but not the
same as that of its parents) will result in a network with a fitness
near the parental fitness (may be worse or better). That means,
whenever the parental fitness is well above the average fitness
(respectively its topology) then the same may be expected for its
offspring (in case using the parental weights). Moreover, our
experiments have shown for the highly evolved 'sparse' topologies that
with random starting weights the gradient descent heuristic did not
find an acceptable local optimum (solving the learning task), but only
by inheriting the parental knowledge and initializing the weights near
the parental weights.
\label{ENZO}
%###########################################################################

\begin{figure}[thb]
\centerline{\begin{minipage}[t]{7.5cm}
\epsfxsize=7.5cm
\epsfysize=5.cm
\epsffile{../bilder/ablauf.eps}
\caption{Evolution cycle of \ENZO}
\label{ablauf}
\end{minipage}}
\end{figure}

Summarizing, our algorithm briefly works as follows (cf. fig.
\ref{ablauf}). Taking into account the user's specifications \ENZO
generates a population of different networks with a given connection
density. Then the evolution cycle starts by selecting a parent,
preferring the ones with a high fitness ranking in the current
population and by generating an offspring as a mutated duplication of
this parent.  If crossover is chosen, one or few hidden features
(=hidden units) of a second parent may be randomly added. Each
offspring is trained by the best available efficient gradient descent
heuristic (RPROP, see \cite {riedmiller93}) using weight decay methods for better
generalization. By removing negligible weights, trained offsprings may
be pruned and then re-trained.  Being evaluated an offspring is
inserted into the sorted population according to its determined
fitness value, thereby removing the last population element. Fitness
values may incorporate any design criterion considered important for
the given problem domain.



%===========================================================================
\subsection {Mutation}
%===========================================================================

Besides of the widely used link mutation we also realized unit
mutation which is well suited to significantly change the network
topology, allowing the evolutionary algorithm to explore the search
space faster and more exhaustive. Our experiments show that unit
mutation efficiently and reliably directs the search towards small
architectures.

Within link mutation every connection can be changed
by chance. The probability for deleting a link should be correlated
with the probability, that this deletion doesn't decrease the
performance significantly.  For that, we prefer links with small
weights for deletion, whereas the probability of adding is equal for
all links:

{\em Weighted Link mutation}: For each link the probability to be
  deleted is $p_{del}*N_\sigma(0,w)$ where $p_{del}$ and $\sigma$ are
  set by the user, $N_\sigma(0,w)$ means a normal distributed value
  with mean $0$ and variance $\sigma$ und $w$ denotes the absolute
  weight value  of the considered link. The probability for
  adding a link is $p_{add}$ constantly.

We may call this soft pruning, since not only the weights below the
threshold $\sigma$ are pruned, but very small weights ($|w|<<\sigma$)
with probability about $p_{del}$, big weights ($|w|>>\sigma$) with
probability about 0 and a soft interpolation in between. By the factor $N_\sigma(0,w)$ we intent
to approximate the probability that the deletion of a link effects no
significant deterioration of the performance.  If we choose $p_{del}$
such that only a few links are deleted by each weight mutation we get
the following heuristic:

{\em Soft pruning}: Test a few weights for pruning preferring small
    weights.  Whenever this pruning effects no significant
    deterioration, this variant will survive and will be subject to
    more pruning, - else this offspring is classified as failure and
    therefore not inserted in the population.

Therefore it is not necessary to estimate the effects of the deletion
of a weight as is done by other heuristics (e.g.  optimal brain
damage, optimal brain surgeon)---just try and test it.

In contrast to link mutation, unit mutation has a strong impact on the
network's performance.  To improve our evolutionary algorithm we
developed two heuristics which support unit mutation: the
prefer-weak-units (PWU) strategy and the bypass algorithm.  The idea
behind the PWU-heuristic is to rank all hidden units of a network
according to their relative connectivity ($\frac{act.
connections}{max. connections}$) and to delete sparsely connected
units with a higher probability than strongly connected ones. This
strategy successfully complements other optimization techniques, like
soft pruning, implemented in \ENZO.

\begin{figure}[thb]
\centerline{\begin{minipage}[t]{7.5cm}
\epsfxsize=7.5cm
\epsfysize=5.0cm
\epsffile{../bilder/bypass_mot.epsf}
\caption{Bypass algorithm: a) original network b) after deletion of the middle unit
c) with added bypass connections}
\label{bypass}
\end{minipage}}
\end{figure}




The bypass algorithm is the second heuristic we realized.  Other than
adding a unit, deletion of a unit can result in a network which is not
able to learn the given training patterns. This can happen because
there are too few hidden units to store the information available in
the training set. But even if there are enough hidden units, deleting
a unit can destroy important data paths within the network
(fig. \ref{bypass}a and b).  For that reason we restore deleted data
paths before training by inserting bypass connections
(fig. \ref{bypass}c). By that, the nonlinear function computed by the
subpart of the neural network, which was connected to the deleted unit
formerly, is now approximated by a linear function using shortcuts
(bypass connections).  The application of the bypass algorithm
significantly increases the proportion of successful generated by unit
mutation.  Both the number of networks with devastated topologies
decreases and the generated nets need less epochs to learn the
training set, because they are not forced to simulate missing bypass
connections by readjusting the weights of the remaining hidden units.


%===========================================================================
\subsection{Benchmarks}
%===========================================================================

Some benchmark problems are distributed with \ENZO, three simple
benchmarks with only a few minutes computing time necessary
(TC-Problem,Encoder,XOR)
and two larger benchmarks (Spirals,Recogintion of digits). 
Some benchmarks are also described in the following.
Before using \ENZO for larger problems, it is worth investigating some
time in parameter tuning of benchmark problems to get an impression
on the influence of single parameters and the dependencies between parameters.



%---------------------------------------------------------------------------
\subsubsection{\bf TC problem}
%---------------------------------------------------------------------------
The task is to correctly classify Ts and Cs given a $4x4$ pixel input matrix
(figure \ref{tc}, see also \cite{McDonnell&Waagen93}). The pattern set contains
all possible Ts and Cs, that is they
can be translated and rotated. In total there are 17 Ts and 23 Cs.
A straightforward network uses the pixel representation as input units, has some
hidden units and one output unit, that classifies Ts with 1 and Cs with 0.
The topology of the network is than $16-16-1$ with full connection, i.e., 288 weights.
Obviously the input layer contains redundant information.
The task for the genetic algorithm is to eliminate redundant input units
and furthermore the topology of the network, without any loss of classification
performance.

\begin{figure}[htb]
\centerline{\psfig{figure=../bilder/tc.eps,width=7cm,height=9cm} }

\caption[The TC - Problem ]{\label{tc}
{\small The figure shows Ts and Cs represented with a
	$4x4$ pixel matrix. In the bottom row, several
	input units were cut off (gray color). The network
	is still able to distinguish every T from every C.
	Can you ?
}}
\end{figure}


 The neural network that originally
had nearly 300 links was already impressively reduced to a 10-2-1 net
with 27 links  left by
\ENZO and in a second approach to 8-2-1
with only 18 links.

\begin{table}[htb]
\centerline{\begin{minipage}[t]{6.5cm}
    \begin{tabular}{|l|c|c|} \hline
      Reference          & topology & \#links \\ \hline \hline
      Original net       & 16-16-1  & 288          \\ \hline
      McDonnell          & 15-7-1   &  60          \\ \hline
      \ENZO94		 & 10-2-1   &  27          \\ \hline
      \ENZO95        	 & 8-2-1    &  18          \\ \hline
    \end{tabular}
    \caption{\sl TC-Problem. Note the decreased number
             of input units due to input-unit-mutation.  \label{tab:tc}}
\end{minipage}}
\end{table}



\subsubsection{\bf Nine Men's Morris}


With \ENZO we investigated networks learning a control strategy for the
endgame of {\bf Nine Men's Morris}.  The table \ref{tab:muehle} shows
the performance of three nets: the first network was the best
hand-crafted network developed just by using backpropagation
(SOKRATES,\cite{braun91b}), a second network was generated by \ENZO
\cite{braun93} and a third network we got by \ENZO additionally rating
the network size in the fitness function \cite{braun94}. Networks
optimized by \ENZO show a significantly better performance than the
original Sokrates net. Further, that superior performance is achieved
with smaller nets.  \ENZO was able to minimize the network to a
14-2-2-1 architecture deleting not only hidden units but also the
majority of the input neurons.

\begin{table}[htb]
\centerline{\begin{minipage}[t]{9.5cm}
 \begin{tabular}{|l|c|r|c|} \hline
System    & topology      & \#weights & performormance\\ \hline \hline
Sokrates & 120-60-20-2-1 &   4222    & 0.826 \\ \hline
\ENZO-1     & 120-60-20-2-1 &   2533    & 1.218 \\ \hline
\ENZO-2  & {\bf 14-2-2-1}&{\bf 24}   & 1.040 \\ \hline
\end{tabular} \hfill 
 \caption{\sl Socrates was the best handcrafted network, the fitness
criteria for \ENZO-1 was performance and for \ENZO-2 additionally network size.
\label{tab:muehle}}
\end{minipage}}
\end{table}


\subsubsection{ Thyroid gland}

Thyroid gland diagnostic is a real-world benchmark we used to
test our algorithm [\ref{Schiffmann 93}]. This task requires a very
good classification, because 92\% of the patterns belong to one
class. So a useful network must classify much better than 92\%. A
further challenge is the large number of training patterns (nearly
3800) which exceeds the size of toy problem's training sets by
far. The evolved network had the same performance as in
[\ref{Schiffmann 93}], but 4 input units less and only 20\% of the
weights.
\begin{table}[htb]
\centerline{\begin{minipage}[t]{8.5cm}
 \begin{tabular}{|c|c|c|c|} \hline
            & topology & \#weights & performance\\ \hline \hline
  Schiffmann& 21-10-3  & 303       & 98.4\%  \\ \hline
  \ENZO   & 17-1-3   &  66       & 98.4\% \\ \hline
\end{tabular}\hfill 
\caption{\sl Thyroid gland diagnostic, - decreasing network size and 
removing redundant input units without deteriorating performance
\label{tab:thyroid}}
\end{minipage}}
\end{table}


%###########################################################
\subsubsection{ Classification of handwritten digits}
%##########################################################

{ Classification of handwritten digits} was the largest problem we
tackled with \ENZO \cite{schaefer95}. We compared the classification performance of our
evolved neural networks with that of a commercially used polynomial
classifier of degree two.  Trained on the same 50,000 pattern subset
of the NIST data base using the same features as the neural net the
polynomial classifier achieves a classification rate of 99.06\%\
correct.  The results in table \ref{tab:nets} shows that both
classification approaches perform equally well. The major difference
is in the number of free parameters: while a $2^{\mbox{\tiny nd}}$
degree polynomial classifier uses 8.610 coefficients our nets range
from less than 1,600 to 3,300 links. As a consequence the
classification time in practical application is reduced to 20\% of
time needed by a PC.  Another advantage is the possibility to obtain a
specialized net for a given time-accuracy tradeoff.  By means of the
fitness-function the user can support the evolution of either nets
with few links, risking a small drop in performance, or more powerful
nets with some links more.

\begin{table}[htb]
\centerline{\begin{minipage}[t]{7.5cm}
\begin{tabular}{|c|c|r@{ : }l|} \hline links & correct &
    \#weights & misclassified \\ \hline\hline 892 & 98.05 & 1 & 2 \\
    \hline 1,648 & 98.71 & 1 & 5 \\ \hline 3,295 & 99.16 & 1 & 1500 \\
    \hline \end{tabular} 
{\caption{\sl Classification of handwritten
    digits, - evolved networks for different ratings of network size
    (\#weights) versus performance (miscl.) in the fitness function}
    \label{tab:nets}}
\end{minipage}}
\end{table}



%###########################################################################

\subsection{Conclusion}

%###########################################################################


\ENZO combines two successful search techniques:
gradient descent for an efficient local weight optimization and
evolution for a global topology optimization.  By that it takes full
advantage of the efficiently computable gradient information without
being trapped by local minima. Through the knowledge transfer by
inheriting the parental weights both learning is speeded up by 1-2
orders of magnitude and the expected fitness of the offspring is far
above the average for its topology. Moreover, \ENZO impressively
thins out the topology by the cooperation of the discrete mutation
operator and the continuous weight decay method. For this the
knowledge transfer is again crucial, because the evolved topologies
are mostly too sparse to be trained with random initial weights.
Additionally, \ENZO tries also to cut off the connections to
eventually redundant input units: For the Nine Men's Morris problem
\ENZO found a network with better performance but only 12\% of the
input units originally used.  Therefore \ENZO not only supports the
user in the network design but also determines the salient
input components.





\section{Who should use \ENZO}

\subsection{History and purpose of \ENZO}

\ENZO was designed to optimize the topology of neural networks as well
as their performance. So far this version supports the optimization of
multilayer perceptrons. Elman networks, TDNNs and RBF networks are currently
under investigation. 


This version of \ENZO uses the Stuttgarter Neural Network Simulator (SNNSv4.1,
kernel and function library)
for manipulating neural networks. All simulators can be supported as long
as they offer a functional interface to manipulate networks.

\ENZO should be a powerfool tool for everybody who uses neural networks
and who is interested in faster, smaller and better networks. 
It is not necessary to have any knowledge about genetic algorithms,
but it makes the system easier to comprehend. See \cite{goldberg89,reeves93,schwefel95} for an
introduction.

The flexible design of \ENZO provides a tool that is usable for many tasks, when
dealing with neural networks. That is, instead of optimizing topologies, one can use
it as well as a batch program to train several networks just  by changing the
command file
in an appropriate way. Adding your own modules allows you to tailor the program
to your desire.

\subsection{Where to get \ENZO}

\ENZO is available via anonymous ftp at the same site as the
SNNS simulator. The host is\\
\\
\centerline {\verb+ftp.informatik.uni-stuttgart.de (129.69.211.2)+}
\\
in the directory\\
\\
\centerline {\verb+/pub/SNNS+.}\\
\\
Check there for further information (Readme.ENZO)
and the  file ENZO.tar.Z or ENZO.tar.gz!
Before extracting the tar-files note that there is no installation script
by now. You should have no problems if
\ENZO (resp. the tar-files) are  located on the same directory level
as SNNSv4.1.
Uncompress the tar-file and extract ENZO with\\
\\
\centerline {\verb+tar -xvf ENZO.tar+}
\\
in the current directory.

The directory ENZO contains a makefile to compile the program.
The subdirectory ENZO/src contains all sources.
The subdirectory ENZO/benchmarks contains some benchmarks.
See also section \ref{bench}.
The subdirectory ENZO/doc contains the documentation (with \LaTeX\  sources). 

\subsection{Mailing list}

There exists a mailing list for \ENZO. If you want to be added to the list,
send a message to\\
\\
\centerline{\verb+enzo-request@ira.uka.de+.}

\vspace*{0.5cm}

Post your messages, questions, comments etc. to\\
\\
\centerline{\verb+enzo@ira.uka.de+.}




\section{Design and Interface of \ENZO}

\begin{figure}[htb]
\psfig{figure=../bilder/system3.eps,width=15cm,height=8cm} 

\caption[Design and Interfaces of \ENZO\ ]{\label{design}
{\small The figure shows the main parts of
\ENZO\ and the interface
to the neural network simulator, e.g., the SNNS simulator. The Interface
contains about 100 functions to perform several network operations.}}
\end{figure}

The design of  \ENZO\ provides a great flexibility.  The specialized knowledge
of how to perform a certain evolution step is located in the modules (right lower corner
of the \ENZO block in figure \ref{design}. They are combinable like toy blocks
and easily extensible. A population manager takes care of handling the individuals
as well as the pattern sets. The neural network simulator is hidden behind a
functional interface. \ENZO also offers the possibility to use the network description
language \CuPit. If one is familiar with \CuPit and interested  in using it
with \ENZO please send an email to \verb+enzo-request@ira.uka.de+.
For more information about \CuPit see \verb+http://wwwipd.ira.uka.de/~hopp/cupit.html+.


\section{Installing and running \ENZO }

\subsection{Installation}

Unfortunately, for this first published version of \ENZO no installation
script exists. You don't have to change any makefiles as
long as \ENZO is located on the same directory level as
the SNNS simulator. (Expected name is SNNSv4.1).
If this is not the  case, use symbolic links or adapt the makefiles.

To install \ENZO do the following:
\begin{itemize}
\item[1.]{Make sure \ENZO is at the same directory level as SNNSv4.1}

\item[2.]{Type \verb+cd ENZO+ and than \verb+make+. That should compile all libraries
	as well as the executable \verb+enzo+. The executable is located in the directory
	\verb+ENZO+ }

\item[3.]{If you want to use the X-history window, type \verb+make xgraf+.
	You may need to adapt the library and include path in the makefile
	in \verb+ENZO/src/history/Xgraf+.}

\end{itemize}



\subsection{Running \ENZO }

\ENZO\  is run as a background (UNIX-) process. For small problems,
a simple X-Window visualization of the fitness function is usable.
The networks can be analyzed using the graphical user interface of SNNS.
In near future, some tools will be provided with each standard history
module, to visualize the results.

To run \ENZO\  one simply types:\\

	\centerline{\tt enzo \em cmd\_file [logfile [seed]]}

If no {\em log file} is given, the output is written to {\em stderr}.

\begin{figure}[htb]
\centerline{\psfig{figure=../bilder/algo.eps,width=12cm,height=7.2cm} }
{\small {\caption[Evolutionary algorithm ]{
\label{algo}
 The figure shows the main loop of \ENZO.
The evolutionary operators are called in the shown sequence.
}}}
\end{figure}


\ENZO\  starts by reading the {\em command file}. A sample command file is given in
chapter \ref{bspcmd}. Via the  command file modules
 can be activated through a key word
and their parameters can be set. 
The genetic operators are called sequentially as  shown in figure \ref{algo}.
Each operator  can consist of several  modules (or be empty).
The modules are combined by specifying their key words in the command file.
They are called in the sequence of the appearance of the key words.
Note that  one module can appear several times in this sequence.
Figure \ref{operator} illustrates the relationship between modules and operators.


\begin{figure}[htb]
\psfig{figure=../bilder/operator.eps,width=15cm,height=6cm} 
\caption[Relationship between modules and operators]{ \label{operator}
{\small {The figure shows the relationship between modules and operators.
The user can specify the key words of the modules in the command file,
which will activate the modules,e.g., the interpreter adds them to a module
list which forms the evolutionary operator.}
}}
\end{figure}




\subsection{The command file \label{steuerdatei}}

All possible key words are defined by the modules. For details see
the description of the modules in chapter \ref{modules}.
A dispatcher passes the key words and possible parameters to
all modules. Each module picks the information it is interested in
and performs necessary actions.
The sequence of key words is only  important in the way
that the functionality of the resulting operator depends on
the order of the keywords, e.g., the optimization operator
in figure \ref{operator} has another functionality if prune would
be called before learnSNNS.

Still it is good style to keep certain entries in different parts: 

\begin{description}

\item[1. Files:]
	You should specify the file names of the networks and patterns in this part.
        Also the prefix for output files should be given.
	This has the advantage that one sees immediately, what task is optimized
     	and which files are involved.

\item[2. Modules:]
	You should specify which modules form the evolutionary operators.
	Every module defines a key word for its activation.
	The key words should be in the typical order, e.g.,
	pre-evolution before selection before crossover etc.
        This part says which modules are to use.
  
\item[3. Parameters:]
	All parameters of all used modules should be set here.
	The order of key words should be the same as for modules.
	If a parameter is set several times the last appearance is used.
	This part  decides how the evolution is done in detail.

\end{description}

A sample command file is shown in chapter \ref{bspcmd}.


\section{Module description\label{modules}}

The following sections describe the modules which are currently available
for \ENZO. Each section corresponds to an operator, each subsection corresponds
to a module. Firstly the key word of the module is given, followed by the description
of its parameters.
Optional parameters are given in brackets.
All modules have sensible default values for their parameters.
Some notes on important parameters can be found in chapter \ref{parameter}.
Each section is closed by a functional description of the module and a sketch
of the algorithm, if necessary.


\subsection{Pre-evolution}       % %W%	%G%
%
%
%
%
%
%
%
%
%
%
%

\begin{moduledoc}{Create an initial population}{initPop}
  \item[\KeyWord{gensize} \optParam{ x } ]~\\
    This parameter sets the maximal number of networks in the parent population.\\
    Default: {\tt POP\_SIZE\_VALUE} (30)
  \item[\KeyWord{popsize} \optParam{ x } ]~\\
    This parameter sets the number of  offsprings to create each generation.\\
    Default: {\tt OFF\_SIZE\_VALUE} (10)
  
  \item[\KeyWord{network} \optParam{ x } ]~\\
    This string contains the filename of the reference net.
    Each created net in the population gets the same topology structure
    as the reference net.\\
    Default:  enzo.net
  
  \item[\KeyWord{initFct} \optParam{ x } ]~\\
    This string contains the name of the SNNS init-function.
    The starting values of the weights and biases will be set by this function.\\
    Default: Randomize\_Weights
  \item[\KeyWord{initParam} \optParam{ x } ]~\\
    These 5 parameters contains the parameters for the init-function.
    For the meaning of these parameters please see the SNNS manual.\\
    Default: -1.0 1.0 0.0 0.0 0.0
\end{moduledoc}

The module {\it initPop} loads the reference net (via SNNS) and copies this
net to all members of the parent population. After that all networks of the population
are initialized with the SNNS initial function.

\algo{9cm}{initPop}
{
Load the reference net;\\
Set the names of all units in the reference net;\\
{\bf forall} (members of the starting population) {\bf do}\\
\hspace*{0.5cm}Copy the reference net to the new net;\\
\hspace*{0.5cm}Initialize the net with {\it initFct} and {\it initParam};\\
\hspace*{0.5cm}Set the {\it initFct} of the net to {\it ENZO\_noinit};\\
}





        
\begin{moduledoc}{Load a starting population}{loadPop}
  \item[\KeyWord{network} \optParam{ x } ]~\\
    This string contains the prefix of the filename, where the networks are  
    stored in.\\ 
    Default: {\tt enzo} 
  \item[\KeyWord{popsize} \optParam{ x } ]~\\
    This parameter $x$ sets the number of networks for the starting population.\\
    Default: {\tt POP\_SIZE\_VALUE} (30)
\end{moduledoc}

The module {\it loadPop} loads networks for the starting population. 
The filename for the reference net consists of the prefix {\it network} and 
the extension {\tt \_ref.net}.
The filename for the other networks consists of the prefix {\it network} 
and an extension containing a number, e.g., \verb+network_0000.net+.
It is important that all  hidden units of the reference net and the
other networks of the starting population get the same unit name in SNNS.
The purpose of this module was to restart an evolution from a stopped
process. The postfix of network names are just in the way \ENZO stored them.
        
\begin{moduledoc}{Creating a population using the nepomuk library}{genpopNepo}
  \item[\KeyWord{popsize} \optParam{ x } ]~\\
    This parameter sets the maximal number of networks in the parent population.\\
    Default: {\tt POP\_SIZE\_VALUE} (30)
  \item[\KeyWord{gensize} \optParam{ x } ]~\\
    This parameter sets the number of  offsprings to create each generation.\\
    Default: {\tt OFF\_SIZE\_VALUE} (10)
\end{moduledoc}

The module creates a population of networks using the nepomuk library.
Three parts of the population are distinguished: the reference net, the parent
population and the offspring population.
Note that only memory for {\it popsize} + {\it gensize} + 1 networks is allocated,
but no networks are created at that time.


  
\begin{moduledoc}{Load standard SNNS pattern sets}
                 {loadSNNSPat}
  \item[\KeyWord{learnpattern} \optParam{ x } ]~\\
    This string contains the filename for the learning patterns.\\ 
    Default: - 
  \item[\KeyWord{testpattern} \optParam{ x } ]~\\
    This string contains the filename for the test patterns.
    The test patterns are used to determine the fitness of the networks for the
    genetic algorithm.\\ 
    Default: -
  \item[\KeyWord{crosspattern} \optParam{ x } ]~\\
    This string contains the filename for the validation patterns.
    The validation patterns are used to determine the efficiency of networks.
    These patterns should not be used in the learning phase or to determine
    the fitness of the networks.\\ 
    Default: -
\end{moduledoc}
The module {\it loadSNNSPat} loads the three pattern sets with the
original SNNS-function.
The filename must contain the extension {\tt .pat} for the 
SNNS pattern files.
The number of pattern sets to be managed is restricted to 3.
To use more sets you have to increase the maximum number defined in
the population manager.


    
\begin{moduledoc}{Learning during the pre-evolution}{initTrain}
  \item[\KeyWord{initLearnfct} \optParam{ x } ]~\\
    This string contains the name of the SNNS learning function.\\
    Default: Rprop
  \item[\KeyWord{initLearnparam} \optParam{ x } ]~\\
    These 5 parameters contains the values of the parameters for the 
    learning function.
    For further informations see the SNNS manual.
    Default: 0.0 0.0 0.0 0.0 0.0
  \item[\KeyWord{initMaxepochs} \optParam{ x } ]~\\
    This parameter $x$ contains the maximum number of periods for the learning 
    algorithm. After this maximum the module {\it initTrain} will automatically stop the 
    learning function.\\
    Default: 50
  \item[\KeyWord{initMaxtss} \optParam{ x } ]~\\
    This parameter indicates the maximal tolerable learning error.
    The error is normalized by the number of learning patterns and the number of output units.
    The module {\it initTrain} will terminate the learning function if the learning error
    is less than this threshold.\\
    Default: 0.5
  \item[\KeyWord{initShuffle} \optParam{ x } ]~\\
    This switch indicates whether the sequence of the learning patterns is changed after
    each learning period or not.
    If the switch is turned on, then the module {\it initTrain} will use the
    SNNS function {\it shuffle}\\
    Default: yes
\end{moduledoc}

The module {\it initTrain} is an alternative version to the standard learning 
module {\it learnSNNS}. It is possible to use different learning functions and 
parameters in the pre-evolution phase than in the optimization phase.
In the sense of lamarckism, where offsprings get the strength of their 
weights directly from their parents, it is useful to have this opportunity.
This leads to less learning epochs for offsprings in comparison to networks
of the starting population, i.e., randomly initialized networks.

\algo{11cm}{initTrain}{
{\bf for} (each net) {\bf do}\\
\taba Set the actual SNNS-learning function {\it initLearnFct};\\
\taba {\bf while} (Epochs $<$ {\it initMaxEpochs}) {\bf and} (tss $>$ {\it initMaxTss})
{\bf do}\\
\tabb Learn one epoch with {\it initLearnParam};\\
}

      
\begin{moduledoc}{Random selection of input units}{inputInit}
  \item[\KeyWord{minNoInput} \optParam{ x } ]~\\
    This parameter sets the lower bound of the number of active input units.\\
    Default: 1 
  \item[\KeyWord{maxNoInput} \optParam{ x } ]~\\
    This parameter sets the upper bound of the number of active input units.\\
    Default: {\it Number of all input units}
\end{moduledoc}

The module {\it inputInit} randomly selects between {\it minNoInput} and
{\it maxNoInput} input units for the net of the starting generation.
The other input units  deactivated in the network structure by deleting 
all incoming and outgoing weights of this units.
The purpose of this module is to increase the diversity of the starting population
by creating different input layer topologies.
      
\begin{moduledoc}{Look for the optimal number of hidden units.}{optInitPop}
  \item[\KeyWord{maxtss} \optParam{ x } ]~\\
    This parameter sets the value for the stop criterion of the learning module.
    In this module it is used to decide if a net can learn the patterns or not.\\
    Default: 0.5
  \item[\KeyWord{learnModul} \optParam{ x } ]~\\
    This string sets the name of the module, which
    contains the learning function during the optimization phase.\\
    Default: {\tt learnSNNS}
 \end{moduledoc}
The module {\it optInitPop} tries to find the minimum number of hidden units,
which are required to learn the given problem properly.
This number of hidden units is computed with a binary search.
When this minimum topology is found, the rest of the networks will be randomly 
created with a number of hidden units between the found number and the maximum number 
of hidden units.
The purpose of this module is to reduce the network size of the parent networks
to a sensible size to speed up the evolution process.


\algo{13cm}{optInitPop}{
Load the reference net;\\
Name the hidden units in the reference net;\\
{\it lowerBound} = 0;\\
{\it upperBound} = \# hidden units;\\
{\it learned}= 0;\\
{\bf repeat} \\
\hspace*{0.5cm}{\it learned}++;\\
\hspace*{0.5cm}{\it hiddenUnits} = ({\it lowerBound} + {\it upperBound}) 
               {\bf div} 2;\\
\hspace*{0.5cm}copy the reference net;\\
\hspace*{0.5cm}delete (\#hidden units in reference net  - {\it hiddenUnits}) 
               from the offspring;\\
\hspace*{0.5cm}train the offspring with {\it learnModul};\\
\hspace*{0.5cm}{\bf if} (offspring is trained well) {\bf then}\\
\hspace*{1.0cm}{\it upperBound} = {\it hiddenUnits};\\
\hspace*{0.5cm}{\bf else}\\
\hspace*{1.0cm}{\it lowerBound} = {\it hiddenUnits} + 1;\\
{\bf until} ({\it upperBound} $\le$ {\it lowerBound}) {\bf or}
({\it learned} = {\it popsize})\\

{\bf for} (i = {\it learned} + 1 {\bf to} {\it popsize}) {\bf do}\\
\hspace*{0.5cm}copy the reference net;\\
\hspace*{0.5cm}{\it hiddenUnits} = Rand({\it lowerBound}, \#hidden units);\\
\hspace*{0.5cm}delete (\#hidden units - {\it hiddenUnits}) from the offspring;\\
}


    
\begin{moduledoc}{Create a population of networks from one special network}
       {startPop}
  \item[\KeyWord{popsize} \optParam{ x } ]~\\
    This parameter sets the number of elements in the population.\\ 
    Default: {\tt POP\_SIZE\_VALUE} (30)
  \item[\KeyWord{network} \optParam{ x } ]~\\
    This string contains the filename of the reference network.\\
    Default:  {}              
  \item[\KeyWord{startnet} \optParam{ x } ]~\\
    This string contains the filename of the master network, which
    structure is copied to all the other networks.\\
    Default: {}
  \item[\KeyWord{initFct} \optParam{ x } ]~\\
    This string contains the name of the SNNS init-function.
    This function is used to initialized all created networks.\\
    Default: {\tt ENZO\_noinit} 
  \item[\KeyWord{initParam} \optParam{ x } ]~\\
    This five parameters contains the values for the function parameters of the 
    SNNS init-function {\it initFct}. 
    For the meaning of the parameters please see the SNNS manual.\\
    Default: -1.0 1.0 1.0 0.0 0.0
\end{moduledoc}

The module {\it startPop} loads the reference network and a special master network.
The topology of the master network is copied to all the other networks in the start-population.
Afterwards all weights and biases of the networks will be initialized with the
{\it initFct}. 
The idea is to take a good network to initialize the genetic search.
Another possibility is to overcome the limitation of the maximal topology by
using a much bigger reference network than master network, with the master network
maybe untrained. If the master network is already locally optimized, one should use
the initialization function {\it ENZO\_noinit}.  

\algo{11cm}{startPop}{
Load the reference and the master net;\\
{\bf forall} (Elements in the start-population) {\bf do}\\
\hspace*{0.5cm}Copy the master network to the network;\\
\hspace*{0.5cm}Initialize the networks randomly with {\it initFct} and {\it initParam};\\
\hspace*{0.5cm}Set {\it initFct} to {\it ENZO\_noinit};
}





       
\begin{moduledoc}{Random selection of weights}{weightInit}
  \item[\KeyWord{weightProb} \optParam{ x } ]~\\
    This parameter sets the probability $p_{exists}$. Each weight will
    be deleted probability $1 - p_{exists}$.
    Default: 1.0
\end{moduledoc}
The module {\it weightInit} deletes weights from a net of the start-population randomly. 
This could by necessary to increase the diversity of the population or to reduce the
free dimensions of the network.

\algo{7cm}{weightInit}{
{\bf forall} (Weights in the net) {\bf do}\\
\hspace*{0.5cm}{\bf if} (Rand(0,1) $> p_{exists}$ ) {\bf then}\\
\hspace*{1.0cm}delete the weight in the net;\\
}
     



\subsection{Stopping condition}  % %W%	%G%
%
%
%
%
%
%
%
%
%
%
%



\begin{moduledoc}{Normal stopping }{stopIt}

\item[\KeyWord{maxGenerations} \optParam{cnt}]~\\
  Stops the evolution after \Param{cnt} generations;
  Is \Param{cnt} not specified, after the first generation.

\end{moduledoc}

   


\begin{moduledoc}{Stopping by error }{stopErr}
  \item[\KeyWord{no parameter}]~
\end{moduledoc}

Stops the evolution, if parents or offsprings are not valid networks.
This usually  happens, if the initialization is missing.


  

\subsection{Selection}           % %W%	%G%
%
%
%
%
%
%
%
%
%
%
%



\begin{moduledoc}{Uniform selection}{unifSel}
  
\item[\KeyWord{NoOfOffsprings} \Param{k}]~\\
   Sets the number of new offspring each generation to \Param{k}.

\item[\KeyWord{selProb} \Param{$p_{sel}$}]~\\
  Sets the probability of selection to 
  \Param{$p_{sel}$}.
  
\end{moduledoc}

The selection probability is \verb+prob+. 
   
\begin{moduledoc}{Selection of  parents preferring the better networks}
                 {preferSel}
  \item[\KeyWord{gensize} \optParam{ gen } ]~\\
  This parameter sets the number of networks in the offspring population.\\
   Default: 10
  \item[\KeyWord{preferfactor} \optParam{ x } ]~\\
  This parameter $x$ sets the bias for selecting fitter networks.
  A value $x \ge 1$ means that the better networks will be preferred,   
  a value $0 \le x\le 1$ means that the poor networks.
  A value $x = 2$ means that the first quarter of the population is as
  often selected as the rest of the population.
  A value $x = 3$ means that the first 12.5\% of the population is as
  often selected as the rest of the population.

    Default: 3.0 
\end{moduledoc}

This  module selects the given number of  parents for the reproduction process
(mutation and crossover).
Better networks are selected with a higher probability. 

\algo{10cm}{preferSel}{
n = number of networks in the parent population;\\
p = preferfactor;\\
$parentNo = (rand(0,1)^{p}) * n$\\
Select the network with  number {\it parentNo};
}
    

\subsection{Mutation}            % %W%	%G%
%
%
%
%
%
%
%
%
%
%
%

\begin{moduledoc}{A simple weight mutation}{simpleMut}

\item[\KeyWord{probadd} \optParam{$p_{add}$}]~\\
This parameter $x$ indicates the probability $p_{add}$ for inserting a
non existing weight.\\
Default: 0.0

\item[\KeyWord{probdel} \optParam{$p_{del}$}]~\\
This parameter $x$ indicates the probability $p_{del}$ for deleting an
existing weight.\\
Default: 0.0

\item[\KeyWord{initRange} \optParam{f}]~\\
The parameter $x$ determines the interval  [-range, range] where
inserted weights are randomly selected from.\\
Default: 0.5;

\end{moduledoc}

The module {\it simpleMut} executes a simple kind of weight mutation. 
Each existing weights of the offspring will be deleted with the probability
$p_{del}$, each weight, that exists in the reference net and not in the offspring 
will be inserted with the probability $p_{add}$.
The inserted weight will be created when both units, the input and the output unit 
exists.
Note that neglecting the selection there exists a equilibrium state of
adding and deleting weights, i.e.,
the number of weights $n$ in the network, that depends
on the values of $p_{add}$ and $p_{del}$:  $n = \frac{p_{add}}{p_{add}*p_{del}} $.




\algo{12cm}{simpleMut}{
{{\bf for all} (Weights of the reference net) {\bf do} \\
\hspace*{0.5cm}Search the appropriate weight in the offspring; \\
\hspace*{0.5cm}{\bf if} ( the weight exists in the offspring ) {\bf then} \\
\hspace*{1.0cm}{\bf if} {( RAND(0,1) $< p_{del}$ )} {\bf then}\\
\hspace*{1.5cm}delete the weight in the offspring;\\
\hspace*{0.5cm}{\bf else if} ( the weight doesn't exist in the offspring) {\bf then}\\
\hspace*{1.0cm}{\bf if} ( both units, start and end-unit exist in the offspring) {\bf then}\\
\hspace*{1.5cm}{\bf if} ( RAND(0,1)  $ < p_{add}$ ) {\bf then}\\
\hspace*{2.0cm}insert weight in the offspring;
}
}









  
\begin{moduledoc}{An other weight mutation}{mutLinks}

\item[\KeyWord{probadd} \optParam{$p_{add}$}]~\\
This parameter $x$ indicates the probability $p_{add}$ for inserting a
non existing weight.\\
Default: 0.0

\item[\KeyWord{probdelStart} \optParam{$p_{delstart}$}]~\\
This parameter $x$ indicates the starting probability $p_{delstart}$ for deleting an
existing weight. All probabilities between the $p_{delstart}$ (first generation)
and $p_{del}$ will be linear interpolated.\\
Default: 0.0

\item[\KeyWord{probdel} \optParam{$p_{del}$}]~\\
This parameter $x$ indicates the height  of the gaussian distribution for deleting an
existing weight (See the algorithm {\it mutLinks}).\\
Default: 0.0

\item[\KeyWord{sigmadel} \optParam{$x$}]~\\
This parameter $x$ indicates the width  of the gaussian distribution for deleting an
existing weight (See the algorithm {\it mutLinks}).\\
Default: 1.0


\item[\KeyWord{probdelEndGen} \optParam{$i$}]~\\
This parameter $i$ indicates the generation in which the linear interpolation of the 
probability $p_{del}$ for deleting an existing weight ends.\\
Default: 0

\item[\KeyWord{initRange} \optParam{f}]~\\
The parameter $x$ determines the interval  [-range, range] where
inserted weights are randomly selected from.\\
Default: 0.5;

\end{moduledoc}

The module {\it mutLinks} executes a mixture of a simple weight mutation 
and pruning, called soft pruning. 
Each weight that exists in the reference net and not in the offspring net
will be inserted with the probability $p_{add}$.
The inserted weight will be created when both units, the input and the output unit 
exists.
Each existing weight will be deleted by an gaussian distribution on the strength of the 
weight. 
This means that the probability to be deleted is for a small weight is greater than
a bigger one.

\algo{10cm}{mutLinks}{
{\bf for all} Weights of the reference net {\bf do} \\
\hspace*{0.5cm}Search the appropriate weight in the offspring \\
\hspace*{0.5cm}{\bf if} ( the weight exists in the offspring ) {\bf then} \\
\hspace*{1.0cm}{\bf if} ( RAND(0,1) $< p_{del}^t e^{\frac{-weight^2}{sigmadel}}$ ) {\bf then}\\
\hspace*{1.5cm}delete the weight in the offspring\\
\hspace*{0.5cm}{\bf else if} ( the weight doesn't exist in the offspring) {\bf then}\\
\hspace*{1.0cm}{\bf if} ( start and end-unit exist in the offspring) {\bf then}\\
\hspace*{1.5cm}{\bf if} ( RAND(0,1) $< p_{add}$ ) {\bf then}\\
\hspace*{2.0cm}insert weight in the offspring
}
   
\begin{moduledoc}{Mutation of hidden neurons}{mutUnits}
  \item[\KeyWord{probMutUnits} \optParam{ x } ]~\\
    The parameter $x$ indicates the probability $p_{mut}$ a mutation takes 
    place.\\
    Default: 0.5
  \item[\KeyWord{probMutUnitsSplit} \optParam{ x } ]~\\
    The parameter $x$ describes the relationship between inserting and deleting
    hidden units.
    Default: 0.5
  \item[\KeyWord{PWU} \optParam{ x } ]~\\
    This switch activates the  {\it Prefer Weak Units} strategy in the case of deleting 
    hidden units.\\
    Default: yes
  \item[\KeyWord{bypass} \optParam{ x } ]~\\
    This switch turns on the {\it bypass} function while deleting a hidden unit.\\
    Default: yes
  \item[\KeyWord{initRange} \optParam{ x } ]~\\
    This parameter $x$ describes the interval [-initRange,initRange] where 
    inserted weights are randomly selected from.
    Default: 0.5
\end{moduledoc}

The module {\it mutUnits} executes a mutation of the hidden units.
The maximum number of deleted or inserted hidden units is limited by one.
All other mutation and optimization modules can only delete units, they can
never insert weights to a deleted hidden unit.
In this case of deleting an activated hidden unit, all weights will be deleted.
In the case of inserting a hidden unit, all possible weights will be inserted.
The number of weights and hidden units which can be inserted is limited by the 
topological structure of the reference net.
The modules uses {\it bypass}-function and the {\it Prefer Weak Unit} strategy.
The {\it bypass}-function is illustrated in figure \ref{bypass}. The idea is to
linearly approximate a non-linear relationship, i.e., approximate the function
computed by a hidden
unit by a linear function. This is done by using shortcut (direct) connections.
The  {\it Prefer Weak Unit} strategy is to check the connectivity of a unit, i.e.,
count its incoming and outgoing weights, and prefer those for deleting which are weaker
connected. This is called soft unit pruning. 

\algo{12cm}{mutUnits}{
{{\bf if} ( RAND(0,1) $> p_{mut}$ ) {\bf then}\\
\hspace*{0.5cm}{\bf if} ( RAND(0,1) $> p_{split}$ ) {\bf then}\\
\hspace*{1.0cm}search a unit in the reference net which does not exist in offspring;\\
\hspace*{1.0cm}insert this unit in the offspring;\\ 
\hspace*{1.0cm}insert all possible weights of the unit;\\
\hspace*{0.5cm}{\bf else}\\
\hspace*{1.0cm}{\bf if} ( PWU set ) {\bf then}\\
\hspace*{1.5cm}select the {\it weakest} hidden unit of the offspring;\\
\hspace*{1.0cm}{\bf else }\\
\hspace*{1.5cm}select accidental a hidden unit of the offspring;\\
\hspace*{1.0cm}{\bf if} ( bypass activated ) {\bf then}\\
\hspace*{1.5cm}delete the selected unit with the bypass function;\\
\hspace*{1.0cm}{\bf else}\\
\hspace*{1.5cm}delete the unit and all its weights;\\
}
}



   
\begin{moduledoc}{Mutation of the input units}{mutInputs}
  \item[\KeyWord{probMutInputs} \optParam{ x } ]~\\
    The parameter $x$ indicates the probability $p_{mut}$ that  
    a mutation takes place.\\
    Default: 0.5
  \item[\KeyWord{probMutInputsSplit} \optParam{ x } ]~\\
    The parameter $x$ indicates 
    the relationship $p_{split}$ between inserting and deleting of 
    input units.\\
    Default: 0.5 
  \item[\KeyWord{initRange} \optParam{ x } ]~\\
    The parameter $x$ determines the interval  [-range, range] where
    inserted weights are randomly selected from.\\
    Default: 0.5
\end{moduledoc}
The module {\it mutInputs} executes a mutation only of the input units.
The maximum number of deleted or inserted input units is limited by one.
All other mutation and optimization modules can only delete units, they can
never insert weights to a dead input unit.

The internal structure of SNNS does not allow to delete the input units like
the hidden units, so the input units are just deactivated.
In this case instead of deleting an activated input unit, all weights will be deleted.
They are marked with the unit name \verb+xxx+. 
If you analyze the network with the graphical user interface of SNNS,
select in the display setup \verb+show name+ to easily identify removed input units.
In the case of inserting a deactivated input unit, all possible weights will be inserted.


\algo{10cm}{mutInputs}{
{\bf if} ( RAND(0,1) $> p_{mut}$ ) {\bf then}\\
\hspace*{0.5cm}{\bf if} ( RAND(0,1) $> p_{split}$ ) {\bf then}\\
\hspace*{1.0cm}insert all possible weights of the deactivated input unit\\
\hspace*{0.5cm}{\bf else}\\
\hspace*{1.0cm}delete all weights of the activated input unit\\
}


  

\subsection{Crossover}           % %W%	%G%
%
%
%
%
%
%
%
%
%
%
%


\begin{moduledoc}{Crossover of the connections between input- and output layer}{linCross}

\item[\KeyWord{probCross} \optParam{ x } ]~\\
    Probability of inserting a connection, that is contained in only one parent.
    Default: 0.5 
\end{moduledoc}


This module does a linear crossover for all weights, which connect
directly the input layer to the output layer, e.g., in nets without
hidden layers or with shortcut connections.
Only those connections are manipulated in the offspring net, no other
units or weights are involved.
If a network does not contain any of these connections, it remains
unchanged.

\algo{10cm}{linCross}{
delete all connections from all offsprings\\
{\bf forall} possible connections {\bf do}\\
\hspace*{0.5cm}{\bf if} ( connection in both parents ) {\bf then}\\
\hspace*{1.0cm}insert connection in offspring;\\
\hspace*{1.0cm}set weight to the mean value of the parents' weights\\
\hspace*{0.5cm}{\bf else if} ( connection only in one parent ){\bf then}\\
\hspace*{1.0cm}{\bf if} ( RAND(0,1) $ < p_{cross}$ ) {\bf then}\\
\hspace*{1.5cm}insert connection in offspring;\\
\hspace*{1.5cm}set weight to the value of the parent's weight;
}
  
\begin{moduledoc}{Implant a feature from the fittest  net in an offspring}{implant}

\item[\KeyWord{implantProb} \optParam{ x } ]~\\
    Probability of selecting a hidden unit of the first hidden layer
	from the best parent network
	and  implanting it in one offspring network.\\
    Default: 0.2 
\end{moduledoc}


This module selects a hidden unit from the first hidden layer
in the fittest parent network
and implants it in an offspring network. The hidden units are
marked with their name to prevent implantation of a feature
twice.

\algo{12cm}{implant}{
{\bf repeat}\\
\hspace*{0.5cm} get hidden unit of the first layer in the fittest network;\\
\hspace*{0.5cm}{\bf if} (hidden unit does not exist in offspring net) {\bf then}\\
\hspace*{1cm} implant (add) unit to offspring net;\\
\hspace*{1cm} {\bf forall} (connections in parent network) {\bf do}\\
\hspace*{1.5cm} {\bf if} (source unit does exist in offspring net) {\bf then}\\
\hspace*{2cm} insert connection with the same weight in offspring net;\\
{\bf until} (a hidden unit was implanted or all hidden units failed)
}
  

\subsection{Optimization}         % %W% %G%
%
%
%
%
%
%
%
%
%
%
%

\begin{moduledoc}{Learning stopped by periods or learning error}{learnSNNS}
  \item[\KeyWord{learnfct} \optParam{ x } ]~\\
    This parameter $x$  contains the name of the SNNS-learning function.\\
    Default: Rprop
  \item[\KeyWord{learnparam} \optParam{ x } ]~\\
    This array of parameters indicates the parameter for the SNNS learning function.
    The meaning of the parameters can differ from learning function to learning function.
    For details please see the SNNS manual.\\
    Default: 0.0 0.0 0.0 5.0 0.0
  \item[\KeyWord{maxepochs} \optParam{ x } ]~\\
    This parameter $x$ contains the maximum number of periods for the learning 
    algorithm. After this maximum the module {\it learnSNNS} will automatically stop the 
    learning function.\\
    Default: 50
  \item[\KeyWord{maxtss} \optParam{ x } ]~\\
    This parameter indicates the maximal tolerable learning error of the network.
    The error is normalized by the number of learning patterns and the number of output units.
    The module {\it learnSNNS} will terminate the learning function if the learning error
    is less then this threshold.\\
    Default: 0.5
  \item[\KeyWord{shuffle} \optParam{ x } ]~\\
    This switch indicates whether the sequence of the learning patterns is changed after
    each learning period or not.
    If the switch is turned on, then the module {\it learnSNNS} will use the
    SNNS-function {\it shuffle}\\
    Default: yes
\end{moduledoc}


The module {\it learnSNNS} is the standard-learn module during the optimization.
The module stops the learning of an offspring network if the learning error is below an 
upper bound or it reaches the maximum number of learning periods.


\algo{10cm}{learnSNNS}{
Set the SNNS learn function {\it learnFct};\\
{\bf for all} (networks of the offspring population) {\bf do}\\ 
\hspace*{0.5cm}{\bf while} (Epochs $<$ {\it maxEpochs}) {\bf and} 
               (tss $>$ {\it maxTss}) {\bf do}\\
\hspace*{1.0cm}Train one period with {\it learnParam};\\

}

 
\begin{moduledoc}{Learning stopped by periods or cross validation error}{learnCV}
  \item[\KeyWord{learnfct} \optParam{ x } ]~\\
    This parameter $x$  contains the name of the SNNS-learning function.\\
    Default: Rprop
  \item[\KeyWord{learnparam} \optParam{ x } ]~\\
    This array of parameters indicates the parameter for the SNNS learning function.
    The meaning of the parameters can differ from learning function to learning function.
    For details please see the SNNS manual.\\
    Default: 0.0 0.0 0.0 5.0 0.0
  \item[\KeyWord{maxepochs} \optParam{ x } ]~\\
    This parameter $x$ contains the maximum number of periods for the learning 
    algorithm. After this maximum the module {\it leranSNNS} will automatically stop the 
    learning function.\\
    Default: 50
  \item[\KeyWord{CVepochs} \optParam{ x } ]~\\
    This parameter $x$ sets after how often  the error on the cross validation set for
cross validation is computed.\\
    Default: 2
  \item[\KeyWord{shuffle} \optParam{ x } ]~\\
    This switch indicates whether the sequence of the learning patterns is changed after
    each learning period or not.
    If the switch is turned on, then the module {\it learnCV} will use the
    SNNS-function {\it shuffle}\\
    Default: yes
\end{moduledoc}


The module {\it learnCV} uses the error an a cross validation pattern set to stop learning.
If it increases again learning is stopped. 
The module stops the learning of an offspring network if the error on a cross vailidation set starts
increasing again  or it reaches the maximum number of learning periods. This is done
by computing the error each {\it CVepochs}. If the current error is larger than the average
of the last four values, learning is stopped.


\algo{12cm}{learnCV}{
Set the SNNS learn function {\it learnFct};\\
{\bf for all} (networks of the offspring population) {\bf do}\\ 
\hspace*{0.5cm}{\bf while} (Epochs $<$ {\it maxEpochs}) {\bf and} 
               ( $ tss(t) < \sum_{k=t-4}^{k=t-1} tss(k) $) {\bf do}\\
\hspace*{1.0cm}Train one period with {\it learnParam};\\

}

   

%    @(#)prune.tex      1.1  2/25/94

\begin{moduledoc}{Delete all weights below a threshold}{prune}
  
\item[\KeyWord{threshold} \optParam{t}]~\\
  The parameter \Param{t} indicates the threshold, underneath all weights of 
  the net will be deleted.\\
  Default: 0.0

\item[\KeyWord{thresholdStart} \optParam{f}]~\\
  The parameter $x$ indicates the start-threshold for the pruning module.\\
  Default: 0.0

\item[\KeyWord{pruneEndGen} \optParam{i}]~\\
  The parameter $i$ indicates th number of the generation until the 
  pruning module will take the origin {\it threshold}
  .\\
  Default: 0
\end{moduledoc}
  
All weights of the net with an abolute strength under the threshold will be deleted.

\vspace*{0.5cm}

% eof

     
\epsfxsize 8cm
\epsffile{../bilder/prune.eps}
\begin{moduledoc}{Adaptive pruning}{adapPrune}
  \item[\KeyWord{threshold} \optParam{ x } ]~\\
    This value gives the threshold used to initialize the parents.\\	
    Default: 0.0
  \item[\KeyWord{deltaThreshold} \optParam{ x } ]~\\
    This value gives the maximal factor the threshold of the offsprings
    is allowed to differ from the parents.\\	
    Default: 0.2
  \item[\KeyWord{aveThreshold} \optParam{ x } ]~\\
    For every network the mean value of the absolute value of the weights
    is computed and the threshold is individually set to this value times
    \Param{aveThreshold}.\\
    Default: 0.0
\end{moduledoc}
The module {\it adapPrune} is an adaptive variation of standard pruning.
Weights which are smaller than the threshold are deleted.
The threshold is set individually for every network, depending on the mean value
of its weights. If the factor {\it aveThreshold} is not set the value {\it threshold}
is used. The distribution of possible changes is realized with a Gaussian
distribution $g(x)$ and a maximal factor: \\
\\
\centerline{($\Delta_{threshold} =   g(x) *  $ {\it deltaThreshold}).}

 
\begin{moduledoc}{Relearning}{relearn}

\item[\KeyWord{relearnfactor} \optParam{f}]~\\
The parameter $x$ indicates the factor with which all weights and biases will
be multiplied with. \\
Default: 1.0 (no change of the weights and biases takes place.)


\end{moduledoc}
In the mind of {\it lamarckism} not only the topology of the parents will 
be transmitted to the offspring, but also the strength of connections inside the 
structure.
In the case of neural networks this will lead to a local minimum in the learning
function. To escape from this local minimum it is necessary to change the weights a little
bit. The module {\it relearn} multiplies  all weights and biases with a given factor.









   
\begin{moduledoc}{Adding random distributed values}{jogWeights}

\item[\KeyWord{jogLimit} \optParam{f}]~\\
The range of values added is given by $[-Param{f},Param{f}]$.
Default: 0.01


\end{moduledoc}
In the mind of {\it lamarckism} not only the topology of the parents will 
be transmitted to the offspring, but also the strength of connections inside the 
structure. To get out of the centre of a local minima 
the module {\it jogWeights} adds  random uniform distributed values to the
connection weights. This is an alternative to the {\it relearn}-module, that multiplies
all weights by a constant factor.









   
\begin{moduledoc}{Cleanup the structure of a net}{cleanup}

  \item[\KeyWord{no parameters}]~\\
    
\end{moduledoc}

The module {\it cleanup} deletes all units and weights which have no direct or
indirect connection to the input or output layer of the net.
   
\begin{moduledoc}{Delete offsprings without links}{nullWeg}

  \item[\KeyWord{no parameter}]~\\
    

\end{moduledoc}
The module {\it nullWeg} deletes all offsprings which contain
no connections. This could otherwise eventually lead to
misbehavior in other modules.

   
%\input{../../src/opt/reinit}    


%eof

\subsection{Evaluation}           %  %W%	%G%
%
%
%
%
%
%
%
%
%
%
%

\begin{moduledoc}{Evaluation of the topology}{topologyRating}

\item[\KeyWord{weightRating} \optParam{f}]~\\
The number of connections in the network is multiplied by this value and
divided through  the maximal number of connections ( in the reference network).
The result is added
to the fitness term.\\
Default: 0.0

\item[\KeyWord{unitRating} \optParam{u}]~\\  
The number of hidden units in the network is multiplied by this value and
divided through  the maximal number of hidden units ( in the reference network).
The result is added
to the fitness term.\\
Default: 0.0

\item[\KeyWord{inputRating} \optParam{k}]~\\ 
The number of input units in the network is multiplied by this value and
divided through  the maximal number of input units ( in the reference network).
The result is added
to the fitness term.\\
Default: 0.0

\end{moduledoc}



The module {\it topologyRating} evaluates the topology of all
offspring networks. Criteria are the number of connections,
the number of hidden units and the number of input units.
These numbers are multiplied by a scaling factor and divided through
the maximal values of the reference network. The result contributes
to the fitness.






  
\begin{moduledoc}{Evaluation of the learning process}{learnRating}

\item[\KeyWord{noLearnRating} \optParam{f}]~\\
This value is added to the fitness term if the network couldn't
learn the patterns properly, e.g. its mean error was higher
than specified by {\it maxtss}\\
Default: 200.0

\item[\KeyWord{epochRating} \optParam{f}]~\\
The number of learning epochs needed is multiplied by this value and added to
the fitness term.\\
Default: 0.0

\item[\KeyWord{tssRating} \optParam{f}]~\\ 
The mean error is multiplied by this value and added to the fitness term.\\
Default: 0.0

\item[\KeyWord{maxtss} \optParam{f}]~\\ 
If the mean error is lower than this value, learning is stopped.
It is necessary to decide if learning was successful or not.\\
Default: 0.5


\end{moduledoc}


The module {\it learnRating} evaluates the learning properties of
all offspring networks. It is possible to evaluate the mean error,
the number of training epochs until the mean error is lower
than a given threshold and additionally punish networks which couldn't learn
the patterns with a specified precision.


     
\begin{moduledoc}{Evaluation using the 40 - 20 - 40 method}{classes}


\item[\KeyWord{crossPattern} \optParam{name}]~\\
Name of the file which contains the set of cross validation patterns.\\
Default: -


\item[\KeyWord{hitRating} \optParam{f}]~\\
Value that is added to the fitness term, in case the network classifies a
pattern correctly.\\
Default: 0.0

\item[\KeyWord{missRating} \optParam{f}]~\\  
Value that is added to the fitness term, in case the network classifies a
pattern wrong.\\
Default: 10.0

\item[\KeyWord{noneRating} \optParam{f}]~\\ 
Value that is added to the fitness term, in case the network doesn't classifies a
pattern.\\
Default: 10.0

\item[\KeyWord{highDesc} \optParam{f}]~\\ 
Maximal output value of the output neuron.\\
Default: 1.0

\item[\KeyWord{lowDesc} \optParam{f}]~\\ 
Minimal output value of the output neuron.\\
Default: -1.0

\item[\KeyWord{decisionThreshold} \optParam{f}]~\\ 
Distance between the output of  {\it min-activation} and
the output of {\it max-activation}.\\
Default: 0.2

\end{moduledoc}

The module is useful for two state classification networks with one output neuron, i.e.,
one value reflecting a positive classification (usually 1),
and the other reflecting the negative classification (usually 0).
The values are set by {\it highDesc} and {\it lowDesc}. The distance
{\it decisionThreshold} is taken around the average. If the output is within
the {\it decisionTreshold} area it is taken as not classified.
     
\begin{moduledoc}{Evaluation using the highest output}{bestGuessHigh}


\item[\KeyWord{crossPattern} \optParam{name}]~\\
Name of the file which contains the set of cross validation patterns.\\
Default: -


\item[\KeyWord{hitRating} \optParam{f}]~\\
Value that is added to the fitness term, in case the network classifies a
pattern correctly.\\
Default: 0.0

\item[\KeyWord{missRating} \optParam{f}]~\\  
Value that is added to the fitness term, in case the network classifies a
pattern wrong.\\
Default: 10.0

\item[\KeyWord{noneRating} \optParam{f}]~\\ 
Value that is added to the fitness term, in case the network doesn't classify a
pattern.\\
Default: 10.0

\item[\KeyWord{hitThreshold} \optParam{f}]~\\ 
The value gives the threshold that needs to be reached, before classifying takes place\\
Default: 0.3

\item[\KeyWord{hitDistance} \optParam{f}]~\\ 
The value gives the distance between the output of two neurons, before a classification
counts as valid.
Default: 0.2


\end{moduledoc}


The module  {\it bestGuessHigh} tests the generalization performance of a network
and increases the fitness. It is useful for a Winner-takes-all output (1 out of n)
properties. The neuron with the highest activity is selected as winner. If its activity
distance to the next highest activated neuron is smaller than {\it hitDistance} 
the pattern is  treated as not classified.














   
\begin{moduledoc}{Evaluation using the lowest output}{bestGuessLow}


\item[\KeyWord{crossPattern} \optParam{name}]~\\
Name of the file which contains the set of cross validation patterns.\\
Default: -


\item[\KeyWord{hitRating} \optParam{f}]~\\
Value that is added to the fitness term, in case the network classifies a
pattern correctly.\\
Default: 0.0

\item[\KeyWord{missRating} \optParam{f}]~\\  
Value that is added to the fitness term, in case the network classifies a
pattern wrong.\\
Default: 10.0

\item[\KeyWord{noneRating} \optParam{f}]~\\ 
Value that is added to the fitness term, in case the network doesn't classifies a
pattern.\\
Default: 10.0

\item[\KeyWord{hitThreshold} \optParam{f}]~\\ 
The Value gives the threshold that needs to be reached, before classifying takes place\\
Default: 0.3

\item[\KeyWord{hitDistance} \optParam{f}]~\\ 
The Value gives the distance between the output of two neurons, before a classification
counts as valid.\\
Default: 0.2


\end{moduledoc}


The module  {\it bestGuessLow} tests the generalization performance of a network
and increases the fitness.
The neuron is selected which has the lowest activity, e.g., a Looser-takes-all
selection.
 If its activity
distance to the next lowest activated neuron is smaller than {\it hitDistance} 
the pattern is  treated as not classified.



    
%\input{../../src/eval/tradingSystem}   
\begin{moduledoc}{Evaluation through a cross validation set}{tssEval}

\item[\KeyWord{crossPattern} \optParam{name}]~\\
Name of the file which contains the set of cross validation patterns.\\
Default: -

\item[\KeyWord{crossTssRating} \optParam{ x } ]~\\
    factor for multiplying the mean error per pattern on the cross validation set.
    Default: 0.0

  \item[\KeyWord{crossHamRating} \optParam{ x } ]~\\
    Value to add for each wrong classified pattern.
    Default: 0.0

  \item[\KeyWord{crossHamThresh} \optParam{ x } ]~\\
    possible distance, that is allowed for the output from the target.
    A pattern is wrong classified, if at least the output of
    one output neuron differs from its target by	
    more than \optParam{x}.
    Default: 0.0
\end{moduledoc}

This module computes a simple fitness term on a cross validation set.
The mean error per pattern as well as the classification performance
can be evaluated.

         

\subsection{History}           %  %W%	%G%
%
%
%
%
%
%
%
%
%
%

\begin{moduledoc}{A simple version of saving all important informations}{histSimple}

\item[\KeyWord{historyFile} \Param{filename}]~\\
This string contains the prefix of the filename where all informations are stored.
The extension of the filename is {\it .simple  }.
\end{moduledoc}

A simple record that writes down all needed informations about the network,
e.g. learning, topology and fitness values.
Each row  contains the informations about one network.
   
\begin{moduledoc}{Saving all informations about fitness of the networks}{histFitness}

\item[\KeyWord{historyFile} \Param{filename}]~\\
This string contains the prefix of the filename where all informations are stored.
The extension of the filename is {\it .fit} for the net informations and
{\it .popfit} for the population informations.
\end{moduledoc}

The modul {\it histFitness} writes all fitness informations of a net in a 
special file (extension {\it .fit}).
Each line of the file accords to one net. 


The file with the {\it .popfit} extension contains informations about the 
fitness of whole population. 
Each line describes the several fitness values (best fitness of all members, 
worst fitness of all members and the average fitness) of the population 
at each generation.  The population file is ready for gnuplot and other
programs.

The fitness values are not computed in this modul, they are computed during 
the {\it evalution} functions and stored in special slots of the 
data structure.
  
\begin{moduledoc}{Saving all topology informations}{histWeights}

\item[\KeyWord{historyFile} \Param{filename}]~\\
This string contains the prefix of the filename where all informations are stored.
The extension of the filename is {\it .weight}.
\end{moduledoc}

A simple record that writes down all needed informations about the topology
of the network, number of weights, number of hidden units and number of
active input units. Each row contains the informations about one 
network.
  
\begin{moduledoc}{Saving all informations about the networks on the cross validation patterns}{histCross}

\item[\KeyWord{historyFile} \Param{filename}]~\\
This string contains the prefix of the filename where all informations are stored.
The extension of the filename is {\it .cross} for the network informations and
{\it .popcross} for the population informations.
\end{moduledoc}

The modul {\it histCross} writes all informations about the network in a special file 
(extension {\it .cross})
Each line of the file accords to one network. It contains information about the number 
of hits (the network classified the pattern correct), miss (the network cassified the pattern wrong)
and nones (the network did not classify the pattern in an unique way). 
It also contains the quotient  hits and misses.

The file with the {\it .popcross} extension contains informations about the whole 
population. Each line describes the several test values (best quotient of all members, 
worst quotient of all members and the average quotient) of the population 
at each generation.  The population file is ready for gnuplot and other
programs.

The values are not computed in this modul, they are computed in the evalution
functions and stored in special slots of the data structure.

This module is only useful in conjunction with  evaluation modules
using classification performance, e.g. {\it bestGuessHigh} or {\it classes}.

     
\begin{moduledoc}{Family tree}{ancestry}
  \item[\KeyWord{historyFile} \optParam{ histfile } ]~\\
    Prefix for the output file\\
    Default: enzo.hst
  \item[\KeyWord{ancestryPS} \optParam{ x } ]~\\
    If the flag is set a postscript figure of the family tree is generated.\\
    Default: NO
\end{moduledoc}
 This module writes the \verb+histID+s of each generation.
It is possible to see when each network is born and how long
it survives in the population. Offsprings which
never enter the population are not shown.
It's possible to generate a postscrip figure from the family tree
by setting the flag to \verb+YES+.
     
\begin{moduledoc}{Simple X-Window history}{Xhist}

  \item[\KeyWord{Xgeometry}
        \optParam{x} \optParam{y} \optParam{width} \optParam{height} ]~\\
    Specifies the size and position of 	the window on the terminal.
    The top left corner is given by  \optParam{(x,y)}. the width and height
by \optParam{width} and \optParam{height}.\\
    Default: 10 10 600 300
  \item[\KeyWord{Xcoord}
        \optParam{xll} \optParam{yll} \optParam{xur} \optParam{yur} ]~\\
	Defines the coordinate system for the graphical representation
	of the fitness values.\\
	The left lower corner is specified by \optParam{(xll, yll)}
        and the right upper corner by optParam{(xur, yur)}.\\
    Default: 0.0 0.0 30 1000.0

\end{moduledoc}

A X-Window with a coordinate system is shown. The best, average and worst
fitness values are plotted vs. generation number. The size of the window
and the coordinates are adjustable by the user.
This module uses the \verb+xgraf+ program, located in the directory \verb+ENZO/src/history/Xgraf+. To compile xgraf type \verb+ make xgraf + in the \verb+ENZO+ directory.
        
%\input{../../src/history/Committee} 
%\input{../../src/history/CrossTS}   
\begin{moduledoc}{Used input units}{histInputs}
  \item[\KeyWord{historyFile} \optParam{ x } ]~\\
    This string contains the prefix of the filename of the output file.\\
    Default: {\tt enzo}
\end{moduledoc}
The modul {\it histInputs} writes for each input unit of a net whether it is
active or not.  The file for the output is named by the prefix
{\it historyFile} and the extension {\it .inputs} .

The information for each network consist of one row, where all the informations are
written down. A {\it D} means that this input unit is absent in the net, a
{\it X} means that the input unit is active.


 



\subsection{Survival}        % %W%	%G%
%
%
%
%
%
%
%
%
%
%




\begin{moduledoc}{Survival of the fittest}{fittestSurvive}
  
\item[\KeyWord{NoOfOffsprings} \Param{k}]~\\
This parameter gives the number of offsprings which are to be inserted
in the population. 
offsprings \Param{k}.

  
\end{moduledoc}

This module inserts better offsprings into
the population  and removes worse parents.
The individuals are sorted by their fitness, the lower the better.
   


\subsection{Post-evolution}       %  %W%	%G%
% 
%
%
%
%
%
%
%
%
%
%
%



\begin{moduledoc}{Storing networks after evolution}{saveAll}
  
\item[\KeyWord{netDestName} \Param{filename}]~\\
Sets the prefix of the filename. Networks are written to files with the
prefix followed by a number for each network and the suffix .net, e.g. {\it filename}\verb+_1.net+ .\\

\item[\KeyWord{saveNetsCnt} \optParam{cnt}]~\\
 Number of networks, which are to be stored.\\
  Default: 99
  
\end{moduledoc}

Saves the given number of  networks  at the end of the evolution. Usually this are
parent networks and the reference net. The networks are numbered with increasing
fitness, e.g. the best network is {\it filename}\verb+_0.net+



\subsection{Sample module}      %  %W%	%G%
%
%
%

\begin{moduledoc}{My\_module title}{mymodule}
  \item[\KeyWord{initialize} \optParam{ x } ]~\\
    Description of this parameter.\\
    Default: -
  \item[\KeyWord{exit} \optParam{ x } ]~\\
    Description of this parameter.\\
    Default: -
  \item[\KeyWord{myParam} \optParam{ x } ]~\\
    Description of this parameters.\\
    Default: -
\end{moduledoc}

Here should follow a description of the functionality of
the whole module: What is it for, when to use an when not to use
it, etc.






\section{Adjusting parameters  \label{parameter}}

You should not be worried about the amount of adjustable parameters.
Most of same are easy to handle, in a way that they have sensible
default values and modifications have little influence on the result.
Still for some problems it might be useful to have the opportunity to
tailor the algorithm in a certain way.

Some parameters need to be set in an intelligent way, i.e., you should
take time and use your knowledge about the problem to adjust them.

Firstly, the parameters of the local optimization  depend heavily on the
problem. That is the mean error {\it maxTss} and the number of epochs
{\it maxEpochs} should be set to values that provide a good solution.
Note that since (in case of our mutation operators) offsprings have some knowledge of their parents,
they need to be trained significantly less\footnote{Since the networks usually are smaller and due to the knowledge transformation they possible speedup is in the range from 10 to 50.}.
If weight decay is used, it should be adjusted ina a way that no overfitting occurs. 

Secondly, the design of the fitness function is important, because
that's our optimization criteria. You should compute all fitness terms
for your reference net (the maximal topology)  and give those higher weights,
that you care about. Be aware that some constraints are maintained, e.g.,
if a network can't learn the training patterns, its fitness should reflect this
clearly. 

The size of the population, the number of offsprings and the number
of generations should be  in a sensible range.
The more generations the better the result (with respect to your fitness function !).
The bigger the population and the higher the number of offsprings created in each
generation, the wider the exploration.
For a given amount of time, you need always to decide the relation of
exploration to exploitation, e.g. creating many offspring each generation vs. creating fewer offsprings but   use more generations.

Sensible values are, if possible, at least 30 generations,
30 networks in the parent population and creating 10 offsprings
each generation.

The probabilities of mutation should be set in a way that at the most $1\%$ to $10\%$
of the links resp. units are mutated. Otherwise the offsprings will loose most of
the knowledge of the parents. 

\subsection*{Acknowledgements}

Several students made valuable contributions to the development of 
ENZO by studiing the evolution of neural networks in their master
thesis \cite{weisbrod92,zagorski94,schaefer94,schubert95}. 
This implementation of \ENZO\ goes back to
the work of \cite{schaefer94,schubert95}.






