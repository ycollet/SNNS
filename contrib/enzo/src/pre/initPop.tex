\begin{moduledoc}{Create an initial population}{initPop}
  \item[\KeyWord{gensize} \optParam{ x } ]~\\
    This parameter sets the maximal number of networks in the parent population.\\
    Default: {\tt POP\_SIZE\_VALUE} (30)
  \item[\KeyWord{popsize} \optParam{ x } ]~\\
    This parameter sets the number of  offsprings to create each generation.\\
    Default: {\tt OFF\_SIZE\_VALUE} (10)
  
  \item[\KeyWord{network} \optParam{ x } ]~\\
    This string contains the filename of the reference net.
    Each created net in the population gets the same topology structure
    as the reference net.\\
    Default:  enzo.net
  
  \item[\KeyWord{initFct} \optParam{ x } ]~\\
    This string contains the name of the SNNS init-function.
    The starting values of the weights and biases will be set by this function.\\
    Default: Randomize\_Weights
  \item[\KeyWord{initParam} \optParam{ x } ]~\\
    These 5 parameters contains the parameters for the init-function.
    For the meaning of these parameters please see the SNNS manual.\\
    Default: -1.0 1.0 0.0 0.0 0.0
\end{moduledoc}

The module {\it initPop} loads the reference net (via SNNS) and copies this
net to all members of the parent population. After that all networks of the population
are initialized with the SNNS initial function.

\algo{9cm}{initPop}
{
Load the reference net;\\
Set the names of all units in the reference net;\\
{\bf forall} (members of the starting population) {\bf do}\\
\hspace*{0.5cm}Copy the reference net to the new net;\\
\hspace*{0.5cm}Initialize the net with {\it initFct} and {\it initParam};\\
\hspace*{0.5cm}Set the {\it initFct} of the net to {\it ENZO\_noinit};\\
}





