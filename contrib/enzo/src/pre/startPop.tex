\begin{moduledoc}{Create a population of networks from one special network}
       {startPop}
  \item[\KeyWord{popsize} \optParam{ x } ]~\\
    This parameter sets the number of elements in the population.\\ 
    Default: {\tt POP\_SIZE\_VALUE} (30)
  \item[\KeyWord{network} \optParam{ x } ]~\\
    This string contains the filename of the reference network.\\
    Default:  {}              
  \item[\KeyWord{startnet} \optParam{ x } ]~\\
    This string contains the filename of the master network, which
    structure is copied to all the other networks.\\
    Default: {}
  \item[\KeyWord{initFct} \optParam{ x } ]~\\
    This string contains the name of the SNNS init-function.
    This function is used to initialized all created networks.\\
    Default: {\tt ENZO\_noinit} 
  \item[\KeyWord{initParam} \optParam{ x } ]~\\
    This five parameters contains the values for the function parameters of the 
    SNNS init-function {\it initFct}. 
    For the meaning of the parameters please see the SNNS manual.\\
    Default: -1.0 1.0 1.0 0.0 0.0
\end{moduledoc}

The module {\it startPop} loads the reference network and a special master network.
The topology of the master network is copied to all the other networks in the start-population.
Afterwards all weights and biases of the networks will be initialized with the
{\it initFct}. 
The idea is to take a good network to initialize the genetic search.
Another possibility is to overcome the limitation of the maximal topology by
using a much bigger reference network than master network, with the master network
maybe untrained. If the master network is already locally optimized, one should use
the initialization function {\it ENZO\_noinit}.  

\algo{11cm}{startPop}{
Load the reference and the master net;\\
{\bf forall} (Elements in the start-population) {\bf do}\\
\hspace*{0.5cm}Copy the master network to the network;\\
\hspace*{0.5cm}Initialize the networks randomly with {\it initFct} and {\it initParam};\\
\hspace*{0.5cm}Set {\it initFct} to {\it ENZO\_noinit};
}





