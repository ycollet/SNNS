\begin{moduledoc}{Mutation of the input units}{mutInputs}
  \item[\KeyWord{probMutInputs} \optParam{ x } ]~\\
    The parameter $x$ indicates the probability $p_{mut}$ that  
    a mutation takes place.\\
    Default: 0.5
  \item[\KeyWord{probMutInputsSplit} \optParam{ x } ]~\\
    The parameter $x$ indicates 
    the relationship $p_{split}$ between inserting and deleting of 
    input units.\\
    Default: 0.5 
  \item[\KeyWord{initRange} \optParam{ x } ]~\\
    The parameter $x$ determines the interval  [-range, range] where
    inserted weights are randomly selected from.\\
    Default: 0.5
\end{moduledoc}
The module {\it mutInputs} executes a mutation only of the input units.
The maximum number of deleted or inserted input units is limited by one.
All other mutation and optimization modules can only delete units, they can
never insert weights to a dead input unit.

The internal structure of SNNS does not allow to delete the input units like
the hidden units, so the input units are just deactivated.
In this case instead of deleting an activated input unit, all weights will be deleted.
They are marked with the unit name \verb+xxx+. 
If you analyze the network with the graphical user interface of SNNS,
select in the display setup \verb+show name+ to easily identify removed input units.
In the case of inserting a deactivated input unit, all possible weights will be inserted.


\algo{10cm}{mutInputs}{
{\bf if} ( RAND(0,1) $> p_{mut}$ ) {\bf then}\\
\hspace*{0.5cm}{\bf if} ( RAND(0,1) $> p_{split}$ ) {\bf then}\\
\hspace*{1.0cm}insert all possible weights of the deactivated input unit\\
\hspace*{0.5cm}{\bf else}\\
\hspace*{1.0cm}delete all weights of the activated input unit\\
}


