\begin{moduledoc}{Learning stopped by periods or cross validation error}{learnCV}
  \item[\KeyWord{learnfct} \optParam{ x } ]~\\
    This parameter $x$  contains the name of the SNNS-learning function.\\
    Default: Rprop
  \item[\KeyWord{learnparam} \optParam{ x } ]~\\
    This array of parameters indicates the parameter for the SNNS learning function.
    The meaning of the parameters can differ from learning function to learning function.
    For details please see the SNNS manual.\\
    Default: 0.0 0.0 0.0 5.0 0.0
  \item[\KeyWord{maxepochs} \optParam{ x } ]~\\
    This parameter $x$ contains the maximum number of periods for the learning 
    algorithm. After this maximum the module {\it leranSNNS} will automatically stop the 
    learning function.\\
    Default: 50
  \item[\KeyWord{CVepochs} \optParam{ x } ]~\\
    This parameter $x$ sets after how often  the error on the cross validation set for
cross validation is computed.\\
    Default: 2
  \item[\KeyWord{shuffle} \optParam{ x } ]~\\
    This switch indicates whether the sequence of the learning patterns is changed after
    each learning period or not.
    If the switch is turned on, then the module {\it learnCV} will use the
    SNNS-function {\it shuffle}\\
    Default: yes
\end{moduledoc}


The module {\it learnCV} uses the error an a cross validation pattern set to stop learning.
If it increases again learning is stopped. 
The module stops the learning of an offspring network if the error on a cross vailidation set starts
increasing again  or it reaches the maximum number of learning periods. This is done
by computing the error each {\it CVepochs}. If the current error is larger than the average
of the last four values, learning is stopped.


\algo{12cm}{learnCV}{
Set the SNNS learn function {\it learnFct};\\
{\bf for all} (networks of the offspring population) {\bf do}\\ 
\hspace*{0.5cm}{\bf while} (Epochs $<$ {\it maxEpochs}) {\bf and} 
               ( $ tss(t) < \sum_{k=t-4}^{k=t-1} tss(k) $) {\bf do}\\
\hspace*{1.0cm}Train one period with {\it learnParam};\\

}

